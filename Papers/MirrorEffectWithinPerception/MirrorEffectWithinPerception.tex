\documentclass[jou,apacite]{apa6}

\usepackage{hanging}

\title{Exploring the extensiveness of the Mirror Effect to the application of SDT model out of Recognition Memory}
\shorttitle{APA style}

\twoauthors{Adriana F. Ch\'{a}vez}{J.M. Ni\~{n}o}
\twoaffiliations{National Autonomous University of Mexico}{National Autonomous University of Mexico}

\abstract{Within recognition memory studies where Signal Detection Theory has been applied to describe subjects’ performance, a pattern of responses known as the Mirror Effect has shown that when comparing subjects’ performance between classes of stimuli that are differentially recognized, this difference appears both for the identification of known and new items. However, the extensiveness of this pattern to other fields has not been explored yet. By using what is known about the Ebbinghaus illusion to design two levels of discriminability, evidence of the Mirror Effect in a detection task, confidence ratings included, that involves perception only is shown.}

\rightheader{APA style}
\leftheader{Author One}

\begin{document}
\maketitle 

\section{Introduction}

El mundo está cargado de ruido e incertidumbre. Los organismos están constantemente expuestos a distintas fuentes de estimulación en su entorno que pueden, o no, dar información relevante sobre el estado de las cosas y las reglas vigentes. Por ello, uno de los primeros grandes problemas de adaptabilidad a los que se enfrentan los organismos es el de ordenar el caos resultante, definiendo relaciones de contingencia que les permitan hacer predicciones sobre la disponibilidad de sucesos biológicamente importantes y ajustar su comportamiento a las restricciones operantes. Una vez establecida la relación entre la presencia u ocurrencia de ciertos estímulos y el acceso a ciertas consecuencias, la detección de éstos se vuelve una tarea importante para que los organismos puedan guiar su comportamiento, (por ejemplo: \textit{'Sé que soy alérgico a las nueces, ¿En este panqué hay nueces? Si hay nueces en el panqué, no me lo como; si no hay nueces en el panqué, sí me la como'}).\\


%La detección de ciertos eventos no es una tarea sencilla. La información a evaluar suele ser ambigua.
La detección no parecería ser un problema importante si asumiéramos que los eventos aparecen con perfecta claridad, o bien, que el organismo interesado en su detección cuenta con sensores altamente precisos que le garantizan el éxito. Sin embargo, la evidencia a partir de la cual juzgamos si algo está o no ocurriendo por lo general es confusa y puede llevarnos a emitir juicios erróneos. A manera de ejemplo cotidiano, imaginemos el caso de un adolescente que quiere conseguir permiso para ir de fiesta y necesita encontrar el momento ideal para pedírselo a su mamá (cuando ella esté de buen humor); los indicadores con que cuenta son imprecisos -los gestos, el tono de voz, las actividades que su madre realice durante el día, etc.- y errar en el diagnóstico del estado emocional de su madre y en consecuencia no obtener el permiso deseado al pedirlo en el momento inadecuado es un riesgo latente, ya sea por una mala lectura de los datos disponibles -que las ansias del adolescente por salir de fiesta le hagan apresurar el momento- o bien porque los datos en sí mismos son poco claros -la mamá podría ser una persona particularmente inexpresiva o, por el contrario, terriblemente variable-.\\ 

%La detección es distinta de la discriminación y la categorización. Todos problemas importantes en un entorno cargado de estimulación. (El problema del embudo)
A diferencia de problemas tales como la discriminación o la categorización, donde la tarea de los organismos es evaluar la evidencia que se les presenta para asignarle una etiqueta (tales como \textit{'¿Es A o es B?'} ó \textit{'De acuerdo a sus propiedades en tales dimensiones, se trata de un caso de...'}), cuando hablamos de un problema de detección nos referimos a situaciones que pueden plantearse en términos de \textit{preguntas Sí/No}, (por ejemplo, \textit{'¿La comida está buena? Sí/No'}, \textit{'¿Ese que viene es mi camión? Sí/No'}, \textit{'¿Este perro es hostil? Sí/No'}) y cuya respuesta permite guiar el comportamiento de los organismos en función de las consecuencias anunciadas (\textit{'Sí, la comida está buena, me la comeré porque es seguro'}, \textit{'No, ese no es mi camión, no me subiré porque acabaré en Ecatepec'}, \textit{'No, no es un perro hostil, puedo acariciarlo'}, etc.).\\ 

\subsection{Signal Detection Theory}

La Teoría de Detección de Señales (SDT en ingles) presenta un modelo estadístico que permite describir las tareas de detección como un problema de decisión al que tienen que enfrentarse los organismos como sistemas inmersos en entornos con incertidumbre -es decir, entornos dinámicos que cambian en el tiempo y presentan variabilidad en la disponibilidad y restricción de ciertos eventos- y que buscan guiar su comportamiento de la manera más óptima posible dada la estructura del mismo. Se trata de uno de los modelos más sólidos y ampliamente estudiados en Psicología, cuyos supuestos son lo suficientemente generales para permitir su aplicación al estudio de distintos fenómenos. Funciona tanto como un modelo estadístico para describir esta clase de problemas, como una herramienta para interpretar la ejecución de  sistemas evaluados experimentalmente y hacer inferencias sobre la precisión con que el estímulo a detectar se distingue del Ruido y la posible preferencia del sistema a responder en favor o en contra de la misma.\\

La SDT le concede a la noción de variabilidad un papel fundamental para entender la detección de señales como un problema de adaptabilidad para los organismos. La idea básica es que las señales cuya detección resulta relevante para los organismos -al igual que cualquier otro estímulo- suelen presentarse y percibirse con cierta variabilidad y además de ello, coexisten en el mundo con otros estímulos (el Ruido) que dentro de su propia variabilidad pueden llegar a presentarse o percibirse con la misma evidencia que lo haría una Señal, pudiendo ser confundidos con la misma. Por ejemplo, imaginemos que queremos detectar si la persona que nos acaba de contestar el teléfono es un adulto. Existe un rango de 'tonos de voz' que estaríamos dispuestos a admitir como pertenecientes a una persona mayor de edad que, a grandes rasgos, tiende a ser más grave de lo que esperaríamos escuchar en un niño. Sin embargo, sabemos que hay adultos que pueden llegar a tener una voz particularmente clara y que puede llegar a confundirse con la de un menor de edad; si la persona que nos contestó el teléfono resulta ser un varón de voz aguda, no podremos estar seguros de si se trata de un adulto -o no- con sólo escuchar su voz y probablemente necesitemos consultarle de manera explícita.\\

La SDT es un modelo de decisión que asume que los organismos no detectan los elementos relevantes de su entorno en respuesta directa a la estimulación que reciben de manera inmediata, sino que ponderan la evidencia con la información que poseen sobre el mismo. Bajo esta visión, los organismos \textit{eligen} el juicio de detección que les permite guiar su comportamiento de la manera más óptima posible tomando en consideración 1) las ganancias y pérdidas en juego (que hacen más, o menos, riesgoso el cometer cierto tipo de error y más, o menos, atractivo el cometer un acierto en particular) y 2) la probabilidad con que dichos eventos se presentan en su entorno.  La noción de los umbrales ampliamente desarrollada en la Psicofísica clásica es reemplazada por la idea del criterio de elección.\\ 

\subsection{SDT in Recognition Memory and the Mirror Effect}

Al aplicar el modelo de Detección de Señales a tareas de Memoria de Reconocimiento, donde los participantes tienen que identificar los elementos ya antes vistos (la Señal) dentro de un conjunto de ítems que incluye elementos presentados en una fase previa y elementos nuevos (el Ruido), se ha encontrado consistentemente un patrón de respuestas cuando se compara la ejecución de los participantes entre dos clases de estímulos, (siendo una de ellas más fácil de reconocer (A) que la otra (B)), que demuestra que los participantes no sólamente son mejores reconociendo los elementos previamente mostrados en la condición A ($Hits(A)>Hits(B)$), sino que también son mejores identificando los estímulos nuevos dentro de esta misma condición ($F.alarm(A)<F.alarm(B)$). Dado que en estos estudios los participantes experimentales no saben que se ha incluido más de una clase de estímulos en la tarea que se les presenta, se asume que utilizan un sólo criterio de elección para emitir sus respuestas y de acuerdo a las tasas reportadas de Hits y Falsas Alarmas por cada clase, se sugiere que las distribuciones de Ruido y Señal de cada clase se distribuyen a lo largo del mismo eje de evidencia de tal forma que parecieran reflejarse entre sí. Por ello, en la literatura en Memoria de Reconocimiento se ha identificado dicho patrón de respuestas bajo el nombre de Efecto Espejo.\\

%3.5*42.5
El Efecto Espejo sólo ha sido estudiado dentro del dominio de la Memoria de Reconocimiento, donde se ha reportado evidencia de su existencia a lo largo de una amplia variedad de procedimientos (tareas Sí/No, tareas de Elección Forzada entre dos Alternativas y protocolos con Escala de Confianza) y variables (palabras comunes vs palabras extrañas; estímulos abstractos vs estímulos concretos; palabras escritas al revés vs palabras bien escritas; imágenes a color o en blanco y negro, etc). Como resultado, gran parte de los modelos y teorías desarrollados para dar cuenta de este fenómeno tienden a hacerlo en términos de la estructura propia de las tareas de reconocimiento, donde se incluye una fase de estudio en la que se asume que los participantes procesan los estímulos para añadirles la \textit{'familiaridad'} necesaria para poder reconocerlos en una segunda fase y donde se asume que tienen origen las diferencias observadas en el desempeño de los participantes -en otras palabras, se asume que las clases de estímulos puestas a prueba son procesadas y atendidas de distinta manera durante la fase de estudio previa a la tarea de reconocimiento-.\\

El interés principal del presente trabajo de tesis fue explorar la generalizabilidad del Efecto Espejo, buscando evidencia del mismo en una tarea de detección ajena a la Memoria de Reconocimiento. Para ello, se presentan dos variaciones de una tarea de detección perceptual (visual) que emula la estructura de los estudios donde dicho fenómenos ha sido reportado en Memoria, construyendo dos niveles de dificultad con base en la literatura en Ilusiones Ópticas. La tarea propuesta fue presentada a los participantes a partir de dos protocolos: una tarea Sí/No y la asignación de un puntaje en una Escala de Confianza. Los resultados e implicaciones de los mismos se discuten en detalle.\\


\section{Method}

\subsection{Experiment 1: A (perceptual) detection task}

\subsection{Experiment 2: A face recognition task}

\section{Results}

\subsection{Experiment 1}

\subsection{Experiment 2}

\section{Discussion}

En el presente trabajo se presentaron los resultados obtenidos en dos experimentos desarrollados como variaciones en la presentación de una misma tarea de detección perceptual (visual), donde se incluyeron dos clases de estímulos A y B construidas con base en una revisión de la literatura en ilusiones ópticas para que representaran dos niveles de discriminabilidad ($d'(A)>d'(B)$) entre los cuales pudiera compararse el desempeño de los participantes. En los experimentos propuestos, los participantes tuvieron que registrar sus respuestas al problema de detección a través de dos protocolos: una tarea Sí/No y una tarea con Escala de Confianza. El diseño experimental propuesto fue elaborado con el propósito de emular la estructura de los estudios en Memoria de Reconocimiento donde se ha reportado evidencia del fenómeno ahora conocido como Efecto Espejo; una regularidad en los patrones de respuesta observados en tareas de reconocimiento con dos clases de estímulos que difieren en la precisión con que sus elementos son reconocidos y que sugiere un orden simétrico en el despliegue de las distribuciones Ruido y Señal de cada clase sobre el eje de la evidencia.\\

Los patrones identificados en la literatura del Efecto Espejo en Memoria de Reconocimiento (el único dominio donde ha sido estudiado) fueron hallados en las tareas perceptuales aquí presentadas, en una proporción significativa contra el azar (entre el $85\%$ y el $90\%$ de los datos analizados).\\

Para validar la pertinencia del análisis de los datos obtenidos en los experimentos realizados como evidencia de la generalizabilidad de Efecto Espejo fuera de la Memoria de Reconocimiento, se realizó un análisis \textit{comprobatorio} que tuvo como objetivo evaluar la eficacia de la manipulación experimental propuesta con base en la literatura para el diseño de las clases de estímulos a comparar. De acuerdo con lo que se esperaba, los estímulos Señal de la clase A propuesta mostraron ser consistentemente más fáciles de detectar respecto del Ruido(A) que los estímulos Señal de la clase B. La robustez de esta afirmación a la luz de los datos fue confirmada tanto mediante la realización de pruebas t de muestras independientes para cada uno de los experimentos realizados, como a partir de la construcción de un modelo bayesiano jerárquico identificado como Modelo Delta, que incorporó al modelo bayesiano estándar de detección de señales una estructura jerárquica sobre los parámetros $d'$ y $c$ y un parámetro determinista encargado de evaluar las diferencias entre las $\mu d'$ estimadas por cada clase de estímulos.\\

Una vez establecida la validez del diseño experimental propuesto como una emulación de las tareas desarrolladas en Memoria de Reconocimiento donde se presenta evidencia del Efecto Espejo, se prosiguió a evaluar la significancia de los patrones de respuesta registrados en los experimentos realizados. El análisis de datos se realizó tanto mediante la replicación de los análisis frecuentistas reportados en la literatura, como mediante la construcción de modelos y la conducción de pruebas estadísticas bayesianas.\\

Primero, en cuanto a la evaluación de las diferencias encontradas entre las tasas de ejecución registradas por cada clase de estímulos, se realizaron un par de pruebas t de muestras independientes por cada Experimento y se desarrolló un modelo bayesiano -identificado como Modelo Tau- que agregó un par de parámetros deterministas $\tau$ al modelo bayesiano estándar, para computar las diferencias entre las tasas de Hits y Falsas Alarmas estimadas. En este primer punto de análisis se pudieron detectar algunas dferencias en la lectura de las conclusiones extraídas a partir de los datos y su robustez derivadas de las discrepancias entre el funcionamiento de los análisis frecuentistas y bayesianos realizados:\\

\begin{enumerate}
\item El análisis frecuentista se desarrolla en torno a la comparación de las tasas de ejecución registradas por los participantes ante cada clase de estímulos, definidas de forma determinsita a partir del número de Hits y Falsas Alarmas obtenidos dentro del total de ensayos con Ruido y Señal.\\

Las pruebas t realizadas para evaluar la significancia estadística de las diferencias encontradas entre los promedios de las tasas computadas, se enfrentan con el problema de que los datos a comparar se encuentran restrigidos dentro de un rango de $0$ a $1$. Más aún, de acuerdo con los supuestos de la SDT sobre el despliegue de las distribuciónes de Ruido y Señal sobre el eje de evidenncia, las tasas de ejecución se presentan dentro de rangos aún más restringidos, siendo que los aciertos (Hits y Rechazos Correctos) caen por arriba del $0.5$, y los errores (Omisiones y Falsas Alarmas), por debajo. Esto representa un problema para la comparación de los resultados encontrados, porque puede darse el caso de que las diferencias entre los datos registrados no sean lo suficientemente grandes como para considerárseles significativas, con independencia de si se presentan de forma consistente en la mayoría de los datos individuales.\\

Para solucionar el problema del efecto de suelo y techo que podría estar mermando la evaluación de los datos obtenidos como evidencia del impacto de la manipulación experimental sobre el desempeño de los participantes, los datos crudos a comparar suelen transformarse a una nueva Escala, de manera que la distancia entre estos aumente, pero manteniendo la proporción de los mismos intacta. En el caso específico de la literatura que aborda el Efecto Espejo en Memoria de Reconocimiento, se ha optado por implementar una transformación arcoseno de las tasas de respuesta registradas antes de someterlas al análisis estadístico.\\

De acuerdo con las pruebas t realizadas en el presente estudio para comparar los promedios de las transformaciones arcoseno de las tasas de ejecución registradas por cada clase de estímulo, se encontraron diferencias significativas en la ejecución de los participantes a la tarea Sí/No dependientes de la clase de estímulo a evaluar en ambos experimentos, reportándose diferencias altamente significativas entre las tasas de Hits y diferencias apenas significativas (justo por debajo de $p=0.05$) para las tasas de Falsas Alarmas.\\

\item El modelamiento bayesiano del problema de detección de señales abandona la interpretación determinista de las tasas de ejecución calculadas a partir de los datos registrados como reflejo directo del área de las distribuciones de Ruido y Señal que caen por encima del criterio de elección. En su lugar, asume que dicha CDF debe ser estimada como una probabilidad oculta que permea la observación del total de Hits y Falsas Alarmas por participante.\\

El Modelo Tau presentado en este trabajo parte del modelo bayesiano estándar de detección de señales, que estima el valor de las CDF de las distribuciones de Ruido y Señal que caen por encima del criterio de elección a partrir del número de Hits y Falsas Alarmas observados en cada participante, y añade un par de paráetros $\tau$ que computan las diferencias entre las probabilidades ocultas estimadas tras la emisión de Hits y Falsas Alarmas. Las estimaciones realizadas por el modelo acerca de las diferencias en el desempeño de los participantes para las distintas clases de estímulos (los valores de $\tau$) toman en cuenta la naturaleza probabilística del modelo de detección de señales, al asumir que los datos observados son extracciones de las distribuciones de Señal y Ruido subyacentes a la tarea. Con ello, el análisis de los datos obtenidos a la luz de los resultados arrojados por el Modelo Tau presenta una ventaja considerable sobre el análisis frecuentista, pues al tratar los datos obtenidos como resultado de un proceso probabilístico, escapa del problema de suelo y techo.\\

Los resultados arrojados por el Modelo Tau confirmaron a grandes razgos lo reportado por las pruebas t frecuentistas: la mayoría de las densidades de probabilidad posterior estimadas para los participantes de los experimentos 1 y 2 se sitúan por encima del punto de \textit{no diferencias} ($\tau = 0$). Sin embargo, y especialmente en términos de las diferencias encontradas en la emisión de Falsas Alarmas, dejó en evidencia la dispersión de las estimaciones de las $\tau$ individuales, siendo que estas presentaban valores no muy alejados de $\tau = 0$.\\
\end{enumerate}

La discrepancia en la precisión con que los análisis llevados a cabos permiten evaluar la evidencia del Efecto Espejo encontrada se atribuye a las diferencias entre la aproximación determinista adoptada por los métodos de análisis frecuentistas clásicos y el enfoque probabilístico de los métodos bayesianos. Mientras que las pruebas t frecuentista requieren de la transformación arcoseno de los datos a promediar para su análisis (las tasas de ejecución), el modelo bayesiano toma como materia prima el número total de Hits y Falsas Alarmas obtenido por los participantes y estima a partir de ellos la proporción de las distribuciones de Señal y Ruido que caen por encima del criterio. \\

Posteriormente, se evaluaron las diferencias entre los promedios de los Puntajes de Confianza asignados a los estímulos con Señal y Ruido de cada clase, mediante la realización de pruebas t de muestras independientes, frecuentistas y bayesianas. En ambos casos, se llegó a la conclusión de que los datos analizados proporcionan evidencia sobre las diferencias entre los puntajes registrados en cada clase de estímulo. En las pruebas t frecuentistas, se observaron \textit{p-values} por debajo de $0.05$ para todas las comparaciones realizadas y en las pruebas t bayesinaas, se computaron valores del Factor de Bayes que confirman la acumulación de evidencia a favor de las diferencias en el desempeño de los participantes. Sin embargo, de acuerdo con los estándares establecidos para la identificar la robustez de la información proporcionada por el Factor de Bayes, la evidencia obtenida en los dos Experimentos, en los estímulos con Señal y Ruido es apenas anecdótica (con excepción de las diferencias en los puntajes asignados a los estímulos Señal del Experimento 2, que difieren notablemente).\\

Finalmente, se revisó que no existieran diferencias en el tiempo de respuesta invertido por los participantes en cada una de las clases de estímulos. Dicha evaluación constituye uno de los análisis de control más frecuentemente reportados en los estudios de Memoria de Reconocimiento que presentan evidencia del Efecto Espejo, como una medida para garantizar que las diferencias en el desempeño de los participantes puedan atribuirse a discrepancias en la precisión con que los elementos que componen cada clase de estímulos son reconocidos y no al cuidado con que los participantes emitieron sus respuestas ante cada una. Para ello, se realizaron múltiples pruebas t de muestras independientes que evaluaron las distintas relaciones que pudieron haberse dado entre los tiempos de respuesta y el tipo de etímulos presentados en los experimentos. De acuerdo con dichas pruebas, no se encontraron diferencias significativas en el tiempo que se tomaron los participantes para responder a la tarea binaria o a la tarea con Escala de Confianza entre clases de estímulos, ni entre los estímulos con Señal y Ruido contenidos en cada clase.\\ 

Los modelos en Memoria desarrollados a partir del modelo de detección de señales, entienden las tareas de reconocimiento como instancias de un problema de detección de señales, donde los participantes tienen que detectar los estímulos que ya le habían sido presentados con anterioridad, \textit{reconociéndolos}. El Efecto Espejo, como un fenómeno exclusivamente reportado en la literatura de Memoria de Reconocimiento, ha sido abordado en la literatura tanto en términos de la  información que podría añadir a lo que se sabe sobre el funcionamiento de la Memoria de Reconocimiento, como un referente para juzgar la adecuación de los modelos en Memoria derivados de la SDT. La evidencia del Efecto Espejo encontrada en los experimentos realizados sugiere que este constituye una regularidad propia de la aplicación de la SDT en el análisis y comparación del desempeño de los participantes sometidos a un protocolo de detección con dos niveles de $d'$. A la luz de los resultados obtenidos, se enfatiza la importancia de abordar el Efecto Espejo en términos de su interacción con los supuestos planteados por la SDT y las implicaciones que trae consigo en cuanto al estudio del las situaciones de detección de señales como un problema adaptativo, común para todos los organismos como sistemas que buscan optimizar su comportamiento a traves de distintos planteamientos de dichos probleas.\\

El presente trabajo es el primero en explorar la generalizabilidad del Efecto Espejo reportado en Memoria de Reconocimiento a otras áreas dentro de la Psicología Experimental donde la SDT ha sido aplicada, siendo también el primero en presentar evidencia a favor de esta. Las implicaciones de la extensividad del Efecto Espejo a diferentes protocolos, clases de estímulos y fenómenos psicológicos en términos de su incorporación a los supuestos establecidos por la SDT deben ser revisados en investigaciones futuras. Por otro lado, también puede atribuirse al presente trabajo el ser el primero en evaluar la evidencia del Efecto Espejo mediante el uso de metodos bayesianos, con la realización de análisis estadísticos y el desarrollo de modelos. Con ello, se presenta un referente empírico sobre las ventajas que ofrece el uso de herramientas derivadas de la estadística bayesiana, especialmente en el análisis de datos obtendos en tareas donde se asume una estructura probabilística que permea en el desempeño de los participantes, como es el caso de las situaciones de detección tal y como se conciben bajo el marco de la SDT.\\ 

\section{Conclusion}
La Teoría de Detección de Señales (SDT) presenta uno de los modelos más sólidos y ampliamente desarrollados dentro de la historia de la Psicología Experimental, que permite dar cuenta de una amplia gama de situaciones donde la tarea principal de los organismos involucrados es la detección de eventos específicos en su entorno que le permitan guiar su comportamiento de manera óptima, en función a las relaciones de contingencia anunciadas. Los supuestos de dicho modelo son lo suficientemente generales para permitir su aplicación al estudio de distintos fenómenos -dentro y fuera de la Psicología-. La SDT funciona tanto como un modelo estadístico para describir la detección de señales como un problema de adaptabilidad, como una herramienta para interpretar la ejecución de sistemas evaluados experimentalmente y hacer estimaciones sobre la precisión con que las Señales se distinguen del Ruido y los posibles sesgos que conlleven a reportar su presencia o ausencia.\\

En estudios de Memoria de Reconocimiento donde el desempeño de los participantes es interpretado como una instancia de un problema de detección, bajo el marco de los principios establecidos en la SDT, se ha reportado consistentemente que al evaluar la ejecución de los participantes entre dos clases de estímulos A y B que difieren en la precisión con que sus elementos son reconocidos (donde $d'(A)$ $>$ $d'(B)$), las respuestas de los participantes sugieren la existencia de un par de distribuciones Ruido-Señal por cada clase de estímulos que se despliegan simétricamente sobre el eje de evidencia, (siendo que las distribuciones de la clase A se sitúan siempre en los extremos y las distribuciones de la clase B se localizan en un punto intermedio). Por esta razón, dichos patrones de respuesta han sido ubicados en la literatura como reflejo de un fenómeno denominado Efecto Espejo, cuyas implicaciones tanto en términos de su relevancia para el estudio de la Memoria de Reconocimiento como de la pertinencia del uso de modelos de detección de señales para dar cuenta de los fenómenos de Memoria, acapararon una parte considerable de la literatura desde su aparición.\\

La evidencia del Efecto Espejo ha sido reportada en una amplia variedad de estudios en Memoria de Reconocimiento donde se han manipulado diversas variables en la definición de las clases A y B a comparar, a través del uso de distintos protocolos de tareas de detección. El presente estudio es el primero en aportar evidencia ajena a cualquier proceso mnémico del Efecto Espejo, en un par de experimentos diseñados para enfrentar a los participantes a un problema de detección perceptual al cual tuvieron que responder a partir de dos protocolos distintos: una tarea binaria y una tarea con Escala de Confianza. El análisis de los datos recopilados íncorporó simultáneamente la replicación de los análisis frecuentistas reportados en la literatura y un análisis bayesiano impulsado por la construcción de modelos y la realización de pruebas estadísticas bayesianas. Ambas aproximaciones identificaron los resultados obtenidos en el presente trabajo como evidencia a favor del Efecto Espejo, pero sólamente el análisis bayesiano  consiguió tomar en cuenta la variabilidad de las estimaciones individuales interpretando los datos como reflejo de un proceso probabilístico -no determinista- y permitiendo una evaluación más completa de la robustez de la evidencia del Efecto Espejo presentada por estos, (por ejemplo, señalando que las diferencias entre los Puntajes de Confianza asignados durante la tarea con Escala de Confianza son apenas anecdóticas).\\

Los resultados encontrados en el presente estudio pueden ser interpretados en dos direcciones: Primero, como evidencia de que el Efecto Espejo no es un fenómeno exclusivo de la Memoria de Reconocimiento y debería ser abordado como una regularidad propia de las situaciones de detección que incorporan distintos niveles de $d'$, con base en la cual debería cotejarse el uso de los modelos de detección de señales para explicar diversos fenómenos psicológicos y no psicológicos. Segudno, como un referente empírico sobre las ventajas que presenta el análisis de datos bayesiano sobre el análisis frecuentista, en tanto que permite un mejor manejo de la incertidumbre contenida en los datos -que podría resultar particularmente útil en modelos como los de detección de señales, que conceden una importancia escencial a la estructura probabilísstica del entorno, tomando como uno de sus supuestos principales la noción de que las respuestas emitidas en situaciones de detección resultan de la interacción entre la variabilidad en el entorno y la variabilidad intrínseca a los organismos-.\\

\section{References}

\begin{hangparas}{.3in}{1}

Gallistel, C. R., Krishan, M., Liu, Y., Miller, R., \& Latham, P. E. (2014). The perception of probability. {\it Psychological review, 121(1),} 96.

Kalman, R. E. (1960). A new approach to linear filtering and prediction problems. {\it Journal of basic Engineering,} 82(1), 35-45.

Miller R.R., Barnet R.C. \& Grahame NJ (1995) Assessment of the Rescolra-Wagner model. {\it Psychological Bulletin} 117: 363–386.

Nassar, M. R., Wilson, R. C., Heasly, B., \& Gold, J. I. (2010). An approximately Bayesian delta-rule model explains the dynamics of belief updating in a changing environment. {\it Journal of Neuroscience,} 30(37), 12366-12378.

Rescorla R.A. \& Wagner A.R. (1972) A Theory of Pavlovian conditioning: variations in the effectiveness of reinforcement and nonreinforcement. In: Black AH, Prokasy WF, editors, Classical conditioning II: current research and theory. New York: Appleton Century Crofts. chapter 3. pp. 64–99. 

Ricci, M. \& Gallistel, R. (2017). Accurate step-hold tracking of smoothly varying periodic and aperiodic probability.{\it Attention, Perception, \& Psychophysics, }1-15.

Schultz, W., Dayan, P., \& Montague, P. R. (1997). A neural substrate of prediction and reward. {\it Science,} 275(5306), 1593-1599.

Speekenbrink, M., \& Shanks, D. R. (2010). Learning in a changing environment. {\it Journal of Experimental Psychology: General,} 139(2), 266.

Sutton, R. S. (1992). Gain adaptation beats least squares. In {\it Proceedings of the 7th Yale workshop on adaptive and learning systems} (Vol. 161168).

Sutton, R. S., \& Barto, A. G. (1998). {\it Introduction to reinforcement learning (Vol. 135)} . Cambridge: MIT Press.

Wilson, R. C., Nassar, M. R., \& Gold, J. I. (2013). A mixture of delta-rules approximation to Bayesian inference in change-point problems. {\it PLoS computational biology,} 9(7), e1003150.




\end{hangparas}


\onecolumn

\bibliography{sample}

\end{document}