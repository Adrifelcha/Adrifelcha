%%%%%%%%%%%%%%%%%%%%%%%%%%%%%%%%%%%%%%%%%
% Classicthesis-Styled CV
% LaTeX Template
% Version 1.0 (22/2/13)
%
% This template has been downloaded from:
% http://www.LaTeXTemplates.com
%
% Original author:
% Alessandro Plasmati
%
% License:
% CC BY-NC-SA 3.0 (http://creativecommons.org/licenses/by-nc-sa/3.0/)
%
%%%%%%%%%%%%%%%%%%%%%%%%%%%%%%%%%%%%%%%%%

%----------------------------------------------------------------------------------------
%	PACKAGES AND OTHER DOCUMENT CONFIGURATIONS
%----------------------------------------------------------------------------------------

\documentclass{scrartcl}

\reversemarginpar % Move the margin to the left of the page 
\newcommand{\MarginText}[1]{\marginpar{\raggedleft\itshape\small#1}} % New command defining the margin text style

\usepackage[nochapters]{classicthesis} % Use the classicthesis style for the style of the document
\usepackage[LabelsAligned]{currvita} % Use the currvita style for the layout of the document
\usepackage[spanish]{babel}
\selectlanguage{spanish}
\usepackage[utf8]{inputenc}
\renewcommand{\cvheadingfont}{\LARGE\color{Violet}} % Font color of your name at the top
\usepackage{hyperref} % Required for adding links	and customizing them
\hypersetup{colorlinks, breaklinks, urlcolor=Plum, linkcolor=Plum} % Set link colors

\newlength{\datebox}\settowidth{\datebox}{September 2017} % Set the width of the date box in each block
\newcommand{\NewEntry}[3]{\noindent\hangindent=2em\hangafter=0 \parbox{\datebox}{\small \textit{#1}}\hspace{1.5em} #2 #3 % D\\\\\\efine a command for each new block - change spacing and font sizes here: #1 is the left margin, #2 is the italic date field and #3 is the position/employer/location field
\vspace{0.5em}} % Add some white space after each new entry
\newcommand{\Description}[1]{\hangindent=2em\hangafter=0\noindent\raggedright\footnotesize{#1}\par\normalsize\vspace{1em}} % Define a command for descriptions of each entry - change spacing and font sizes here

%----------------------------------------------------------------------------------------

\begin{document}
\thispagestyle{empty} % Stop the page count at the bottom of the first page

%----------------------------------------------------------------------------------------
%	NAME AND CONTACT INFORMATION SECTION
%----------------------------------------------------------------------------------------
%\noindent
%\vspace{1em}

\begin{cv}{\textbf{\spacedallcaps{Adriana Felisa Chávez De la Peña}}}\vspace{1.5em} % Your name

\hrule{}\vspace{1.5em}

\noindent\spacedsmallallcaps{\textbf{PERSONAL INFORMATION}}\vspace{0.1em}\\ % Personal information heading

\NewEntry{Born in}{\textit{Mexico,}}{7 March 1993}\\ % Birthplace and date
\NewEntry{email}{\href{mailto:adrifelcha@gmail.com}{adrifelcha@gmail.com}}\\ % Email address
\NewEntry{github}{\href{https://github.com/Adrifelcha}{https://github.com/Adrifelcha}}\\ % Personal website
\NewEntry{phone (H)}{01 52 - 5686 3442}\\ %\ \ $\cdotp$\ \ (M) 015255 32051765} % Phone number(s)
\NewEntry{phone (M)}{01 52 - 55 3205 1765}\\%\vspace{1.5em}\\ %\ \ $\cdotp$\ \ (M) 015255 32051765} % Phone number(s)

\hrule{}\vspace{1.5em}

\noindent\spacedsmallallcaps{\textbf{ABOUT}}\vspace{0.1em}\\ % Personal information heading

\Description{Undergrad experimental psychologist interested in the study of Perception and Cognitive processes.}\vspace{1em} % Goal text

\hrule{}\vspace{1.5em}

%----------------------------------------------------------------------------------------
%	EDUCATION
%----------------------------------------------------------------------------------------


\textbf{\spacedsmallallcaps{EDUCATION}}\vspace{1em}

\NewEntry{2012-2016}{National Autonomous University of Mexico}

\Description{\MarginText{Bachelor of Psychology}GPA: 9.79/10.0\ \ $\cdotp$\ \ \textit{Degree pending}\ \ $\cdotp$\ \ School: Faculty of Psychology\newline 
Thesis: \textit{Estudios en Detección de Señales (Studies on Signal Detection )}\newline
Description: This thesis explored the extension of a phenomenon reported within Recognition Memory studies where Signal Detection Theory is applied to describe subjects' performance, to a perceptual detection task.\newline
Advisors: Dr.~Arturo \textsc{Bouzas Riaño} \& Rev. Dr.~Germán \textsc{Palafox Palafox}}

\vspace{1em} % Extra space between major sections


%----------------------------------------------------------------------------------------
%	SCHOLARSHIPS
%----------------------------------------------------------------------------------------

\textbf{\spacedsmallallcaps{FULL-SCHOLARSHIPS}}\vspace{1em}

\NewEntry{Oct-Dec, 2014}{Exchange Abroad Program student at \textsc{University of California, Santa Barbara}}

\Description{\MarginText{University of California,\newline Santa Barbara} Studied one-quarter at the UCSB as part of an Exchange Abroad Program, sponsored by the UNAM and by BANAMEX Group, as a high GPA benefit.}

\vspace{1em} % Extra space between major sections

%----------------------------------------------------------------------------------------
%	ACADEMIC EXPERIENCE
%----------------------------------------------------------------------------------------

\textbf{\spacedsmallallcaps{ACADEMIC EXPERIENCE}}\vspace{1em}

%\NewEntry{2015--2017}{1\textsuperscript{st} English teacher, \textsc{Instituto Mexicano Norteamericano de Cultura}}
%\Description{\MarginText{IMNAC} \\ Reference: John \textsc{McDonald}\ \ $\cdotp$\ \ +1 (000) 111 1111\ \ $\cdotp$\ \ \href{mailto:john@lehman.com}{john@lehman.com}}

\NewEntry{2014 - 2016}{\textsc{Student Council}}

\Description{\MarginText{UNAM: \newline Faculty of Psychology} Elected by the students of the Faculty of Psychology to represent them within the period of May, 2014 to May 2016.}

\NewEntry{2015--Present}{Member of Lab25, with \textsc{Dr. Arturo Bouzas}}

\Description{\MarginText{UNAM: \newline Faculty of Psychology}I have participated in the following projects\newline $\cdotp$ \underline{\textit{PAPIIT IN307214}}.\newline Project name: \textit{Adaptative Learning in Dynamic Environments} \newline $\cdotp$ \underline{\textit{PAPIME PE310016}}\newline Project name: \textit{Development of virtual tools for teaching cognitive and behavioral sciences.}}

\vspace{1em} % Extra space between major sections

%----------------------------------------------------------------------------------------
%	WORK EXPERIENCE
%----------------------------------------------------------------------------------------

\textbf{\spacedsmallallcaps{WORK EXPERIENCE}}\vspace{1em}

\NewEntry{Jun--Nov 2017}{Summer intern at \textsc{Asociación Ciencias de la Conducta} -- Mexico City}

\Description{\MarginText{ACC:\newline Asociación Ciencias de la Conducta}Worked at ACC (stands for Behavioral Sciences Association, in Spanish), applying psychometric instruments to children with learning, attentional and behavioral problems.}

\NewEntry{Jun-Jul 2017}{\textsc{El Buen Socio} --- Veracruz, México}

\Description{\MarginText{El Buen Socio}Worked on a project developed through the Metlife foundation support, in association with El Observatorio de Desarrollo Regional y Promoción Social and the Faculty of psychology, applying psychometric instruments and sociodemographic questionnaires on the people of several villages and cities across the state of Veracruz. \\ Reference: Javier \textsc{Alfaro}\ \ $\cdotp$\ \ +152 (55) 5401 6575\ \ $\cdotp$\ \ \href{mailto:javieralfa@gmail.com}{javieralfa@gmail.com}}

\vspace{1em} % Extra space between major sections

%----------------------------------------------------------------------------------------
%	TEACHING EXPERIENCE
%----------------------------------------------------------------------------------------

\textbf{\spacedsmallallcaps{TEACHING EXPERIENCE}}\vspace{1em}

\NewEntry{2015--2017}{English teacher, \textsc{Instituto Mexicano Norteamericano de Cultura}}

\Description{\MarginText{Instituto Mexicano NorteAmericano de Cultura}Several four-skill courses for undergraduate and highschool students at basic and intermediate levels. Skills covered: reading, writing, listening and speaking.}

\NewEntry{2015--Present}{Teacher Assistant of \textsc{Dr. Arturo Bouzas Riaño}}

\Description{\MarginText{UNAM:\newline Faculty of Psychology}Teacher assistant for the following courses:\newline $\cdotp$ \underline{Learning and Adaptative Behavior I} (2015) \newline $\cdotp$ \underline{Learning, Motivation and Cognition III} (2015 and 2016) \newline $\cdotp$ \underline{Research practices I, II and III} (2015-2017)}

\NewEntry{June 2017}{\textsc{Courses} --- Faculty of Psychology UNAM}

\Description{\MarginText{UNAM:\newline Faculty of Psychology}$\cdotp$ Introducción al Pensamiento Estadístico (\textit{Introduction to statistical thinking}).\newline A one-week course (4 hours a day) given by Adriana F. Chávez De la Peña and José Manuel Niño García\newline $\cdotp$ Introducción al Modelamiento Bayesiano (\textit{Introduction to Bayesian modeling}).\newline A one-week course, (4 hours a day) given by Uriel O. González Bravo, Adriana F. Chávez De la Peña and José Manuel Niño García\newline $\cdotp$ Un laboratorio en Python para Ciencias del Comportamiento (\textit{A virtual laboratory on Python for Behavioral Sciences})\newline A one-week course (4 hours a day) given by Adriana F. Chávez De la Peña, Uriel O. González Bravo and José Manuel Niño García}

%about the main basis of statistics (data recolection, data analysis, presentation of results and conclusion construction) and its use and misuse on everyday media
% which covered the basis of Bayesian modeling and its application to cognitive and behavioral sciences
% where the codes and virtual tools developed as part of a project developed within Lab25 (under the guidance of Dr. Arturo Bouzas) were presented in detail

\vspace{1em} % Extra space between major sections

%----------------------------------------------------------------------------------------

%\spacedlowsmallcaps{Publications}\vspace{1em}

%\NewEntry{January 2013}{Publication Title}
%\Description{\MarginText{Full Journal Name}Lorem ipsum dolor sit amet, consectetur adipiscing elit. Ut nisl tellus, sodales non pulvinar in, adipiscing sit amet purus. Suspendisse sed facilisis diam. Sed ornare sem nec justo adipiscing nec venenatis lectus commodo. Mauris non neque ligula. Pellentesque sed quam eu felis iaculis iaculis ac a leo. Suspendisse neque neque, placerat id adipiscing et, elementum eu sem.\\ Authors: John \textsc{Smith}, ~James \textsc{Smith}}

%\vspace{1em} % Extra space between major sections

%------------------------------------------------

%------------------------------------------------


\spacedsmallallcaps{\textbf{POSTER PRESENTATIONS}}\vspace{1em}

%\NewEntry{November 2015}{La Sensibilidad como fuente de Sesgo en una tarea de detección de señales usando la ilusión de Ebbinghaus}

%\Description{\MarginText{Simposium International of Behavior and its Applications V}\textit{About:} Classical Signal Detection Theory distinguishes between discriminability ($d'$) and system's bias ($\beta$) as two independent factors involved in the production of detection judgments. Lynn and Feldman (2014) suggested a relation between these factors that was explored by means of an experiment where the information gathered about Optical Illusions were used to construct different levels of discriminability. \\ Author: Adriana F. \textsc{Chávez De la Peña}}


\NewEntry{November 2015}{\MarginText{International Symposium of Behavior and its Applications V}\newline La Sensibilidad como fuente de Sesgo en una tarea de detección de señales usando la ilusión de Ebbinghaus}

\Description{\textit{(Sensibility as a source of Bias in a detection task using the Ebbinghaus illusion)} \newline Author: Adriana F. \textsc{Chávez De la Peña}}

%\NewEntry{Nov, 2016}{The Mirror Effect within Perception: Not another Recognition Memory study}

%\Description{\MarginText{Object, Perception, Attention and Memory meeting}\textit{About:} A perceptual detection task which uses what is known of the Ebbinghaus illusion to design two classes of stimuli -as levels of $d'$-, is presented to explore the extension of a pattern of response known as the Mirror Effect, reported within Recognition Memory studies where Signal Detection Theory has been applied.\\ Author: Adriana F. \textsc{Chávez De la Peña}}

\NewEntry{November 2016}{\MarginText{Object, Perception, Attention and Memory meeting}\newline The Mirror Effect within Perception: Not another Recognition Memory study}

\Description{Author: Adriana F. \textsc{Chávez De la Peña}}


\vspace{1em} % Extra space between major sections

%------------------------------------------------

\textbf{\spacedsmallallcaps{CONFERENCES AND SYMPOSIA}}\vspace{1em}

%\NewEntry{Nov, 2017}{El Efecto Espejo en Percepción: No es otro estudio de Memoria de Reconocimiento}

%\Description{\MarginText{Simposium International of Behavior and its Applications VI}\textit{Abstract:} The extensiveness of the Mirror Effect reported in Recognition Memory is explored by the results obtained in a perceptual task, which were assessed both through the replication of the analyses conducted in the literature and with the development of bayesian models.\\ Author: Adriana F. \textsc{Chávez De la Peña}}

\NewEntry{November 2017}{\MarginText{International Symposium of Behavior and its Applications VI}\newline El Efecto Espejo en Percepción: No es otro estudio de Memoria de Reconocimiento}

\Description{\textit{(The Mirror Effect within Perception: Not another Recognition Memory Study)} \newline Author: Adriana F. \textsc{Chávez De la Peña}}

\vspace{1em} % Extra space between major sections

%----------------------------------------------------------------------------------------

\textbf{\spacedsmallallcaps{SKILLS}}\vspace{1em}

\Description{\MarginText{Psychometry} Construction and validation of psychological instruments}
\Description{\MarginText{Communication} Teaching experience; Poster and Conferences presentations; Student council}
\Description{\MarginText{Team Work} Worked at Lab25 among several other students in different research projects}
%\Description{\MarginText{Experimental Framework}\textsc{Python} (PsychoPy)}
\vspace{1em} % Extra space between major sections


\textbf{\spacedsmallallcaps{COMPUTER SKILLS}}\vspace{1em}

\Description{\MarginText{Typesetting} \LaTeX, OpenOffice}
\Description{\MarginText{Operative Systens}\textsc{Mac}, \textsc{Windows}}
\Description{\MarginText{Data Analysis and plotting}\textsc{R} (IDE: RStudio), \textsc{Python} (IDE: Spyder), \textsc{SPSS} and \textsc{JASP}}
\Description{\MarginText{Experimental Framework}\textsc{Python} (PsychoPy)}
\vspace{1em} % Extra space between major sections

%------------------------------------------------

\textbf{\spacedsmallallcaps{OTHER INFORMATION}}\vspace{1em}

%\Description{\MarginText{\textbf{Awards}}2012\ \ $\cdotp$\ \ Grade perfection for a full year}
%\vspace{-0.5em} % Negative vertical space to counteract the vertical space between every \Description command
%\Description{2010\ \ $\cdotp$\ \ Top Achiever Award -- Commerce}
%\vspace{1em}
%\Description{\MarginText{\textbf{Communication Skills}}2010\ \ $\cdotp$\ \ Oral Presentation at the California Business Conference}
%\vspace{-0.5em} % Negative vertical space to counteract the vertical space between every \Description command
%\Description{2009\ \ $\cdotp$\ \ Poster at the Annual Business Conference in Oregon}
%\vspace{1em}
\newlength{\langbox} % Create a new length for the length of languages to keep them equally spaced
\settowidth{\langbox}{English} % Length equals the length of "English" - if you have a longer language in your list put it here
\Description{\MarginText{\textbf{Languages}}\parbox{\langbox}{\textsc{Spanish}}\ \ $\cdotp$\ \ \ Mothertongue}
\vspace{-0.5em} % Negative vertical space to counteract the vertical space between every \Description command
\Description{\parbox{\langbox}{\textsc{English}}\ \ $\cdotp$\ \ \ Advanced}
\vspace{1em} % Negative vertical space to counteract the vertical space between every \Description command
%\Description{\parbox{\langbox}{\textsc{Dutch}}\ \ $\cdotp$\ \ \ Basic (simple words and phrases only)}
%\vspace{1em} % Negative vertical space to counteract the vertical space between every \Description command

%------------------------------------------------

\Description{\MarginText{\textbf{Interests}}Cognitive and Experimental Psychology\ \ $\cdotp$\ \ Psychophysics\ \ $\cdotp$\ \ Decission Making\ \ $\cdotp$\ \ Cognitive Development\ \ $\cdotp$\ \ Evolutionary Psychology}


\begin{center}
\underline{\textbf{All probatory documents are available upon request}}
\end{center}

%----------------------------------------------------------------------------------------

\end{cv}

\end{document}