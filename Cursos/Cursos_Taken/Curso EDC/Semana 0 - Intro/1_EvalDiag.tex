%%%%%%%%%%%%%%%%%%%%%%%%%%%%%%%%%%%%%%%%%
% Cleese Assignment (For Students)
% LaTeX Template
% Version 2.0 (27/5/2018)
%
% This template originates from:
% http://www.LaTeXTemplates.com
%
% Author:
% Vel (vel@LaTeXTemplates.com)
%
% License:
% CC BY-NC-SA 3.0 (http://creativecommons.org/licenses/by-nc-sa/3.0/)
% 
%%%%%%%%%%%%%%%%%%%%%%%%%%%%%%%%%%%%%%%%%

%----------------------------------------------------------------------------------------
%	PACKAGES AND OTHER DOCUMENT CONFIGURATIONS
%----------------------------------------------------------------------------------------

\documentclass[11pt]{article}

\usepackage[utf8]{inputenc} % Required for inputting international characters
\usepackage[T1]{fontenc} % Output font encoding for international characters
\usepackage{graphics}
\usepackage{utopia} % Use the Palatino font by default
\usepackage[spanish]{babel}
\usepackage[latin1]{inputenc}
\usepackage{hyperref}


\input{structure.tex} % Include the file specifying the document structure and custom commands

%----------------------------------------------------------------------------------------
%	ASSIGNMENT INFORMATION
%----------------------------------------------------------------------------------------

% Required
\newcommand{\assignmentQuestionName}{Exploración Diagnóstica; Actividad} % The word to be used as a prefix to question numbers; example alternatives: Problem, Exercise
\newcommand{\assignmentClass}{EDC} % Course/class
\newcommand{\assignmentTitle}{Actividad\ \#1} % Assignment title or name
\newcommand{\assignmentAuthorName}{Adriana Felisa Chávez De la Peña} % Student name

% Optional (comment lines to remove)
\newcommand{\assignmentClassInstructor}{Dr. Juan Carlos Pérez Morán} % Intructor name/time/description
\newcommand{\assignmentDueDate}{Lunes,\ Septiembre\ 17,\ 2018} % Due date

%----------------------------------------------------------------------------------------

\begin{document}

%----------------------------------------------------------------------------------------
%	TITLE PAGE
%----------------------------------------------------------------------------------------

\maketitle % Print the title page

\thispagestyle{empty} % Suppress headers and footers on the title page

\newpage

%----------------------------------------------------------------------------------------
%	QUESTION 1
%----------------------------------------------------------------------------------------

\begin{question}

\questiontext{1.- Nombre:}  Adriana Felisa Chávez De la Peña\\

\questiontext{2.- Institución o Dirección de Adscripción:}\\  
Instituto Nacional para la Evaluación de la Educación (\textit{desde Mayo, 2018})\\
Dirección General para la Evaluación de Docentes y Directivos.\\
(Equipo de Educación Básica; a cargo de la Mtra. Sandra Conzuelo Serrato)\\

Facultad de Psicología, UNAM.\\
Laboratorio de Comportamiento Adaptable (Lab 25), dirigido por el Dr. Arturo Bouzas\\
\href{http://www.bouzaslab25.com/}{www.bouzaslab25.com}
\textit{Miembro activo desde Enero, 2015}.\\


\questiontext{3.-Puesto:}\\
Jefe de Proyecto A.

\questiontext{4.-Formación y Universidad de Procedencia:}\\
Licenciada en Psicología, por la Facultad de Psicología de la UNAM.

\questiontext{5.- Experiencia en proyectos de evaluación educativa de pequeña y gran escala:}\\
Muy poca.\\

\questiontext{6.- ¿En tu trabajo te asignarán o descargarán horas para realizar las actividades de este curso?}\\
Tengo entendido que, con excepción de las sesiones denominadas "presenciales", el resto de las actividades tendrán que cubrirse en los ratos libres de los que podamos disponer en la oficina, cuando baje la carga de trabajo.\\

\questiontext{7.- Días y horarios de los que vas a disponer para las actividades del curso:}\\
De lunes a viernes, podría dedicar cualquier momento en que baje mi carga de trabajo dentro de mi horario laboral (10 am a 7 pm). Los sábados y domingos aprovechaía mi tiempo libre para ponerme al corriente con las lecturas y actividades que hayan quedado pendientes durante la semana.\\

\questiontext{8.- ¿Cuentas con computadora en tu casa y/u oficina en donde podrás trabajar en las actividades de este curso? En caso de responder sí, especifica qué tipo de computadora.}\\
Sí, cuento con una computadora de escritorio (tanto en casa, como en la oficina).\\

\questiontext{9.- ¿Cuentas con algún servicio de internet en tu casa? En caso de responder sí, indica qué tipo de servicio recibes y su calidad de conectividad:}\\
Sí. Desconozco los detalles técnicos, pero me parece que es el servicio estándar de Telmex.\\

\questiontext{10.- Describe el motivo por el cual tomas este curso:}\\
Laboralmente (en términos del por qué me inscribieron al curso), como capacitación.\\

Personalmente, yo ya tenía interés e intención de participar en el curso desde que me enteré que se iba a impartir. Principalmente, porque me intrigó la sugerencia de que se trabajaría con modelos de mezclas latentes.\\

\questiontext{11.- Describe brevemente cuál es tu expectativa del curso:}\\
¡No estoy muy segura!

Por un lado, cuando leí el documento de trabajo con que se presentó el curso, me intrigó mucho la promesa de adquirir una herramienta que permitiera discernir el proceso cognitivo detrás de la emisión de respuestas individuales en un test. Inmediatamente pensé que se trabajaría con modelamiento cognitivo bayesiano, más específicamente, a partir de la construcción de un modelo de mezclas latentes. La perspectiva de aprender cómo una herramienta estadística de este tipo podría aplicarse al ámbito de la psicometría, en un escenario de impacto nacional, me emocionó -y siendo sincera, debo confesar que también me intimidó- bastante.\\

Por otro, con base en la clase introductoria, me parece que el curso tiene un tinte más bien introductorio y que posiblemente hará más énfasis en las distintas formas en que la Evaluación Diagnóstica Cognoscitiva puede aplicarse -y aportar algo nuevo- a la evaluación del sistema educativo nacional. Espero que el curso no se quede sólo en la interpretación/lectura de los resultados que puedan resultar de la EDC, sino en los principios, supuestos y métodos de análisis que operan detrás de ellos. (Espero poder decir que sé cómo hacer EDC al terminar el curso).\\

\questiontext{12.- Describe brevemente en qué y en dónde aplicarás las competencias que este curso te aportará:}\\
La respuesta más honesta que podría dar es: donde se me permita hacerlo.\\

En la Dirección donde trabajo no suelen realizarse los análisis psicométricos, sino que estos se realizan en la Dirección General de Medición y Tratamiento de Datos y nosotros nos limitamos a interpretar los estadísticos resultantes.\\

Además de ello, cabe destacar que soy personal eventual de relativamente poca antigüedad, por lo que, por más que me gustaría, siempre corro el riesgo de que no me recontraten el próximo año y no pueda formar parte de los futuros proyectos del Instituto. \\

\questiontext{13.- ¿Cuál es tu nivel de pericia de navegación en internet?}\\
Alta, espero.\\

\questiontext{14.- ¿Qué herramientas, redes sociales, aplicaciones y otros beneficios de internet para el aprendizaje manejas (por ejemplo, Docs, Tradukka, Moddle,Blackboard, Classroom, etc.)? y ¿cuál es tu nivel de dominio en cada uno de ellos?}\\
Conozco las plataformas de Schoology y Moddle, que utilizamos como materiales de apoyo para la impartición de los cursos de licenciatura del Dr. Bouzas\\

También estoy familiarizada con el uso de otros servidores que permiten -o intentan- facilitar el trabajo colaborativo, como GitHub, Slack, Open Science Framework y Overlief para trabajar con Latex.\\

\questiontext{15.- ¿Has tomado algún curso por internet en el pasado? ¿Cuál? Describe brevemente:}\\
Sí y no.\\

Estoy familiarizada con sitios como Coursera, Khan Academy, iTunes University (y ciertos canales de Youtube) donde en general he tomado distintos cursos en materia de cálculo, análisis estadístico, y modelamiento bayesiano. No obstante, en su mayoría son recursos libres por los cuales no he recibido ningún tipo de certificación.\\

\questiontext{16.- ¿Cuál es tu nivel de inglés en cuanto a lectura de comprensión y traducción?}\\
Avanzado.\\

\questiontext{17.- ¿Cuál es tu habilidad para trabajar en equipo? Describe brevemente:}\\
Con total honestidad, en términos de preferencia, me gusta más trabajar de forma individual porque me permite tener un mayor control sobre mis tiempos y la profundidad con que se desarrollan los materiales. Sin embargo, si se tiene que trabajar en equipo, creo honestamente que puedo adaptarme a ello.\\

\questiontext{18.- Describe en media cuartilla las actividades y proyectos en los que hayas participado en el tema de evaluación educativa:}\\
\answer{Entré al Instituto hace cuatro meses. Antes de ello, trabajaba como asistente de investigación en el Laboratorio de Comportamiento Adaptable -"Lab 25" para los cuates- dirigido por el Dr. Arturo Bouzas. Mis actividades ahí consistían principalmente en apoyar como adjunta en las clases del doctor, impartir cursos intersemestrales introductorios a la estadística y el modelamiento bayesiano, el desarrollo de proyectos de investigación con el objetivo de que éstos se presentaran en congresos nacionales e internacionales y coordiar un proyecto PAPIME en el que se diseñaron varias herramientas virtuales que facilitaran el aprendizaje y la enseñanza de la Psicología Experimental (ver \href{http://www.bouzaslab25.com/LabVirtual25}{www.bouzaslab25.com/LabVirtual25}).\\

En general, estoy familiarizada con el rigor del diseño experimental, el uso de diverss análisis estadísticos y la construcción de modelos bayesianos. No obstante, no había tenido oportunidad de vincular estos conocimientos con el "mundo real" hasta que entré al Instituto.\\

La experiencia que tengo en proyectos de evaluación educativa es muy poca. Desde mi llegada al INEE, he contribuído como parte de un gran equipo de trabajo, en el funcionamiento del Servicio Profesional Docente, revisando descriptores, especificaciones y reactivos. Además de ello, he trabajado en dos grandes proyectos de la dirección: primero, la adaptación de los cuestionarios aplicados en la Etapa 1 de la Evaluación del Desempeño docente a los tipos de servicio Indígena, Multigrado y Telesecundaria; y segundo, en la elaboración de la matriz de especificaciones para la construcción de las Evaluaciones Articuladas, en el ámbito de la Gestión de la enseñanza y el aprendizaje.\\}

\questiontext{19.- ¿Conoces alguna definición de EVALUACIÓN DIAGNÓSTICA? ¿Cuál?}\\
No conozco ningún referente teórico particular, sólo se me ocurre alguna definición de "sentido común": En el ámbito educativo, refiere a la valoración del nivel de dominio que se tiene sobre ciertos contenidos de interés, así como de las características personales y socioculturales que describen al estudiante, a fin de diseñar estrategias didácticas acordes con sus características.\\

\questiontext{20.- Menciona el tipo de clasificación de los tipos de evaluación que domines a la perfección:}\\
No me queda claro a qué se refiere la pregunta (y tampoco estoy segura de que exista tal cosa como un "tipo de evaluación" que yo domine a la perfección).\\

\questiontext{21.- En escala del 1 al 5 ¿cuál es tu dominio en temas relacionados con teoría de la medición, psicometría, teoría del constructo, teoría de la validez, estadística descriptiva y estadística inferencial?}\\

Teoría de la Medición - 4\\
Psicometría - 4\\
Teoría del Constructo - 3\\
Teoría de la Validez - 3\\
Estadística descriptiva - 5\\
Estadística Inferencial - 4\\

\questiontext{22.- ¿Qué temáticas conoces de teoría y técnica de la medida (TTM)?}\\
Estoy familiarizada con la Teoría de la Medida y la influencia que ha tenido en distintos modelos matemáticos dentro y fuera de la psicometría (mi tesis de licenciatura giró en torno a la Teoría de Detección de Señales desarrollada por Green y Swets, que recupera las nociones básicas de la teoría de la medida).\\

\questiontext{23.- ¿Puedes interpretar los parámetros de la TCT de un test y comprender en una curva característica del ítem los parámetros de la TRI?}\\
La respuesta corta es sí. Conozco y comprendo los supuestos que componen la Teoría Clásica de los Test, y estoy familiarizada con la lectura de ICC's, y otras curvas en general.\\

\questiontext{24.- ¿Dominas algún tipo de análisis psicométrico?}\\
Después de tres años intentando dedicarme a la investigación, he aprendido a cuidar de no autonombrarme "experto" o "con dominio de" las cosas...¡Siempre hay algo nuevo que aprender!\\

Estoy familiarizada y entiendo los marcos conceptuales y conceptos que rigen la CTT y la IRT, y estoy bastante confiada de que podría realizar cualquier análisis que se me requiriera en R. No obstante, no tengo la experticia necesaria para, por ejemplo, contribuir al desarrollo de los modelos per se.\\

Además de ello, vuelvo a enfatizar que mi experiencia en el mundo laboral (a.k.a. ``el mundo real'') es limitada y, aunque tengo los conocimientos teóricos, aún no practico del todo su aplicación a un contexto real, de escala nacional.\\

\questiontext{25.- ¿Conoces algún software o programa para el análisis psicométrico de pruebas psicológicas y educativas?}\\
En general, en el análisis de datos, tengo experiencia y preferencia por el uso de R. Pero también he trabajado con SPSS. 

\questiontext{26.- ¿Conoces los diferentes tipos de niveles de medición y sus requisitos?}\\
Sí.\\

En el nivel nominal se nombran categorías, pero no se establece ningún tipo de relación entre ellas.\\
En el nivel ordinal se nombran categorías, que siguen un orden.\\
En el nivel intervalar, se presentan valores separados por un intervalo  homogéneo.\\
En el nivel de razon, se presentan valores intervalares, que además, parten de un cero absoluto de referencia.\\

Definir el nivel de medición que caracteriza la variable latente a evaluar (a.k.a. el constructo) repercute en el tipo de supuestos con los que se puede trabajar.

\questiontext{27.- ¿Cuál es tu dominio de los estándares para el desarrollo y validación de pruebas psicológica y educativa?}\\

Meramente teórico.



\end{question}
\end{document}
