% Chapter Template

\chapter{Memoria de Reconocimiento} % Main chapter title

\label{Cap_RM} % Change X to a consecutive number; for referencing this chapter elsewhere, use \ref{ChapterX}

%----------------------------------------------------------------------------------------
%	SECTION 1
%----------------------------------------------------------------------------------------

\section{Main Section 1}

Lorem ipsum dolor sit amet, consectetur adipiscing elit. Aliquam ultricies lacinia euismod. Nam tempus risus in dolor rhoncus in interdum enim tincidunt. Donec vel nunc neque. In condimentum ullamcorper quam non consequat. Fusce sagittis tempor feugiat. Fusce magna erat, molestie eu convallis ut, tempus sed arcu. Quisque molestie, ante a tincidunt ullamcorper, sapien enim dignissim lacus, in semper nibh erat lobortis purus. Integer dapibus ligula ac risus convallis pellentesque.

%-----------------------------------
%	SUBSECTION 1
%-----------------------------------
\subsection{Subsection 1}

Nunc posuere quam at lectus tristique eu ultrices augue venenatis. Vestibulum ante ipsum primis in faucibus orci luctus et ultrices posuere cubilia Curae; Aliquam erat volutpat. Vivamus sodales tortor eget quam adipiscing in vulputate ante ullamcorper. Sed eros ante, lacinia et sollicitudin et, aliquam sit amet augue. In hac habitasse platea dictumst.

%-----------------------------------
%	SUBSECTION 2
%-----------------------------------

\subsection{Subsection 2}
Morbi rutrum odio eget arcu adipiscing sodales. Aenean et purus a est pulvinar pellentesque. Cras in elit neque, quis varius elit. Phasellus fringilla, nibh eu tempus venenatis, dolor elit posuere quam, quis adipiscing urna leo nec orci. Sed nec nulla auctor odio aliquet consequat. Ut nec nulla in ante ullamcorper aliquam at sed dolor. Phasellus fermentum magna in augue gravida cursus. Cras sed pretium lorem. Pellentesque eget ornare odio. Proin accumsan, massa viverra cursus pharetra, ipsum nisi lobortis velit, a malesuada dolor lorem eu neque.

%----------------------------------------------------------------------------------------
%	SECTION 2
%----------------------------------------------------------------------------------------

\section{El Efecto Espejo}

“is usually interpreted in terms of (unequal variance) signal detection theory (SD) in which case it implies that the order of the underlying old item distributions mirrors the order of the new item distributions” (DeCarlo, L.,  2007)\\
Teoría de Atención /Verosimilitud: Un modelo de marcaje de rasgos, determinado por un muestreo diferencial dada la condición (H-frequency, L-frequency)\\
Teoría de Atención / Verosimilitud; demasiado complicada, sus supuestos no son necesarios (Decarlo, 2007; Hintzman, 1994; Murdock, 1998) Intercambio de papers Hintzman-Glanzer\\
‘The mixture model’ (DeCarlo, 2007) – Extensión de la SDT, una extensión mezclada.\\
Between vs Within condition discussion (Listas separadas o mezcladas)\\
Between condition: Problemas (1) No se puede descartar la posibilidad de que el criterio de respuesta difiera a lo largo de las condiciones. Y (2) las distribuciones subyacentes no necesariamente están escaladas de la misma forma a lo largo de las dos condiciones.\\
“one cannot compare the values of d’ across the two conditions without further assuming that the variance of the reference distributions (LN and HN) are the same, which does not appear to be the case. (DeCarlo,2007)  
