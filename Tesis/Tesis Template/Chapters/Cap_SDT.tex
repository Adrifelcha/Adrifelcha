% Chapter 1

\chapter{Teoría de Detección de Señales} % Main chapter title

\label{Cap_SDT} % For referencing the chapter elsewhere, use \ref{Chapter1} 

%----------------------------------------------------------------------------------------

% Define some commands to keep the formatting separated from the content 
\newcommand{\keyword}[1]{\textbf{#1}}
\newcommand{\tabhead}[1]{\textbf{#1}}
\newcommand{\code}[1]{\texttt{#1}}
\newcommand{\file}[1]{\texttt{\bfseries#1}}
\newcommand{\option}[1]{\texttt{\itshape#1}}

%----------------------------------------------------------------------------------------

%Importancia del capitulo: Presentar a detalle la Teoría de Detección de Señales
En este capítulo se expone a detalle el modelo estadístico que protagoniza la presente tesis. Se trata de uno de los modelos más importantes en el desarrollo de la psicología científica y que, aún a la fecha, sigue formando parte importante del estudio de fenómenos psicológicos donde los organismos deben detectar casos particulares en su entorno antes de decidir cómo responder al mismo.\\

%Estructura del capítulo: Exposición de supuestos, parámetros y relevancia actual.
La revisión del modelo de Detección de Señales comienza por una exposición a detalle de los supuestos generales de los que parte la teoría, así como de los parámetros y la interpretación que estos tienen en términos de la descripción de tareas de detección. Una vez explicado el modelo, se aborda el papel queha tenido en el desarrollo de la psicología científica, situándolo en el contexto de los problemas a los que la disciplina intentaba dar respuesta en varios ámbitos, tales como la psicofísica y la teoría de decisión, y cómo estosfueron construyendo y dando sentido al modelo de detección de señales.

\section{El problema de la Detección y la Incertidumbre}

%Detectar ciertos estados en el mundo es importante para guiar nuestro comportamiento
Uno de los problemas más frecuentes a los que se enfrentan los organismos como sistemas inmersos en entornos variables es la detección de estados o eventos particulares. Los organismos están constantemente expuestos a distintas fuentes y tipos de estimulación, que pueden -o no- aportar información relevante sobre el estado del mundo y las relaciones de contingencia vigentes. Todo sistema que tienda a la optimización de su comportamiento debe identificar propiedades específicas en su entorno para actuar de acuerdo con las consecuencias sugeridas por las mismas. \\

%La detección es distinta de la discriminación y la categorización. Todos problemas importantes en un entorno cargado de estimulación. (El problema del embudo)
Para una mejor comprensión del modelo aquí expuesto y el esquema general de los fenómenos o situaciones cuya descripción y estudio permea, es importante definir con claridad el problema que representa la detección de casos particulares. A diferencia de problemas tales como la discriminación o la categorización, situaciones donde los organismos evaluan la evidencia que se les presenta para asignarle una etiqueta (e.g. '¿Es A o es B?' (discriminación), 'De acuerdo a sus propeidades en tales dimensiones, se trata de un caso de...' (categorización)), las tareas de detección son suceptible a parafrasearse como una pregunta 'Sí/No' (e.g. '¿La comida es buena? Sí/No', '¿Ese que viene es mi camión? Sí/No'), de cuya respuesta depende la acción que decidamos tomar, en función a las consecuencias que se anuncian (e.g. 'Sí, la comida está buena, me la comeré porque es seguro', 'No, ese no es mi camión, no me subiré porque acabaré en Ecatepec').\\ 

%La detección de ciertos eventos no es una tarea sencilla. La información a evaluar suele ser ambigua.
Detectar 'algo' no parecería ser un problema importante si asumimos que dichos casos aparecen con perfecta claridad y son inconfundibles, o bien, que los sistemas comprometidos en su detección cuentan con detectores altamente sensibles que garantizan su identificación. Sin embargo, el 'mundo real' este es difícilmente el caso. Comúnmente, la información a partir de la cual queremos juzgar si algo está o no ocurriendo es confusa y puede llevarnos a incurrir en juicios erróneos (i.e. Falsos Positivos donde se 'detectan' casos cuando no los hay, o Falsos Negativos donde se descarta la evidencia erróneamente). Como un ejemplo cotidiano, imaginemos el caso de un adolescente que quiere tener permiso para salir de fiesta y necesita encontrar el momento ideal para pedírselo a su mamá, cuando ella esté de buen humor; los indicadores con que cuenta son imprecisos (e.g. los gestos, el tono de voz, las actividades que su madre realice durante el día, etc.), equivocar el diagnóstico del estado emocional de su madre y en consecuencia no obtener el permiso deseado es un riesgo latente, ya sea por una mala lectura de los datos disponibles (e.g. que las ansias del adolescente por salir de fiesta le hagan apresurar el momento) o bien porque los datos en sí mismos son poco claros (e.g. la mamá podría ser una persona particularmente inexpresiva).\\ 

%El mundo está cargado de ruido y los organismos tienen que lidiar con la incertidumbre resultante mediante inferencias probabilísticas, relaciones causales (contingencia), experiencia. ??????????
El mundo está cargado de ruido e incertidumbre. Los organismos están siendo constantemente bombardeados por todo tipo de estimulación, dejándoles la tarea fundamental de ordenar el caos resultante, detectando las contingencias que permitan definir relaciones de causalidad, juicios de probabilidad y asignación de crédito a los comportamientos emitidos en relación a las consecuencias obsercadas, para proveer al mundo de cierto sentido. Todo este conocimiento adquirido con la experiencia sobre las reglas de ocurrencia de eventos específicos en el mundo es utilizado para compensar la incertidumbre presente en las tareas de detección, la poca claridad con que aparecen los estímulos y la incapacidad propia del organismo para evaluarlos, se compensa con la información que tiene el sistema sobre la probabilidad de que se trate de un caso u otro, y sobre las consecuencias que están en juego.\\ 

\section{Teoría de Detección de Señales}

Las nociones de la variabilidad en el entorno y la incertidumbre resultante permeando el comportamiento de los organismos, han sido uno de los ingredientes principales en el desarrollo de la Psicología como ciencia. Desde que Fechner extendiera las ideas originalmente planteadas por Gauss sobre la incertidumbre contenida en toda medición (i.e. Toda medición realizada contiene un 'error de medida' que se añade al valor 'verdadero' de lo que se quiere medir, alterando su estimado y cargándolo de incertidumbre ), al estudio de la percepción y la sensación (i.e. Nuestros sistemas sensoriales perciben las cualidades 'verdaderas' más un cierto 'error' en cada observación), se sentaron las bases que dieron lugar a una amplia gama de modelos matemáticos en psicofísica orientados a  .\\ 

La Teoría de Detección de Señales (TDS o SDT, por sus siglas en inglés) constituye un modelo sólido, que sirve como marco teórico  

se desarrolla originalmente en el ámbito del estudio de las señales eléctricas tras la Segunda Guerra Mundial  (peterson birdsall and fox 1954)

La TDS 



%La TDS admite la importancia de la incertidumbre: El ruido y la señal.
La Teoría de Detección de Señales (TDS o SDT, por sus siglas en inglés) plantea que la información que  interesa  detectar  (Sea un evento en particular o un estado o categoría más general, identificada como una señal)  suele  presentarse  en  conjunto  con  otro  tipo  de  estimulación (i.e. ruido),  cargándola  de  incertidumbre  y  haciendo de  la  percepción  un  proceso  de  toma  de decisiones  donde  el  sistema  debe  formular  un  juicio  de  detección que  le  permita  guiar  su comportamiento. Es importante precisar  que la TDS no es exclusiva del estudio de la percepción visual  u  otras  modalidades  de  detección  sensorial,  sino  que  también  puede  referirse,  en  un sentido más abstracto, a la detección de información dentro de un conjunto de datos ambiguos; (e.g.  estudios  de  memoria  donde  se  solicita  al  participante  detectar  los  elementos  que  ya  se  le habían mostrado antes, o bien, la interpretación de baterías clínicas). (Wei Ji Ma, 2012)\\

\subsection{Supuestos generales del modelo}

La TDS funciona como un modelo descriptivo, que traduce el desempeño de sistemas en tareas de detección (aciertos y errores), en inferencias sobre la precisión con que se están distinguiendo la señal del ruido (i.e. la discriminabilidad) y la posible preferencia a responder en favor o en contra de la detección (i.e. el sesgo)¿'089.\\

\begin{itemize}
  \item{Siempre hay incertidumbre}

%Aportacion de Fechner: De la teoría de Gauss a una teoría mentalista de percepción

    \begin{itemize}
      \item{Variabilidad en la Señal}\\

%

La idea de que la variabilidad forma parte fundamental de toda tarea de detección comienza situándola en la propia presentación de la señal. 

La idea central de variabilidad radica en la noción de que ningún estímulo se presenta ni se percibe exactamente igual cada vez que nos encontramos con él.  Es decir, cada vez que nos encontramos con la  señal en el mundo, ésta puede hacerlo dentro de un rango de posibilidades con cierta probabilidad. Esta idea se muestra gráficamente en la Figura 1, con la distribución normal azul identificada bajo la etiqueta de ‘Señal’. La idea es que la señal va adoptar una cierta forma de entre los puntos que abarca la distribución de probabilidad; siendo unas más probables que otras, conforme se aproximan a la media.\\

La variabilidad en la señal puede interpretarse en términos de dos fuentes: la percepción del sistema que ejecuta la tarea de detección, o la propia presentación estímulo en sí mismo. En el primer caso, se asume que cada vez qu e vemos un mismo estímulo que se mantiene constante en términos de sus propiedades físicas,  (e.g. una luz o un tono),  este puede ser percibido de manera distinta en cada presentación (i.e. unas veces parecerá un poco más intenso y otras, un poco menos). En el segundo caso, se asume que la señal puede tomar más de una forma, (e.g. si la señal es el enojo de un amigo, existen ciertos rasgos que son más o menos comúnmente asociados a su enfado; pero no siempre se va a ver exactamente igual).\\

      \item{Variabilidad en el Entorno: Ruido}

Por otro lado, es importante tomar en cuenta que las señales que interesa detectar coexisten en el mundo con otros estímulos; algunos de los cuales pueden llegar a producir una evidencia similar a la de nuestra señal y ser, por tanto, confundidos con la misma. Esta idea se representa en la Figura 1 con la distribución normal negra identificada bajo el nombre de ruido, que se traslapa con cierta probabilidad con la distribución de señal.\\


El soporte de las distribuciones, identificado en la Figura 1 bajo el nombre de ‘Evidencia’ rara vez se define con precisión,  teniendo una concepción más bien abstracta; La idea general es que cuando queremos detectar una señal particular, comenzamos a recolectar un tipo de evidencia específico a la tarea ante la que nos encontramos. Lo más importante, es que la señal siempre va a estar asociada en mayor medida con dicha evidencia, distribuyéndose siempre en valores situados por encima (a la derecha, en la Figura 1) del ruido.\\


Este primer supuesto de variabilidad, como algo inherente a todo estímulo y sistema, nos lleva a hablar de la discriminabilidad de la señal, o bien, de la sensibilidad del sistema ante la señal, que el modelo de detección de señales va a representar con un mismo parámetro: d’, que corresponde a la distancia entre las medias de las distribuciones de ruido y señal, y cuyo cómputo abordaremos más afondo más adelante con ayuda de nuestro graficador en Python.\\ 

     \end{itemize}
  \item{La detección es decisión}

La TDS define toda tarea de detección como una tarea de decisión, donde el fin último por el cual el organismo se interesa en determinar si la señal está o no presente, es el de guiar su curso de acción. Es decir, el comportamiento de cualquier organismo va a depender de las señales que este detecta en su entorno.\\
Una consecuencia directa de la variabilidad involucrada en el entorno de decisión, es que el desempeño de todo sistema de detección es propenso a cometer errores y emitir un juicio de presencia o ausencia de la señal, que puede no coincidir con el estado del mundo. Dependiendo la correspondencia entre el estado del mundo y el juicio emitido por el sistema de detección, la TDS maneja las clasificaciones de respuesta mostradas en la Tabla 1; donde las celdas 2 y 3, corresponden a los errores posibles.\\

La TDS asume que el organismo fija un criterio de elección a lo largo del eje de la Evidencia, que va a determinar a partir de cuánta evidencia va a juzgar la señal como presente. Dicho criterio se va a representar como una línea transversal que atraviesa ambas distribuciones en una determinada altura, y se le va a identificar con el parámetro k. La TDS asume que los organismos van a fijar esta regla de elección, ponderando la información a la que tienen acceso con la información que poseen sobre la estructura de la tarea (i.e. cómo suele presentarse la señal, qué tan probable es que se presente, etc.)\\

    \begin{itemize}
      \item{Los errores cuestan y los aciertos pagan: Matrices de pago}\\

      \item{Sesgo o Preferencial}\\

 Sin embargo, no todos los errores tienen el mismo costo. Imaginemos el caso de una presa en potencia que busca determinar si el sonido que acaba de escuchar en la maleza corresponde o no con el de un depredador; no hay tiempo que perder, y el costo que dicho organismo tendría que pagar por cometer una falsa alarma (gasto innecesario de energía) o una omisión (morir devorado) es sustancialmente diferente. En este escenario particular, es muy probable que la presa sea mucho más propensa a correr por su vida, juzgando la presencia del depredador a partir de valores menores de evidencia.\\

Esta discrepancia en el peso que se le da a las consecuencias posibles de emitir una u otra respuesta y obtener uno de los cuatro posibles resultados, suele representarse en términos de una matriz de pagos, que nos ayude a definir cuáles son las consecuencias que el organismo buscará evitar o promover, según sea el caso, en mayor medida.\\

Ya sea por los distintos pesos que tengan las posibles consecuencias para el organismo, o porque se tiene una preferencia o predisposición inherente a decretar la presencia o ausencia de la señal, la TDS asume que el desempeño de los organismos que se enfrentan a tareas de detección de señales va a depender tanto de la calidad de la información a la que se tiene acceso (dentro de lo que se incluye la importancia de la variabilidad, que determina tanto la discriminabilidad de la señal como la sensibilidad del sistema ante la misma), como de un sesgo de elección.\\

La localización del criterio en nuestro eje de evidencia recolectada va a estar altamente influida por el sesgo que tenga nuestro sistema. Podemos hablar entonces de dos tipos distintos de sesgo: conservador y liberal. El primero, favorece la emisión de respuestas negativas al desplazar el criterio a la derecha y requerir al sistema la recolección de mayores niveles de evidencia antes de dar por detectada la señal. El segundo, promueve la detección de la señal, situando el criterio de elección hacia la izquierda, emitiendo un juicio de detección con valores menores de evidencia. Nótese que un sistema carente de sesgo, sería aquel que situara su criterio de elección justo en el punto en que las dos distribuciones se juntan, donde la probabilidad de cometer cualquiera de los tipos de acierto y errores, son iguales entre sí.\\

Para cuantificar el sesgo del sistema, la TDS nos proporciona dos medidas: la primera de ellas corresponde a la distancia entre el punto de sesgo-neutro (i.e. el punto donde se interceptan ambas distribuciones) y la localización del criterio (c) y la segunda, a la razón entre el punto en que el criterio toca la distribución de la señal y la distribución de ruido ($\beta$). Dichos parámetros no sólo permiten saber cuán grande es el sesgo del sistema, sino que facilitan su clasificación en las categorías previamente expuestas, siendo el caso que si $\beta<1$ o C<0, sabemos se trata de un sesgo liberal y si $\beta>1$, C>0, hablamos de un sesgo conservador.\\
     \end{itemize}
\end{itemize}



La TDS ha sido utilizada para describir un amplio número de fenómenos. Cuando hablamos de detección de señales, podemos referirnos a la señal tanto como un estímulo sensorial concreto (e.g. una luz, tono, u objeto particular), como una categoría más abstracta (e.g. una enfermedad, una emoción o un estado).  (Ver la sección de lecturas recomendadas para más ejemplos).\\


\subsection{Parámetros del modelo}

Como se mencionó previamente, al realizar una tarea de detección existen dos posibles tipos de aciertos: al detectar la señal (Hits) y al rechazar el ruido (Rechazos), y dos posibles tipos de errores: los falsos positivos (Falsas alarmas) y los falsos negativos (Omisiones). La materia prima con base en la cual funciona el modelo propuesto por la TDS, son las tasas de aciertos y errores cometidos durante la tarea, de manera que por cada participante que pasa por una tarea de detección, tenemos cuatro tasas que describen su ejecución:

La Tabla 2 ilustra el cómputo de las cuatro tasas de ejecución, como una relación entre el resultado obtenido y el tipo de ensayo con base en el que se le definió como tal. Es decir, tenemos dos tasas definidas en relación al número total de ensayos con la señal (la tasa de hits y la tasa de omisiones) que nos dicen qué proporción de los ensayos con señal fueron detectados correctamente y cuáles se dejaron pasar; y tenemos dos tasas definidas en relación al total de ensayos con ruido (la tasa de falsas alarmas y la tasa de rechazos correctos) que nos describen la relación de los ensayos con ruido que fueron discriminados correctamente y aquellos que se confundieron con la señal.

Para realizar el análisis de datos, bajo el marco de la TDS, sólo necesitaremos un par de estas tasas: la tasa de hits y la tasa de falsas alarmas. Esto bajo el entendido de que las tasas de omisión y rechazos correctos no son más que su complemento, respectivamente, y que estas dos tasas contienen toda la información que necesitamos sobre el desempeño de los participantes.

La idea general de la importancia de estas tasas de ejecución, es que cada una representa el área de las distribuciones de ruido y señal que cae a la izquierda o derecha del criterio de decisión.

Fig. 2. El Graficador de Tasas de Ejecución ilustra la idea de que, dependiendo la localización del criterio de decisión que esté usando el participante, cambia la proporción de aciertos y errores que se puedan cometer en función al área bajo la curva. En la parte superior del simulador se muestra la proporción de cada distribución que cae bajo cada clasificación hecha por el modelo. El slider colocado en la parte inferior del graficador permite al usuario alterar la posición del criterio sobre el eje de decisión y alterar así la probabilidad de obtener cada outcome posible.  Con fines ilustrativos, en este graficador el valor de d’ se mantiene constante y lo único que se altera es la localización de la línea que atraviesa ambas distribuciones (y que simula al criterio de elección).

La Fig. 2 presenta una vista previa del Graficador. En ella, se puede observar cómo la distribución de señal y la distribución de ruido se dividen a ambos lados del criterio, en los aciertos y errores correspondientes. El supuesto descriptivo que hace la teoría, es que el organismo computa la evidencia que observa con la información que tiene sobre las consecuencias de cometer uno u otro posible error para colocar un criterio de elección que maximice sus ganancias, o bien, minimice sus pérdidas.  Esto se ilustra en el Graficador con el slider ubicado en la parte inferior, con el que se puede alterar la posición del criterio sobre el eje de decisión y observar los cambios en la probabilidad de cometer ciertos aciertos o ciertos errores que se dan en consecuencia.

Para la estimación paramétrica se utiliza la misma lógica, pero se sigue el procedimiento inverso. Dado que no podemos observar ni cuantificar de manera directa el criterio usado por los participantes para responder a la tarea, qué tan juntas o separadas se encuentran las distribuciones de ruido y señal para cada participante o qué tipo de sesgo pudieran estar siguiendo, utilizamos las tasas de ejecución para hacer inferencias sobre la localización del criterio, la diferencia entre las medias de ambas distribuciones y el grado en que una respuesta se favorece sobre otra. 

A partir de ahora comenzaremos a hablar sobre cómo se calculan cada uno de los parámetros del modelo, de acuerdo a la teoría clásica que sigue los supuestos estadísticos previamente descritos.  Es importante aclarar que el Graficador de Tasas previamente expuesto no representa la teoría con entera precisión; el propósito de ese primer Graficador es simplemente ilustrar cómo describe la TDS el comportamiento de un sistema que se enfrenta ante una tarea de detección, donde existen dos distribuciones que se sobreponen. El Graficador permite manipular directamente la localización del criterio, con la simpleza que implicaría desplazar una línea vertical sobre el eje de decisión y ver qué consecuencias tiene sobre la probabilidad de obtener un tipo particular de acierto o error.


Antes de ahondar a detalle en los parámetros, hay que declarar un par de supuestos formales que hace la Teoría para facilitar la representación gráfica del modelo y la estimación paramétrica:

\begin{enumerate}
\item En su forma clásica, la TDS asume que las distribuciones de ruido y señal son distribuciones normales.
\item La TDS asume equivarianza entre las distribuciones de ruido y señal. Es decir, asume que la dispersión de ambas distribuciones es la misma, fijando la desviación estándar a 1.
\item Para facilitar la estimación paramétrica, a la distribución de ruido (que por definición debe aparecer siempre a la izquierda de la señal) se le asigna una media de 0.
\end{enumerate}


\begin{itemize}
\item Discriminabilidad $(d')$



Para encontrar la distancia entre las medias de la distribución de ruido y señal, necesitamos saber el punto en que el criterio toca cada distribución. Para ello, calculamos las probabilidades complementarias a las tasas de hits y falsas alarmas y las traducimos a puntajes Z (Ver Fig. 3). Dado que el puntaje Z funciona como una medida de dispersión de la media, basta con restar el puntaje Z de la intersección del criterio con la distribución de señal a el puntaje Z de intersección con la distribución de ruido para conocer la localización de la media de la señal. Por definición, d’ sólo puede tener valores positivos ya que la teoría asume que la distribución de señal siempre está a la derecha de la distribución de ruido porque contiene una mayor cantidad de la evidencia con base en la cual se hace el juicio de detección de la señal.



\item Criterio  $k$

Una vez que hemos resumido el desempeño de nuestro participante en la tarea de detección, el parámetro cuya estimación resulta más sencilla y directa es el Criterio (k). Entender cómo se computa el parámetro nos requiere únicamente de mantener presente el supuesto de que el Ruido se distribuye normalmente y se va a localizar siempre a la izquierda de la señal, por lo que le asignamos una media de cero para tener un punto de referencia para estimar el espacio en que se desarrollan el resto de los parámetros. \\

Para calcular el criterio lo único que necesitamos es conocer la tasa de Falsas Alarmas, que tal y como mencionábamos en el segmento anterior, nos indica qué proporción de la distribución de ruido cae a la derecha del criterio. Dado que a la distribución de ruido, le fue asignada arbitrariamente una media de cero, podemos asignar un valor al punto en que el criterio corta la distribución de ruido y define las tasas de Rechazos y Falsas Alarmas obtenidas por el participante. Conociendo el área de la distribución de Ruido que cae bajo el criterio, (el complemento de la tasa de Falsas Alarmas, o bien, la Tasa de Rechazos correctos), y sabiendo que la distribución tiene una desviación estándar de 1, podemos convertir el valor de la tasa (que corresponde a la probabilidad de cometer un rechazo correcto, de acuerdo al área bajo la curva) en Puntajes Z y conocer la localización del criterio.\\

 El parámetro k, por lo general, va estar representado por un número natural (un número positivo), que indica en términos de Puntajes Z  la posición del criterio sobre el eje de decisión, relativo a la distribución de ruido con media cero. El criterio sólo tiene valores positivos, porque normalmente se espera que la tasa de falsas alarmas nunca tenga un valor mayor a 0.5 (las consecuencias de una tasa de Falsas Alarmas tan alta, se expondrán con más claridad en el apartado correspondiente a la d’. \\


\item Sesgo - $\beta$

La razón de la densid

$\beta = \frac{p(Signal)}{p(Noise)}$


\item Sesgo - $C$


\end{itemize}



Una revisión un poco más profunda en la literatura, (sobre todo en literatura más formal y especializada) nos demuestra que, si se cuenta con información suficiente, los supuestos 1 y 2 pueden violarse. Por ejemplo, en estudios de memoria de reconocimiento, donde se les pide a los participantes que discriminen entre estímulos que les fueron presentados en una etapa previa y estímulos completamente nuevos,  los resultados demuestran consistentemente que la distribución de señal (de estímulos previamente vistos), tiene una mayor varianza que la distribución de ruido; dicho eso, si se piensa utilizar la TDS como modelo de referencia para una tarea de memoria de reconocimiento, se puede hacer caso omiso del supuesto de equivarianza.\\

%----------------------------------------------------------------

\subsection{Tareas de detección}

\begin{itemize}
\item Tareas de detección binaria 

En el laboratorio, la  TDS se estudia a partir  de tareas de detección donde se expone a un  sujeto  a  N  número  de  ensayos,  (comprendidos  por  n  ensayos con  sólo  ruido  y  n  ensayos donde  el  ruido  viene  acompañado  de  la  señal)  ante  los  que  se  le  pide  al  participante  que responda eligiendo una de dos opciones: Sí está la señal o No está la señal. En estos escenarios controlados,  el  experimentador  decide  la  proporción  de  ensayos  con  y  sin  señal  que  se presentarán, así como la matriz de pagos que definirán la utilidad de sus aciertos y errores. \\


\item Tareas con escala de confianza


\item Tarea con elección forzada entre dos alternativas.
\end{itemize}

