% Chapter 1

\chapter{Teoría de Detección de Señales} % Main chapter title

\label{Cap_SDT} % For referencing the chapter elsewhere, use \ref{Chapter1} 

%----------------------------------------------------------------------------------------

% Define some commands to keep the formatting separated from the content 
\newcommand{\keyword}[1]{\textbf{#1}}
\newcommand{\tabhead}[1]{\textbf{#1}}
\newcommand{\code}[1]{\texttt{#1}}
\newcommand{\file}[1]{\texttt{\bfseries#1}}
\newcommand{\option}[1]{\texttt{\itshape#1}}

%----------------------------------------------------------------------------------------

\section{Introducción: El problema de la Detección}

Con frecuencia nos enfrentamos a situaciones en las cuales debemos decidir si ‘algo’ está o no ocurriendo para poder actuar en consecuencia, (e.g. ‘¿mi mamá está enojada?, ¿mi perro está enfermo?, ¿esta comida está pasada?'). No parecería tratarse de un gran problema si asumiéramos que somos



somos infalibles en la detección de dichos casos, o bien, que aquello que nos interesa detectar es un evento tan particular que es completamente inconfundible con nada más en el mundo. Desafortunadamente, el mundo siempre está cargado de ruido e incertidumbre: tanto la información con base en la cual buscamos tomar una decisión, como la precisión con que nuestro sistema es capaz de evaluarla, son imperfectos. 


\section{El papel de la incertidumbre}


\section{Teoría de Detección de Señales}

\subsection{Supuestos generales del modelo}

La Teoría de Detección de Señales (TDS o SDT, por sus siglas en inglés) plantea que la información que  interesa  detectar  (i.e. señal)  suele  presentarse  en  conjunto  con  otro  tipo  de  estimulación (i.e. ruido),  cargándola  de  incertidumbre  y  haciendo de  la  percepción  un  proceso  de  toma  de decisiones  donde  el  sistema  debe  formular  un  juicio  de  detección que  le  permita  guiar  su comportamiento. Es importante precisar  que la TDS no es exclusiva del estudio de la percepción visual  u  otras  modalidades  de  detección  sensorial,  sino  que  también  puede  referirse,  en  un sentido más abstracto, a la detección de información dentro de un conjunto de datos ambiguos; (e.g.  estudios  de  memoria  donde  se  solicita  al  participante  detectar  los  elementos  que  ya  se  le habían mostrado antes, o bien, la interpretación de baterías clínicas). (Wei Ji Ma, 2012)\\

En el laboratorio, la  TDS se estudia a partir  de tareas de detección donde se expone a un  sujeto  a  N  número  de  ensayos,  (comprendidos  por  n  ensayos con  sólo  ruido  y  n  ensayos donde  el  ruido  viene  acompañado  de  la  señal)  ante  los  que  se  le  pide  al  participante  que responda eligiendo una de dos opciones: Sí está la señal o No está la señal. En estos escenarios controlados,  el  experimentador  decide  la  proporción  de  ensayos  con  y  sin  señal  que  se presentarán, así como la matriz de pagos que definirán la utilidad de sus aciertos y errores. \\


If you are writing a thesis (or will be in the future) and its subject is technical or mathematical (though it doesn't have to be), then creating it in \LaTeX{} is highly recommended as a way to make sure you can just get down to the essential writing without having to worry over formatting or wasting time arguing with your word processor.

\LaTeX{} is easily able to professionally typeset documents that run to hundreds or thousands of pages long. With simple mark-up commands, it automatically sets out the table of contents, margins, page headers and footers and keeps the formatting consistent and beautiful. One of its main strengths is the way it can easily typeset mathematics, even \emph{heavy} mathematics. Even if those equations are the most horribly twisted and most difficult mathematical problems that can only be solved on a super-computer, you can at least count on \LaTeX{} to make them look stunning.

\subsection{Parámetros del modelo}

%----------------------------------------------------------------------------------------

\section{La Teoría de Detección de Señales en el desarrollo de la Psicología}

\LaTeX{} is not a \textsc{wysiwyg} (What You See is What You Get) program, unlike word processors such as Microsoft Word or Apple's Pages. Instead, a document written for \LaTeX{} is actually a simple, plain text file that contains \emph{no formatting}. You tell \LaTeX{} how you want the formatting in the finished document by writing in simple commands amongst the text, for example, if I want to use \emph{italic text for emphasis}, I write the \verb|\emph{text}| command and put the text I want in italics in between the curly braces. This means that \LaTeX{} is a \enquote{mark-up} language, very much like HTML.

\section{Aplicaciones}

\subsection{Psicofísica}

\subsection{Teoría de la Decisión}

\subsection{Modelos de decisión perceptual}

If you are new to \LaTeX{}, there is a very good eBook -- freely available online as a PDF file -- called, \enquote{The Not So Short Introduction to \LaTeX{}}. The book's title is typically shortened to just \emph{lshort}. You can download the latest version (as it is occasionally updated) from here:
\url{http://www.ctan.org/tex-archive/info/lshort/english/lshort.pdf}

It is also available in several other languages. Find yours from the list on this page: \url{http://www.ctan.org/tex-archive/info/lshort/}

It is recommended to take a little time out to learn how to use \LaTeX{} by creating several, small `test' documents, or having a close look at several templates on:\\ 
\url{http://www.LaTeXTemplates.com}\\ 
Making the effort now means you're not stuck learning the system when what you \emph{really} need to be doing is writing your thesis.
