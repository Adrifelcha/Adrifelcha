% Chapter 1

\chapter{Teoría de Detección de Señales} % Main chapter title

\label{Cap_SDT} % For referencing the chapter elsewhere, use \ref{Chapter1} 

%----------------------------------------------------------------------------------------

% Define some commands to keep the formatting separated from the content 
\newcommand{\keyword}[1]{\textbf{#1}}
\newcommand{\tabhead}[1]{\textbf{#1}}
\newcommand{\code}[1]{\texttt{#1}}
\newcommand{\file}[1]{\texttt{\bfseries#1}}
\newcommand{\option}[1]{\texttt{\itshape#1}}

%----------------------------------------------------------------------------------------

\section{Introducción: El problema de la Detección}

Uno de los problemas más frecuentes a los que se enfrenta cualquier sistema que se encuentre inmerso en un ambiente variable es la detección de estados particulares, asociados de manera especial a una conducta específica. Dicho en otras palabras, con frecuencia los organismos que buscan distribuir su conducta de manera óptima se encuentran a sí mismos tratando de 'decidir' si ‘algo’ está o no ocurriendo en el mundo, para poder guiar su comportamiento en consecuencia. 





(e.g. ‘¿mi mamá está enojada?, ¿mi perro está enfermo?, ¿esta comida está pasada?'). No parecería tratarse de un gran problema si tuvieramos entera confianza en nuestra capacidad para identificar dichos eventos, sin embargo, este no parece ser el caso casi nunca.\\


Desafortunadamente, el mundo siempre está cargado de ruido e incertidumbre: tanto la información con base en la cual buscamos tomar una decisión, como la precisión con que nuestro sistema es capaz de evaluarla, son imperfectos. 


\section{El papel de la incertidumbre}

\section{Teoría de Detección de Señales}

\subsection{Supuestos generales del modelo}

La Teoría de Detección de Señales (TDS o SDT, por sus siglas en inglés) plantea que la información que  interesa  detectar  (i.e. señal)  suele  presentarse  en  conjunto  con  otro  tipo  de  estimulación (i.e. ruido),  cargándola  de  incertidumbre  y  haciendo de  la  percepción  un  proceso  de  toma  de decisiones  donde  el  sistema  debe  formular  un  juicio  de  detección que  le  permita  guiar  su comportamiento. Es importante precisar  que la TDS no es exclusiva del estudio de la percepción visual  u  otras  modalidades  de  detección  sensorial,  sino  que  también  puede  referirse,  en  un sentido más abstracto, a la detección de información dentro de un conjunto de datos ambiguos; (e.g.  estudios  de  memoria  donde  se  solicita  al  participante  detectar  los  elementos  que  ya  se  le habían mostrado antes, o bien, la interpretación de baterías clínicas). (Wei Ji Ma, 2012)\\

En el laboratorio, la  TDS se estudia a partir  de tareas de detección donde se expone a un  sujeto  a  N  número  de  ensayos,  (comprendidos  por  n  ensayos con  sólo  ruido  y  n  ensayos donde  el  ruido  viene  acompañado  de  la  señal)  ante  los  que  se  le  pide  al  participante  que responda eligiendo una de dos opciones: Sí está la señal o No está la señal. En estos escenarios controlados,  el  experimentador  decide  la  proporción  de  ensayos  con  y  sin  señal  que  se presentarán, así como la matriz de pagos que definirán la utilidad de sus aciertos y errores. \\

\subsection{Parámetros del modelo}


\begin{itemize}
\item Discriminabilidad
\item Criterio
\item Sesgo - Beta
\item Sesgo - C
\end{itemize}

%----------------------------------------------------------------

\section{La Teoría de Detección de Señales en el desarrollo de la Psicología}



\section{Areas de impacto}

\subsection{Psicofísica}

\subsection{Teoría de la Decisión}

\subsection{Modelos de decisión perceptual}

