% Chapter Template

\chapter{Experimento: Buscando el Efecto Espejo en una tarea Perceptual} % Main chapter title

\label{Cap_Exp} % Change X to a consecutive number; for referencing this chapter elsewhere, use \ref{ChapterX}

%----------------------------------------------------------------------------------------
%	SECTION 1
%----------------------------------------------------------------------------------------
\section{Planteamiento general}

%Se propone buscar evidencia del Efecto Espejo en una tarea fuera de Memoria de reconocimiento


%Se propone una tarea perceptual ya que carece de una fase de preparacion
El interés principal del trabajo aquí expuesto es evaluar la posibilidad de que el Efecto Espejo y los patrones de respuesta identificados como parte del mismo, sean un producto del uso del modelo de detección de señales en la comparación del desempeño de los participantes a lo largo de dos condiciones de dificultad cualesquiera y no así de una discrepancia en el grado en que se tratan dichas condiciones a nivel de ciertos procesos superiores de atención y memoria alterando el orden en que dichos estímulos se distribuyen en el eje de la evidencia. Para ello se diseñó una tarea de detección meramente perceptual, donde las condiciones de dificultad fueron construidas con base en su discriminabilidad perceptual. 

%Se trabaja con ilusiones opticas, dado que la literatura en ellas permite anticiparnos a la d' y proponer dos niveles de dificultad.
Las ilusiones ópticas constituyen un fenómeno atractivo para la psicología cognoscitiva interesada en el estudio de nuestr os sistemas perceptuales, en tanto que han permitido dar evidencia de cómo se puede engañar a dichos sistemas 

%De encontrarse el efecto espejo, se sugeriría que éste es un producto de la aplicación del TDS y no así de procesos superiores.

\subsection{Objetivo}

%Buscar evidencia del efecto espejo en una tarea de detección que no implique el reconocimiento de estimulos ya conocidos.
Poner a prueba la existencia de los patrones de respuesta identificados como parte del Efecto Espejo en una tarea de detección perceptual con dos niveles de dificultad, (i.e. una tarea de detección que no pertenezca a la familia de tareas de reconocimiento y, principalmente, que carezca de una etapa previa a la fase experimental donde los participantes tuvieran ocasión de manipular los estímulos y hacer alguna distinción entre niveles de dificultad).


\section{Construcción de la Tarea}

Por tratarse de estímulos no antes probados en una tarea de detección como la aquí propuesta, se decidió correr dos experimentos que bien podrían pensarse como dos variaciones del mismo procedimiento experimental. 

De acuerdo a \parencite{Massaro1971}, la 

\subsection{Estímulos}

\begin{itemize}
\item Experimento 1 : Circulo de referencia aislado vs Figura de Ebbinghaus
\item Experimento 2 : Dos figuras de Ebbinghaus
\end{itemize}


Las figuras de Ebbinghaus construidas de acuerdo a un diseño factorial de 5x2x2 (Ver Figura). 

\subsection{Materiales}
La tarea fue programada y ejecutada a partir de Psychopy v.12. Se controló que la distancia que separase los estímulos presentados se encontrara dentro de un ángulo de $X grados$ del campo visual de los participantes, así como que estos realizaran la tarea en una distancia de 1 m respecto del monitor.

\subsection{Participantes}

Un total de cuarenta y un estudiantes de la Facultad de Psicología participaron en alguno de los dos experimentos presentados; el Experimento 1 se llevó a cabo con veinte participantes y el Experimento 2, con veintiuno. Ambos experimentos se llevaron a cabo simultáneamente por lo que la asignación de los participantes a cualquiera de los mismos se hizo de manera alternada, procurando garantizar que la muestra de parti

\section{Procedimiento}



\subsection{Registro de respuestas}

El experimento fue programado de manera tal que por cada participante se obtuviera un documento en formato csv (i.e. comma separated value) que contuviera información, ensayo a ensayo, sobre la respuesta dada por 
