% Chapter Template

\chapter{Experimento: Buscando el Efecto Espejo en otras áreas} % Main chapter title

\label{Cap_Exp} % Change X to a consecutive number; for referencing this chapter elsewhere, use \ref{ChapterX}

%----------------------------------------------------------------------------------------
%	SECTION 1
%----------------------------------------------------------------------------------------


A perceptual task was designed to compare subjects’ responses across two levels of discriminability that were constructed according to the literature that has explored the variables involved in the Ebbinghaus illusion. The focus was on the number of external circles, which has shown to be directly related to the intensity of the illusion (Massaro, 1971). The two levels were defined as follows:
• High accuracy (A): Ebbinghaus illusions with 2 or 3 surrounding circles.
• Low accuracy (B): Ebbinghaus illusions with 7 or 8 surrounding circles.
Two experiments were conducted, on each one participants had to indicate, pressing one of two keys, whether two circles appearing on screen were the same size (Signal) or not (Noise); these circles were presented on a bright color and identified by the name of ‘central circles’. In Experiment 1 both circles were constructed as Ebbinghaus illusions, varying the number of surrounding circles on each trial.
Experiment 2 consisted of a single Ebbinghaus illusion-circle that had to be compared with an aisle, constant, reference circle. Both experiments included underestimation and overestimation-inducing Ebbinghaus illusion.
Each experiment included a total of 640 trials (320 trials for each class of stimuli, A or B, with 120 signal and noise trials respectively) presented at random. On each trial, stimuli were shown for only 1.5 seconds to prevent habituation to the illusion. Participants could enter their response (‘Yes, circles are the same size’ or ‘No, they’re not’) either before or after stimuli disappeared from screen.
After the first response was given, a scale containing numbers from 1 to 3 was presented to indicate participants to grade how confident they were from their previous response by pressing one of the three possible response keys, (1-Low, 2-Medium, 3-High). However, the program registered these responses as part of a larger continuum going from 1 to 6, distinguishing ‘yes’ from ‘no’ responses:
a) If participants chose ‘No’ and pressed 3, it would be registered as 1 (Very sure noise), and so on.
b) If participants chose ‘Yes’ first and pressed 3, it would be registered as 6 (Very sure signal), and so on.
Participants had to press the space bar to indicate that they were ready to move from trial to trial. Response time was also registered.


\section{Objetivo}

El interés principal del trabajo aquí expuesto, fue el de evaluar la posibilidad de que el Efecto Espejo y los patrones de respuesta identificados como parte del mismo, sean un producto directo del uso del modelo de detección de señales en la comparación del desempeño de los participantes a lo largo de dos condiciones de dificultad cualesquiera y no así de una discrepancia en el grado en que se tratan dichas condiciones a nivel de ciertos procesos superiores de atención y memoria alterando el orden en que dichos estímulos se distribuyen en el eje de la evidencia. Para ello se diseñó una tarea de detección meramente perceptual, donde las condiciones de dificultad fueron construidas con base en su discriminabilidad perceptual. 

Las ilusiones ópticas constituyen un fenómeno atractivo para la psicología cognoscitiva interesada en el estudio de nuestros sistemas perceptuales, en tanto que han permitido dar evidencia de cómo se puede engañar a dichos sistemas 


%-----------------------------------
%	SUBSECTION 1
%-----------------------------------
\subsection{Construcción de la Tarea}

Por tratarse de estímulos no antes probados en una tarea de detección como la aquí propuesta, se decidió correr dos experimentos que bien podrían pensarse como dos variaciones del mismo procedimiento experimental. 

De acuerdo a 



%-----------------------------------
%	SUBSECTION 2
%-----------------------------------

\subsection{Materiales}
La tarea fue programada y ejecutada a partir de Psychopy v.12. Se controló que la distancia que separase los estímulos presentados se encontrara dentro de un ángulo de $X grados$ del campo visual de los participantes, así como que estos realizaran la tarea en una distancia de 1 m respecto del monitor.

%----------------------------------------------------------------------------------------
%	SECTION 2
%----------------------------------------------------------------------------------------

\section{Procedimiento}

Sed ullamcorper quam eu nisl interdum at interdum enim egestas. Aliquam placerat justo sed lectus lobortis ut porta nisl porttitor. Vestibulum mi dolor, lacinia molestie gravida at, tempus vitae ligula. Donec eget quam sapien, in viverra eros. Donec pellentesque justo a massa fringilla non vestibulum metus vestibulum. Vestibulum in orci quis felis tempor lacinia. Vivamus ornare ultrices facilisis. Ut hendrerit volutpat vulputate. Morbi condimentum venenatis augue, id porta ipsum vulputate in. Curabitur luctus tempus justo. Vestibulum risus lectus, adipiscing nec condimentum quis, condimentum nec nisl. Aliquam dictum sagittis velit sed iaculis. Morbi tristique augue sit amet nulla pulvinar id facilisis ligula mollis. Nam elit libero, tincidunt ut aliquam at, molestie in quam. Aenean rhoncus vehicula hendrerit.

\subsection{Instrucciones}



\subsection{Registro de respuestas}