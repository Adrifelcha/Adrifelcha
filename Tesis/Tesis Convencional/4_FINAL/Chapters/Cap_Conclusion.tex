% Chapter 1

\chapter{Conclusión} % Main chapter title

\label{Cap_Conclusion} % For referencing the chapter elsewhere, use \ref{Chapter1} 

La Teoría de Detección de Señales (SDT) presenta uno de los modelos más sólidos y ampliamente desarrollados dentro de la historia de la Psicología Experimental, que permite dar cuenta de una amplia gama de situaciones donde la tarea principal de los organismos involucrados es la detección de eventos específicos en su entorno que le permitan guiar su comportamiento de manera óptima, en función a las relaciones de contingencia anunciadas. Los supuestos de dicho modelo son lo suficientemente generales para permitir su aplicación al estudio de distintos fenómenos -dentro y fuera de la Psicología-. La SDT funciona tanto como un modelo estadístico para describir la detección de señales como un problema de adaptabilidad, como una herramienta para interpretar la ejecución de sistemas evaluados experimentalmente y hacer estimaciones sobre la precisión con que las Señales se distinguen del Ruido y los posibles sesgos que conlleven a reportar su presencia o ausencia.\\

En estudios de Memoria de Reconocimiento donde el desempeño de los participantes es interpretado como una instancia de un problema de detección, bajo el marco de los principios establecidos en la SDT, se ha reportado consistentemente que al evaluar la ejecución de los participantes entre dos clases de estímulos A y B que difieren en la precisión con que sus elementos son reconocidos (donde $d'(A)$ $>$ $d'(B)$), las respuestas de los participantes sugieren la existencia de un par de distribuciones Ruido-Señal por cada clase de estímulos que se despliegan simétricamente sobre el eje de evidencia, (siendo que las distribuciones de la clase A se sitúan siempre en los extremos y las distribuciones de la clase B se localizan en un punto intermedio). Por esta razón, dichos patrones de respuesta han sido ubicados en la literatura como reflejo de un fenómeno denominado Efecto Espejo, cuyas implicaciones tanto en términos de su relevancia para el estudio de la Memoria de Reconocimiento como de la pertinencia del uso de modelos de detección de señales para dar cuenta de los fenómenos de Memoria, acapararon una parte considerable de la literatura desde su aparición.\\

La evidencia del Efecto Espejo ha sido reportada en una amplia variedad de estudios en memoria de reconocimiento donde se han manipulado diversas variables en la definición de las clases A y B a comparar, a través del uso de distintos protocolos de tareas de detección. El presente estudio es el primero en aportar evidencia ajena a cualquier proceso mnémico del Efecto Espejo, en un par de experimentos diseñados para enfrentar a los participantes a un problema de detección perceptual al cual tuvieron que responder a partir de dos protocolos distintos: una tarea binaria y una tarea con Escala de Confianza. El análisis de los datos recopilados íncorporó simultáneamente la replicación de los análisis frecuentistas reportados en la literatura y un análisis bayesiano impulsado por la construcción de modelos y la realización de pruebas estadísticas bayesianas. Ambas aproximaciones identificaron los resultados obtenidos en el presente trabajo como evidencia a favor del Efecto Espejo, pero sólamente el análisis bayesiano  consiguió tomar en cuenta la variabilidad de las estimaciones individuales interpretando los datos como reflejo de un proceso probabilístico -no determinista- y permitiendo una evaluación más completa de la robustez de la evidencia del Efecto Espejo presentada por estos, (por ejemplo, señalando que las diferencias entre los Puntajes de Confianza asignados durante la tarea con Escala de Confianza son apenas anecdóticas).\\

Los resultados encontrados en el presente estudio pueden ser interpretados en dos direcciones: Primero, como evidencia de que el Efecto Espejo no es un fenómeno exclusivo de la Memoria de Reconocimiento y debería ser abordado como una regularidad propia de las situaciones de detección que incorporan distintos niveles de $d'$, con base en la cual debería cotejarse el uso de los modelos de detección de señales para explicar diversos fenómenos psicológicos y no psicológicos. Segundo, como un referente empírico sobre las ventajas que presenta el análisis de datos bayesiano sobre el análisis frecuentista, en tanto que permite un mejor manejo de la incertidumbre contenida en los datos -que podría resultar particularmente útil en modelos como los de detección de señales, que conceden una importancia escencial a la estructura probabilísstica del entorno, tomando como uno de sus supuestos principales la noción de que las respuestas emitidas en situaciones de detección resultan de la interacción entre la variabilidad en el entorno y la variabilidad intrínseca a los organismos-.\\
