% Chapter 1

\chapter{Conclusión} % Main chapter title

\label{Cap_Conclusion} % For referencing the chapter elsewhere, use \ref{Chapter1} 

La Teoría de Detección de Señales (SDT) presenta uno de los modelos más sólidos y ampliamente desarrollados en Psicología Experimental, que permite dar cuenta de una amplia gama de situaciones donde los organismos se enfrentan a la tarea de detectar ciertos eventos en su entorno para guiar su comportamiento de manera óptima, en función a las relaciones de contingencia anunciadas. Los supuestos de dicho modelo son lo suficientemente generales para permitir su aplicación al estudio de distintos fenómenos, dentro y fuera de la psicología, funcionando tanto como un modelo estadístico para describir la detección como problema de adaptabilidad, como una herramienta para interpretar la ejecución de sistemas evaluados experimentalmente.\\

En estudios donde el desempeño de los participantes en tareas de memoria de reconocimiento es evaluado con base en la SDT, se ha reportado consistentemente que al comparar su ejecución entre dos clases de estímulos A y B que difieren en la precisión con que sus elementos son reconocidos ($d'(A)$ $>$ $d'(B)$), las respuestas de los participantes sugieren que las distribuciones de Ruido y Señal de las clases A y B se despliegan simétricamente sobre el eje de evidencia, (las distribuciones de la clase A se encuentran en los extremos y las distribuciones de la clase B, en el área intermedia). Los patrones de respuesta relacionados con este fenómeno han sido identificados como Efecto Espejo y sus implicaciones han sido abordadas tanto en términos de su relevancia para el estudio de la memoria como de la validez que pueden tener los modelos de memoria basados en detección de señales.\\

La evidencia del Efecto Espejo ha sido reportada en una gran variedad de estudios donde las clases A y B son construidas a partir de distintas variables y donde se utilizan diferentes protocolos de detección. El presente estudio es el primero en aportar evidencia del Efecto Espejo en tareas de detección perceptual, con un par de experimentos compuestos por una tarea binaria con Escala de Confianza. El análisis de datos se realizó tanto a partir de los análisis frecuentistas reportados usualmente en la literatura, como de un análisis bayesiano sustentado en la construcción de modelos y la realización de pruebas estadísticas bayesianas. Ambas aproximaciones confirman la importancia de este proyecto como evidencia de que el Efecto Espejo puede encontrarse fuera de la memoria de reconocimiento. Sin embargo, el análisis bayesiano permitio una evaluación más precisa de su robustez, (por ejemplo, señalando que las diferencias entre los Puntajes de Confianza asignados durante la tarea con Escala de Confianza son apenas anecdóticas).\\

Los resultados obtenidos en el presente trabajo pueden ser interpretados en dos direcciones. Primero, como evidencia de que el Efecto Espejo no es un fenómeno exclusivo de la memoria de reconocimiento y debe ser abordado como una regularidad en tareas de detección con más de un nivel de $d'$. Segundo, como un referente sobre las ventajas que presenta el análisis de datos bayesiano sobre el análisis frecuentista, en tanto que permite un mejor manejo de la incertidumbre contenida en los datos, algo particularmente útil en situaciones donde se concibe una estructura probabilística tanto en el entorno como en las respuestas de los organismos.\\
