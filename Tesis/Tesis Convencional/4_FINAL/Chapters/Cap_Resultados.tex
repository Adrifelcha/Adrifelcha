% Chapter 1

\chapter{Resultados} % Main chapter title
\label{Cap_Res} % For referencing the chapter elsewhere, use \ref{Chapter1} 

\section{Datos recolectados}

%SDatos duros antes del análisis.
Antes de realizar análisis estadísticos para determinar si se encontró -o no- evidencia sólida del Efecto Espejo en nuestros experimentos, los datos recopilados se exploraron de manera exhaustiva, graficando la relación entre la ejecución de los participantes y diferentes variables.\\ 

%Se destaca la implrtancia de revisar los datos antes de someterlos a analisis estadísticos
Graficar los datos antes de someterlos al análisis estadístico, constituye una práctica recomendable en tanto que 1) permite evaluar la pertinencia del diseño experimental a la luz de las respuestas registradas y 2) proporciona un filtro para comprobar que los participantes estuvieran respondiendo de manera congruente y consistente con las tareas presentadas, permitiendo una mayor confianza en las conclusiones que resulten de su análisis.\\

%Presentación de los controles graficados: Atención, El efecto del paso del tiempo y las variables externas en los estímulos. 
A continuación se presentan las distintas gráficas realizadas para explorar tres grandes fuentes de ruido que potencialmente podrían haber influido en el desempeño de los participantes (contaminando los datos obtenidos). Dichas fuentes son:\\

\begin{enumerate}
	\item \textbf{¿Las respuestas se emiten en trenes?} (\textit{Evaluando la atención}).\\

Para evaluar que los participantes estuvieran poniendo atención a las tareas presentadas, se revisaron las respuestas emitidas a lo largo de los ensayos.\\

Durante los experimentos los estímulos con Ruido o Señal de las clases de estímulos diseñadas eran presentados de manera aleatoria, por lo que se esperaría encontrar mucha variabilidad en las respuestas emitidas por los participantes (y que usaran todas las opciones de respuesta). En caso contrario, la persistencia en la emisión de una misma respuesta de manera consecutiva (\textit{trenes de respuesta}), podría sugerir una falta de atención a los etímulos en pantalla y una posible dependencia entre respuestas.\\

	\item \textbf{¿El Aprendizaje o la Fatiga alteran la ejecución de los participantes?}  (\textit{Evaluando cambios en el desempeño a lo largo del tiempo}).\\

Un segundo filtro consistió en la evaluación de los posibles efectos que el paso del tiempo (y el avance entre ensayos) pudo haber tenido sobre el desempeño de los participantes. Por un lado, dado que los experimentos estuvieron compuestos por un amplio número de ensayos -cada uno con un par de tareas de detección-, la \textbf{Fatiga} es un riesgo latente. Por otro lado, dado que la tarea experimental consistió en la presentación de ilusiones ópticas, es posible que la exposición repetida a estas redujera su impacto por efecto de \textbf{Habituación}, mejorando el desempeño de los participantes.\\

	\item \textbf{¿Los participantes tienen alguna preferencia hacia la emisión de ciertas respuestas ante cierto tipo de estímulos?} (\textit{Evaluando el efecto de las variables manipuladas en la construcción de los estímulos}).\\

Finalmente, se evaluó el impacto que las variables manipuladas para diseñar los estímulos pudieran haber tenido sobre 1) la intensidad de la ilusión óptica y 2) la emisión de respuestas de los participantes.\\
\end{enumerate}

Idealmente, el desempeño de los participantes no debería cambiar en función a ninguna de las fuentes de ruido presentadas, y sólo presentar diferir significativamente entre las clases A y B diseñadas.\\












\subsection{Control 1: ¿Los participantes estaban poniendo atención a la tarea al emitir sus respuestas?}

Los experimentos realizados estuvieron compuestos de 640 ensayos, a lo largo de los cuales los participantes tuvieron que 1) decidir si los estímulos presentados cumplían con la condición que se les solicitó detectar y 2) valorar su certidumbre sobre esta primer respuesta y asignarle un puntaje. Dado lo demandante y extenso del procedimiento, la primer preocupación respecto de la validez de los datos obtenidos fue determinar si los participantes habían -o no- respondido a la tarea con atención. Para ello, se revisaron las respuestas emitidas ensayo a ensayo para garantizar todas las opciones de respuesta se hubieran usado y que la variabilidad en la emisión de respuestas correspondiera con la naturaleza aleatoria con que se presentaban los distintos tipos de estímulo.\\

\begin{itemize}
	\item Emisión de respuestas 'Sí/No' a lo largo del experimento.

Primero se graficaron las respuestas emitidas durante la tarea de detección binaria ('Sí, los círculos son iguales' o 'No, los círculos son diferentes'), ensayo a ensayo. Con estos gráficos se buscó descartar la presentación de trenes de respuesta lo suficientemente largos como para sugerir, dada la aleatoriedad con que los estímulos fueron presentados por el programa, una preferencia en el participante a responder a una tecla en particular, con independencia del estímulo a evaluar en pantalla.\\

%Participante representativo: Respuestas 'No' por 80 ensayos
La Figura~\ref{fig:Resp_E2_P1} presenta un ejemplo particularmente ilustrativo de la importancia que tiene revisar los datos antes de realizar el análisis estadístico. El gráfico muestra las respuestas emitidas en la tarea de detección binaria por el Participante 1 del Experimento 2, quien pasó los primeros 80 ensayos del experimento presionando persistentemente la tecla de respuesta 'No'. Este tren de respuesta fue considerado lo suficientemente largo como para cuestionar la atención con que el Participante 1 estuvo respondiendo a la tarea.\\ 

\begin{figure}[th]
\centering
\includegraphics[width=0.80\textwidth]{Figures/Response_Exp2_P1} 
%\decoRule
\caption[Respuesta emitida por ensayo: Ejemplo de participante sesgado]{Respuestas Sí/No emitidas en los 640 ensayos del Experimento 1 por el Participante 1. En la gráfica se aprecia un tren de respuesta, que se extiende a lo largo de 80 ensayos, donde el Participante 1 sólo utilizó una de las opciones de respuesta disponibles.}
\label{fig:Resp_E2_P1}
\end{figure}

%Las gráficas correspondientes al resto de los participantes en los Experimentos 1 y 2, se muestran en las Figuras~\ref{fig:Response_P1} y \ref{fig:Response_E2}, respectivamente.\\

	\item Correlación entre las respuestas 'Sí/No' emitidas y el tipo de estímulo presentado en cada ensayo.

Como paso siguiente, se añadieron indicadores que señalaran las características de los estímulos presentados en cada ensayo -es decir, si se trataba de una Señal o Ruido, o si se trataba de un estímulo de la clase Fácil o Difícil-.\\ 

Retomando el caso del Participante 1 del Experimento 2 presentado en la Figura~\ref{fig:Resp_E2_P1}, la Figura~\ref{fig:BiasResp_E1_P1} permite evaluar la posibilidad de que las respuestas registradas estuvieran relacionadas con los estímulos presentados en pantalla, (por ejemplo, en el improbable -pero posible- caso de que se le hubiera presentado una gran proporción de estímulos con Ruido durante los primeros 80 ensayos del experimento). De acuerdo con esta gráfica, parece ser que el tren de 80 respuestas 'No' consecutivas se mantuvo con independencia del tipo de estímulo presentado. Con base en ello, se decidió eliminar a dicho participante de la muestra a analizar, al encontrarse evidencia suficiente para cuestionar la atención con que respondió a la tarea.\\

\begin{figure}[th]
\centering
\includegraphics[width=0.80\textwidth]{Figures/BiasResp_Exp2_P1} 
%\decoRule
\caption[Respuesta por Tipo de Estímulo; ejemplo de participante sesgado]{Respuestas registradas por el Participante 1 en cada uno de los 640 ensayos del Experimento 2, indicando con diferentes colores el tipo de estímulo que se le mostraba en cada ocasión. En el panel superior se señala con colores violeta y azul si el estímulo presentado pertenecía a la clase Difícil o Fácil, respectivamente, y en el panel inferior, si se trataba de una Señal o Ruido con los colores verde y rojo, respectivamente.}
\label{fig:BiasResp_E1_P1}
\end{figure}

%Las Figuras~\ref{fig:BiasResp_E1} y () muestran las gráficas correspondientes al resto de los participantes en el Experimento 1 y 2, respectivamente.\\

	\item Asignación de puntajes de confianza, (\textit{'1'},\textit{'2'} ó \textit{'3'}).

En la segunda fase del problema de detección presentado en cada ensayo, los participantes tenían que elegir entre tres opciones de respuesta (teclas 1, 2 y 3) para señalar \textit{cuánta} confianza tenían sobre la respuesta recién emitida (\textit{'poco seguro'}, \textit{'más o menos seguro'} o \textit{'muy seguro'}, respectivamente). Las respuestas fueron registradas por el programa como parte de una Escala mayor (con valores del 1 al 6), que permite diferenciar entre la confianza de haber rechazado correctamente un estímulo con Ruido (e.g. \textit{'1, estoy muy seguro de que los círculos eran diferentes'}) y la confianza de haber identificado correctamente un estímulo con Señal (e.g \textit{'6, estoy muy seguro de que los círculos son iguales'}), asignando los valores intermedios (\textit{3} y \textit{4}) a los puntajes asignados para señalar una confianza baja en la respuesta emitida (e.g. \textit{'3, poco seguro de que los círculos eran diferentes'} y \textit{'4, poco seguro de que los círculos eran iguales'}).\\

Tal y como se hizo con la tarea de detección binaria, se graficaron los puntajes de confianza registrados por los participantes en cada uno de los 640 ensayos que conformaron cada experimento. A manera de ejemplo, la Figura~\ref{fig:Rating_E2_P4} muestra los puntajes emitidos por el Participante 15 del Experimento 2 a lo largo de la tarea. Este participante, de acuerdo con lo que se esperaría encontrar, hizo uso de las tres teclas de respuesta, registradas de acuerdo a su correspondencia con la respuesta dada a la tarea binaria.\\ 
 
\begin{figure}[th]
\centering
\includegraphics[width=0.80\textwidth]{Figures/Rating_Exp2_P15} 
%\decoRule
\caption[Asignación Puntaje de confianza: Ejemplo]{Puntajes de confianza asignados por el Participante 15 del Experimento 2 a las respuestas emitidas en la tarea binaria en cada uno de los 640 ensayos. El panel superior se muestran los puntajes asignados en los primeros 320 ensayos del experimento; en el panel inferior, los 320 restantes.}
\label{fig:Rating_E2_P4}
\end{figure}

%El registro ensayo a ensayo de los puntajes de confianza asignados por el resto de los participantes en los Experimentos 1 y 2 se muestran en las Figuras~\ref{fig:Rating_E1} y \ref{fig:Rating_E2}, respectivamente.\\

\end{itemize}










\subsection{Control 2: ¿La duración del experimento tuvo un impacto en la ejecución de los participantes?}

La fatiga causada por la extensión del experimento y la posible habituación a la ilusión óptica, fueron dos de las fuentes de ruido externo que se pensó podrían tener un efecto sobre la ejecución de los participantes, mermando la validez de los datos obtenidos. Para preveer su influencia, se incluyeron un par de controles desde el diseño experimental, tales como el agregar una pantalla de espera entre los ensayos que diera oportunidad a los participntes de descansar e indicar cuando se sintieran listos para atender un nuevo par de estímulos, o el restringir el tiempo durante el que se mostraban los estímulos a comparar. El siguiente conjunto de gráficas fueron realizadas para comprobar que ni la Fatiga ni la Habituación hubieran contaminado la ejecuciónde los participantes, esperando que su desempeño no mostrara signos de decaimiento o mejora con el paso de los ensayos.\\ 

\begin{itemize}
	\item Aciertos y errores a lo largo del tiempo

Primero, se graficó la clasificación de las respuestas emitidas por los participantes como acierto o error, en cada ensayo. Con ello se buscó explorar visualmente si hubieron cambios significativos en el desempeño de los participantes conforme adquirían más experiencia en la tarea.\\

La Figura~\ref{fig:Success_E2_P14} muestra como ejemplo el desempeño del Participante 14 durante el Experimento 2. La gráfica superior presenta el registro acumulativo de los aciertos y errores cometidos a lo largo de todo el experimento y en los paneles inferiores se muestra -ensayo a ensayo- si sus respuestas fueron registradas como acierto o error.\\

\begin{figure}[th]
\centering
\includegraphics[width=0.60\textwidth]{Figures/SuccessCumulative_Exp2_P14} \\
\includegraphics[width=0.80\textwidth]{Figures/Success_Exp2_P14}
%\decoRule
\caption[Aciertos y errores a lo largo del tiempo: Participante ejemplar]{Aciertos y errores cometidos por el Participante 14 del Experimento 2 a lo largo de la tarea. En el panel superior se muestra el registro acumulativo de estos a lo largo del experimento. Los paneles inferiores muestran la clasificación ensayo a ensayo de las respuestas emitidas por el participante como Acierto o Error (el panel intermedio presenta la primera mitad del experimento y el panel inferior, el resto).}
\label{fig:Success_E2_P14}
\end{figure}

%Las Figuras~\ref{fig:Success_E1} y \ref{fig:Success_E2} muestran los aciertos y errores cometidos a lo largo de los experimentos por el resto de los participantes en los Experimentos 1 y 2, respectivamente.\\


	\item Tipos de acierto y error a lo largo del tiempo

Para tener más información sobre los aciertos y errores cometidos, se realizaron gráficas que señalaran qué resultado habían obtenido los participantes en cada ensayo (Hit, Falsa Alarma, Rechazo y Omisión). En la Figura~\ref{fig:Outcome_E2_P14} se vuelven a presentar los datos del Participante 14 del Experimento 2, señalando el tipo de resultado obtenido. El panel superior muestra el registro acumulativo de cada uno de los cuatro posibles resultados a lo largo del experimento y el panel inferior, los resultados obtenido en cada ensayo.\\ 

\begin{figure}[th]
\centering
\includegraphics[width=0.80\textwidth]{Figures/Outcome_Exp2_P14}
%\decoRule
\caption[Resultado obtenido a lo largo del tiempo: Ejemplo]{Resultados obtenidos por el Participante 14 del Experimento 2 a lo largo del Experimento. En el panel superior se muestra la frecuencia acumulada de los Hits, Falsas Alarmas, Rechazos y Omisiones obtenidos durante el experimento, mientras que en el panel inferior se presenta el resultado obtenido por cada ensayo.}
\label{fig:Outcome_E2_P14}
\end{figure}


\end{itemize}










\subsection{Control 3: ¿El diseño de los estímulos afectó el desempeño de los participantes?}

Durante la construcción de los estímulos a presentar en los experimentos se manipularon dos variables: 1) El número de círculos externos incluidos en las figuras de Ebbinghaus y 2) el color en que se presentaron las figuras. La primera constituye la variable experimental, y con ella que se buscó definir las dos clases de estímulos entre las que se compararía el desempeño de los participantes. Por otro lado, las variaciones en la segunda variable mencionada fueron incluidas arbitrariamente, en un intento por hacer la tarea menos tediosa y más dinámica, buscando preveer los efectos de fatiga y habituación. Es decir, se esperaba que el desempeño de los participantes variara exclusivamente en función a los cambios en la variable experimental.\\

A continuación se presenta un último conjunto de gráficos, realizados para descartar la posibilidad de que el desempeño de los participantes se hubiera visto influido por el color en que los estímulos fueron presentados.\\

\begin{itemize}
	\item El efecto del color sobre la intensidad de la ilusión.

Una primer forma en que el color de las figuras pudo haber afectado el desempeño de los participantes fue alterando la intensidad de la ilusión de Ebbinghaus. Para evaluar dicha posibilidad, se exploró la posible relación entre el número total de Hits y Falsas Alarmas cometidos y los colores en que aparecían las figuras.\\

\begin{figure}[th]
\centering
\includegraphics[width=0.85\textwidth]{Figures/Color_Exp1_P14}
%\decoRule
\caption[Hits y Falsas Alarmas por Color; Ejemplo]{Número de Hits (panel izquierdo) y Falsas Alarmas (panel derecho) cometidas por el Participante 14 del Experimento 1 por cada uno de los colores en que se presentaron las figuras. Se omite el número de Omisiones y Rechazos Correctos obtenidos, al tratarse del complemento de las cifras presentadas.}
\label{fig:Color_E1_P14}
\end{figure}

Por ejemplo, la Figura~\ref{fig:Color_E1_P14} muestra la cantidad de Hits y Falsas Alarmas cometidas por el Participante 14 a lo largo del Experimento 1 por cada color empleado en el diseño de las figuras (panel izquierdo y derecho, respectivamente). De acuerdo a lo que se observa, no parece que el color haya tenido un efecto sobre la precisión con que este participante respondía a la tarea.\\


	\item El Efecto del Color sobre la emisión de respuestas en los participantes.

Una segunda forma en que el color de los estímulos pudo haber alterado el desempeño de los participantes, es si estos hubieran tenido un sesgo o preferencia a responder de cierta forma ante alguno de los colores utilizados con independencia del resto de las características de las figuras. Para explorar esta posibilidad, se graficó la relación entre el color de las figuras y la proporción de respuestas afirmativas y negativas emitidas por cada participante. La Figura~\ref{fig:BiasCol_E1_P13} presenta un ejemplo de este tipo de gráficas, donde se muestra la proporción de Respuestas Sí/No emitidas por el Participante 13 del Experimento 1 para cada uno de los diferentes colores en que se presentaron los estímulos. Como se puede ver en la figura, parece ser que este participante mantuvo constante la proporción de respuestas afirmativas y negativas emitidas a lo largo de los distintos colores utilizados, por lo que no parece que el color hubiera tenido un efecto sobre su propensión a responder de manera afirmativa o negativa.\\

\begin{figure}[th]
\centering
\includegraphics[width=0.80\textwidth]{Figures/BiasColor_Exp1_P13}
%\decoRule
\caption[Proporción de Respuestas Sí/No por color; Ejemplo]{Se muestra la proporción de respuestas afirmativas y negativas emitidas por el Participante 14 del Experimento 1 en función al color en que se presentaron los estímulos.}
\label{fig:BiasCol_E1_P13}
\end{figure}
\end{itemize}

%De cualqueir forma, se presentó siempre un mismo número de figuras por cada color, para cada uno de los estímulos construidos. Es decir, que aún cuando se hubiera encontrado evidencia de variaciones en el desempeño de los participantes en función de los colores utilizados, se esperaría que dicho efecto se distribuyera de manera homogénea entre las dos clases de estímulos entre las cuales se compararía su ejecución.\\

Las gráficas correspondientes al desempeño de cada uno de los participantes en los Experimentos 1 y 2, así como los códigos empleados para su graficación, se pueden consultar en línea en el siguiente repositorio de GitHub: \href{http://github.com/Adrifelcha/MirrorEffect-Adrifelcha}{http://github.com/Adrifelcha/MirrorEffect-Adrifelcha}.\\




































\section{Análisis estadísticos}

En los experimentos realizados, los patrones de respuesta identificados como parte del Efecto Espejo se presentaron en más de tres cuartas partes de los participantes, en al menos uno de los dos protocolos de detección empleados. De los veinte participantes en el Experimento 1, diecisiete ($85\%$) mostraron el patrón de respuesta esperado en la tarea binaria y dieciocho ($90\%$), en la Escala de Confianza. A su vez, en el Experimento 2, dieciocho de los veinte participantes incluidos en la muestra ($90\%$) presentaron los patrones del Efecto Espejo en ambas tareas. Estas proporciones resultan estadísticamente significativas al compararlas contra el azar con una prueba binomial simple (ver Tabla~\ref{Tabla_Binom}).\\

\begin{table}
\caption[Pruebas Binomiales que comparan la proporción de patrones de Efecto Espejo encontrados]{Se presentan los resultados de la comparación entre la proporción de casos con el Efecto Espejo encontrados en cada experimento y el azar ($p=0.5$), de acuerdo con la realización de pruebas binomiales.}
\label{Tabla_Binom}
\centering
\begin{tabular}{l l | c c c}
\toprule
%\tabhead{Groups} & \tabhead{Treatment X} & \tabhead{Treatment Y} \\
\textbf{} & \textbf{Tarea} & \textbf{Proporción} & \textbf{P value}\\
\midrule
Exp 1 & Sí/No & 17/20 & 0.0025 \\
Exp 1 & Escala & 18/20 & 0.0004\\
Exp 2 & Sì/No & 18/20 & 0.0004\\
Exp 2 & Escala & 18/20 & 0.0004\\
\bottomrule
\end{tabular}
\end{table}


Tomando en cuenta que las clases de estímulos a comparar se diseñaron de manera exploratoria y que la tarea de detección planteada se presentó en términos de dos protocolos distintos, el análisis de datos se llevó a cabo en el siguiente orden:\\

\begin{enumerate}
\item \textbf{Verificar que realmente existiera una diferencia entre las clases de estímulos construidas}\\

Dado que las clases de estímulos a comparar fueron construidas con base en la literatura revisada sobre el funcionamiento de la ilusión de Ebbinghaus, se comenzó por evaluar que realmente existienran distintos niveles distintos de discriminabilidad entre éstas y que dicha diferencia se presentara en la dirección esperada. Para ello, se compararon los valores de $d'$ calculados a partir de los subconjuntos de Hits y Falsas Alarmas obtenidos por cada clase, para corroborar que se satisfaciera la siguiente relación:\\

\begin{center}
 $d'(A) > d'(B)$\\
 donde A y B representan las clases de estímulos construidas en función del número de círculos externos contenidos en las figuras: la clase A con \textit{pocos} círculos (2 o 3 círculos) y la clase B con \textit{muchos} círculos (7 u 8).\\
\end{center}

\item \textbf{Comparar las tasas de Hits y Falsas Alarmas entre condiciones.}\\

Una vez corroborada la diferencia entre las clases de estímulos construidas, se evaluaron las diferencias encontradas entre las tasas de Hits y Falsas Alarmas registradas por cada clase de estímulo para comprobar que dicha discrepancia fuera significativa, cumpliendo con el siguiente patrón reportado en estudios de memoria de reconocimiento:\\

\begin{center}
$H(B) < H(A)$\\
$FA(A) < FA(B)$\\
donde $H$ y $FA$ representan las tasas de Hits y Falsas Alarmas obtenidas en cada clase de estímulos a evaluar. Es decir:\\
$FA(A) < FA(B) < H(B) < H(A)$\\
\end{center}

\item \textbf{Comparar el puntaje de confianza promedio asignado a cada tipo de ensayo (con ruido o señal) entre cada clase a comparar}\\

Dado que la clase A es más discriminable que B, esperaríamos que los participantes hubieran reportado un promedio mayor de confianza al responder a los estímulos de la clase A, ($PuntajeCrudo(A) > PuntajeCrudo(B)$). Pero tomando en cuenta que los experimentos fueron programados de manera que los puntajes crudos emitidos por los puntajes (del 1 al 3) fueran transformados en una escala más grande que distinguiera entre la confianza en sus respuestas afirmativas y negativas, la relación entre los promedios de los puntajes transformados debería mostrar el mismo patrón que se reporta en la literatura en memoria:

%Posteriormente, se compararon los puntajes de confianza asignados a cada tipo de ensayo (Señal o Ruido) por cada clase de estímulo (Fácil o Difícil).\\

%Primero, en función de los puntajes de confianza crudos asignados por los participantes -el registro de las teclas de respuesta 1, 2 y 3-, se esperaba encontrar la siguiente relación en función de los niveles de dificultad:

%\begin{center}
%$P(A) > P(B)$\\
%donde $P$ refiere al promedio de los puntajes de confianza asignados a las respuestas emitidas por los participantes durante la tarea binaria, por cada condición de dificultad.\\
%\end{center}

%Después, tomando en cuenta que los experimentos fueron programados de tal manera que los puntajes registrados por los participantes -para indicar qué tan seguros se sentían respecto de la respuesta emitida en ese mismo ensayo a la tarea de detección binomial- fueran \textit{transformados} a valores dentro de una Escala mayor que distingue entre la confianza en las respuestas negativas y la confianza en las respuestas afirmativas, se espera encontrar la misma relación reportada en Memoria de Reconocimiento entre los promedios de los puntajes de confianza asignados por cada tipo y clase de estímulos:\\

\begin{center}
$P(BS) < P(AS)$\\
$P(AN) < P(BN)$\\
donde nuevamente $P$ refiere al promedio de los puntajes de confianza asignados a cada tipo de ensayo por cada clase de estímulo. Es decir:\\
$P(AN) < P(BN) < P(BS) < P(AS)$\\
\end{center}

\item \textbf{Réplica de controles reportados en la literatura.}\\

Además de evaluar las diferencias encontradas en la ejecución de los participantes entre las clases de estímulos construidas, en la literatura del Efecto Espejo también suelen reportarse algunos análisis estadísticos de control para confirmar la solidez del efecto encontrado, (Glanzer y Adams, \citeyear{Glanzer1990}). En el presente trabajo se retomó la revisión de la posible correlación entre los tiempos de respuesta invertidos por los participantes y las clases de estímulos a comparar. De acuerdo con la literatura, descartar esta correlación es importante en tanto que permite descartar que los resultados encontrados sean producto de una diferencia en el tiempo invertido por parte de los participantes al responder a cada clase.\\
\end{enumerate}

Todos los análisis previamente descritos fueron realizados desde dos enfoques distintos:\\

\begin{itemize}
\item Como una réplica de los análisis reportados en la literatura. \textit{(ANOVA's y Pruebas T)}.\\

Tomando en cuenta que el objetivo princial del trabajo de investigación realizado fue evaluar la generalizabilidad de los patrones de respuesta identificados en memoria de reconocimiento como Efecto Espejo en una tarea de detección perceptual, se consideró necesario someter los datos obtenidos a los mismos análisis reportados en la literatura. Para ello, se utilizó como guía un artículo que reporta evidencia del Efecto Espejo en cinco experimentos de memoria de reconocimiento donde se manipulan distintas variables para la definición de las clases A y B, (Glanzer y Adams, \citeyear{Glanzer1990}).\\

\item El desarrollo de modelos Bayesianos para la estimación paramétrica y la evaluación de la evidencia encontrada.\\

La estadística bayesiana es una herramienta flexible para el análisis de datos, que permite formalizar y evaluar las hipótesis que se tiene sobre el funcionamiento de ciertos procesos psicológicos especificando una distribución prior y una función de verosimilitud que describa la relación entre los datos obtenidos y los modelos estadísticos-cognitivos que dan cuenta de los fenómenos psicológicos que subyacen a su generación, (Lee, \citeyear{Lee2011}). En la presente Tesis, además de realizar los análisis estadísticos frecuentistas comúnmente reportados en los estudios en memoria de reconocimiento, se incluyeron desarrollaron modelos y análisis bayesianos para evaluar con mayor detalle la variabilidad contenida en las respuestas emitidas por los participantes.\\
\end{itemize}










\subsection{Evaluación de las diferencias entre las clases de estímulos propuestas}

En los experimentos realizados como parte de la presente tesis, las clases A y B fueron definidas de acuerdo con la literatura que ha explorado el impacto que tienen los distintos elementos que componen las figuras de Ebbinghaus en la intensidad de la ilusión evocada, (Massaro y Anderson, \citeyear{Massaro1971}), de la siguiente forma: \\

\begin{itemize}
\item \underline{Clase A}, (\textit{Fácil}): Figuras de Ebbinghaus compuestas por 2 o 3 círculos externos.\\

\item \underline{Clase B}, (\textit{Difícil}): Figuras de Ebbinghaus con 7 u 8 círculos externos.\\
\end{itemize}

Para evaluar la eficacia del diseño de las clases de estímulos propuestas, comprobando que estas realmente difirieron en términos de qué tan difícil resultó para los participantes discriminar entre los estímulos con Señal y aquellos con Ruido, se compararon los valores de $d'$ estimados para cada una de ellas, de acuerdo con el registro de Hits y Falsas Alarmas obtenidos por todos los participantes en los experimentos realizados.\\

\begin{figure}[th]
\centering
\includegraphics[width=0.90\textwidth]{Figures/Diff_D_E1yE2}
%\decoRule
\caption[Diferencias entre las $d'$ de los niveles de dificultad propuestos]{Por cada uno de los experimentos realizados se muestra la relación entre las $d'$ estimadas de acuerdo a la ejecución de cada participante. En ambos experimentos, se observa una tendencia sistemática a obtener niveles mayores de $d'$ en la clase A (la condición con pocos círculos en las figuras de Ebbinghaus).}
\label{fig:Diff_D}
\end{figure}

La Figura~\ref{fig:Diff_D} presenta de manera gráfica la comparación entre los valores de $d'$ estimados por cada nivel de dificultad de acuerdo con los datos obtenidos en los Experimentos 1 y 2, relacionando con una línea los pares de $d'$ estimados por cada participante ($d'(A) y d'(B)$). De acuerdo con esta figura, parece ser claro que las clases de estímulos propuestas cumplieron su objetivo, en tanto que puede apreciarse consistentemente la misma tendencia entre participantes a presentar valores mayores de $d'$ en la clase A que en la clase B.\\

A continuación se presentan los análisis realizados para determinar si las discrepancias que parecieran ser evidentes en la Figura~\ref{fig:Diff_D}, son estadísticamente significativas.\\

\textbf{Análisis Frecuentista: Prueba T para comparar las medias de $d'$ por nivel de dificultad}\\

Tal y como se reporta en los estudios que reportan evidencia del Efecto Espejo en Memoria de Reconocimiento, se realizó una prueba t para comparar la precisión con que los participantes discriminaron los estímulos Señal y los estímulos Ruido de cada clase de estímulos construida. Para ello, se computaron los valores de $d'$ correspondientes al desempeño de los participantes en cada nivel de dificultad y se comparó la diferencia entre los promedios de los valores obtenidos.\\

La Tabla~\ref{Tabla_t-Dprimas} presenta el promedio de las $d'$ computadas por cada nivel de dificultad en cada uno de los experimentos realizados, la diferencia en estas y un indicador sobre su significancia estadística. Como se puede apreciar, la diferencia entre las medias de $d'$ estimadas es significativa en ambos experimentos.\\

\begin{table}
\caption[Prueba T para evaluar las diferencias entre las medias de $d'$ por nivel de dificultad]{Pruebas t para evaluar las diferencias entre las medias de $d'$ computadas por nivel de dificultad}
\label{Tabla_t-Dprimas}
\centering
\begin{tabular}{l | c c c c}
\toprule
%\tabhead{Groups} & \tabhead{Treatment X} & \tabhead{Treatment Y} \\
\textbf{Experimento} & \textbf{$\mu d'(A)$} & \textbf{$\mu d'(B)$} & \textbf{T}  & \textbf{P value}\\
\midrule
Experimento 1 & 3.240 & 2.448 & -3.0587 & 0.0020 \\
Experimento 2 & 1.981 & 1.038 & -3.4131 & 0.0007 \\
\bottomrule
\end{tabular}
\end{table}


\textbf{Análisis Bayesiano: Modelo jerárquico bayesiano para evaluar las diferencias en $d'$}\\

Partiendo del modelo bayesiano estándar que describe los supuestos realizados por la SDT, (Lee y Wagenmakers, \citeyear{LeeBook}), se desarrolló un modelo jerárquico bayesiano (identificado dentro del presente trabajo como Modelo Delta), que asume que tanto los valores de $d'$, como el sesgo $c$ estimado por cada nivel de dificultad se distribuyen de acuerdo a una distribución normal, de donde se extraen los valores computados por cada participante. Bajo este supuesto, el modelo utiliza las inferencias realizadas sobre los valores de $d'$ y $c$ que subyacen al desempeño de cada sujeto para estimar los parámetros que definen la distribución normal de donde se asume que éstos son extraidos (la media ($\mu$) y la desviación estándar ($\sigma$)). Finalmente, el modelo incorpora un parámetro $\delta$ que computa la diferencias entre las medias de las $d'$ estimadas por concidión.\\

\begin{figure}[th]
\centering
\includegraphics[width=1.1\textwidth]{Figures/Model_Delta_Diff_D}
%\decoRule
\caption[Modelo Delta: Modelo jerárquico bayesiano para revisar las diferencias en $d'$ entre clases de estímulos]{Modelo jerárquico bayesiano que asume que la $d'$ y la medida de sesgo $c$ computada por cada participante en cada clase de estímulo proviene de una distribución normal. El modelo incorpora un parámetro ($\delta$) que estima la diferencia entre las medias de $d'$ estimadas por cada condición. El modelo tiene priors no informativas.}
\label{fig:Mod_Delta}
\end{figure}

La Figura~\ref{fig:Mod_Delta} ilustra las relaciones y parámetros definidos en el modelo desarrollado. A este tipo de representaciones, se les conoce como \textit{Modelos Gráficos} y en el caso del Modelo Delta aquí presentado, se presentan los siguientes elementos:\\

\begin{itemize}
\item \underline{Nodos sombreados que representan los datos.}\\

En los modelos gráficos, los nodos representan las variables que el modelo a representar asume tienen una influencia sobre el proceso cognitivo de interés. Dependiendo cómo estén dibujados, los nodos proporcionan información sobre la naturaleza de dichas variables en torno a tres grandes puntos: 1) la variable adopta valores contínuos (nodo circular) o discretos (nodo cuadrado); 2) el valor de la variable es conocido (nodo sombreado) o necesita ser inferido a partir de los datos (nodo claro), y 3) la variable es probabilística (nodo simple) o está determinada por el valor inferido de otras variables (doble nodo), (Lee y Wagenmakers, \citeyear{LeeBook}).\\

En el caso de nuestros experimentos, los datos registrados de manera directa -a partir de los cuales se desarrolla todo el análisis- son el número de ensayos con Ruido ($n$) y Señal ($s$), y los resultados obtenidos en la tarea: el número total de Hits ($H_ij$) y Falsas Alarmas (($Fa_ij$) cometidos por cada participante ($i$) en cada condición ($j$).\\

\item \underline{Las tasas de Hits y Falsas Alarmas como una probabilidad oculta.}\\ 

Mientras que en la literatura clásica en SDT las tasas registradas de Hits y Falsas Alarmas son interpretadas como reflejo directo del área de las distribuciones de Señal y Ruido que caen por encima del criterio de elección, (Wickens, \citeyear{Wickens1}; Gescheider, \citeyear{Gescheider}; Stainslaw y Todorov, \citeyear{Stainslaw1999}, bajo el marco del modelamiento bayesiano, el total registrado de Hits y Falsas Alarmas ($H_ij$ y $Fa_ij$) se toma una instancia de un conteo de casos específicos encontrados dentro de un conjunto de observaciones ($s$ y $n$, respectivamente) con cierta probabilidad ($\theta^H_{ij}$ y $\theta^F_{ij}$).\\

En otras palabras, el modelo bayesiano propuesto asume que $H_ij$ y $Fa_ij$ representan un \textit{'número de éxitos'} extraídos con una probabilidad oculta de un conjunto definido de observaciones, de acuerdo a una distribución binomial con parámetros $p=$ ($\theta^H_{ij}$ o $\theta^F_{ij}$) y $n=$ ($s$ o $n$). Es decir:\\

\begin{center}
$H_{ij}\sim \mathrm{Binomial}\bigl(\theta^H_{ij}, s)$
y
$F_{ij}\sim \mathrm{Binomial}\bigl(\theta^F_{ij}, n)$\\
\end{center}

\item \underline{Sesgo y Discriminabilidad}\\

De acuerdo con la SDT, la probabilidad que determina las tasas registradas de Hits y Falsas Alarmas ($\theta^H_{ij}$ y $\theta^F_{ij}$) representan el área de las distribuciones de Señal y Ruido que caen por encima de los criterios de elección utilizados por los participantes. Es decir, que dicha probabilidad está definida como la función de densidad acumulada (CDF, por sus siglas en inglés, típicamente representada con el parámetro $\phi$) en una distribución normal estándar para la ubicación de $x$ que corresponde a la diferencia entre $\frac{1}{2}D_{ij}$ (en el caso de los Hits) o $\frac{1}{2}D_{ij}$ (en el caso de las Falsas Alarmas), y la medida de sesgo $C_{ij}$. Es decir:\\

\begin{center}
$\theta^H_{ij}\gets \phi (\frac{1}{2}D_{ij}-C_{ij})$
y
$\theta^F_{ij}\gets \phi (-\frac{1}{2}D_{ij}-C_{ij})$\\
\end{center}

\CONSIDERAR \AGREGAR \UNA \FIGURA \ILUSTRATIVA

\item \underline{Plato de participantes}\\

En los modelos gráficos bayesianos se utilizan platos para representar lo que en cualquier lenguaje de programación se conoce como un \textit{ciclo for}, (Lee y Wagenmakers, \citeyear{LeeBook}). Es decir, los platos delimitan conjuntos de parámetros cuyo cómputo se realizará tantas veces como casos -o conjuntos de datos- represente el plato. Por ejemplo, un plato $a$ que contiene los parámetros $p_1, p_2... p_n$ indica que el cómputo de estos se realizará por cada caso contenido en $a$.\\

En el caso del modelo desarrollado, el primer plato ($i$ $participantes$) señala que el cómputo de las probabilidades ocultas tras la emisión de cada par de Hits y Falsas Alarmas registrado ($\theta^H_{ij}$ y $\theta^F_{ij}$) y la estimación del valor de los parámetros $D_{ij}$ y $C_{ij}$ que mejor permitan dar cuenta de dichas probabilidades a partir de la CDF de un par de distribuciones normales, se va a repetir y realizar por cada uno de los participantes ($i$) incluidos en el experimento -es decir, por cada par de Hits y Falsas Alarmas recolectado-.\\

No hay necesidad de incluir los parámetros $n$ y $s$, que representan el total de ensayos con Ruido y Señal contenidos en el experimento, ya que estos permanecen constantes para todos los participantes ($i$) y para todas las clases de estímulos ($j$).\\

\item \underline{Estructura jerárquica: Distribuciones que describen los datos individuales}\\

Como ya se mencionó, la cualidad esencial del modelo desarrollado es que asume que los parámetros $D_{j}$ y $C_{j}$ computados por cada sujeto $i$ provienen de distribuciones normales que describen la discriminabilidad ($d'$) y el sesgo en las respuestas ($c$) asociado a cada clase de estímulos $j$. Dichas distribuciones están definidas por los parámetros $\mu^D_{j}$, $\mu^C_{j}$ y $\sigma^D_{j}$, $\sigma^C_{j}$, que el modelo infiere a partir de los datos ($H_ij$ y $Fa_ij$). Es decir:\\

\begin{center}
$D_{ij}\sim \mathrm{Gaussian}\bigl(\mu^D_{j},\lambda^D_{j})$\\
y\\
$C_{ij}\sim \mathrm{Gaussian}\bigl(\mu^C_{j},\lambda^C_{j})$\\
\end{center}

\item \underline{Plato de Condición}\\

Un segundo plato ($j$ $condiciones$) señala que el cómputo de los parámetros que definen las distribuciones normales asociadas al sesgo y la discriminabilidad, se va a realizar de manera independiente para cada clase de estímulo, a partir del par de Hits y Falsas Alarmas registrado para cada uno por cada participante ($plato$ $i$ $participantes$).\\

\item \underline{Parámetro Delta}\\

Finalmente, se incluye un parámetro Delta ($\delta$) que computa directamente la diferencia entre las inferencias realizadas acerca del valor de las medias de $d'$ por cada nivel de dificultad, ($\mu^D_{A}$ y $\mu^D_{B}$). Es decir:\\

\begin{center}
$\delta \gets \mu^D_{A}-\mu^D_{B}$\\
\end{center}

El parámetro $\delta$ está representado con un doble nodo para señalar que es un parámetro determinado por los valores estimados para otras variables ($\mu^D_{A}$ y $\mu^D_{B}$) de manera directa -no probabilística-.\\
\end{itemize} 

En la Figura~\ref{fig:Delta_Joints} se presentan las densidades de probabilidad posterior marginales y conjunta, para las inferencias realizadas en cuanto al valor de las medias de las distribuciones de $d'$ y $c$ por cada condición, en cada uno de los Experimentos llevados a cabo, (Experimento 1 en el panel superior y Experimento 2 en el inferior). En los extremos de cada gráfica se presenta la densidad de probabilidad marginal computada para la media de cada parámetro a lo largo de distintos valores, distinguiendo con colores las clases de estímulos diseñadas (en azul se presenta la clase fácil A y en púrpura, la clase difícil B). El panel central presenta las densidades de probabilidad posterior conjuntas entre distintos pares de valores para $\mu d'$ y $\mu c$. Los colores empleados en estos gráficos serán utilizados recurrentemente durante la presentación de los resultados y su análisis para diferenciar la clases A y B.\\ 

\begin{figure}[th]
\centering
\includegraphics[width=0.7\textwidth]{Figures/MDelta_Joint_E1}\\
\includegraphics[width=0.7\textwidth]{Figures/MDelta_Joint_E2}\\
%\decoRule
\caption[Modelo Delta: Distribuciones posteriores marginales y conjuntas para $\mu d'$ y $\mu c$ por cada clase de estímulo]{Se presenta la densidad posterior conjunta y marginal computada por el Modelo Delta para las medias de $d'$ y $c$ de cada clase, por cada experimento realizado.}
\label{fig:Delta_Joints}
\end{figure}

De acuerdo con la Figura~\ref{fig:Delta_Joints} parece ser que, en general, los valores estimados por el modelo para las medias de $d'$ en cada clase de estímulo son de hecho diferentes. Esto se puede observar tanto en las distribuciones de densidad marginal trazadas por cada clase de estímulo, que se sobrelapan sobre un rango pequeño de valores y con baja probabilidad, como en la distribución de los puntos trazados por cada pareja de valores inferidos de $d'$ y $c$ por cada clase de estímulo, donde se puede distinguir con facilidad dos aglomeraciones de puntos estimados: las inferencias realizadas para la clase A, que tiende a tener valores más bajos de $\mu d'$ y las inferencias sobre la clase B, con valores $\mu d'$ más altos. La distancia entre los valores de $\mu d'(A)$ y $\mu d'(B)$ estimados es mayor en el Experimento 1, donde sólo se presentó una ilusión de Ebbinghaus por ensayo.\\


Es interesante señalar que, de acuerdo con las inferencias presentadas en la Figura~\ref{fig:Delta_Joints}, los valores estimados para $\mu c$ por cada clase de estímulo se despliegan en un mismo rango de valores, siendo que las densidades posteriores marginales se encuentran completamente sobrelapadas. Más interesante aún es notar que, de acuerdo con las estimaciones del modelo, los participantes en el Experimento 1 no mostraron sesgo alguno al responder a los estímulos de la tarea. Por otro lado, en el Experimento 2 se aprecian diferencias en el sesgo promedio estimado para cada clase de estímulos, siendo que para la clase A presenta un sesgo liberal, mientras que para la clase B se sugiere un sesgo neutro -tal y como se reportó en el Experimento 1-.\\

Para evaluar la variabilidad de los valores de $d'$ y $c$ computados por cada participante para cada clase de estímulo, las Figuras~\ref{fig:Delta_Dprima} y \ref{fig:Delta_Cbias} presentan las densidades de probabilidad posterior computadas por cada par de Hits y Falsas Alarmas registrado en cada una de las clases de estímulos. En estas Figuras se presenta además, con una línea más gruesa, la densidad de probabilidad de los valores estimados para las medias de cada parámetro para las clases A y B.\\

\begin{figure}[th]
\centering
\includegraphics[width=0.6\textwidth]{Figures/MDelta_Dprima_E1}\\
\includegraphics[width=0.6\textwidth]{Figures/MDelta_Dprima_E2}\\
%\decoRule
\caption[Modelo Delta: Densidades posteriores de los valores de $d'$ estimados individualmente y en promedio, por experimento]{Densidades posteriores de los valores de $d'$ y estimados por el Modelo Delta por cada clase de estímulo (azul A, purpura B), individualmente (líneas delgadas) y en promedio (líneas gruesas)}
\label{fig:Delta_Dprima}
\end{figure}

En la Figura~\ref{fig:Delta_Dprima} se presentan las estimaciones realizadas por cada participante acerca del valor de $d'$ que mejor describe su desempeño ante cada clase de estímulo, distinguiendo en paneles distintos los valores estimados en el Experimento 1 (panel superior) y 2 (panel inferior). De acuerdo con las densidades de probabilidad posterior estimadas por cada par de Hits y Falsas Alarmas registrado, puede observarse que en el extremo derecho -que corresponde con valores de $d'$ altos que indican una alta discriminabilidad- se encuentran una mayor cantidad de inferencias realizadas para la clase fácil A, en tanto que en el extremo izquierdo -valores bajos de $d'$ que sugieren una baja discriminabilidad- se concentra una mayor cantidad de distribuciones de densidad correspondientes al cómputo realizado para la clase difícil B. De manera adicional, la figura presenta con líneas más gruesas los valores estimados para las medias de $d'$ en cada condición, permitiendo que su exploración sea realizada a la luz de la variabilidad capturada por las estimaciones individuales.\\

\begin{figure}[th]
\centering
\includegraphics[width=0.6\textwidth]{Figures/MDelta_Cbias_E1}\\
\includegraphics[width=0.6\textwidth]{Figures/MDelta_Cbias_E2}\\
%\decoRule
\caption[Modelo Delta: Densidades posteriores de los valores de $c$ estimados individualmente y en promedio, por experimento]{Densidades posteriores de los valores de $c$ y estimados por el Modelo Delta por cada clase de estímulo (azul A, purpura B), individualmente (líneas delgadas) y en promedio (líneas gruesas)}
\label{fig:Delta_Cbias}
\end{figure}

Por su parte, en cuanto a los valores del parámetro $c$ estimados por cada participante, en la Figura~\ref{fig:Delta_Cbias} se pueden apreciar diferencias en términos de las inferencias realizadas a partir de los datos recopilados en cada experimento. En el Experimento 1, pese a la variabilidad en las estimaciones realizadas, no parece haber ninguna relación entre el tipo de estímulo y los valores de $c$ estimados, tal y como se ilustra con el sobrelape total en que se presentan las densidades de probabilidad posterior estimadas para las medias de ambos grupos. Por otro lado, en el caso del Experimento 2 sí parece haber una ligera diferencia en términos del sesgo con que los participantes responden a cada clase de estímulos, siendo que en general se observan valores de $c$ por debajo de 0 (sesgo liberal) en los estímulos de la clase A, mientras que en la clase B se observa la misma ausencia de sesgo ($c$ cercano a $0$) que en los estímulos presentados en el Experimento 1.\\

Finalmente, la Figura~\ref{fig:Delta} presenta la densidad de probabilidad posterior computada para las diferencias entre las medias de $d'$ computadas por cada condición, que se traducen en distintos valores del parámetro $\delta$. Tal como se puede apreciar en la figura, parece ser que a la luz de los datos registrados en los dos experimentos realizados, es muy poco probable que la diferencia entre las $d'$ asociadas a cada clase de estímulo sea 0. De acuerdo con las gráficas presentadas, los picos más altos de densidad de probabilidad aparecen cerca de 1, lo que sugeriría que la media de las $d'$ asociadas a cada clase de estíimulos difieren alrededor de una unidad de desviación estándar.\\

\begin{figure}[th]
\centering
\includegraphics[width=0.6\textwidth]{Figures/MDelta_DensidadDelta_E1}\\
\includegraphics[width=0.6\textwidth]{Figures/MDelta_DensidadDelta_E2}\\
%\decoRule
\caption[Modelo Delta: Densidad posterior de los valores estimados para el parámetro Delta en cada Experimento]{Densidad posterior computada en cada experimento respecto de los valores de Delta, que refiere a la diferencia entre las medias estimadas de $d'$ por cada clase de estímulos.}
\label{fig:Delta}
\end{figure}

En conjunto, los análisis de datos realizados arrojaron evidencia a favor de la manipulación experimental implementada en el diseño de los estímulos construidos para formar los dos niveles de dificultad a comparar. Se confirmó la existencia de una diferencia significativa en la discriminabilidad ($d'$) asociada a cada clase de estímulos, que además se reportó de acuerdo a lo esperado con base en la literatura revisada para el diseño de los estímulos: las figuras de Ebbinghaus compuestas por un número menor de círculos externos (clase A), tuvieron valores de $d'$ más grandes que las figuras de la clase B, formados por un número mayor de círculos externos.\\

Una vez comprobada la relación $d'(A) > d'(B)$, se prosiguió a evaluar la presentación de los patrones de respuesta identificados como Efecto Espejo en Memoria de Reconocimiento, en los datos obtenidos en el presente trabajo.\\












\subsection{Diferencias en las Tasas de Hits y Falsas Alarmas}

Como se recordará de la revisión presentada en el Capítulo 1 acerca del fenómeno identificado como Efecto Espejo en Memoria de Reconocimiento, la evidencia en favor del mismo se presenta en Tareas de respuesta binaria Sí/No a partir del siguiente patrón de respuestas:\\
 
\begin{center}
$FA(A) < FA(B) < H(B) < H(A)$\\
\end{center}
\begin{center}
donde $FA$ y $H$ señalan las tasas de Hits y Falsas Alarmas observadas durante la tarea en cada clase, (Glanzer y cols., \citeyear{Glanzer1993}).\\
\end{center}

Para evaluar la presentación de dicho patrón en los datos obtenidos en el presente estudio, se comparó el número de Hits y Falsas Alarmas cometidos en cada una de las clases de estímulos construidas por cada participante.\\

Una forma sencilla de explorar dicha relación fue mediante la realización de gráficas de barras que presentaran las diferencias en los resultados obtenidos por los participantes para los distintos tipos de estímulos presentados. Como un caso ejemplar, en la Figura~\ref{fig:MirrorRate_E2_P4} se muestra la frecuencia absoluta de Hits y Falsas Alarmas obtenidas por el Participante 4 en el Experimento 2, por cada clase de estímulos construido, distinguiendo la clase fácil A en color azul y la clase difícil B, en púrpura; presentando el patrón de respuestas identificado como Efecto Espejo. Las gráficas correspondientes al desempeño del resto de los participantes en los Experimentos 1 y 2, pueden consultarse en los Apéndices.\\
\\

\begin{figure}[th]
\centering
\includegraphics[width=0.60\textwidth]{Figures/MirrorRate_Exp2_P4}
%\decoRule
\caption[Diferencias entre Hits y Falsas Alarmas por Condición; Participante ejemplar]{Gráfica de barras que presenta el número de Hits y Falsas Alarmas cometidos por el Participante 4 del Experimento 2 en cada clase de estímulo, (clase fácil A en azul y clase difícil B en púrpura). Los resultados se presentan de acuerdo al órden ascendente entre los resultados reportados en el Efecto Espejo.}
\label{fig:MirrorRate_E2_P4}
\end{figure}


La comparación global entre las tasas de Hits y Falsas Alarmas registradas por cada participante de los Experimentos 1 y 2 en cada tipo de estímulo se presenta en la Figura~\ref{fig:Diff_Rate}. De acuerdo con el patrón de respuestas reportado como parte del Efecto Espejo, se espera observar una pendiente descendente para la comparación de las tasas de Hits registradas por cada condición (paneles izquierdos) y una pendiente ascendente en el registro de tasas de Falsas Alarmas (paneles derechos). Como se puede observar, parece ser que dichas tendencias aparecen en la mayoría de los participantes, presentándose de manera más evidente en los datos registrados en el Experimento 1. Sin embargo, resulta poco claro si las diferencias entre las tasas registradas en una y otra clase de estímulos son lo suficientemente grandes como para considerarse estadísticamente significativas.\\

\begin{figure}[th]
\centering
\includegraphics[width=0.65\textwidth]{Figures/Diff_Rate_E1}\\ 
\includegraphics[width=0.65\textwidth]{Figures/Diff_Rate_E2}\\
%\decoRule
\caption[Diferencias entre las Tasas de Hits y Falsas Alarmas registradas en cada clase de estímulos]{Se presenta la comparación, participante a participante, entre las tasas de Hits y Falsas Alarmas registradas para cada clase de estímulo, (paneles izquierdos y derechos, respectivamente). El panel superior muestra las comparaciones correspondientes al Experimento 1 y el panel inferior, al Experimento 2.}
\label{fig:Diff_Rate}
\end{figure}

A continuación se presentan los análisis realizados para determinar si los Hits y las Falsas Alarmas registrados por cada clase de estímulos son estadísticamente diferentes.\\

\textbf{Análisis Frecuentista: Pruebas T para comparar las tasas de Hits y Falsas Alarmas registradas por cada clase de estímulos}\\

De acuerdo con la literatura, la manera más apropiada de evaluar la significancia estadística del patrón de respuestas reportado como Efecto Espejo en Tareas de respuesta Sí/No, es mediante la realización de dos pruebas-T que evalúen de manera independiente las diferencias entre las medias de las tasas de Hits reportadas por cada clase de estímulo y entre las tasas de Falsas Alarmas, (Glanzer y Adams, \citeyear{Glanzer1990}).\\

%Las pruebas t realizadas para evaluar la significancia estadística de las diferencias encontradas entre los promedios de las tasas computadas, se enfrentan con el problema de que los datos a comparar se encuentran restrigidos dentro de un rango de $0$ a $1$. Más aún, de acuerdo con los supuestos de la SDT sobre el despliegue de las distribuciónes de Ruido y Señal sobre el eje de evidenncia, las tasas de ejecución se presentan dentro de rangos aún más restringidos, siendo que los aciertos (Hits y Rechazos Correctos) caen por arriba del $0.5$, y los errores (Omisiones y Falsas Alarmas), por debajo. Esto representa un problema para la comparación de los resultados encontrados, porque puede darse el caso de que las diferencias entre los datos registrados no sean lo suficientemente grandes como para considerárseles significativas, con independencia de si se presentan de forma consistente en la mayoría de los datos individuales.\\

%Para solucionar el problema del efecto de suelo y techo que podría estar mermando la evaluación de los datos obtenidos como evidencia del impacto de la manipulación experimental sobre el desempeño de los participantes, los datos crudos a comparar suelen transformarse a una nueva Escala, de manera que la distancia entre estos aumente, pero manteniendo la proporción de los mismos intacta. En el caso específico de la literatura que aborda el Efecto Espejo en Memoria de Reconocimiento, se ha optado por implementar una transformación arcoseno de las tasas de respuesta registradas antes de someterlas al análisis estadístico.\\

%De acuerdo con las pruebas t realizadas en el presente estudio para comparar los promedios de las transformaciones arcoseno de las tasas de ejecución registradas por cada clase de estímulo, se encontraron diferencias significativas en la ejecución de los participantes a la tarea Sí/No dependientes de la clase de estímulo a evaluar en ambos experimentos, reportándose diferencias altamente significativas entre las tasas de Hits y diferencias apenas significativas (justo por debajo de $p=0.05$) para las tasas de Falsas Alarmas.\\

Además, para prevenir el efecto de suelo y techo causado por el hecho de que los datos a comparar éstán restringidos dentro de un rango entre $0$ y $1$, en la literatura se recomienda hacer una transformación arcoseno de las tasas registradas de Hits y Falsas Alarmas por cada clase de estímulos, (Glanzer y Adams, \citeyear{Glanzer1990}).\\ 

En la Tabla~\ref{Tabla_t-HitsyFA} se presentan los resultados obtenidos en la prueba-T de muestras independientes realizada para evaluar las diferencias entre las medias de las tasas de Hits y Falsas Alarmas obtenidas entre cada clase de estímulo, por cada Experimento realizado. En la tabla se muestra tanto la media de las tasas crudas registradas, como de su transformación arcoseno.\\

\begin{table}
\caption[Prueba T para evaluar diferencias entre las tasas de ejecución (Hits y F. Alarmas) promedio registradas por cada condición]{Se presentan los resultados de las pruebas t de una muestra realizadas para la comparación del promedios de las transformaciones arcoseno de las tasas registradas por cada clase de estímulos, en los Experimentos 1 y 2.}
\label{Tabla_t-HitsyFA}
\centering
\begin{tabular}{l l | c c c c c c}
\toprule
%\tabhead{Groups} & \tabhead{Treatment X} & \tabhead{Treatment Y} \\
\textbf{} & \textbf{Tasa} & \textbf{$\mu(A)$} & \textbf{$arcsin(\mu(B))$} & \textbf{$\mu(B)$} & \textbf{$arcsin(\mu(B))$} &\textbf{T} & \textbf{P value}\\
\midrule
Exp 1 & Hits & 0.922 & 1.314 & 0.860 & 1.209 & -2.4348 & 0.0098 \\
Exp 1 & FA & 0.08 & 0.247 & 0.143 & 0.353 & 1.872 & 0.0345 \\
Exp 2 & Hits & 0.857 & 1.219 & 0.681 & 0.994 & -3.3595, & 0.0009 \\
Exp 2 & FA & 0.266 & 0.524 & 0.336 & 0.611 & 1.7223 & 0.0468 \\
\bottomrule
\end{tabular}
\end{table}

De acuerdo con los resultados de las pruebas T presentadas en la Tabla~\ref{Tabla_t-HitsyFA}, las diferencias entre el promedio de las transformaciones arcoseno de las tasas de Hits y Falsas Alarmas registradas por cada clase de estímulo, son estadísticamente significativas en ambos experimentos.\\

\textbf{Análisis Bayesiano: Modelo bayesiano desarrollados para comparar las tasas de Hits y Falsas Alarmas entre clases de estímulos}\\

Cuando las mismas pruebas T presentadas en la Tabla~\ref{Tabla_t-HitsyFA} se realizan sin la transformación arcoseno de los datos, las diferencias entre las Tasas de Falsas Alarmas no resultan significativas en ningún experimento ($t=1.6536$, $p=0.0533>0.05$ para el Experimento 1 y $t=1.6577$, $p=0.0534>0.05$, para el Experimento 2). Esta discrepancia entre las conclusiones sugeridas por las pruebas t realizadas puede llegar a generar dudas acerca de su confiabilidad, aún cuando la transformación arcoseno se realiza de manera justificada, como una forma de prevenir el efecto de suelo y techo que pudiera afectar la evaluación de diferencias entre datos que se encuentren en la cercanía de los límites superior e inferior.\\

El desarrollo de un modelo bayesiano para evaluar las diferencias entre los Hits y las Falsas Alarmas registradas por cada clase de estímulo, presenta como ventaja principal que no requiere manipular los datos en forma alguna, en tanto que la estimación paramétrica se hace a partir de inferencias probabilísticas que evalúan qué valores de cada parámetro podrían dar cuenta de los datos registrados en conjunto con el resto de los paràmetros, con mayor probabilidad. Para ello, se desarrolló un modelo bayesiano (identificado en el presente trabajo como Modelo Tau), que incorpora un par de parámetros Tau ($\tau$) que computan las diferencias entre las estimaciones realizadas por cada clase de estímulos ($j$) acerca de las probabilidades ocultas tras la emisión de los Hits y Falsas Alarmas obtenidos por cada participante ($i$).\\ 

En la Figura~\ref{fig:Mod_Tau} se presenta el modelo gráfico correspondiente al Modelo Tau desarrollado para evaluar la evidencia del Efecto Espejo encontrada en los experimentos realizados. El modelo cuenta con los mismos elementos que el Modelo Delta desarrollado para comparar las medias de $d'$ por cada clase de estímulo, con las siguientes excepciones:\\

\begin{itemize}
\item \underline{El modelo Tau \textbf{no} asume una estructura jerárquica}\\

El Modelo Tau evalúa la diferencia entre las probabilidades ocultas tras la emisión de Hits y Falsas Alarmas en las clases A y B, por cada individuo. Este modelo no requiere de una estructura jerárquica porque tiene su énfasis en el cómputo determinista de las diferencias entre dos parámetros (las probabilidades ocultas tras los Hits y Falsas Alarmas de cada clase $\theta^H_{ij}$ y $\theta^F_{ij}$) que son a su vez definidos de manera determinista por la interacción de $D_{ij}$ y $C_{ij}$.\\

\item \underline{Parámetros Tau}\\

De la misma forma en que el Modelo Delta incorporaba un parámetro $\delta$ que computaba las diferencias entre las medias estimadas de $d'$ por cada clase de estímulo, el Modelo Tau recibe su nombre tras la inclusión de un par de parámetros Tau ($\tau^H_{i}$ y $\tau^F_{i}$) que determinan cuál es la diferencia entre las probabilidades ocultas tras la emisión de Hits y Falsas Alarmas por cada clase de estímulos:\\

\begin{center}
$\tau^H_{i}\gets \theta^H_{iA}-\theta^H_{iB}$\\
y\\
$\tau^F_{i}\gets \theta^F_{iB}-\theta^F_{iA}$\\
\end{center}

\end{itemize}

\begin{figure}[th]
\centering
\includegraphics[width=1.1\textwidth]{Figures/Model_Tau_Diff_Tetas}
%\decoRule
\caption[Modelo Tau: Modelo Bayesiano para evaluar las diferencias entre las tasas de Hits y Falsas Alarmas obtenidas por cada clase de estímulos]{Modelo bayesiano desarrollado para evaluar las diferencias entre las probabilidades ocultas tras la emisión de Hits y Falsas Alarmas para cada clase de estímulos}
\label{fig:Mod_Tau}
\end{figure}

En las Figuras~\ref{fig:Tau_Hits} y \ref{fig:Tau_FA} se presentan las inferencias realizadas por el Modelo Tau acerca de las probabilidades ocultas tras la producción de Hits y Falsas Alarmas de cada individuo, por cada clase de estímulos, en los Experimentos 1 y 2. \\

\begin{figure}[th]
\centering
\includegraphics[width=0.6\textwidth]{Figures/MTau_Hits_E1}\\
\includegraphics[width=0.6\textwidth]{Figures/MTau_Hits_E2}\\
%\decoRule
\caption[Modelo Tau: Inferencias individuales acerca de las probabilidades ocultas tras la emisión de Hits por clase de estímulos; Experimentos 1 y 2]{Se presentan las densidades de probabilidad posterior obtenidas individualmente para las probabilidades ocultas tras la emisión de Hits por cada clase de estímulos (clase fácil A en azul y clase difícil B, en púrpura).}
\label{fig:Tau_Hits}
\end{figure}

\begin{figure}[th]
\centering
\includegraphics[width=0.6\textwidth]{Figures/MTau_FA_E1}\\
\includegraphics[width=0.6\textwidth]{Figures/MTau_FA_E2}\\
%\decoRule
\caption[Modelo Delta: Inferencias individuales acerca de las probabilidades ocultas tras la emisión de Falsas Alarmas por clase de estímulos; Experimentos 1 y 2]{Se presentan las estimaciones individuales acerca de las probabilidades ocultas tras la emisión de Falsas Alarmas ante cada clase de estímulo (clase fácil A en azul y clase difícil B en púrpura).}
\label{fig:Tau_FA}
\end{figure}

Finalmente, en la Figura~\ref{fig:Tau} se presentan las densidades de probabilidad posterior para los valores de Tau ($\tau^H_{i}$ en tonalidades verdes y $\tau^F_{i}$ en tonos rojizos) computados por cada individuo en cada Experimento realizado. Cada panel incluído en esta Figura contiene una línea punteada que señala $\tau = 0$,  . Como se puede apreciar en la figura, las densidades de probabilidad estimadas para $\tau^H_{i}$ caen por encima del punto de \textit{'no diferencias'} para la mayoría de los participantes, en tanto que en el caso de las densidades de $\tau^F_{i}$ parece haber una mayor dispersión en el rango de valores estimados por cada participante y una mayor concentración de los mismos cae cerca del punto de \textit{'no diferencias'}.\\

\begin{figure}[th]
\centering
\includegraphics[width=0.6\textwidth]{Figures/MTau_DensidadTau_E1} \& \includegraphics[width=0.6\textwidth]{Figures/MTau_DensidadTau_E2}\\
%\decoRule
\caption[Modelo Tau: Densidades de probabilidad posterior para los valores de los parámetros Tau; Experimentos 1 y 2]{Se presentan las densidades de probabilidad posterior para las inferencias realizadas por el Modelo Tau acerca de las diferencias entre las probabilidades ocultas tras la emisión de Hits (Tau-Hits) y Falsas Alarmas (Tau-Falsas Alarmas) por cada clase de estímulos.}
\label{fig:Tau}
\end{figure}

Los resultados arrojados por el Modelo Tau confirman los hallazgos reportados con las pruebas t: la diferencia en el desempeño de los participantes ante cada clase de estímulos parece clara cuando se trata de comparar los Hits, pero no lo es tanto cuando se evalúan las Falsas Alarmas.\\

En el caso de las pruebas t realizadas para el análisis frecuentista, las tasas de Hits y Falsas Alarmas registradas son interpretadas directamente como un reflejo del área de las distribuciones de Ruido y Señal que caen por encima del criterio de elección (Stainslaw y Todorov, \citeyear{Stainslaw1999}) y la transformación arcoseno de las mismas aparece como un \textit{'mal necesario'} que, aún implicando una manipulación de los datos, compensa el impacto de los efectos de suelo y techo sobre la evaluación de las diferencias entre las tasas registradas (Glanzer y Adams, \citeyear{Glanzer1990}). \\ 

Por su parte, el modelamiento bayesiano de detección de señales permite una mayor flexibilidad en la interpretación de la ejecución de los participantes, pues el número de Hits y Falsas Alarmas son interpretadas como extracciones realizadas bajo la influencia de una probabilidad oculta, (Lee y Wagenmakers, \citeyear{LeeBook}). Por ello, los datos recopilados sobre la ejecución de los participantes no son tratados de manera determinista, sino probabilística, como el número de casos observados en un proceso binomial y no es necesario llevar a cabo ninguna manipulación adicional sobre los mismos para compensar la rigidez de su interpretación.\\

En cuanto al análisis de datos conducido en el presente estudio para evaluar la evidencia del Efecto Espejo encontrada en los experimentos realizados, resaltan diferencias importantes: De acuerdo con la significancia estadística sugerida por las pruebas t realizadas tras la trasnformación arcoseno de los datos, los datos obtenidos en el presente estudio presentan evidencia sólida sobre el Efecto Espejo en tareas de detección perceptual; sin embargo, de acuerdo a las inferencias realizadas a partir de los datos crudos con el modelo bayesiano desarrollado, parece poco claro si el patrón de respuestas identificado como Efecto Espejo se presenta con suficiente claridad y consistencia, sobre todo en términos de la comparación de las Falsas Alarmas.\\

En cualquier caso, con independencia de si los Hits y las Falsas Alarmas registrados por cada clase de estímulo resultan significativamente diferentes o no, resulta claro que el patrón de respuestas del Efecto Espejo (el orden de los resultados obtenidos por cada participante) aparece en una proporción significativa de los participantes, (ver Tabla~\ref{Tabla_Binom}).\\ 















\subsection{Diferencias en la asignación de Puntajes de Confianza}

La evidencia del Efecto Espejo presentada en la tarea con Escala de Confianza se evaluó en función a la prevalencia y significancia del patrón de respuestas identificado en la literatura en Memoria de Reconocimiento:\\
 
\begin{center}
$P(NuevoA) < P(NuevoB) < P(ViejoB) < P(ViejoA)$\\
\end{center}
\begin{center}
donde $P$ es el puntaje promedio asignado a los estímulos Nuevos y Viejos de cada clase de estímulo $A y B$, dentro de una Escala de Confianza donde los valores más altos señalan una mayor confianza en el juicio \textit{'Viejo'} y los valores más bajos, en \textit{'Nuevo'}, (Glanzer y cols., \citeyear{Glanzer1993}).\\
\end{center}

Para evaluar el Efecto Espejo en los datos obtenidos con los Experimentos realizados, se compararon los puntajes de confianza asignados en promedio a los estímulos con Ruido y Señal de cada clase. La Figura~\ref{fig:MirrorRating_E1_P10} presenta una ejemplo de la exploración visual de dicha relación donde se muestra el promedio de los puntajes de confianza asignados por el Participante 10 del Experimento 1 a los estímulos pertenecientes a cada clase A y B. En la figura se puede apreciar que este participante presentó claramente la tendencia identificada en estudios de Memoria de Reconocimiento. Las gráficas correspondientes al desempeño del resto de los participantes en los Experimentos 1 y 2, pueden consultarse en los Apéndices.\\

\begin{figure}[th]
\centering
\includegraphics[width=0.60\textwidth]{Figures/MirrorRating_Exp1_P10}
%\decoRule
\caption[Comparación entre Puntajes de Confianza asignados por Clase; Ejemplo]{Se muestra el promedio de los Puntajes de Confianza asignados por el Participante 10 del Experimento 1 a los estímulos con Señal y Ruido de cada clase.}
\label{fig:MirrorRating_E1_P10}
\end{figure}

La revisión global de los Puntajes de Confianza promedio asignados por cada participante a los estímulos con Señal y Ruido de la clase A o B se presenta en la Figura~\ref{fig:Diff_Ratings}. Una vez más, de acuerdo con el patrón de respuestas identificado con el Efecto Espejo, se espera observar una pendiente descendente para las comparaciones presentadas en los páneles izquierdos y una pendiente ascendente en los derechos. En la figura se puede apreciar que la mayoría de los participantes presentan diferencias entre los Puntajes de Confianza emitidos que coincíden con el Efecto Espejo, sin embargo, resulta difícil determinar mediante la pura exploración visual de los datos si estas son lo suficientemente grandes como para considerarse un patrón significativo.\\

\begin{figure}[th]
\centering
\includegraphics[width=0.65\textwidth]{Figures/Diff_Rating_E1}\\ 
\includegraphics[width=0.65\textwidth]{Figures/Diff_Rating_E2}\\
%\decoRule
\caption[Diferencias entre los Puntajes de Confianza promedio asignados a los estímulos con Señal y Ruido de cada clase]{Se presenta la comparación, participante a participante, entre los puntajes de confianza asignados en promedio a los estímulos con Ruido y Señal por cada clase de estímulos. El panel superior muestra las comparaciones correspondientes al Experimento 1 y el panel inferior, al Experimento 2.}
\label{fig:Diff_Ratings}
\end{figure}

A continuación se presentan los análisis realizados para determinar si los Hits y las Falsas Alarmas registrados por cada clase de estímulos son estadísticamente diferentes.\\

\textbf{Análisis Frecuentista: Pruebas t para comparar la emisión de puntajes de confianza por cada clase de estímulo.}\\

Para evaluar las diferencias encontradas entre los promedios de los Puntajes de Confianza asignados a los estímulos con Señal y Ruido de cada clase se realizaron  dos pruebas-t de muestras independientes, (Glanzer y Adams, \citeyear{Glanzer1990}). Para este análisis no es necesario recurrir a la transformación de los datos a una nueva Escala, puesto que la comparación radica en el promedio de los puntajes asignados.\\

En la Tabla~\ref{Tabla_t-Confidence} se presentan los Puntajes de Confianza promedio asignados a cada clase de estímulos con Señal y Ruido y los resultados obtenidos en las pruebas-t de muestras independientes realizadas para su comparación. De acuerdo con este análisis, las diferencias son signifivativas ($p<0.05$) en ambos Experimentos tanto para los Puntajes emitidos  en los ensayos con Señal como en los ensayos con Ruido. Sin embargo, con excepción de las diferencias observadas en el Experimento 2 en los estímulos con Señal, las diferencias reportadas obtienen valores $p$ apenas por debajo del nivel de significancia estadística estándar ($p=0$.$05$).\\

\begin{table}
\caption[Prueba T para evaluar diferencias en las medias de los puntajes de confianza asigandos entre condiciones]{Se presentan los resultados de las pruebas t de muestras independientes realizadas para comparar el promedio de los puntajes de confianza asignados por los participantes a los estímulos con Ruido y Señal de cada clase de estímulo, en los Experimentos 1 y 2.}
\label{Tabla_t-Confidence}
\centering
\begin{tabular}{l l |  c c c c}
\toprule
%\tabhead{Groups} & \tabhead{Treatment X} & \tabhead{Treatment Y} \\
\textbf{Experimento} & \textbf{Ensayo} & \textbf{$\mu$P(A)} & \textbf{$\mu$P(B)} & \textbf{T} & \textbf{P value}\\
\midrule
Exp 1 & Señal & 5.4453 & 5.2128 & -1.7778, & 0.0418 \\
Exp 1 & Ruido & 1.5428 & 1.8834 & -1.7208 & 0.0472 \\
Exp 2 & Señal & 5.2009 & 4.3618  & -3.5126, & 0.0006 \\
Exp 2 & Ruido & 2.3853 & 2.7578 & -1.7544 & 0.0432 \\
\bottomrule
\end{tabular}
\end{table}

\textbf{Análisis Bayesiano: Prueba t bayesiana para comparar la emisión de puntajes de confianza entre clases de estímulos.}\\

El complemento bayesiano de las pruebas-t frecuentistas ya presentadas se llevó a cabo mediante la realización de su homólogo bayesiano, que ofrece una alternativa a los $p-values$ como criterio estadístico para juzgar la significancia del efecto encontrado, a partir del uso del Factor de Bayes, (Lee, \citeyear{Lee2016}). En otras palabras, mientras que los análisis estadísticos frecuentistas se desarrollan en torno a la Comparación de Hipótesis, utilizando los $p-values$ como un indicador de qué tan probable sería observar los datos evaluados \textit{por puro azar}, el Factor de Bayes funciona a partir de la Comparación de Modelos y señala la razón entre la densidad de probabilidad posterior obtenida con los datos para el punto de \textit{'no diferencias'} y la densidad prior que corresponde con la hipótesis inicial (en otras palabras: \textit{'¿Cuántas veces es más probable que, a la luz de los datos, la diferencia entre las muestras evaluadas sea igual a $0$?'}).\\

La versión bayesiana de las pruebas t de muestras independientes se realizó mediante el uso del Software especializado JASP (Jeffreys's Amazing Statistics Program), diseñado para facilitar la realización de análisis estadísticos frecuentistas y bayesianos a partir de la lectura de archivos CSV, proporcionando tablas y gráficos informativos de manera automatizada.\\

El software JASP permite visualizar los datos ingresados para su análisis mediante la presentación de diversos gráficos descriptivos. Por ejemplo, en la Figura~\ref{fig:JASP_MeanRatings} se presentan un par de gráficos donde se señalan las medias de los Puntajes de Confianza asignados a los estímulos con Ruido y Señal de cada clase, y su dispersión. Esta figura confirma la evidencia presentada por la Figura~\ref{fig:Diff_Ratings}. Los promedios computados por cada clase de estímulos difieren en el sentido esperado de acuerdo al Efecto Espejo, pero la dispersión de los datos registrados presenta un sobrelape importante, cubriendo rangos de valores muy cercanos.\\

\begin{figure}[th]
\centering
\includegraphics[width=0.85\textwidth]{Figures/JASP_MeanRatings}\\ 
%\decoRule
\caption[Dispersión en las Tasas de Hits y Falsas Alarmas registradas en cada clase de estímulos]{Se presentan los promedios y dispersión de los puntajes de confianza asignados a los estímulos con Señal (parte superior) y Ruido (parte inferior) de cada clase de estímulo, en los Experimentos 1 y 2 (paneles izquierdos y derechos, respectivamente). Este gráfico fue generado por el software JASP.}
\label{fig:JASP_MeanRatings}
\end{figure}

En cuanto a los resultados obtenidos tras la realización de las pruebas-t bayesianas, en la Tabla~\ref{Tabla_t-Bayesian} se presentan nuevamente los Puntajes de Confianza promedio asignados por cada clase de estímulo en los ensayos con Señal y Ruido, y el Factor de Bayes computado para cada comparación. Como se puede observar, los Factores de Bayes computados son bastante cercanos para la mayoría de las comparaciones realizadas, con la excepción de los puntajes promedios registrados para las diversas clases de estímulos con Señal.\\

\begin{table}
\caption[Prueba T bayesiana para evaluar diferencias en las medias de los puntajes de confianza asigandos entre condiciones]{Se muestra el Factor de Bayes computado a partir de las diferencias entre los puntajes de confianza asignados a los estímulos con Ruido y Señal de cada clase, de acuerdo con pruebas t bayesianas de muestras independientes.}
\label{Tabla_t-Bayesian}
\centering
\begin{tabular}{l l |  c c c c}
\toprule
%\tabhead{Groups} & \tabhead{Treatment X} & \tabhead{Treatment Y} \\
\textbf{Experimento} & \textbf{Ensayo} & \textbf{$\mu$P(A)} & \textbf{$\mu$P(B)} & \textbf{$BF_{10}$} \\
\midrule
Exp 1 & Signal & 5.445 & 5.213 & 1.989 \\
Exp 1 & Noise & 1.543 & 1.883 & 1.832 \\
Exp 2 & Signal & 5.201 & 4.362  & 54.983 \\
Exp 2 & Noise & 2.385 & 2.758 & 1.923 \\
\bottomrule
\end{tabular}
\end{table}

Una forma más ilustrativa de presentar la información proporcionada por los Factores de Bayes reportados en la Tabla~\ref{Tabla_t-Bayesian} es con los gráficos que se muestran en la Figura~\ref{fig:JASP_Tbayesian}. En esta figura se presentan los resultados obtenidos por las pruebas t Bayesianas en la comparación de los puntajes asignados a los estímulos con Señal (en los cuatro paneles superiores) y los estímulos con Ruido (los cuatro inferiores), para cada uno de los Experimentos realizados. La figura está compuesta por gráficos que presentan la siguiente información:\\

\begin{itemize}
\item \textbf{Comparación entre las densidades Prior y Posterior}, \textit{(Paneles izquierdos).}\\

Las gráficas presentadas en los paneles izquierdos presentan la comparación entre la densidad de la distribución prior (con una línea punteada) especificada con base en la hipótesis inicial (el patrón identificado como Efecto Espejo) y la densidad de probabilidad posterior (con una línea sólida) computada a partir de los datos registrados. En estos gráficos se marca la ubicación $\delta = 0$ porque señala el punto de \textit{'no diferencia'} entre las muestras comparadas; si la distribución posterior tuviera una mayor densidad que la distribución prior en este punto, querría decir que los datos analizados proporcionan más evidencia sobre la homogeneidad entre las muestras comparadas de lo que se estimaba en un inicio. En estas gráficas, $\delta$ recibe el nombre de \textit{tamaño del efecto} (\textit{\textit{Size effect}}) porque por default se asume que las diferencias entre las muestras comparadas con la prueba t son un reflejo del efecto que tuvo la manipulación experimental sobre los datos obtenidos.\\ 

En estas mismas gráficas se incluyen los siguientes indicadores:\\

\begin{itemize}
	\item \textit{El Intervalo de Credibilidad y la Mediana.} En el extremo superior derecho se señalan la mediana y el intervalo de credibilidad que cubre el $95\%$ de los valores estimados de $\delta$, (\textit{$95\%$ CI}).\\

	\item \textit{Factor de Bayes en dos direcciones.} En el extremo superior izquierdo se presenta el Factor de Bayes computado a partir de las posibles definiciones de la razón a estimar: 1) ¿Cuántas veces es más probable la Hipótesis Nula ($H0$) que la Hipótesis Alterna? ($BF_{0+}$); y 2) ¿Qué tantas veces es más probable la Hipótesis Alterna ($H+$) que la Hipótesis Nula? ($BF_{+0}$).\\

	\item \textit{Representación gráfica del Factor de Bayes} En la parte superior, junto a las estimaciones del Factor de Bayes, se presenta una representación gráfica de la proporción de evidencia acumulada en favor de una y otra hipótesis, de acuerdo con el Factor de Bayes de interés ($BF_{+0}$).\\
\end{itemize}

\item \textbf{Evaluación de la robustez de la evidencia presentada}, \textit{(Paneles derechos).}\\

Por su parte, las gráficas presentadas en los páneles derechos permiten interpretar los Factores de Bayes computados en términos de qué tanta evidencia arrojan a favor de alguna de las Hipótesis.\\

\end{itemize}

\begin{figure}[th]
\centering
\includegraphics[width=0.7\textwidth]{Figures/JASP_Tbayesian_Signals}\\ 
\includegraphics[width=0.7\textwidth]{Figures/JASP_Tbayesian_Noise}\\ 
%\decoRule
\caption[Diferencias entre las Tasas de Hits y Falsas Alarmas registradas en cada clase de estímulos]{Se presenta la comparación entre las densidades de probabilidad prior y posterior (paneles izquierdos) en el punto de \textit{no diferencia} ($\delta = 0$) que subyace al cómputo de cada Factor de Bayes, y la evaluación de este (paneles derechos) en términos de la robustez de la evidencia presentada (paneles derechos), para las comparaciones hechas entre los estímulos con Señal (panel superior) y Ruido (panel inferior) por cada experimento. Estos gráficos fueron elaborados con el sofware JASP.}
\label{fig:JASP_Tbayesian}
\end{figure}

Como se puede observar en la Figura~\ref{fig:JASP_Tbayesian}, cuando se comparan las densidades de probabilidad prior y posterior, consistentemente se encuentra que la densidad posterior cae por debajo de la prior en el punto de \textit{no diferencias}. Esto sugiere que en general, a la luz de los datos obtenidos es más probable que el valor de $\delta$ sea diferente de $0$, situando el punto de máxima densidad por encima de este (en el caso de los Estímulos con Señal) o por debajo (Estímulos con Ruido). Sin embargo, de acuerdo con la evaluación de la robustez de los Factores de Bayes, parece ser que la evidencia en favor de la diferencia entre las muestras sólo es sólida para los datos obtenidos acerca de Estímulos con Señal del Experimento 2; el resto de las comparaciones arrojan sólo evidencia Anecdótica.\\















\subsection{Réplica de controles reportados en la literatura}

Finalmente, en el presente trabajo se replicó uno de los análisis de control sugeridos en la literatura para evaluar la \textit{'no-trivialidad'} del Efecto Espejo encontrado en los datos: la revisión de una posible correlación entre el tiempo que los participantes se tomaron en responder las tareas planteadas (RT, \textit{Response Time}) y las clases de estímulos entre las cuales se comparó su desempeño, (Glanzer y Adams, \citeyear{Glanzer1990}). La idea detrás de este análisis es que, de encontrarse una correlación entre RT's y clases de estímulos a comparar, se tendría que descartar el supuesto básico que asume que la única diferencia entre las clases A y B es la precisión con que se detectan las Señales y sugeriría que las diferencias en la ejecución de los participantes se deben simplemente a diferencias en el cuidado con que estos responden a la tarea.\\

En los Experimentos realizados y presentados en la presente Tesis se registraron dos Tiempos de Respuesta por ensayo:\\

\begin{itemize}
\item \textbf{RT1:} \textit{Tiempo de Respuesta a la Tarea Binaria.}  Contabilizado desde la presentación de las figuras de Ebbinghaus en pantalla, hasta la emisión de una respuesta a la tarea Sí/No por el participante, con independencia de si la figura seguía presentándose en pantalla o no.\\

\item \textbf{RT2:} \textit{Tiempo de Respuesta a la Escala de Confianza.} Contabilizado desde la aparición de la Escala de Confianza hasta que el participante hubiera registrado una respuesta.\\
\end{itemize}

Primero, en las Tablas~\ref{Tabla_RT1_AB} y ~\ref{Tabla_RT2_AB} se presentan las pruebas t de muestras independientes realizadas para evaluar las diferencias globales entre los Tiempos de Respuesta registrados en promedio en las tareas presentadas (RT1 y RT2, respectivamente) por cada clase de estímulo, en los Experimentos 1 y 2. De acuerdo con las tablas, no se encontraron diferencias significativas en ninguno de los experimentos.\\

Después, por cada clase de estímulos se evaluaron las diferencias en el Tiempo de Respuesta de los Participantes entre los estímulos con Señal y Ruido.\\

Las Tablas~\ref{Tabla_RT1_A} y ~\ref{Tabla_RT2_A} presentan las pruebas t de muestras independientes realizadas por cada Experimento para comparar los promedios de RT1 y RT2 registrados para los estímulos con Ruido y Señal de la clase A, y en las Tablas~\ref{Tabla_RT1_B} y ~\ref{Tabla_RT2_B} se presentan los resultados correspondientes a las pruebas t realizadas sobre la clase B. De acuerdo con estos resultados, no se encontraron diferencias significativas entre los Tiempos de Respuesta registrados para los estímulos con Ruido y Señal de las clases A y B, en ninguno de los experimentos llevados a cabo.\\



%RT1  A vs B
\begin{table}
\caption[Tiempo de Respuesta a la tarea Sí/No por clase de estímulo]{Prueba t para comparar el Tiempo de Respuesta a la Tarea Sí/No entre clases de estímulos.}
\label{Tabla_RT1_AB}
\centering
\begin{tabular}{l |  c c c c}
\toprule
%\tabhead{Groups} & \tabhead{Treatment X} & \tabhead{Treatment Y} \\
\textbf{Experimento} & \textbf{$\mu$RT(A)} & \textbf{$\mu$RT(B)} & \textbf{T} & \textbf{P value}\\
\midrule
Exp 1 & 1.0462 & 1.1175 & -0.4618 & 0.6468 \\
Exp 2 & 0.7262 & 0.7871 & -0.6315 & 0.5315 \\
\bottomrule
\end{tabular}
\end{table}

%RT2  A vs B
\begin{table}
\caption[Tiempo de Respuesta a la Escala de Confianza por clase de estímulo]{Prueba t para comparar el Tiempo de Respuesta a la Tarea con Escala de Confianza entre clases de estímulos}
\label{Tabla_RT2_AB}
\centering
\begin{tabular}{l |  c c c c}
\toprule
%\tabhead{Groups} & \tabhead{Treatment X} & \tabhead{Treatment Y} \\
\textbf{Experimento} & \textbf{$\mu$RT(A)} & \textbf{$\mu$RT(B)} & \textbf{T} & \textbf{P value}\\
\midrule
Exp 1 & 0.7893 & 0.8341 & -0.3535 & 0.7257 \\
Exp 2 & 0.7636 & 0.7680 & -0.0532 & 0.9578 \\
\bottomrule
\end{tabular}
\end{table}

%RT1  AS vs AN
\begin{table}
\caption[Tiempo de Respuesta a la tarea Sí/No para la clase A (Señal vs Ruido)]{Prueba t para comparar el Tiempo de Respuesta a la Tarea Sí/No entre los estímulos con Señal y Ruido de la clase A}
\label{Tabla_RT1_A}
\centering
\begin{tabular}{l |  c c c c}
\toprule
%\tabhead{Groups} & \tabhead{Treatment X} & \tabhead{Treatment Y} \\
\textbf{Experimento} & \textbf{$\mu$RT(AS)} & \textbf{$\mu$RT(AN)} & \textbf{T} & \textbf{P value}\\
\midrule
Exp 1 & 0.7964 & 0.6573 & -1.4808 & 0.1469 \\
Exp 2 & 1.1025 & 3.7078 & 1 & 0.3299 \\
\bottomrule
\end{tabular}
\end{table}

%RT2  AS vs AN
\begin{table}
\caption[Tiempo de Respuesta a la tarea Sí/No para la clase A (Señal vs Ruido)]{Prueba t para comparar el Tiempo de Respuesta a la Tarea con Escala de Confianza entre los estímulos con Señal y Ruido de la clase A}
\label{Tabla_RT2_A}
\centering
\begin{tabular}{l | c c c c}
\toprule
%\tabhead{Groups} & \tabhead{Treatment X} & \tabhead{Treatment Y} \\
\textbf{Experimento} & \textbf{$\mu$RT(AS)} & \textbf{$\mu$RT(AN)} & \textbf{T} & \textbf{P value}\\
\midrule
Exp 1 & 0.7724 & 0.8019 & 0.2409 & 0.810 \\
Exp 2 & 0.7598 & 0.7676 & 0.0915 & 0.9275  \\
\bottomrule
\end{tabular}
\end{table}


%RT1  BS vs BN
\begin{table}
\caption[Tiempo de Respuesta a la tarea Sí/No para la clase A (Señal vs Ruido)]{Prueba t para comparar el Tiempo de Respuesta a la Tarea Sí/No entre los estímulos con Señal y Ruido de la clase B}
\label{Tabla_RT1_B}
\centering
\begin{tabular}{l |  c c c c}
\toprule
%\tabhead{Groups} & \tabhead{Treatment X} & \tabhead{Treatment Y} \\
\textbf{Experimento} & \textbf{$\mu$RT(BS)} & \textbf{$\mu$RT(BN)} & \textbf{T} & \textbf{P value}\\
\midrule
Exp 1 & 0.8334 & 0.7214 & -0.9483 & 0.3498 \\
Exp 2 & 1.2142 & 1.0203 & -1.2342 & 0.225  \\
\bottomrule
\end{tabular}
\end{table}

%RT2  BS vs BN
\begin{table}
\caption[Tiempo de Respuesta a la tarea Sí/No para la clase A (Señal vs Ruido)]{Prueba t para comparar el Tiempo de Respuesta a la Tarea con Escala de Confianza entre los estímulos con Señal y Ruido de la clase B}
\label{Tabla_RT2_B}
\centering
\begin{tabular}{l |  c c c c}
\toprule
%\tabhead{Groups} & \tabhead{Treatment X} & \tabhead{Treatment Y} \\
\textbf{Experimento} & \textbf{$\mu$RT(BS)} & \textbf{$\mu$RT(BN)} & \textbf{T} & \textbf{P value}\\
\midrule
Exp 1 & 0.8010 & 0.8672 & 0.4916 & 0.625 \\
Exp 2 & 0.7458 & 0.7758 & 0.3576 & 0.7226  \\
\bottomrule
\end{tabular}
\end{table}
