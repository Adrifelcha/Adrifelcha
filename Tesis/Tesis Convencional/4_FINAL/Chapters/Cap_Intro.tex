% Chapter 1

\chapter{Introducción} % Main chapter title

\label{Chapter1} % For referencing the chapter elsewhere, use \ref{Chapter1} 

%----------------------------------------------------------------------------------------

% Define some commands to keep the formatting separated from the content 
\newcommand{\keyword}[1]{\textbf{#1}}
\newcommand{\tabhead}[1]{\textbf{#1}}
\newcommand{\code}[1]{\texttt{#1}}
\newcommand{\file}[1]{\texttt{\bfseries#1}}
\newcommand{\option}[1]{\texttt{\itshape#1}}

%----------------------------------------------------------------------------------------

El mundo está cargado de ruido e incertidumbre. Los organismos están constantemente expuestos a distintas fuentes de estimulación en su entorno que pueden proporcionar, o no, información relevante sobre el estado de las cosas y las reglas vigentes. Por ello, uno de los primeros grandes problemas de adaptabilidad a los que se enfrentan los organismos consiste en ordenar el caos resultante, definiendo relaciones de contingencia que les permitan hacer predicciones sobre la disponibilidad de ciertos sucesos biológicamente importantes y ajustar su comportamiento a las restricciones operantes. Una vez establecida la relación entre la presencia u ocurrencia de ciertos estímulos y el acceso a ciertas consecuencias, la detección de éstos se vuelve una tarea importante para que los organismos puedan guiar su comportamiento, (por ejemplo: \textit{'Sé que soy alérgico a las nueces, ¿En este panqué hay nueces? Si hay nueces en el panqué, no lo como; si no hay nueces en el panqué, sí lo como.'}).\\


%La detección de ciertos eventos no es una tarea sencilla. La información a evaluar suele ser ambigua.
Detectar algo no parecería ser un problema importante si asumiéramos que todo evento aparece con perfecta claridad, o bien, que el organismo interesado en su detección cuenta con sensores altamente precisos que le garantizan el éxito. Sin embargo, la evidencia a partir de la cual juzgamos si algo está ocurriendo o no, por lo general es confusa y puede llevarnos a emitir juicios erróneos. Como un ejemplo cotidiano, imaginemos el caso de un adolescente que quiere conseguir permiso para ir a una fiesta y necesita encontrar el momento ideal para pedírselo a su mamá (\textit{cuando ella esté de buen humor}). Los indicadores con que cuenta son imprecisos (los gestos, el tono de voz, las actividades que su madre realice durante el día, etc.), por lo que puede errar en el diagnóstico del estado emocional de su madre y no obtener el permiso deseado, ya sea por una mala lectura de los datos disponibles (que las ansias del adolescente por salir de fiesta le hagan apresurar el momento) o bien porque los datos en sí mismos son poco claros (la mamá podría ser una persona particularmente inexpresiva o, por el contrario, altamente variable).\\ 

%La detección es distinta de la discriminación y la categorización. Todos problemas importantes en un entorno cargado de estimulación. (El problema del embudo)
%A diferencia de problemas tales como la discriminación o la categorización, donde la tarea de los organismos es evaluar la evidencia que se les presenta para asignarle una etiqueta (tales como \textit{'¿Es A o es B?'} ó \textit{'De acuerdo a sus propiedades en tales dimensiones, se trata de un caso de...'}), cuando hablamos de un problema de detección nos referimos a situaciones que pueden plantearse en términos de \textit{preguntas Sí/No}, (por ejemplo, \textit{'¿La comida está buena? Sí/No'}, \textit{'¿Ese que viene es mi camión? Sí/No'}, \textit{'¿Este perro es hostil? Sí/No'}) y cuya respuesta permite guiar el comportamiento de los organismos en función de las consecuencias anunciadas (\textit{'Sí, la comida está buena, me la comeré porque es seguro'}, \textit{'No, ese no es mi camión, no me subiré porque acabaré en Ecatepec'}, \textit{'No, no es un perro hostil, puedo acariciarlo'}, etc.).\\ 

La Teoría de Detección de Señales (SDT en ingles) presenta un modelo estadístico que describe las tareas de detección como un problema de decisión al que tienen que enfrentarse los organismos, sistemas inmersos en entornos con incertidumbre -es decir, entornos dinámicos que presentan variabilidad en la disponibilidad y presentación de ciertos eventos-, para guiar su comportamiento de la manera más óptima posible dada la estructura del mismo (Peterson, Birdsall y Fox, \citeyear{Peterson1954}; Tanner y Swets, \citeyear{Tanner1954}; Kileen, \citeyear{Killeen2014}). La SDT funciona tanto como un modelo estadístico para describir esta clase de problemas, como una herramienta para interpretar la ejecución de sistemas evaluados experimentalmente y hacer inferencias sobre la precisión con que los eventos a detectar (las señales) se distinguen del ruido que les rodea y la posible preferencia del sistema a responder en favor o en contra de su detección (Stainslaw y Todorov, \citeyear{Stainslaw1999}). Se trata de uno de los modelos más sólidos y ampliamente estudiados en Psicología Experimental, cuyos supuestos son lo suficientemente generales para permitir su aplicación al estudio de distintos fenómenos. \\

La SDT le concede a la noción de variabilidad un papel fundamental en su definición de la detección de señales como un problema de adaptabilidad. La idea básica es que las señales cuya detección resulta relevante para los organismos suelen presentarse y percibirse con cierta variabilidad y que, además de ello, coexisten en el mundo con otros estímulos (el ruido) que, dada su propia variabilidad, pueden llegar a ser confundidos con éstas (Tanner y Swets, \citeyear{Tanner1954}; Swets, \citeyear{Swets1973}). Por ejemplo, imaginemos que queremos detectar si la persona que nos acaba de contestar el teléfono es un adulto. Existe un rango de tonos de voz que asociamos con las personas adultas y que, a grandes rasgos, es más grave de lo que esperaríamos escuchar en un niño. Sin embargo, sabemos que hay adultos que pueden tener voces particularmente agudas y que pueden confundirse con la de un menor de edad: si quien nos contestó el teléfono tiene una voz aguda, no podremos estar seguros de si se trata de un adulto -o no- y tendremos que actuar conforme lo que nos parezca más probable.\\

La SDT opera como un modelo de decisión, en tanto que no asume que los organismos detectan los elementos relevantes en su entorno como respuesta directa a la estimulación que reciben momento a momento (como ocurre en las Teorías de Umbral que le preceden), sino que esta evidencia es ponderada por el sistema detector con la información que posee sobre el escenario en que se encuentra (Killeen, \citeyear{Killeen2014}). Bajo esta visión, los organismos \textit{eligen} el juicio de detección que les permite guiar su comportamiento de la manera más óptima posible tomando en cuenta: 1) las ganancias y pérdidas que están en juego -y que hacen más o menos importante el cometer cierto tipo de acierto o evitar cierto tipo de error- y 2) la probabilidad con que dichos eventos se presentan en su entorno. De esta forma, la noción de los umbrales ampliamente desarrollada en la Psicofísica clásica es reemplazada por los criterios de elección, (Wickens \citeyear{Wickens1}).\\ 

Al aplicar la SDT en tareas de memoria de reconocimiento -donde los participantes tienen que identificar los elementos ya antes vistos (las señales) dentro de un conjunto de ítems que contiene tanto elementos de una fase previa como elementos nuevos (el ruido)- para comparar la ejecución de los participantes entre dos clases de estímulos A y B, (siendo que A es más fácil de reconocer que B), se ha encontrado consistentemente un patrón de respuestas que demuestra que los participantes no sólamente son mejores reconociendo las señales de la condición A ($Hits(A)>Hits(B)$), sino que también son mejores identificando los estímulos con ruido de esta misma condición ($F.alarm(A)<F.alarm(B)$). En este tipo de estudios, los estímulos de las clases A y B son presentados de manera aleatoria, sin que se les informe a los participantes de la diferencia entre estos y por tanto, se asume que sus respuestas son emitidas en función a un único criterio de elección. De acuerdo con las tasas de Hits y Falsas Alarmas registradas por cada clase, parece ser que las distribuciones de Ruido y Señal de las clases A y B, se despliegan a lo largo de un mismo eje de evidencia \textit{reflejándose} entre sí (es decir, las distribuciones de la clase A aparecen a los extremos del eje, con muy poco sobrelape entre sí, y las distribuciones de la clase B se sitúan  en el centro, presentan un área de intersección mayor), razón por la que dicho patrón de respuestas ha sido referido como Efecto Espejo (Glanzer y Bowles, \citeyear{Glanzer1976}; Glanzer, Adams, Iverson y Kim, \citeyear{Glanzer1993}).\\

%3.5*42.5
El Efecto Espejo sólo ha sido estudiado dentro del dominio de la memoria de reconocimiento, donde se ha reportado a lo largo de una amplia variedad de procedimientos y variables (Glanzer y Adams, \citeyear{Glanzer1990}). Como resultado, gran parte de los intentos por dar cuenta de este fenómeno se han desarrollado en torno a la estructura de las tareas de reconocimiento, por lo general, asumiendo que existen diferencias importantes en la forma en que cada clase de estímulo es atendida, procesada y/o evaluada durante la fase de estudio (Glanzer, Adams, Iverson y Kim, \citeyear{Glanzer1993}; Glanzer, Kim y Adams, \citeyear{Glanzer1998}; Glanzer, Hilford y Maloney, \citeyear{Glanzer2009}). Es decir, que bajo esta perspectiva el Efecto Espejo ha sido tratado como reflejo de los procesos cognitivos compreometidos en las tareas memoria de reconocimiento.\\

El interés principal del presente trabajo de tesis fue explorar la generalizabilidad del Efecto Espejo, buscando evidencia del mismo en una tarea de detección ajena a la memoria de reconocimiento. Para ello, se presentan dos variaciones de una tarea de detección perceptual (visual) que emula la estructura de los estudios donde ha sido reportado, construyendo dos niveles de dificultad con base en la literatura en ilusiones ópticas. La tarea elaborada fue presentada a los participantes a partir de dos protocolos: una tarea Sí/No y la asignación de puntajes en una Escala de Confianza. Los resultados e implicaciones de los mismos se discuten en detalle.\\