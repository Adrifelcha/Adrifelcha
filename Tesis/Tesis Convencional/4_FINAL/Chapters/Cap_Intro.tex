% Chapter 1

\chapter{Introducción} % Main chapter title

\label{Chapter1} % For referencing the chapter elsewhere, use \ref{Chapter1} 

%----------------------------------------------------------------------------------------

% Define some commands to keep the formatting separated from the content 
\newcommand{\keyword}[1]{\textbf{#1}}
\newcommand{\tabhead}[1]{\textbf{#1}}
\newcommand{\code}[1]{\texttt{#1}}
\newcommand{\file}[1]{\texttt{\bfseries#1}}
\newcommand{\option}[1]{\texttt{\itshape#1}}

%----------------------------------------------------------------------------------------

El mundo está cargado de ruido e incertidumbre. Los organismos están constantemente expuestos a distintas fuentes de estimulación en su entorno que pueden, o no, dar información relevante sobre el estado de las cosas y las reglas vigentes. Por ello, uno de los primeros grandes problemas de adaptabilidad a los que se enfrentan los organismos consiste en ordenar el caos resultante, definiendo relaciones de contingencia que les permitan hacer predicciones sobre la disponibilidad de ciertos sucesos biológicamente importantes y ajustar su comportamiento a las restricciones operantes. Una vez establecida la relación entre la presencia u ocurrencia de ciertos estímulos y el acceso a ciertas consecuencias, la detección de éstos se vuelve una tarea importante para que los organismos puedan guiar su comportamiento, (por ejemplo: \textit{'Sé que soy alérgico a las nueces, ¿En este panqué hay nueces? Si hay nueces en el panqué, no me lo como; si no hay nueces en el panqué, sí lo como.'}).\\


%La detección de ciertos eventos no es una tarea sencilla. La información a evaluar suele ser ambigua.
Detectar algo no parecería ser un problema importante si asumiéramos que todo evento aparece con perfecta claridad, o bien, que el organismo interesado en su detección cuenta con sensores altamente precisos que le garantizan el éxito. Sin embargo, la evidencia a partir de la cual juzgamos si algo está o no ocurriendo por lo general es confusa y puede llevarnos a emitir juicios erróneos. A manera de ejemplo cotidiano, imaginemos el caso de un adolescente que quiere conseguir permiso para ir de fiesta y necesita encontrar el momento ideal para pedírselo a su mamá (\textit{cuando ella esté de buen humor}); los indicadores con que cuenta son imprecisos -los gestos, el tono de voz, las actividades que su madre realice durante el día, etc.- y errar en el diagnóstico del estado emocional de su madre y en consecuencia no obtener el permiso deseado al pedirlo en el momento inadecuado es un riesgo latente, ya sea por una mala lectura de los datos disponibles -que las ansias del adolescente por salir de fiesta le hagan apresurar el momento- o bien porque los datos en sí mismos son poco claros -la mamá podría ser una persona particularmente inexpresiva o, por el contrario, terriblemente variable-.\\ 

%La detección es distinta de la discriminación y la categorización. Todos problemas importantes en un entorno cargado de estimulación. (El problema del embudo)
%A diferencia de problemas tales como la discriminación o la categorización, donde la tarea de los organismos es evaluar la evidencia que se les presenta para asignarle una etiqueta (tales como \textit{'¿Es A o es B?'} ó \textit{'De acuerdo a sus propiedades en tales dimensiones, se trata de un caso de...'}), cuando hablamos de un problema de detección nos referimos a situaciones que pueden plantearse en términos de \textit{preguntas Sí/No}, (por ejemplo, \textit{'¿La comida está buena? Sí/No'}, \textit{'¿Ese que viene es mi camión? Sí/No'}, \textit{'¿Este perro es hostil? Sí/No'}) y cuya respuesta permite guiar el comportamiento de los organismos en función de las consecuencias anunciadas (\textit{'Sí, la comida está buena, me la comeré porque es seguro'}, \textit{'No, ese no es mi camión, no me subiré porque acabaré en Ecatepec'}, \textit{'No, no es un perro hostil, puedo acariciarlo'}, etc.).\\ 

La Teoría de Detección de Señales (SDT en ingles) presenta un modelo estadístico que describe las tareas de detección como un problema de decisión al que tienen que enfrentarse los organismos, sistemas inmersos en entornos con incertidumbre -es decir, entornos dinámicos que presentan variabilidad en la disponibilidad y presentación de ciertos eventos-, para guiar su comportamiento de la manera más óptima posible dada la estructura del mismo. La SDT funciona tanto como un modelo estadístico para describir esta clase de problemas, como una herramienta para interpretar la ejecución de sistemas evaluados experimentalmente y hacer inferencias sobre la precisión con que los eventos a detectar (las señales) se distinguen del ruido que les rodea y la posible preferencia del sistema a responder en favor o en contra de su detección. Se trata de uno de los modelos más sólidos y ampliamente estudiados en Psicología Experimental, cuyos supuestos son lo suficientemente generales para permitir su aplicación al estudio de distintos fenómenos, dentro y fuera de la Psicología. \\

La SDT le concede a la noción de variabilidad un papel fundamental para entender la detección de señales como un problema de adaptabilidad. La idea básica es que las señales cuya detección resulta relevante para los organismos suelen presentarse y percibirse con cierta variabilidad y que, además de ello, coexisten en el mundo con otros estímulos (el ruido) que, dada su propia variabilidad, pueden llegar a presentarse o percibirse con la misma evidencia que lo haría una señal, pudiendo ser confundidos con la misma. Por ejemplo, imaginemos que queremos detectar si la persona que nos acaba de contestar el teléfono es un adulto. Existe un rango de 'tonos de voz' que estaríamos dispuestos a admitir como pertenecientes a una persona mayor de edad que, a grandes rasgos, tiende a ser más grave de lo que esperaríamos escuchar en un niño. Sin embargo, sabemos que hay adultos que pueden llegar a tener una voz particularmente clara y que puede llegar a confundirse con la de un menor de edad; si la persona que nos contestó el teléfono resulta ser un varón de voz aguda, no podremos estar seguros de si se trata de un adulto -o no- y tendremos que elegir la opción que nos parezca más probable.\\

La SDT opera como un modelo de decisión, en tanto que no asume que los organismos detectan los elementos relevantes en su entorno como respuesta directa a la estimulación que reciben momento a momento -como ocurre en las Teorías de Umbral que le preceden-, sino que la evidencia evaluada es ponderada por el sistema detector con la información que posee sobre el escenario en que se encuentra. Bajo esta visión, los organismos \textit{eligen} el juicio de detección que les permite guiar su comportamiento de la manera más óptima posible tomando en cuenta 1) las ganancias y pérdidas que están en juego -y que hacen más o menos riesgoso el cometer cierto tipo de error, y más o menos atractivo el cometer cierto tipo de acierto- y 2) la probabilidad con que dichos eventos se presentan en su entorno. Bajo esta perspectiva, la noción de los umbrales ampliamente desarrollada en la Psicofísica clásica es reemplazada por los criterios de elección.\\ 

Al aplicar la SDT en la interpretación de datos obtenidos en tareas de memoria de reconocimiento, donde los participantes tienen que identificar los elementos ya antes vistos (las señales) dentro de un conjunto de ítems que contiene tanto elementos presentados en una fase previa como elementos nuevos (el ruido), cuando se compara la ejecución de los participantes entre dos clases de estímulos A y B, (siendo una de ellas más fácil de reconocer (A) que la otra (B)), se ha encontrado consistentemente un patrón de respuestas que demuestra que los participantes no sólamente son mejores reconociendo las señales de la condición A ($Hits(A)>Hits(B)$), sino que también son mejores identificando los estímulos con ruido de esta misma condición ($F.alarm(A)<F.alarm(B)$). Dado que en estos estudios los participantes experimentales no saben que se ha incluido más de una clase de estímulos en la tarea que se les presenta, se asume que utilizan un sólo criterio de elección para emitir sus respuestas y de acuerdo a las tasas reportadas de Hits y Falsas Alarmas por cada clase, se sugiere que las distribuciones de Ruido y Señal de cada clase se distribuyen a lo largo del mismo eje de evidencia de tal forma que parecieran reflejarse entre sí. Es por esto que en la literatura en memoria de reconocimiento se ha identificado dicho patrón de respuestas bajo el nombre de Efecto Espejo.\\

%3.5*42.5
El Efecto Espejo sólo ha sido estudiado dentro del dominio de la memoria de reconocimiento, donde se ha reportado evidencia de su existencia a lo largo de una amplia variedad de procedimientos (tareas Sí/No, tareas de Elección Forzada entre dos alternativas y protocolos con Escala de Confianza) y variables (palabras comunes vs palabras extrañas; estímulos abstractos vs estímulos concretos; imágenes a color vs en blanco y negro, etc). Como resultado, gran parte de los modelos y teorías desarrollados para dar cuenta de este fenómeno tienden a hacerlo en términos de la estructura propia de las tareas de reconocimiento, donde se incluye una fase de estudio en la que se asume que los participantes procesan los estímulos para añadirles la \textit{'familiaridad'} que les permitiría reconocerlos posteriorente y donde se asume que tienen origen las diferencias observadas en el desempeño de los participantes. En otras palabras, el Efecto Espejo ha sido tratado como evidencia de que las clases de estímulos puestas a prueba son procesadas y atendidas de distinta manera durante la fase de estudio previa a la tarea de reconocimiento.\\

El interés principal del presente trabajo de tesis fue explorar la generalizabilidad del Efecto Espejo, buscando evidencia del mismo en una tarea de detección ajena a la Memoria de Reconocimiento. Para ello, se presentan dos variaciones de una tarea de detección perceptual (visual) que emula la estructura de los estudios donde dicho fenómenos ha sido reportado, construyendo dos niveles de dificultad con base en la literatura en Ilusiones Ópticas. La tarea propuesta fue presentada a los participantes a partir de dos protocolos: una tarea Sí/No y la asignación de puntajes en una Escala de Confianza. Los resultados e implicaciones de los mismos se discuten en detalle.\\