% Chapter Template

\chapter{Discusión} % Main chapter title

\label{Cap_Disc} % Change X to a consecutive number; for referencing this chapter elsewhere, use \ref{ChapterX}

En el presente trabajo se evaluó la generalizabilidad del fenómeno identificado como Efecto Espejo en estudios de memoria de reconocimiento donde se compara el desempeño de los participantes a través de dos clases de estímulos A y B, (siendo que $d'(A)>d'(B)$), bajo el marco de la SDT. Para ello, se diseñó una tarea de detección perceptual donde los participantes tenían que identificar los ensayos en que dos círculos presentados en pantalla tuvieran el mismo tamaño, respondiendo a cada ensayo en términos de una tarea 'sí/no' y de un protocolo con Escala de Confianza. Se corrieron dos experimentos para la presentación de dicha tarea: en el Experimento 1, uno de los círculos a comparar estaba construido como el círculo central de una figura de Ebbinghaus y en el Experimento 2, los dos círculos a comparar constituían una ilusion de Ebbinghaus. En cada experimento, las clases de estímulos A y B estuvieron definidas por el número de círculos externos contenidos en cada figura de Ebbinghaus (2 o 3 círculos externos en la clase A y 7 u 8, en la clase B).\\

%En el presente trabajo se presentaron los resultados obtenidos en dos experimentos desarrollados como variaciones en la presentación de una misma tarea de detección perceptual (visual), donde se incluyeron dos clases de estímulos A y B construidas con base en una revisión de la literatura en ilusiones ópticas para que representaran dos niveles de discriminabilidad ($d'(A)>d'(B)$) entre los cuales pudiera compararse el desempeño de los participantes. En los experimentos propuestos, los participantes tuvieron que registrar sus respuestas al problema de detección a través de dos protocolos: una tarea Sí/No y una tarea con Escala de Confianza. El diseño experimental propuesto fue elaborado con el propósito de emular la estructura de los estudios en Memoria de Reconocimiento donde se ha reportado evidencia del fenómeno ahora conocido como Efecto Espejo; una regularidad en los patrones de respuesta observados en tareas de reconocimiento con dos clases de estímulos que difieren en la precisión con que sus elementos son reconocidos y que sugiere un orden simétrico en el despliegue de las distribuciones Ruido y Señal de cada clase sobre el eje de la evidencia.\\

Los patrones identificados como parte del Efecto Espejo en memoria de reconocimiento fueron hallados en al menos el $85\%$ de los datos aquí reportados, una proporción significativa contra el azar.\\

Para validar la evidencia del Efecto Espejo encontrada en el presente trabajo, se comenzó por verificar que las clases A y B diseñadas difirieran en sus niveles de $d'$ en la dirección esperada. Para ello, por cada experimento realizado se realizó una prueba t de muestras independientes y se empleó un modelo jerárquico bayesiano -identificado como Modelo Delta- para evaluar las diferencias entre las medias de $d'$  ($\mu d'$) estimadas por cada clase de estímulos. Los resultados de dichos análisis confirmaron que los ensayos-Señal donde las figuras de Ebbinghaus tenían 2 o 3 círculos (clase A) fueron más fáciles de detectar que aquellos donde las figuras tenían 7 u 8 (clase B), cumpliendose así la regla $d'(A) > d'(B)$.\\

%Para validar la pertinencia del análisis de los datos obtenidos en los experimentos realizados como evidencia de la generalizabilidad de Efecto Espejo fuera de la Memoria de Reconocimiento, se realizó un análisis \textit{comprobatorio} que tuvo como objetivo evaluar la eficacia de la manipulación experimental propuesta con base en la literatura para el diseño de las clases de estímulos a comparar. De acuerdo con lo que se esperaba, los estímulos Señal de la clase A propuesta mostraron ser consistentemente más fáciles de detectar respecto del Ruido(A) que los estímulos Señal de la clase B. La robustez de esta afirmación a la luz de los datos fue confirmada tanto mediante la realización de pruebas t de muestras independientes para cada uno de los experimentos realizados, como a partir de la construcción de un modelo bayesiano jerárquico identificado como Modelo Delta, que incorporó al modelo bayesiano estándar de detección de señales una estructura jerárquica sobre los parámetros $d'$ y $c$ y un parámetro determinista encargado de evaluar las diferencias entre las $\mu d'$ estimadas por cada clase de estímulos.\\

Una vez establecida la validez de las clases A y B diseñadas, se prosiguió a evaluar la robustez de los patrones de respuesta encontrados, en torno a tres grandes ejes:\\

\begin{enumerate}
	\item Evaluación de las diferencias encontradas entre las tasas de Hits y Falsas Alarmas obtenidas por cada clase de estímulo durante la tarea 'Sí/No'.\\
	\item Comparar el promedio de los puntajes registrados en los ensayos-Señal y los ensayos-Ruido de cada clase de estímulo durante el protocolo de Escala de Confianza.\\
	\item Revisar que no hubiera relación entre el desempeño de los participantes en cada clase de estímulo y distintos tiempos de respuesta, para descartar ésta como una explicación alterna a las regularidades encontradas.\\
\end{enumerate}

Los análisis realizaron tanto mediante análisis frecuentistas comúnmente reportados en los estudios de memoria de reconocimiento con el Efecto Espejo, como mediante la realización de modelos y pruebas estadísticas bayesianas.\\

Primero, en cuanto a las tasas de Hits y Falsas Alarmas registradas por cada clase de estímulos, se realizó un par de pruebas t de muestras independientes para evaluar las diferencias entre el promedio de las tasas registradas (con transformación arcoseno para evitar el problema de suelo y techo) en cada experimento, encontrándose diferencias significativas tanto entre las tasas de Hits como de Falsas Alarmas por clase. Además de ello, se desarrolló un modelo bayesiano -identificado como Modelo Tau- para computar las diferencias entre las tasas de Hits y Falsas Alarmas estimadas, agregando un parámetro determinista $\tau$ al modelo estándar. El Modelo Tau confirmó lo reportado por las pruebas t: para la mayoría de los participantes, el parámetro $\tau$ tiene muy poca densidad de probabilidad posterior en $0$ (el punto de "no diferencias"), sin embargo, la evidencia parece ser menos sólida en el caso de las Falsas Alarmas. \\ %En este primer punto de análisis se pudieron detectar algunas dferencias en la lectura de las conclusiones extraídas a partir de los datos y su robustez derivadas de las discrepancias entre el funcionamiento de los análisis frecuentistas y bayesianos realizados:\\

%Las pruebas t realizadas para evaluar la significancia estadística de las diferencias encontradas entre los promedios de las tasas computadas, se enfrentan con el problema de que los datos a comparar se encuentran restrigidos dentro de un rango de $0$ a $1$. Más aún, de acuerdo con los supuestos de la SDT sobre el despliegue de las distribuciónes de Ruido y Señal sobre el eje de evidenncia, las tasas de ejecución se presentan dentro de rangos aún más restringidos, siendo que los aciertos (Hits y Rechazos Correctos) caen por arriba del $0.5$, y los errores (Omisiones y Falsas Alarmas), por debajo. Esto representa un problema para la comparación de los resultados encontrados, porque puede darse el caso de que las diferencias entre los datos registrados no sean lo suficientemente grandes como para considerárseles significativas, con independencia de si se presentan de forma consistente en la mayoría de los datos individuales.\\

%Para solucionar el problema del efecto de suelo y techo que podría estar mermando la evaluación de los datos obtenidos como evidencia del impacto de la manipulación experimental sobre el desempeño de los participantes, los datos crudos a comparar suelen transformarse a una nueva Escala, de manera que la distancia entre estos aumente, pero manteniendo la proporción de los mismos intacta. En el caso específico de la literatura que aborda el Efecto Espejo en Memoria de Reconocimiento, se ha optado por implementar una transformación arcoseno de las tasas de respuesta registradas antes de someterlas al análisis estadístico.\\

%De acuerdo con las pruebas t realizadas en el presente estudio para comparar los promedios de las transformaciones arcoseno de las tasas de ejecución registradas por cada clase de estímulo, se encontraron diferencias significativas en la ejecución de los participantes a la tarea Sí/No dependientes de la clase de estímulo a evaluar en ambos experimentos, reportándose diferencias altamente significativas entre las tasas de Hits y diferencias apenas significativas (justo por debajo de $p=0.05$) para las tasas de Falsas Alarmas.\\

%El modelamiento bayesiano del problema de detección de señales abandona la interpretación determinista de las tasas de ejecución calculadas a partir de los datos registrados como reflejo directo del área de las distribuciones de Ruido y Señal que caen por encima del criterio de elección. En su lugar, asume que dicha CDF debe ser estimada como una probabilidad oculta que permea la observación del total de Hits y Falsas Alarmas por participante.\\

%El Modelo Tau presentado en este trabajo parte del modelo bayesiano estándar de detección de señales, que estima el valor de las CDF de las distribuciones de Ruido y Señal que caen por encima del criterio de elección a partrir del número de Hits y Falsas Alarmas observados en cada participante, y añade un par de paráetros $\tau$ que computan las diferencias entre las probabilidades ocultas estimadas tras la emisión de Hits y Falsas Alarmas. Las estimaciones realizadas por el modelo acerca de las diferencias en el desempeño de los participantes para las distintas clases de estímulos (los valores de $\tau$) toman en cuenta la naturaleza probabilística del modelo de detección de señales, al asumir que los datos observados son extracciones de las distribuciones de Señal y Ruido subyacentes a la tarea. Con ello, el análisis de los datos obtenidos a la luz de los resultados arrojados por el Modelo Tau presenta una ventaja considerable sobre el análisis frecuentista, pues al tratar los datos obtenidos como resultado de un proceso probabilístico, escapa del problema de suelo y techo.\\

%Los resultados arrojados por el Modelo Tau confirmaron a grandes razgos lo reportado por las pruebas t frecuentistas: la mayoría de las densidades de probabilidad posterior estimadas para los participantes de los experimentos 1 y 2 se sitúan por encima del punto de \textit{no diferencias} ($\tau = 0$). Sin embargo, y especialmente en términos de las diferencias encontradas en la emisión de Falsas Alarmas, dejó en evidencia la dispersión de las estimaciones de las $\tau$ individuales, siendo que estas presentaban valores no muy alejados de $\tau = 0$.\\
%\end{enumerate}

La discrepancia en la precisión con que los análisis realizados evalúan las diferencias encontradas se atribuye a una diferencia entre la interpretación determinista que el análisis frecuentista clásico tiene sobre las tasas de ejecución y el enfoque probabilístico adoptado por los métodos bayesianos. Es decir, mientras que las pruebas t frecuentistas requieren de la transformación arcoseno de los datos a promediar para su análisis (las tasas de ejecución), el modelo bayesiano trabaja con el número total de Hits y Falsas Alarmas obtenidos por los participantes para estimar la proporción de las distribuciones de Señal y Ruido que caen por encima del criterio. \\

En segundo lugar, se evaluaron las diferencias entre los promedios de los puntajes asignados en la Escada de Confianza a los estímulos con Señal y Ruido de cada clase. Para ello se realizaron pruebas t de muestras independientes frecuentistas y bayesianas que señalaron diferencias entre los puntajes registrados. Las pruebas t frecuentistas arrojaron \textit{p-values} por debajo de $0.05$ para todas las comparaciones realizadas. Por su parte, las pruebas t bayesianas encontraron que la evidencia obtenida en ambos experimentos fue apenas anecdótica, de acuerdo con los estándares de evaluación del Factor de Bayes.\\

Finalmente, para descartar la posibilidad de que los resultados encontrados fueran efecto de diferencias en el tiempo de respuesta entre las clases de estímulos, se realizaron múltiples pruebas t de muestras independientes para evalaur toda posible discrepancia que pudiera haberse observado en los tiempos de respuesta a través de los distintos tipos de ensayo presentados. Los análisis llevados a cabo no encontraron diferencias significativas en los tiempos de respuesta en ninguno de los protocolos utilizados en ambos experimentos.\\

%se revisó que no existieran diferencias en el tiempo de respuesta invertido por los participantes en cada una de las clases de estímulos. Dicha evaluación constituye uno de los análisis de control más frecuentemente reportados en los estudios de Memoria de Reconocimiento que presentan evidencia del Efecto Espejo, como una medida para garantizar que las diferencias en el desempeño de los participantes puedan atribuirse a discrepancias en la precisión con que los elementos que componen cada clase de estímulos son reconocidos y no al cuidado con que los participantes emitieron sus respuestas ante cada una. Para ello, se realizaron múltiples pruebas t de muestras independientes que evaluaron las distintas relaciones que pudieron haberse dado entre los tiempos de respuesta y el tipo de etímulos presentados en los experimentos. De acuerdo con dichas pruebas, no se encontraron diferencias significativas en el tiempo que se tomaron los participantes para responder a la tarea binaria o a la tarea con Escala de Confianza entre clases de estímulos, ni entre los estímulos con Señal y Ruido contenidos en cada clase.\\ 

Los modelos en memoria desarrollados desde el marco de la SDT entienden las tareas de reconocimiento como instancias de un problema de detección, donde los organismos tienen que detectar (\textit{reconocer}) los elementos antes vistos en su entorno. El Efecto Espejo ha sido abordado tanto en términos de lo que pudiera sugerir sobre los procesos superiores involucrados, como de un referente para juzgar la adecuación de la aplicación de la SDT al estudio de la memoria. La evidencia del Efecto Espejo encontrada en el presente trabajo sugiere que se trata de una regularidad propia de la aplicación de la SDT en la comparación del desempeño de los participantes en tareas de detección con dos niveles de $d'$. Los resultados obtenidos señalan la importancia de estudiar el Efecto Espejo en relación a los supuestos planteados por la SDT y las implicaciones que tiene en el estudio de la detección de señales como un problema de adaptabilidad.\\

El presente trabajo es el primero en explorar la generalizabilidad del Efecto Espejo a otra área dentro de la Psicología Experimental donde la SDT ha sido aplicada, presentando evidencia a favor de esta. Las implicaciones de la extensividad de dicho fenómeno a distintos protocolos y fenómenos psicológicos de detección deben ser revisadas en investigaciones futuras. Por otro lado, este trabajo también es el primero en evaluar el Efecto Espejo mediante metodos bayesianos y con ello, presenta un referente sobre las ventajas que presenta la estadística bayesiana en el análisis de datos obtendos en tareas donde se asume una estructura probabilística, como es el caso de las tareas de detección bajo el marco de la SDT.\\









