%%%%%%%%%%%%%%%%%%%%%%%%%%%%%%%%%%%%%%%%%
% Masters/Doctoral Thesis 
% LaTeX Template
% Version 2.4 (22/11/16)
%
% This template has been downloaded from:
% http://www.LaTeXTemplates.com
%
% Version 2.x major modifications by:
% Vel (vel@latextemplates.com)
%
% This template is based on a template by:
% Steve Gunn (http://users.ecs.soton.ac.uk/srg/softwaretools/document/templates/)
% Sunil Patel (http://www.sunilpatel.co.uk/thesis-template/)
%
% Template license:
% CC BY-NC-SA 3.0 (http://creativecommons.org/licenses/by-nc-sa/3.0/)
%
%%%%%%%%%%%%%%%%%%%%%%%%%%%%%%%%%%%%%%%%%

%----------------------------------------------------------------------------------------
%	PACKAGES AND OTHER DOCUMENT CONFIGURATIONS
%----------------------------------------------------------------------------------------

\documentclass[
12pt, % The default document font size, options: 10pt, 11pt, 12pt
%oneside, % Two side (alternating margins) for binding by default, uncomment to switch to one side
spanish, % ngerman for German
onehalfspacing, % Single line spacing, alternatives: onehalfspacing or doublespacing or singlespacing
%draft, % Uncomment to enable draft mode (no pictures, no links, overfull hboxes indicated)
%nolistspacing, % If the document is onehalfspacing or doublespacing, uncomment this to set spacing in lists to single
%liststotoc, % Uncomment to add the list of figures/tables/etc to the table of contents
%toctotoc, % Uncomment to add the main table of contents to the table of contents
%parskip, % Uncomment to add space between paragraphs
%nohyperref, % Uncomment to not load the hyperref package
headsepline, % Uncomment to get a line under the header
%chapterinoneline, % Uncomment to place the chapter title next to the number on one line
%consistentlayout, % Uncomment to change the layout of the declaration, abstract and acknowledgements pages to match the default layout
]{MastersDoctoralThesis} % The class file specifying the document structure

\usepackage[utf8]{inputenc} % Required for inputting international characters
\usepackage[T1]{fontenc} % Output font encoding for international characters
\usepackage{graphics}
\usepackage{utopia} % Use the Palatino font by default
\usepackage[spanish]{babel}
\usepackage[latin1]{inputenc}
\usepackage{hyperref}
%\usepackage{arial}
%
%\usepackage{natbib}
%\setcitestyle{authoryear,open={((},close={))}}

%\usepackage{apacite}

%\usepackage[backend=bibtex,style=authoryear,natbib=true]{biblatex} % Use the bibtex backend with the authoryear citation style (which resembles APA)

\usepackage[natbibapa]{apacite}
\setcitestyle{authoryear,open={},close={}}
\bibliographystyle{apacite}

%\addbibresource{Bib_Adrifelcha.bib} % The filename of the bibliography

%\bibliographystyle{apacite}
%\bibliography{Bib_Adrifelcha.bib}

%\usepackage[autostyle=true]{csquotes} % Required to generate language-dependent quotes in the bibliography

%----------------------------------------------------------------------------------------
%	MARGIN SETTINGS
%----------------------------------------------------------------------------------------
%\usepackage[pass,paperwidth=6.7in,paperheight=9in]{geometry}

\geometry{
%paperwidth=6.7in,
%paperheight=9in
paper=letterpaper,
inner=3.5cm, % Inner margin
outer=3cm, % Outer margin
%bindingoffset=.5cm, % Binding offset
top=2.5cm, % Top margin
bottom=2.5cm, % Bottom margin
%showframe, % Uncomment to show how the type block is set on the page
}

%\geometry{
%	paper=a4paper, % Change to letterpaper for US letter
%	inner=2.5cm, % Inner margin
%	outer=3.8cm, % Outer margin
%	bindingoffset=.5cm, % Binding offset
%	top=1.5cm, % Top margin
%	bottom=1.5cm, % Bottom margin
	%showframe, % Uncomment to show how the type block is set on the page
%}

%----------------------------------------------------------------------------------------
%	THESIS INFORMATION
%----------------------------------------------------------------------------------------

\thesistitle{Estudios con Detección de Señales} % Your thesis title, this is used in the title and abstract, print it elsewhere with \ttitle
\supervisor{Dr. Arturo Bouzas Riaño}%\textsc{Bouzas}} % Your supervisor's name, this is used in the title page, print it elsewhere with \supname
\examiner{Dr. Germán Palafox Palafox}%\textsc{Palafox}} % Your examiner's name, this is not currently used anywhere in the template, print it elsewhere with \examname
\degree{Licenciatura en Psicología} % Your degree name, this is used in the title page and abstract, print it elsewhere with \degreename
\author{Adriana Felisa Chávez De la Peña}%\textsc{Chávez De la Peña}} % Your name, this is used in the title page and abstract, print it elsewhere with \authorname
\addresses{} % Your address, this is not currently used anywhere in the template, print it elsewhere with \addressname

\subject{Psicología} % Your subject area, this is not currently used anywhere in the template, print it elsewhere with \subjectname
\keywords{Teoría de Detección de Señales, Memoria de Reconocimiento, Efecto Espejo, Ilusiones Ópticas, Modelamiento Bayesiano} % Keywords for your thesis, this is not currently used anywhere in the template, print it elsewhere with \keywordnames
\university{Universidad Nacional Autónoma de México} % Your university's name and URL, this is used in the title page and abstract, print it elsewhere with \univname
\department{Laboratorio 25 de Comportamiento Adaptable}
% Your department's name and URL, this is used in the title page and abstract, print it elsewhere with \deptname
\group{Laboratorio de Comportamiento Adaptable} % Your research group's name and URL, this is used in the title page, print it elsewhere with \groupname
\faculty{Facultad de Psicología} % Your faculty's name and URL, this is used in the title page and abstract, print it elsewhere with \facname

\sinodalA{Dr. Óscar Zamora Arévalo}
\sinodalB{Mtro. Miguel Herrera Ortíz}
\sinodalC{Lic. José Luis Baroja Manzano}

\city{Ciudad de México}
\degreemonth{Noviembre}
\degreeyear{2017}

\proyectopapime{PAPIME PE310016}
\proyectopapiit{PAPIIT IN307214}

\AtBeginDocument{
\hypersetup{pdftitle=\ttitle} % Set the PDF's title to your title
\hypersetup{pdfauthor=\authorname} % Set the PDF's author to your name
\hypersetup{pdfkeywords=\keywordnames} % Set the PDF's keywords to your keywords
}

\begin{document}

\frontmatter % Use roman page numbering style (i, ii, iii, iv...) for the pre-content pages

\pagestyle{plain} % Default to the plain heading style until the thesis style is called for the body content

%----------------------------------------------------------------------------------------
%	TITLE PAGE
%----------------------------------------------------------------------------------------

\begin{titlepage}

\begin{minipage}[c][8.5in][s]{1in}
\centering
\hspace*{-0.2in} \includegraphics[width=1.1in]{Escudo-UNAM}\\[10pt]
\hskip 2pt\vrule width 2pt height 6.1in
\hskip 1mm\vrule width 1pt height 6.1in\\[10pt]
\hspace*{-0.2in} \includegraphics[width=1.2in]{PSI}
\end{minipage}\hskip 10pt
% Right layout - Titles
\begin{minipage}[c][\textheight][s]{5.125in}
\centering
% University, institute, department and title
{\Large\scshape\univname}
\vspace{3mm}\hrule height2pt
\vspace{1mm}\hrule height1pt
\vspace{3mm}
%\@ifundefined{\facname}{\relax}{{\large\scshape\facname}\\[3pt]}
{\scshape\facname}\par
% Title
\vfill\vfill
{\def\baselinestretch{1}\LARGE\scshape\ttitle\par}
\vfill\vfill
% Degree, author, supervisor and date
\makebox[8cm][s]{\Huge T E S I S}\\[8pt]
QUE PARA OBTENER EL GRADO DE:\\[8pt]
{\scshape\degreename}\\[16pt]
{\huge P  R  E  S  E  N  T  A:}\\[8pt]
{\large {\scshape\authorname}}\par
\vfill
{\small DIRECTOR DE TESIS:\\{\scshape\supname}}\par
\vfill
{\small REVISOR:\\{\scshape\examname}}\par
\vfill
{\small SINODALES:\\{\scshape\SinodalA}\\{\scshape\SinodalB}\\{\scshape\SinodalC}}\par
\vfill
{\small Con el apoyo de:\\Proyecto {\scshape\PAPIIT} y \\Proyecto {\scshape\PAPIME}}\par
\vfill
%{\today}
{\hspace*{0.2in}\scshape\city\hfill\today}
\end{minipage}
\end{titlepage}



\begin{titlepage}
\centering\large
{\def\baselinestretch{1.2}\Large\bfseries\ttitle\par}\\
\vspace{12mm}
por\par
\vspace{12mm}
{\Large\authorname}
\vspace{25mm}
\par
Tesis presentada para obtener la\par
\vspace{12mm}
\degreename\par
\vspace{12mm}
en la\par 
\vspace{12mm}
\facname
\par
\vspace{12mm}
{\Large\scshape\univname}
\vspace{25mm}
\par
\city, \today
\end{titlepage}}



%----------------------------------------------------------------------------------------
%	DECLARATION PAGE
%----------------------------------------------------------------------------------------

\begin{declaration}
\addchaptertocentry{\authorshipname} % Add the declaration to the table of contents
\noindent Yo, \authorname, declaro que la tesis aquí presentada bajo el título \enquote{\ttitle}, es de mi entera autoría, aclarando que:\\

\begin{itemize} 
\item La presente tesis fue desarrollada en el Laboratorio 25 de la Facultad de Psicología de la Universidad Nacional Autónoma de México, bajo la tutela del Dr. Arturo Bouzas Riaño. 
\item Ningún dato aquí presentado ha sido utilizado con anterioridad para recibir un grado académico ni en ésta ni en ninguna otra Universidad. 
\item Las ideas cuya autoría no me corresponde están clara y adecuadamente señaladas en el texto. 
\item Así mismo, he señalado y dado crédito a toda fuente y material de apoyo consultados (lenguajes de programación, códigos base y manuales).
\item Todas las figuras que se presentan a lo largo de la presente tesis fueron elaboradas por la autora de la misma, por medio de los IDE's RStudio (para R) y Spyder (para Python), excepto cuando se señala lo contrario.
%\item Cualquier porción del trabajo aquí expuesto que se haya realizado en colaboración directa o indirecta con un tercero, es señalada y presentada con claridad.
\item El presente proyecto de investigación fue realizado con el apoyo de los proyectos PAPIIT IN307214 y PAPIME PE310016.\\
\end{itemize}
 
%\noindent Firma:\\
%\rule[0.5em]{25em}{0.5pt} % This prints a line for the signature
 
%\noindent Fecha:\\
%\rule[0.5em]{25em}{0.5pt} % This prints a line to write the date
\end{declaration}

\cleardoublepage

%----------------------------------------------------------------------------------------
%	QUOTATION PAGE
%----------------------------------------------------------------------------------------

\vspace*{0.2\textheight}

\noindent\enquote{\itshape Research is what I'm doing when I don't know what I'm doing.}\bigbreak

%My experiences with science led me to God. They challenge science to prove the existence of God. But must we really light a candle to see the sun?
%1972

%Read more at: https://www.brainyquote.com/quotes/authors/s/steven_pinker_2.html
\hfill Wernher von Braun, 1957.\\

\vspace*{0.3\textheight}

\hfill(Apparently, I've been doing research my whole life\ldots)

%----------------------------------------------------------------------------------------
%	ABSTRACT PAGE
%----------------------------------------------------------------------------------------

\begin{abstract}
\addchaptertocentry{\abstractname} % Add the abstract to the table of contents

En estudios de memoria de reconocimiento donde se ha aplicado la Teoría de Detección de Señales para comparar el desempeño de participantes experimentales entre clases de estímulos que se distinguen por la precisión con que sus elementos se reconocen al ser presentados más de una vez -habiendo una clase que se reconoce con mayor facilidad (\textit{A}) que en la otra (\textit{B})-, se ha encontrado evidencia consistente de que dicha discrepancia se refleja simultáneamente en el número de Hits y Falsas Alarmas cometidas en cada una ($F.Alarmas(A) < F.Alarmas (B) < Hits(B) < Hits(A)$) un patrón que sería identificado en la literatura como Efecto Espejo. Sin embargo, la extensión de este patrón a otras áreas no ha sido explorada todavía y su estudio e interpretación se ha restringido al dominio específico de las tareas de reconocimiento. En el presente trabajo se reportan los patrones de respuesta asociados con el Efecto Espejo en una tarea de detección perceptual donde se incluyeron protocolos de respuesta binaria y de escala de confianza, con dos niveles de discriminabilidad construidos con base en la literatura sobre la ilusión de Ebbinghaus. Dicha evidencia fue evaluada tanto a partir de la replicación del análisis clásico (pruebas t y ANOVAs), como con la construcción de modelos bayesianos. 
\end{abstract}



\cleardoublepage
\vspace*{0.11\textheight}
\begin{center}
	{\huge\textit{Abstract} \par}
	\bigskip
	\bigskip
	{\normalsize\bfseries Studies on Signal Detection \par} % Thesis title
	\medskip
	{\normalsize by \authorname \par} % Author name
	\bigskip
\end{center}
On recognition memory studies where Signal Detection Theory has been aplied to compare subjects' performance across two classes -where one of these classes is known to be easier to recognize (A) than the other (B)-, certain patterns of response have consistently been found, showing that differences between classes A and B are reflected both on the identification of old stimuli as old, and new stimuli as New ($F.Alarms(A) < F.Alarms (B) < Hits(B) < Hits(A)$). This pattern has been identified within the memory literature as the Mirror Effect. However, neither this phenomenom nor its implications have been explored outside this particular field. In the present thesis project we report evidence of these same patterns of response on a detection task that involves perception only. In the task presented, participants' responses were captured across two differnt procedures: a Yes/No task and by Confidence Rates, and the classes A and B were constructed based on what is known about the functioning of the Ebbinghaus illusion. The results obtained were analyzed both by means of the replication of what has been reported in the literature so far (t tests and ANOVAs), and by bayesian means, developing some bayesian cognitive models and conducting some bayesian statistical tests.\\

%----------------------------------------------------------------------------------------
%	ACKNOWLEDGEMENTS
%----------------------------------------------------------------------------------------

\begin{acknowledgements}
\addchaptertocentry{\acknowledgementname} % Add the acknowledgements to the table of contents

\textbf{1. A mi familia}\\

Las primeras personas en quienes pienso al leer la palabra \textit{Agradecimiento} son mis padres, \textbf{Sandra Amada De la Peña Cortina} y \textbf{David Chávez Granados}, ¡a ustedes les debo cuanto soy y tengo! Gracias por toda la paciencia y apoyo que me brindaron durante los dos años que me tomó llegar a la impresión de esta tesis. Mamá, gracias por enseñarme a no conformarme nunca y a dar siempre un poquito más, porque sé que -aunque no siempre te lo diga- es gracias a ti que soy quien soy. Y papá, gracias por siempre confiar en mí y por el eterno apoyo que me has dado; tener a mi Superman en casa me ha enseñado a creer que existe un lado 'bueno' en el mundo por el que vale la pena luchar. \\

Gracias a mi hermana \textbf{Angélica} -¡'la Enana'!- por ser la mejor compañera de aventuras, cómplice y confidente que pude haber deseado desde su llegada al mundo. La vida sería una carga mucho más pesada sino tuviera a la enana -pegada a su computadora- esperando en nuestro cuarto, dispuesta a escuchar las patoaventuras y dilemas existenciales de su hermana mayor.\\

%%%Gracias Juan Pablo Aguayo Chávez -¡JUANCHO!- por ser mi hermano mayor. Me llena de un profundo orgullo verte crecer en lo personal y profesionalmente. Tu hambre de éxito y aventura te llevará lejos, ¡y me hará muy feliz ser testigo de ello!\\

%Gracias a la \textbf{familia Mejía-Chávez} y a la \textbf{familia Jiménez-Lloyd} por todo su cariño e infinito apoyo. Los llevo en mi corazon siempre.\\

%%%Gracias a la familia que me ha acompañado siempre. Gracias a mis abuelitos Lupe y Juan que siempre han estado al pendiente de a su nieta más distante; a mis padrinos, Grisel y Tomás, por cada muestra de amor y apoyo que me han obsequiado; a mis 'hermanos', JuanMa y Moni, que pese a la distancia y la brecha generacional me han hecho sentir su apoyo y cariño incondicional; a mi primo-hermano (literalmente) Juancho, con quien crecí y quien año con año cuando me hace admirarlo más cuando lo veo realizar sus sueños y proyectos; y a Peter por ser siempre un ejemplo a seguir, el hermano mayor que siempre quise tener.\\ 

\textbf{2. A Jaime}\\

Muchísimas gracias a \textbf{Jaime Osvaldo Islas Farias} -mi amor, mi vida, ¡mi compañero de equipo!- por todo, absolutamente todo lo que hemos compartido. Las palabras que vienen a mi mente para expresar cuán en deuda estoy con la vida por permitirme encontrarte no hacen justicia a la emoción que siento, aún después de cuatro años juntos, cada vez que me das un beso en la frente. Gracias por toda tu paciencia, apoyo, comprensión y cuidados; por ser el mejor compañero de equipo que la vida pudo haber puesto en mi camino; por todo el tiempo que hemos compartido.\\

Encontrar al hombre perfecto y conquistar su corazón, siempre será el mayor logro de mi vida. ¡¡Te amo!!.\\


\textbf{3. Al Lab25}\\

Primero, gracias a los 'big guns' del Lab25 (\textbf{Manuel Villareal}, \textbf{Melisa Chávez}, \textbf{Darío Trujano} y \textbf{José Luis Baroja}, quien es también mi sinodal) que con su ejemplo, han mantenido al resto del Lab 25 motivado a seguir trabajando y esforzándose por hacer justicia a los estándares que ustedes han forjado. Realmente no creo que tengan una idea de cuánto los he admirado desde mi llegada al lab, o de cuánto me costó perder el miedo a interactuar con ustedes. Siempre será un orgullo para mí decir que compartí Lab con personas tan francamente brillantes, ¡gracias, muchachos!\\

Gracias a \textbf{Elena Villalobos}, por ser un recordatorio constante de cuán importante es la constancia y la pasión por lo que se hace. En estos tres años he aprendido mucho de ti y de mí misma, a raíz de todas esas pláticas, fiestas y, también, los desacuerdos ocasionales.\\

Gracias a \textbf{Karina Ruíz García}, \textbf{Ángel Garduño}, \textbf{Álvaro Zacarías} y \textbf{Mauricio Hernández} por todo el apoyo y la alegría que me dan. Me es muy agradable pensar que a quienes hace algunos años conocí como 'nuestros chicos de ACA', ahora son ¡el futuro del Lab 25!.\\

Gracias, a los miembros del Lab25 que tuvieron que aguantar la presentación de mis avances en numerosas ocasiones durante nuestros seminarios: \textbf{Victoria Torres, Alfonso Medina, Yuznhio Sierra, Paulina Eckerle, Itzel Laurel, Carlos Velázquez,  Stéphane Lejars} y \textbf{Diana Álvarez}.\\

Finalmente, con especial énfasis y muchísimo cariño, quiero agradecer a \textbf{José Manuel Niño} y \textbf{Uriel O. González Bravo} por todo su apoyo. De corazón, gracias por regresarme las ganas de levantarme por la mañana. ¡Los quiero muchísimo!\\

\textbf{4. A mis amigos}\\

\textbf{Niño}, gracias INFINITAS por ser el primero en revisar el presente trabajo, en el mismo instante en que di el \textit{-primer-} punto final; por siempre hacerte un tiempo para ayudarme a dar sentido al desorden que suele haber en mi cabeza; por el tiempo y la atención que has dedicado a conocerme y ayudarme a cuidarme hasta de mí misma, y sobretodo,  gracias por enseñarme a reírme de mis tragedias y demostrarme que no hay problema en la vida que no pueda terminar en un 'bueno, ¿y qué piensas hacer al respecto?'. Amigo mío, sé que no te gusta este 'título' y que tú no se lo concedes a nadie, por ello te agradezco con todo mi corazón lo que haz hecho por mí. Sabes que eres alguien a quien tengo en muy, muy alta estima: admiro mucho la optimabilidad con que tomas decisiones y lo increíblemente bien que te adaptas a cualquier cosa. En fin, como ya te he dicho antes: si la vida fuera un libro, tu serías mi personaje favorito.\\

\textbf{Uri}, gracias por ser parte del ruido blanco del Lab 25 mientras intentaba trabajar en la tesis... ¡Mentira! ¡Sabes que te quiero muchísimo! Gracias por llenar mis días de luz con tu adorable presencia. Admiro muchísimo de ti la versatilidad de tu mente, que puede hacer que las cosas más complejas del mundo parezcan lo más simple e intuitivo (y al mismo tiempo, complicarse la existencia con las cosas más sencillas...).\\

Gracias a \textbf{Edgar Vázquez Silva}. Primero, por existir tal cual eres, y segundo, por tomar cada una de las decisiones que has tomado en tu vida, que eventualmente dieron pie a que nuestros caminos se juntaran. Edgar, eres un ser humano excepcional y es para mí un gusto enorme poder llamarme tu amiga. Gracias por ser tan bueno, conmigo y con el mundo, siempre. La bondad en tu corazón y la capacidad que tienes para asombrarte con el mundo, es algo que admiro profundamente. Estoy convenida de que el mundo sería un lugar mejor si tuviéramos más 'Edgars' en él.\\

Gracias a \textbf{Alejandro Vázquez Calderón} -¡Ale!- por estos ocho años de apoyo incondicional y cariño. Gracias por todas las tardes de café en que aguantaste mi parlanchinería y me permitiste desahogar mis conflictos existenciales. Gracias por todo, gracias por tanto. Pero sobre todas las cosas, gracias por demostrarme que cuando se quiere a alguien, se hace todo por estar ahí, no importa cómo. Espero que para este punto de nuestras vidas quede claro que siempre tendrás un lugar muy especial para mí.\\

Gracias a \textbf{Sol Fernández Rodríguez} -Solesiwi-, por regresarle el sentido a la expresión \textit{'mejor amiga'}. Gracias por toda tu paciencia y empatía; gracias por esa don casi mágico que tienes para aterrizar mis pies en la tierra y traducir mis tormentos en soluciones concretas; gracias por siempr tener tiempo para escucharme, y sobretodo, gracias por ser el excelente ser humano que eres, e iluminar a quienes te rodeamos con tu ejemplo.\\


\textbf{5. A los académicos e investigadores que me apoyaron como sinodales.}\\

Gracias al \textbf{Dr. Germán Palafox Palafox} por permitirme contar con su apoyo como Revisor de la presente tesis. A mi breve paso por su Laboratorio le debo mi interés por el estudio de la Percepción y otros fenómenos cognitivos que nos permiten dar sentido al mundo que nos rodea.\\

Gracias al \textbf{Dr. Óscar Zamora Arévalo} por ser uno de los académicos más comprometidos y abiertos a ayudar a la comunidad estudiantil que he conocido en mi paso por la Facultad. Muchas gracias por acceder a apoyarme como sinodal de la presente tesis.\\

Gracias al \textbf{Mtro. Miguel Herrera Ortiz} por todo el apoyo que me ha brindado a lo largo de toda la carrera. La perspectiva tan amplia y la narrativa tan mágica con que usted suele presentar el estudio de la Cognición y la Toma de Decisiones, fue la principal razón por la que decidí orientar mi formación hacia la Psicología Experimental.\\

Gracias a \textbf{José Luis Baroja Manzano} por acceder a ser uno de mis sinodales. José Luis, eres alguien a quien admiro profundamente. Desde que entré al Lab25 en octavo semestre, por allá del 2015, has sido una figura muy importante para mí. Siempre me ha maravillado la forma en que puedes traducir los modelos más complejos en ejemplos escandalosamente cotidianos. El primer código que hice en R, fue gracias a que decidiste regalar una tarde de viernes a la 'nueva del lab', que francamente se encontraba completamente perdida. Siempre he pensado en tí como un genio de nuestro siglo: admiro muchísimo tu pasión por buscar la pregunta perfecta y, al mismo tiempo, cuestionar todo intento por dar respuesta a ella. En pocas palabras, es extremadamente significativo para mí (statistical pun intended) que formes parte de mi jurado de titulación.\\

\textbf{6. Last but not least... al Doc.}\\

Y sobre todo, "last but not least", agradezco infinitamente al \textbf{Dr. Arturo Bouzas Riaño} por brindarme la oportunidad de trabajar bajo su guía, por permitirme el honor de llamarme su estudiante y por todas las oportunidades de crecimiento que puso a mi alcance. Sé que el tema que decidí trabajar no es precisamente 'su mero mole' y sin embargo, siempre me dio libertad de trabajar en lo que a mí me gustaba. Por éso y por la confianza que depositó en mí al permitirme colaborar en los diversos proyectos del laboratorio, es que siempre será para mí EL Doc.\\

%\ldots
\end{acknowledgements}

%----------------------------------------------------------------------------------------
%	LIST OF CONTENTS/FIGURES/TABLES PAGES
%----------------------------------------------------------------------------------------

\tableofcontents % Prints the main table of contents

\listoffigures % Prints the list of figures

\listoftables % Prints the list of tables

%----------------------------------------------------------------------------------------
%	ABBREVIATIONS
%----------------------------------------------------------------------------------------

\begin{abbreviations}{ll} % Include a list of abbreviations (a table of two columns)

\textbf{AUC} & \textbf{A}rea-\textbf{U}nder (the) \textbf{C}urve\\
\textbf{CDF} & \textbf{C}umulative \textbf{D}ensity \textbf{F}unction\\
\textbf{CSV} & \textbf{C}omma \textbf{S}eparated \textbf{V}alues\\
\textbf{JASP} & \textbf{J}effrey's \textbf{A}mazing \textbf{S}tatistics \textbf{P}rogran\\
\textbf{SDT} & \textbf{S}ignal \textbf{D}etection \textbf{T}heory\\
\textbf{ROC} & \textbf{R}eceiver-\textbf{O}perating \textbf{C}haracteristic curve\\
\textbf{RT} & \textbf{R}esponse \textbf{T}ime\\
\textbf{MOC} & \textbf{M}emory-\textbf{O}perating \textbf{C}haracteristic curve\\

\end{abbreviations}


%----------------------------------------------------------------------------------------
%	DEDICATION
%----------------------------------------------------------------------------------------

\dedicatory{\textit{En memoria de...}\\\\
\vspace{30mm}
Mi tía,\\
María Eugenia Leticia De la Peña Cortina\\
(1956-2011)\\\\
\vspace{20mm}
Mi padrino,\\
Tomás Munguía Ramírez\\
(1958-2017)\\\\
\vspace{10mm}
 y\\\\
\vspace{10mm}
Mi abue,\\
Guadalupe Cortina González\\
(1928-2018)} 


%\vspace*{0.2\textheight}

%\noindent\enquote{\itshape Arriésgate, equivócate, cáete... pero vive.}\bigbreak

%\hfill Mi tía Letty, \\Mayo del 2009.


%----------------------------------------------------------------------------------------
%	THESIS CONTENT - CHAPTERS
%----------------------------------------------------------------------------------------

\mainmatter % Begin numeric (1,2,3...) page numbering

\pagestyle{thesis} % Return the page headers back to the "thesis" style

% Include the chapters of the thesis as separate files from the Chapters folder
% Uncomment the lines as you write the chapters

% Chapter 1

\chapter{Introducción} % Main chapter title

\label{Chapter1} % For referencing the chapter elsewhere, use \ref{Chapter1} 

%----------------------------------------------------------------------------------------

% Define some commands to keep the formatting separated from the content 
\newcommand{\keyword}[1]{\textbf{#1}}
\newcommand{\tabhead}[1]{\textbf{#1}}
\newcommand{\code}[1]{\texttt{#1}}
\newcommand{\file}[1]{\texttt{\bfseries#1}}
\newcommand{\option}[1]{\texttt{\itshape#1}}

%----------------------------------------------------------------------------------------

El mundo está cargado de ruido e incertidumbre. Los organismos están constantemente expuestos a distintas fuentes de estimulación en su entorno que pueden, o no, dar información relevante sobre el estado de las cosas y las reglas vigentes. Por ello, uno de los primeros grandes problemas de adaptabilidad a los que se enfrentan los organismos es el de ordenar el caos resultante, definiendo relaciones de contingencia que les permitan hacer predicciones sobre la disponibilidad de sucesos biológicamente importantes y ajustar su comportamiento a las restricciones operantes. Una vez establecida la relación entre la presencia u ocurrencia de ciertos estímulos y el acceso a ciertas consecuencias, la detección de éstos se vuelve una tarea importante para que los organismos puedan guiar su comportamiento, (por ejemplo: \textit{'Sé que soy alérgico a las nueces, ¿En este panqué hay nueces? Si hay nueces en el panqué, no me lo como; si no hay nueces en el panqué, sí me la como'}).\\


%La detección de ciertos eventos no es una tarea sencilla. La información a evaluar suele ser ambigua.
La detección no parecería ser un problema importante si asumiéramos que los eventos aparecen con perfecta claridad, o bien, que el organismo interesado en su detección cuenta con sensores altamente precisos que le garantizan el éxito. Sin embargo, la evidencia a partir de la cual juzgamos si algo está o no ocurriendo por lo general es confusa y puede llevarnos a emitir juicios erróneos. A manera de ejemplo cotidiano, imaginemos el caso de un adolescente que quiere conseguir permiso para ir de fiesta y necesita encontrar el momento ideal para pedírselo a su mamá (cuando ella esté de buen humor); los indicadores con que cuenta son imprecisos -los gestos, el tono de voz, las actividades que su madre realice durante el día, etc.- y errar en el diagnóstico del estado emocional de su madre y en consecuencia no obtener el permiso deseado al pedirlo en el momento inadecuado es un riesgo latente, ya sea por una mala lectura de los datos disponibles -que las ansias del adolescente por salir de fiesta le hagan apresurar el momento- o bien porque los datos en sí mismos son poco claros -la mamá podría ser una persona particularmente inexpresiva o, por el contrario, terriblemente variable-.\\ 

%La detección es distinta de la discriminación y la categorización. Todos problemas importantes en un entorno cargado de estimulación. (El problema del embudo)
A diferencia de problemas tales como la discriminación o la categorización, donde la tarea de los organismos es evaluar la evidencia que se les presenta para asignarle una etiqueta (tales como \textit{'¿Es A o es B?'} ó \textit{'De acuerdo a sus propiedades en tales dimensiones, se trata de un caso de...'}), cuando hablamos de un problema de detección nos referimos a situaciones que pueden plantearse en términos de \textit{preguntas Sí/No}, (por ejemplo, \textit{'¿La comida está buena? Sí/No'}, \textit{'¿Ese que viene es mi camión? Sí/No'}, \textit{'¿Este perro es hostil? Sí/No'}) y cuya respuesta permite guiar el comportamiento de los organismos en función de las consecuencias anunciadas (\textit{'Sí, la comida está buena, me la comeré porque es seguro'}, \textit{'No, ese no es mi camión, no me subiré porque acabaré en Ecatepec'}, \textit{'No, no es un perro hostil, puedo acariciarlo'}, etc.).\\ 

La Teoría de Detección de Señales (SDT en ingles) presenta un modelo estadístico que permite describir las tareas de detección como un problema de decisión al que tienen que enfrentarse los organismos como sistemas inmersos en entornos con incertidumbre -es decir, entornos dinámicos que cambian en el tiempo y presentan variabilidad en la disponibilidad y restricción de ciertos eventos- y que buscan guiar su comportamiento de la manera más óptima posible dada la estructura del mismo. Se trata de uno de los modelos más sólidos y ampliamente estudiados en Psicología, cuyos supuestos son lo suficientemente generales para permitir su aplicación al estudio de distintos fenómenos. Funciona tanto como un modelo estadístico para describir esta clase de problemas, como una herramienta para interpretar la ejecución de  sistemas evaluados experimentalmente y hacer inferencias sobre la precisión con que el estímulo a detectar se distingue del Ruido y la posible preferencia del sistema a responder en favor o en contra de la misma.\\

La SDT le concede a la noción de variabilidad un papel fundamental para entender la detección de señales como un problema de adaptabilidad para los organismos. La idea básica es que las señales cuya detección resulta relevante para los organismos -al igual que cualquier otro estímulo- suelen presentarse y percibirse con cierta variabilidad y además de ello, coexisten en el mundo con otros estímulos (el Ruido) que dentro de su propia variabilidad pueden llegar a presentarse o percibirse con la misma evidencia que lo haría una Señal, pudiendo ser confundidos con la misma. Por ejemplo, imaginemos que queremos detectar si la persona que nos acaba de contestar el teléfono es un adulto. Existe un rango de 'tonos de voz' que estaríamos dispuestos a admitir como pertenecientes a una persona mayor de edad que, a grandes rasgos, tiende a ser más grave de lo que esperaríamos escuchar en un niño. Sin embargo, sabemos que hay adultos que pueden llegar a tener una voz particularmente clara y que puede llegar a confundirse con la de un menor de edad; si la persona que nos contestó el teléfono resulta ser un varón de voz aguda, no podremos estar seguros de si se trata de un adulto -o no- con sólo escuchar su voz y probablemente necesitemos consultarle de manera explícita.\\

La SDT es un modelo de decisión que asume que los organismos no detectan los elementos relevantes de su entorno en respuesta directa a la estimulación que reciben de manera inmediata, sino que ponderan la evidencia con la información que poseen sobre el mismo. Bajo esta visión, los organismos \textit{eligen} el juicio de detección que les permite guiar su comportamiento de la manera más óptima posible tomando en consideración 1) las ganancias y pérdidas en juego (que hacen más, o menos, riesgoso el cometer cierto tipo de error y más, o menos, atractivo el cometer un acierto en particular) y 2) la probabilidad con que dichos eventos se presentan en su entorno.  La noción de los umbrales ampliamente desarrollada en la Psicofísica clásica es reemplazada por la idea del criterio de elección.\\ 

Al aplicar el modelo de Detección de Señales a tareas de Memoria de Reconocimiento, donde los participantes tienen que identificar los elementos ya antes vistos (la Señal) dentro de un conjunto de ítems que incluye elementos presentados en una fase previa y elementos nuevos (el Ruido), se ha encontrado consistentemente un patrón de respuestas cuando se compara la ejecución de los participantes entre dos clases de estímulos, (siendo una de ellas más fácil de reconocer (A) que la otra (B)), que demuestra que los participantes no sólamente son mejores reconociendo los elementos previamente mostrados en la condición A ($Hits(A)>Hits(B)$), sino que también son mejores identificando los estímulos nuevos dentro de esta misma condición ($F.alarm(A)<F.alarm(B)$). Dado que en estos estudios los participantes experimentales no saben que se ha incluido más de una clase de estímulos en la tarea que se les presenta, se asume que utilizan un sólo criterio de elección para emitir sus respuestas y de acuerdo a las tasas reportadas de Hits y Falsas Alarmas por cada clase, se sugiere que las distribuciones de Ruido y Señal de cada clase se distribuyen a lo largo del mismo eje de evidencia de tal forma que parecieran reflejarse entre sí. Por ello, en la literatura en Memoria de Reconocimiento se ha identificado dicho patrón de respuestas bajo el nombre de Efecto Espejo.\\

%3.5*42.5
El Efecto Espejo sólo ha sido estudiado dentro del dominio de la Memoria de Reconocimiento, donde se ha reportado evidencia de su existencia a lo largo de una amplia variedad de procedimientos (tareas Sí/No, tareas de Elección Forzada entre dos Alternativas y protocolos con Escala de Confianza) y variables (palabras comunes vs palabras extrañas; estímulos abstractos vs estímulos concretos; palabras escritas al revés vs palabras bien escritas; imágenes a color o en blanco y negro, etc). Como resultado, gran parte de los modelos y teorías desarrollados para dar cuenta de este fenómeno tienden a hacerlo en términos de la estructura propia de las tareas de reconocimiento, donde se incluye una fase de estudio en la que se asume que los participantes procesan los estímulos para añadirles la \textit{'familiaridad'} necesaria para poder reconocerlos en una segunda fase y donde se asume que tienen origen las diferencias observadas en el desempeño de los participantes -en otras palabras, se asume que las clases de estímulos puestas a prueba son procesadas y atendidas de distinta manera durante la fase de estudio previa a la tarea de reconocimiento-.\\

El interés principal del presente trabajo de tesis fue explorar la generalizabilidad del Efecto Espejo, buscando evidencia del mismo en una tarea de detección ajena a la Memoria de Reconocimiento. Para ello, se presentan dos variaciones de una tarea de detección perceptual (visual) que emula la estructura de los estudios donde dicho fenómenos ha sido reportado en Memoria, construyendo dos niveles de dificultad con base en la literatura en Ilusiones Ópticas. La tarea propuesta fue presentada a los participantes a partir de dos protocolos: una tarea Sí/No y la asignación de un puntaje en una Escala de Confianza. Los resultados e implicaciones de los mismos se discuten en detalle.\\
% Chapter 1
\chapter{Marco Teórico} % Main chapter title

\label{Cap_SDT} % For referencing the chapter elsewhere, use \ref{Chapter1} 

%----------------------------------------------------------------------------------------

% Define some commands to keep the formatting separated from the content 
\newcommand{\keyword}[1]{\textbf{#1}}
\newcommand{\tabhead}[1]{\textbf{#1}}
\newcommand{\code}[1]{\texttt{#1}}
\newcommand{\file}[1]{\texttt{\bfseries#1}}
\newcommand{\option}[1]{\texttt{\itshape#1}}

%----------------------------------------------------------------------------------------

\section{Teoría de Detección de Señales}

%Detectar ciertos estados en el mundo es importante para guiar nuestro comportamiento
Uno de los problemas más frecuentes a los que se enfrentan los organismos como sistemas inmersos en entornos variables que buscan optimizar su comportamiento, es la detección de estados o eventos específicos que les informen sobre las restricciones y relaciones de contingencia vigentes.\\

%La detección es distinta de la discriminación y la categorización. Todos problemas importantes en un entorno cargado de estimulación. (El problema del embudo)
A diferencia de problemas tales como la discriminación o la categorización, donde la tarea de los organismos es evaluar la evidencia que se les presenta para asignarle una etiqueta (e.g. "¿Es A o es B?"; "¿De acuerdo a sus propiedades en tales dimensiones, se trata de un caso de..."), cuando hablamos de un problema de detección nos referimos a situaciones que pueden plantearse en términos de preguntas "Sí/No" (e.g. "¿La comida está buena? Sí/No", "¿Ese que viene es mi camión? Sí/No", "¿Este perro es hostil? Sí/No") y cuya respuesta permite guiar el comportamiento de los organismos en función de las consecuencias anunciadas (e.g. "Sí, la comida está buena, me la comeré porque es seguro", "No, ese no es mi camión, no me subiré porque acabaré en Ecatepec", "No, no es un perro hostil, puedo acariciarlo", etc.).\\ 

%Origen y expansión de la Teoría de Detección de Señales en la psicología y otras áreas
La Teoría de Detección de Señales (TDS o SDT, por sus siglas en inglés) aparece por primera vez en 1954 -como tantos otros avances científicos y tecnológicos motivados por las necesidades planteadas por la Segunda Guerra Mundial- en el contexto del estudio y desarrollo de radares para detectar señales eléctricas específicas \parencite{Peterson1954}. Muy poco tiempo después, los psicólogos John A. Swets y Wilson P. Tanner contribuyeron a la expansión de la teoría a un contexto psicológico, como un modelo para estudiar la percepción de los organismos, \parencite{Tanner1954, Swets1961}. Desde entonces, la TDS constituye uno de los modelos más estudiados, desarrollados y ampliamente aplicados en Psicología, extendiéndose desde su foco inicial en el estudio de la percepción \parencite{Rosenholtz2001, Pessoa2005, Wallis2007} hacia el estudio de cualquier fenómeno o tarea donde los organismos se enfrenten al problema de emitir -y guiar su comportamiento en función a- juicios de detección; por ejemplo, en materia de la emisión de diagnósticos clínicos \parencite{Grossberg1978, Swets2000, Boutis2010}, en el estudio de ciertas condiciones clínicas \parencite{Westermann2010, Bonnel2003, Brown1994, Naliboff1981}, en el estudio de la identificación visual de testigos \parencite{Gronlund2014, Wixted2014, Wixted2016} y un muy amplio 'etcétera' \parencite{Gordon1974, Nuechterlein1983, Harvey1992, Verghese2001}.\\ 


%La Teoría de Detección de señales como un modelo descriptivo para el problema de la detección que admite la importancia de la incertidumbre, como parte del entorno y como motor en el uso de sesgos de respuesta.
La TDS constituye un modelo estadístico que describe el problema al que se enfrentan los organismos inmersos en situaciones de detección en ambientes con incertidumbre, donde las señales -los estímulos cuya ocurrencia interesa detectar- 
coexisten con ruido -estímulos que no son la señal pero que pueden confundirse con esta-. Se trata de un modelo de decisión que entiende la detección como una tarea de elección, donde los organismos no responden simplemente con base en lo que perciben, sino que eligen el juicio de detección que les permita guiar su comportamiento de la manera mas óptima posible dada la información que poseen sobre la estructura del entorno -probabilidades y consecuencias involucradas- a la luz de la evidencia presente.\\

La generalizabilidad del modelo de la TDS al estudio de distintos fenómenos y tareas de detección se debe a lo abstracto de sus elementos: la 'señal' que interesa detectar puede ser desde un estímulo concreto -una luz o un tono- hasta la pertenencia a una categoría -una enfermedad o amenaza- y el 'ruido' es simplemente todo elemento presente en el entorno de la tarea que no sea la señal.\\ 

Los organismos compensan la incertidumbre contenida en las tareas de detección con la información que poseen sobre el entorno. En términos generales, ésta puede ser de dos tipos: 1) información probabilística y 2) información sobre las consecuencias comprometidas. Sin importar a cuál de estas categorías pertenezca, los organismos construyen la 'información sobre el entorno de decisión' con base en su experiencia con el mismo y en la información que han recibido sobre este a lo largo de su vida. Por ejemplo, imaginemos el caso de un médico que trata de decidir si los resultados obtenidos en cierta prueba clínica son evidencia suficiente para diagnosticar una enfermedad 'X' a un paciente 'Y'. La evidencia con la que el médico cuenta es imprecisa: toda prueba clínica tiene un margen de error y su lectura debe complementarse con información extraída de su historia clínica por un médico especialista. El resultado de la prueba no es lo suficientemente informativo en sí mismo y el médico debe juzgar la evidencia en función de distintos factores, por ejemplo: ¿Qué tan confiable es la prueba?, ¿cuál es su tasa de aciertos y errores?; ¿qué tan común es la enfermedad o condición cuya presencia se intenta determinar?; de acuerdo con la historia clínica del paciente, ¿qué tanto correlacionan sus características con los factores de riesgo asociados a dicha enfermedad o condición?. En otras palabras, el médico tiene que compensar la incertidumbre implícita en la evidencia a evaluar con toda la información probabilística de la que dispone. Y la historia no termina aquí.  Pese a que la inferencia probabilística contribuye a identificar la conclusión más probable, el médico no puede estar completamente seguro de su respuesta. Para optimizar su comportamiento y tomar la mejor decisión posible, el médico también debe tomar en consideración la información que posee sobre las consecuencias asociadas a cada escenario posible: a) Si el paciente tiene la enfermedad y el médico la detecta acertadamente, podrá tratarse a tiempo; b) Si tiene la enfermedad y el médico falla en detectarla, podría poner en riesgo su vida; c) Si no tiene la enfermedad y el médico le dice que sí, se gastarán recursos innecesarios en solucionar un problema que no existe, corriendo el riesgo de que el tratamiento le haga daño y d) Si no tiene la enfermedad y el médico decide no darle el diagnóstico, todo permanecerá igual. La tarea del médico es mucho más compleja de lo que parecía en un principio, puesto que no se limita a la lectura de una prueba clínica, sino a ponderar lo que sugieren los resultados de la misma con toda la información que posee sobre la probabilidad de las interpretaciones posibles y las consecuencias comprometidas.\\


\subsection{Supuestos generales del modelo}

%La TDS distingue dos grandes factores en la emisión de un juicio o respuesta: La discriminabilidad y el sesgo.
La TDS funciona como una herramienta -o marco de análisis- para traducir el desempeño observado en tareas de detección en inferencias sobre la precisión con que la señal se distingue del ruido (la discriminabilidad) y la posible preferencia -o tendencia- del sistema detector a responder en favor o en contra de esta (el sesgo). Esta distinción entre la Discriminabilidad de los estímulos comprometidos y el Sesgo del sistema, como factores que interactúan en la emisión de juicios de detección, es una de las principales propiedades de la TDS cuya importancia e implicaciones se discuten a continuación:\\

\textbf{1.- El papel de la Discriminabilidad: Siempre hay incertidumbre}\\

%Hay variabilidad en todos los estímulos implicados en las tareas de detección (en la señal y en los estímulos no-señal)
Se habla de la detección de señales como un problema de adaptación porque se asume que la variabilidad en la presentación y percepción de los estímulos en el ambiente merma la capacidad de los organismos de emitir juicios de detección que reflejen el estado del mundo con certeza. Y dado que los estímulos-señal coexisten en el mundo con estímulos-ruido, saber qué tan salientes son las señales respecto del ruido es uno de los factores más importantes para determinar qué tan difícil es su detección para los organismos. En términos de la TDS, se habla de dicha dificultad como 'la discriminabilidad' de los estímulos comprometidos en la tarea.\\

De acuerdo con la TDS, la discriminabilidad constituye el primer gran componente en la emisión de juicios de detección óptimos que reflejen el verdadero estado del mundo y permitan al organismo actuar conforme a las consecuencias vigetes. Suele explicarse en términos de:\\ %  1) la variabilidad intrínseca en la presentación de las señales y 2) el ruido con que ésta coexiste.\\

\underline{a) La Variabilidad en la Señal}\\

%Existe variabilidad en la forma en que percibimos los estímulos que nos rodean. Los sistemas sensoriales y perceptuales se comportan como instrumentos de medición (error de medida)
La noción de variabilidad ha sido uno de los principales motores para el desarrollo de modelos estadísticos en Psicología. Desde que Fechner extendiera las ideas planteadas por Gauss sobre la incertidumbre contenida en toda medición -la idea de que toda medición realizada contiene el valor 'verdadero' de aquello que se quiere medir más un 'error' aleatorio que la carga de incertidumbre- al estudio de la percepción -conceptualizando nuestros sistemas sensoriales y perceptuales como 'instrumentos de medición' que perciben las cualidades 'verdaderas' de los estímulos más un 'error' en cada observación- \parencite{Fechner, Gauss}, se sentaron las bases para el desarrollo de una amplia gama de modelos matemáticos y estadísticos en Psicofísica orientados a estudiar la relación entre las cualidades físicas -'reales'- de los estímulos y la magnitud o intensidad con que se perciben psicológicamente \parencite{Link1994}.\\

%Variabilidad en la percepción de un mismo estímulo.
En el marco de la TDS, la variabilidad se considera una propiedad intrínseca de las señales a detectar bajo el supuesto de que ningún estímulo se percibe o se presenta de manera idéntica en cada exposición. Por ejemplo, imaginemos los siguientes casos: \\

\begin{itemize}

\item Una persona es expuesta a un mismo tono en cien ocasiones distintas y tras cada presentación, asigna un valor a la intensidad percibida. El valor reportado en cada ensayo será una mezcla entre el valor real del tono y un error aleatorio; es decir, los reportes se acercarán bastante al valor real de estímulo pero presentarán cierta variabilidad en torno al mismo. Esta idea se captura en la Figura~\ref{fig:Senal_percepcion}. Imaginemos que el estímulo presentado tiene un valor real de 10 (la unidad de medición no importa para fines de este ejemplo): Es muy probable que el valor percibido y reportado coincida con -o se acerque bastante a- su valor real -la media de la distribución, $\mu$, señalada con una línea vertical roja-, pero también habrá ensayos en que aún tratándose del mismo estímulo, el valor percibido caiga por encima o por debajo de su valor real con cierta dispersión -las colas de la distribución-. Este ejemplo pretende capturar la noción de que hay variabilidad intrínseca a la percepción de los estímulos en nuestro entorno.\\

\item Supongamos el caso de una escala clínica diseñada para aportar evidencia para el diagnóstico de la Depresión. Por lo general, las pruebas clínicas sugieren que el diagnóstico se haga con base en rangos de valores. No todas las personas con depresión van a obtener exactamente el mismo puntaje. La Figura~\ref{fig:Senal_presentacion} representa de manera gráfica esta idea: Existe una serie de posibles puntajes a obtener en la prueba de depresión (los valores en el eje de las x), y se sabe que las personas con depresión suelen obtener puntajes dentro de un rango específico con cierta probabilidad (la distribución azul), habiendo puntajes más comúnes -la media de la distribución, $\mu$, señalada en rojo- que otros. Hay variabilidad en la presentación de ciertos estímulos en el entorno.\\

\end{itemize}

\begin{figure}[th]
\centering
\includegraphics[width=0.80\textwidth]{Figures/Signal_Perception} 
%\decoRule
\caption[Variabilidad en la percepción de los estímulos]{Para ilustrar la idea de que las señales a detectar no son percibidas de la misma forma en cada presentación, se plantea el ejemplo de un estímulo con intensidad de 10 (la elección de valores y la omisión de unidades de medida es arbitraria). Es muy probable que el estímulo sea percibido de acuerdo a su valor real, (la media de la distribución, $\mu$) sin embargo y con menor probabilidad, también es posible que sea percibido como ligeramente más, o menos, intenso, (siendo que los valores que más se alejan del valor 'real' son menos probables que los cercanos).}
\label{fig:Senal_percepcion}
\end{figure}

\begin{figure}[th]
\centering
\includegraphics[width=0.80\textwidth]{Figures/Signal_Presentation} 
%\decoRule
\caption[Variabilidad en la presentación de los estímulos]{Para ilustar la idea de que las señales a detectar no se presentan de la misma forma en cada ocasión, se plantea como ejemplo la detección de una condición clínica. La figura captura la noción de que en una prueba clínica diseñada para detectar casos de Depresión, esta no se identifica a partir de un sólo puntaje sino que existe un rango de valores que se asocian a dicha condición con mayor o menor probabilidad, al rededor de un valor promedio ($\mu$, señalado en rojo). Los valores representados son arbitrarios.}
\label{fig:Senal_presentacion}
\end{figure}

En general, las Figuras~\ref{fig:Senal_percepcion} y \ref{fig:Senal_presentacion} representan el supuesto más elemental descrito en la TDS: la variabilidad es intrínseca a la presentación de los estímulos, ya sea porque nuestros sistemas sensoriales no los capturan igual en cada presentación, o porque los estímulos no se nos presentan exactamente de la misma forma en cada ocasión. En otras palabras, las señales que los organismos necesitan detectar en su entorno para guiar su comportamiento son variables, en tanto que no se presentan ni son percibidas exactamente iguales en cada ocasión.\\

    \underline{b) La variabilidad en el Entorno: Ruido}\\

%La señal coexiste con el ruido y puede llegar a confundirse con el mismo.
Además de la variabilidad contenida en las señales, es necesario tomar en cuenta que coexisten en el mundo con otros estímulos o estados, mismos que pueden producir evidencia similar a las señales y ser por tanto, confundidos con las mismas. Por ejemplo, retomemos el caso de la prueba clínica para detectar casos de Depresión. Las pruebas clínicas psicológicas se aplican para distinguir entre las personas con cierta condición -la señal- y las que no la tienen -el ruido-. Y habíamos mencionado ya que las personas con depresión no obtienen un mismo puntaje en la prueba, sino que existe un rango de valores asociados a dicha condición con mayor o menor probabilidad. De la misma forma, al aplicar la prueba a personas que no tienen depresión tampoco se obtiene siempre el mismo puntaje, sino que el resultado obtenido suele presentarse dentro de un rango de valores distinto con su propia distribución de probabilidad. Esta extensión del ejemplo original se presenta en la Figura~\ref{fig:Noise}, donde nuevamente tenemos una distribución de probabilidad que señala los puntajes asociados con la condición a detectar (en azul) y una nueva distribución que señala los puntajes que las personas sin depresión suelen obtener al resolver la prueba con distinta probabilidad (en negro).\\

La Figura~\ref{fig:Noise} ilustra otros dos puntos claves considerados en la TDS. El primero, es que las señales no se presentan de manera aislada en el entorno, sino que suelen estar acompañadas de otros estímulos o estados del mundo -que constituyen lo que se conoce como 'ruido'-. El segundo, es que el ruido -al igual que las señales- es variable y puede llegar a presentarse o percibirse de la misma forma en que lo puede hacer la Señal. Este último punto se ilustra en la Figura con el área de las distribuciones presentadas donde estas se sobrelapan, relacionando los mismos valores con ambos estados del mundo -Depresión vs No depresión- con distinta probabilidad. Por ejemplo, parece ser que es posible observar un puntaje de 55 en una persona con o sin depresión, sin embargo, es mucho más probable observar este puntaje en ausencia de dicha condición (según la intersección entre el puntaje y cada una de las distribuciones); lo mismo ocurre con un puntaje de 65, es posible observarlo en ambos casos y sin embargo, es mucho más probable que se presente en una persona con depresión.\\

\begin{figure}[th]
\centering
\includegraphics[width=0.80\textwidth]{Figures/Noise} 
%\decoRule
\caption[Variabilidad en la señal y en el ruido]{Continuando con el ejemplo de la prueba de depresión, se incluyen una distribución para representar el rango de puntajes asociados con dicha condición clínica (en azul) y una segunda distribución que representa el rango de puntajes que las personas sin depresión suelen obtener (en negro). Esta figura representa la noción de que los posibles estados de mundo -la señal y el ruido- se presentan y perciben con cierta variabilidad entre cada ocurrencia. Los valores utilizados son arbitrarios.}
\label{fig:Noise}
\end{figure}

Cabe señalar que la distribución-Señal se sitúa por encima de la distribución-Ruido ya que, como se mencionó previamente, las tareas de detección implican dar respuesta a una pregunta binaria 'Sí/No' respecto a la presencia o ausencia de 'algo' -una señal- en el entorno que permite a los organismos guiar su comportamiento de manera óptima. Sea cual sea la evidencia con base en la cual se forman los juicios de detección -los valores en el eje X sobre el cual se despliegan las distribuciones ruido y señal-, se espera que la Señal tenga 'más' de dicha evidencia que el Ruido (en tanto que este último implica su ausencia). Por ejemplo, en el caso ilustrado en la Figura~\ref{fig:Noise}, tiene sentido que las personas con Depresión obtengan puntajes más altos dado que la prueba fue hecha para evaluar la presencia de dicha condición y el puntaje final obtenido suele ser reflejo de cuántas de las respuestas proporcionadas coinciden con lo que se esperaría en una persona con Depresión.\\

La variabilidad en la presentación y percepción de los posibles estados del entorno -la presencia o ausencia de la señal- constituye el elemento base sobre el cual se desarrolla la TDS y que nos lleva a pensar en la detección de señales como una tarea cargada de incertidumbre, donde los organismos no pueden confiar completamente en la evidencia que se les presenta -y no pueden emitir un juicio de detección basándose únicamente en esta- puesto que dicha evidencia puede corresponder a -o haber sido producida por- cualquiera de las interpretaciones posibles -'la señal está presente' o 'no, sólo hay ruido'-. Por ello, la TDS asume que el primer factor que incide en la ejecución y emisión de juicios de detección por parte de los sistemas inmersos en este tipo de tareas, es la discriminabilidad de la señal en relación al ruido.\\

Hablar de la discriminabilidad en tareas de detección implica cuestionar '¿con qué probabilidad la señal y el ruido producen la misma evidencia?' -o bien, '¿qué tan probable es que la señal y el ruido se confundan?'-. En términos de la representación gráfica de las distribuciones de probabilidad de ruido y señal, implica preguntarnos -y tratar de evaluar- qué tanto se sobrelapan las distribuciones. En general, el área de sobrelape de las distribuciones es un reflejo de la incertidumbre contenida en la tarea. Por ejemplo, tal y como se ilustra en la Figura~\ref{fig:Overlap}, si las distribuciones de ruido y señal están muy separadas, el sobrelape entre estas será pequeño y podemos hablar de un entorno con poca incertidumbre -alta discriminabilidad- donde ambos estados del mundo comparten -con poca probabilidad- muy poca evidencia (panel a); por otro lado, si las distribuciones están más juntas, el sobrelape en estas será mayor, indicándonos que el rango de valores-evidencia que puede relacionarse a las dos interpretaciones posibles es más grande -discriminabilidad baja- (panel b).\\

\begin{figure}[th]
\centering
\includegraphics[width=0.55\textwidth]{Figures/Overlap_Small}\\ 
\includegraphics[width=0.55\textwidth]{Figures/Overlap_Big} 
%\decoRule
\caption[El sobrelape Ruido-señal como reflejo de la incertidumbre contenida en las tareas de detección]{La incertidumbre contenida en las tareas de detección depende de la distancia que existe entre las distribuciones de ruido y señal implicadas, puesto que esta determina el área de sobrelape que existe entre las mismas, que refleja el rango de evidencia que puede ser producido y asociado con ambos estados del mundo -con su propia probabilidad-. El panel a presenta un ejemplo donde el sobrelape es pequeño al estar muy separadas las distribuciones, sugiriendo una tarea con poca incertidumbre; en el panel b, se presenta un escenario hipotético donde las distribuciones están más cerca y comparten, por tanto, más evidencia -hay más sobrelape-, cargando la tarea de una mayor incertidumbre.}
\label{fig:Overlap}
\end{figure}

Podemos pensar en la discriminabilidad como producto de la variabilidad con que la TDS asume que los posibles estados del mundo se presentan y perciben por los sistemas detectores. Se dice entonces que la discriminabilidad, como cualidad intrínseca a toda tarea de detección, depende tanto de las propiedades intrínsecas de los estímulos a evaluar -¿qué tanto comparten los estímulos con la señal vs los estímulos sin esta?- como de la precisión con que los sistemas detectores son capaces de discernir entre dichas instancias. Por ejemplo, \\

El soporte de las distribuciones -identificado en la Figura~\ref{fig:Overlap} bajo el nombre de ‘Evidencia’ ( y en las Figuras~\ref{fig:Senal_presentacion} y \ref{fig:Noise}, como 'Puntaje en prueba clínica' e 'Intensidad' en la Figura~\ref{fig:Senal_percepcion}- rara vez se define con precisión,  teniendo una concepción más bien abstracta; La idea general es que cuando queremos detectar una señal particular, comenzamos a recolectar un tipo de evidencia específico a la tarea ante la que nos encontramos. Lo más importante, es que la señal siempre va a estar asociada en mayor medida con dicha evidencia, distribuyéndose siempre en valores situados por encima (a la derecha, en la Figura 1) del ruido.\\
 
  \textbf{2.- El papel del Sesgo: La detección es decisión}\\

Una consecuencia directa de la variabilidad involucrada en el entorno de decisión, es que el desempeño de todo sistema de detección es propenso a cometer errores y emitir un juicio de presencia o ausencia de la señal, que puede no coincidir con el estado del mundo. Dependiendo la correspondencia entre el estado del mundo y el juicio emitido por el sistema de detección, la TDS maneja las clasificaciones de respuesta mostradas en la Tabla~\ref{fig:Mat_Output}; donde las celdas 2 y 3, corresponden a los errores posibles.\\

\begin{figure}[th]
\centering
\includegraphics[width=0.60\textwidth]{Figures/Matriz_Outputs} 
%\decoRule
\caption[Posibles Resultados en una Tarea de Detección]{}
\label{fig:Mat_Output}
\end{figure}

La TDS asume que el organismo fija un criterio de elección a lo largo del eje de la Evidencia, que va a determinar a partir de cuánta evidencia va a juzgar la señal como presente. Dicho criterio se va a representar como una línea transversal que atraviesa ambas distribuciones en una determinada altura, y se le va a identificar con el parámetro k. La TDS asume que los organismos van a fijar esta regla de elección, ponderando la información a la que tienen acceso con la información que poseen sobre la estructura de la tarea (i.e. cómo suele presentarse la señal, qué tan probable es que se presente, etc.)\\


    \begin{itemize}
      \item{Los errores cuestan y los aciertos pagan: Matrices de pago}\\

      \item{Estimados de Probabilidad}

     \end{itemize}

El sesgo se define como la preferencia del sistema a emitir respuestas de un tipo particular (i.e. 'Sí, detecto la señal' o 'No, no está'). La TDS cuenta con dos parámetros dedicados a la medición y evaluación del sesgo de los sistemas sometidos a tareas de detección.

Sin embargo, no todos los errores tienen el mismo costo. Imaginemos el caso de una presa en potencia que busca determinar si el sonido que acaba de escuchar en la maleza corresponde o no con el de un depredador; no hay tiempo que perder, y el costo que dicho organismo tendría que pagar por cometer una falsa alarma (gasto innecesario de energía) o una omisión (morir devorado) es sustancialmente diferente. En este escenario particular, es muy probable que la presa sea mucho más propensa a correr por su vida, juzgando la presencia del depredador a partir de valores menores de evidencia.\\

Esta discrepancia en el peso que se le da a las consecuencias posibles de emitir una u otra respuesta y obtener uno de los cuatro posibles resultados, suele representarse en términos de una matriz de pagos, que nos ayude a definir cuáles son las consecuencias que el organismo buscará evitar o promover, según sea el caso, en mayor medida.\\

Ya sea por los distintos pesos que tengan las posibles consecuencias para el organismo, o porque se tiene una preferencia o predisposición inherente a decretar la presencia o ausencia de la señal, la TDS asume que el desempeño de los organismos que se enfrentan a tareas de detección de señales va a depender tanto de la calidad de la información a la que se tiene acceso (dentro de lo que se incluye la importancia de la variabilidad, que determina tanto la discriminabilidad de la señal como la sensibilidad del sistema ante la misma), como de un sesgo de elección.\\

La localización del criterio en nuestro eje de evidencia recolectada va a estar altamente influida por el sesgo que tenga nuestro sistema. Podemos hablar entonces de dos tipos distintos de sesgo: conservador y liberal. El primero, favorece la emisión de respuestas negativas al desplazar el criterio a la derecha y requerir al sistema la recolección de mayores niveles de evidencia antes de dar por detectada la señal. El segundo, promueve la detección de la señal, situando el criterio de elección hacia la izquierda, emitiendo un juicio de detección con valores menores de evidencia. Nótese que un sistema carente de sesgo, sería aquel que situara su criterio de elección justo en el punto en que las dos distribuciones se juntan, donde la probabilidad de cometer cualquiera de los tipos de acierto y errores, son iguales entre sí.\\


\subsection{Parámetros del modelo}

Como se mencionó previamente, al realizar una tarea de detección existen dos posibles tipos de aciertos: al detectar la señal (Hits) y al rechazar el ruido (Rechazos), y dos posibles tipos de errores: los falsos positivos (Falsas alarmas) y los falsos negativos (Omisiones). La materia prima con base en la cual funciona el modelo propuesto por la TDS, son las tasas de aciertos y errores cometidos durante la tarea, de manera que por cada participante que pasa por una tarea de detección, tenemos cuatro tasas que describen su ejecución:

La Tabla 2 ilustra el cómputo de las cuatro tasas de ejecución, como una relación entre el resultado obtenido y el tipo de ensayo con base en el que se le definió como tal. Es decir, tenemos dos tasas definidas en relación al número total de ensayos con la señal (la tasa de hits y la tasa de omisiones) que nos dicen qué proporción de los ensayos con señal fueron detectados correctamente y cuáles se dejaron pasar; y tenemos dos tasas definidas en relación al total de ensayos con ruido (la tasa de falsas alarmas y la tasa de rechazos correctos) que nos describen la relación de los ensayos con ruido que fueron discriminados correctamente y aquellos que se confundieron con la señal.

Para realizar el análisis de datos, bajo el marco de la TDS, sólo necesitaremos un par de estas tasas: la tasa de hits y la tasa de falsas alarmas. Esto bajo el entendido de que las tasas de omisión y rechazos correctos no son más que su complemento, respectivamente, y que estas dos tasas contienen toda la información que necesitamos sobre el desempeño de los participantes.

La idea general de la importancia de estas tasas de ejecución, es que cada una representa el área de las distribuciones de ruido y señal que cae a la izquierda o derecha del criterio de decisión.

Para la estimación paramétrica se utiliza la misma lógica, pero se sigue el procedimiento inverso. Dado que no podemos observar ni cuantificar de manera directa el criterio usado por los participantes para responder a la tarea, qué tan juntas o separadas se encuentran las distribuciones de ruido y señal para cada participante o qué tipo de sesgo pudieran estar siguiendo, utilizamos las tasas de ejecución para hacer inferencias sobre la localización del criterio, la diferencia entre las medias de ambas distribuciones y el grado en que una respuesta se favorece sobre otra. 

A partir de ahora comenzaremos a hablar sobre cómo se calculan cada uno de los parámetros del modelo, de acuerdo a la teoría clásica que sigue los supuestos estadísticos previamente descritos.  Es importante aclarar que el Graficador de Tasas previamente expuesto no representa la teoría con entera precisión; el propósito de ese primer Graficador es simplemente ilustrar cómo describe la TDS el comportamiento de un sistema que se enfrenta ante una tarea de detección, donde existen dos distribuciones que se sobreponen. El Graficador permite manipular directamente la localización del criterio, con la simpleza que implicaría desplazar una línea vertical sobre el eje de decisión y ver qué consecuencias tiene sobre la probabilidad de obtener un tipo particular de acierto o error.


Antes de ahondar a detalle en los parámetros, hay que declarar un par de supuestos formales que hace la Teoría para facilitar la representación gráfica del modelo y la estimación paramétrica:

\begin{enumerate}
\item En su forma clásica, la TDS asume que las distribuciones de ruido y señal son distribuciones normales.
  \begin{itemize}
  \item 
  \end{itemize}
\item En su forma estándar, se asume que las distribuciones de ruido y señal son equivariantes, (compartiendo una desviación estándar de 1).
  \begin{itemize}
  \item 
  \end{itemize}
\item La distribución de ruido tiene su media en 0. 
  \begin{itemize}
  \item La estimación de todos los parámetros del modelo de detección de señales se hace tomando como referencia la distribución del ruido (con  media 0 y desviación estándar de 1)
  \end{itemize}
\end{enumerate}



\begin{itemize}
\item Discriminabilidad $(d')$

La disc

La discriminabilidad se representa en los modelos de detección de señales con un parámetro $d'$, que representa la distancia entre las medias de las distribuciones de ruido y señal. 

Para encontrar la distancia entre las medias de la distribución de ruido y señal, necesitamos saber el punto en que el criterio toca cada distribución. Para ello, calculamos las probabilidades complementarias a las tasas de hits y falsas alarmas y las traducimos a puntajes Z (Ver Fig. 3). Dado que el puntaje Z funciona como una medida de dispersión de la media, basta con restar el puntaje Z de la intersección del criterio con la distribución de señal a el puntaje Z de intersección con la distribución de ruido para conocer la localización de la media de la señal. Por definición, d’ sólo puede tener valores positivos ya que la teoría asume que la distribución de señal siempre está a la derecha de la distribución de ruido porque contiene una mayor cantidad de la evidencia con base en la cual se hace el juicio de detección de la señal.



\item Criterio  $k$

Una vez que hemos resumido el desempeño de nuestro participante en la tarea de detección, el parámetro cuya estimación resulta más sencilla y directa es el Criterio (k). Entender cómo se computa el parámetro nos requiere únicamente de mantener presente el supuesto de que el Ruido se distribuye normalmente y se va a localizar siempre a la izquierda de la señal, por lo que le asignamos una media de cero para tener un punto de referencia para estimar el espacio en que se desarrollan el resto de los parámetros. \\

Para calcular el criterio lo único que necesitamos es conocer la tasa de Falsas Alarmas, que tal y como mencionábamos en el segmento anterior, nos indica qué proporción de la distribución de ruido cae a la derecha del criterio. Dado que a la distribución de ruido, le fue asignada arbitrariamente una media de cero, podemos asignar un valor al punto en que el criterio corta la distribución de ruido y define las tasas de Rechazos y Falsas Alarmas obtenidas por el participante. Conociendo el área de la distribución de Ruido que cae bajo el criterio, (el complemento de la tasa de Falsas Alarmas, o bien, la Tasa de Rechazos correctos), y sabiendo que la distribución tiene una desviación estándar de 1, podemos convertir el valor de la tasa (que corresponde a la probabilidad de cometer un rechazo correcto, de acuerdo al área bajo la curva) en Puntajes Z y conocer la localización del criterio.\\

 El parámetro k, por lo general, va estar representado por un número natural (un número positivo), que indica en términos de Puntajes Z  la posición del criterio sobre el eje de decisión, relativo a la distribución de ruido con media cero. El criterio sólo tiene valores positivos, porque normalmente se espera que la tasa de falsas alarmas nunca tenga un valor mayor a 0.5 (las consecuencias de una tasa de Falsas Alarmas tan alta, se expondrán con más claridad en el apartado correspondiente a la d’. \\


\item Sesgo - $\beta$

El parámetro más comúnmente utilizado en la literatura para evaluar el sesgo de los participantes en estudios donde se aplica el modelo de detección de señales al análisis de tareas experimentales, es Beta ($\beta$). Se define como una razón entre la probabilidad con que la evidencia  \\

$\beta = \frac{p(Signal)}{p(Noise)}$

si $\beta<1$ o C<0, sabemos se trata de un sesgo liberal y si $\beta>1$, C>0, hablamos de un sesgo conservador.\\

\item Sesgo - $C$


\end{itemize}   %Terminan los parametros



%----------------------------------------------------------------

\subsection{Tareas de detección}

\begin{itemize}
\item Tareas de detección binaria 

En el laboratorio, la  TDS se estudia a partir  de tareas de detección donde se expone a un  sujeto  a  N  número  de  ensayos,  (comprendidos  por  n  ensayos con  sólo  ruido  y  n  ensayos donde  el  ruido  viene  acompañado  de  la  señal)  ante  los  que  se  le  pide  al  participante  que responda eligiendo una de dos opciones: Sí está la señal o No está la señal. En estos escenarios controlados,  el  experimentador  decide  la  proporción  de  ensayos  con  y  sin  señal  que  se presentarán, así como la matriz de pagos que definirán la utilidad de sus aciertos y errores. \\


\item Tareas con escala de confianza

Un segundo procedimiento comúnmente utilizado en tareas de detección implica añadir un paso adicional a la tarea de los participantes. 

\parencite{McNicol}

\item Tarea con elección forzada entre dos alternativas.



\end{itemize}














\section{Panorama general del estudio de Memoria de Reconocimiento}

\subsection{Teoría del Umbral}

Lorem ipsum dolor sit amet, consectetur adipiscing elit. Aliquam ultricies lacinia euismod. Nam tempus risus in dolor rhoncus in interdum enim tincidunt. Donec vel nunc neque. In condimentum ullamcorper quam non consequat. Fusce sagittis tempor feugiat. Fusce magna erat, molestie eu convallis ut, tempus sed arcu. Quisque molestie, ante a tincidunt ullamcorper, sapien enim dignissim lacus, in semper nibh erat lobortis purus. Integer dapibus ligula ac risus convallis pellentesque.

%-----------------------------------
%	SUBSECTION 1
%-----------------------------------
\subsection{Teoría del Procesamiento Dual}

La Teoría del Procesamiento Dual (TPD; o PDT por sus siglas en inglés) sostiene que existen dos procesos fundamentales involucrados en todo juicio de reconocimiento: la recolección y el análisis de familiaridad. El primero corresponde a la extracción de rasgos y detalles específicos del estímulo a evaluar, (i.e. el estímulo que queremos determinar si se ha visto antes, o no), un proceso que requiere cierto tiempo; el segundo, es entendido como un fenómeno más o menos automático y casi instantáneo donde el sistema identifica el estímulo como 'familiar' y decide que lo ha reconocido de alguna experiencia previa. 

%-----------------------------------
%	SUBSECTION 2
%-----------------------------------

\section{Teoría de Detección de Señales en Memoria de Reconocimiento}
Aplicar la TDS al estudio de la Memoria de Reconocimiento implica asumir que existe tal cosa como una 'fuerza de memoria' ('memory strength', en inglés) que refleja el grado en que un estímulo cualquiera es percibido como 'familiar' para el sistema que busca emitir un juicio de detección. La fuerza de memoria evocada por cada estímulo se compara con un criterio de elección para que el sistema pueda decidir si lo reconoce, o no, como 'elemento antes visto'.\\ 

\begin{figure}[th]
\centering
\includegraphics[width=0.60\textwidth]{Figures/RM_SDT_1} 
\decoRule
\caption[SDT en Memoria de Reconocimiento]{Modelo de Detección de Señales aplicado al estudio de Memoria de Reconocimiento}
\label{fig:RM_SDT_1}
\end{figure}

La Figura~\ref{fig:RM_SDT_1} ilustra la forma en que los supuestos de la TDS, expuestos a detalle en el Capítulo 1, se aplican al estudio de la Memoria de Reconocimiento. Las idea central se mantiene en escencia y simplemente sustituimos algunos conceptos: al tratarse de una tarea de reconocimiento, decimos que la 'señal' a detectar es cualquier estímulo antes visto (i.e. 'estímulos viejos', como suelen identificarse en la literatura); el 'ruido' son los estímulos nuevos, que podrían -o no- confundirse con los primeros; el 'eje de decisión' a lo largo del cual se sitúan las dos distribuciones de ruido y señal, se convierte en un 'eje de familiaridad' que va a representar distintos grados de lo que parece ser una 'fuerza de memoria'. También se mantienen las ideas centrales propuestas por el modelo: los estímulos ya antes vistos tendrán valores más altos de 'familiaridad' que aquellos nunca antes vistos, admitiendo sin embargo la posibilidad de que éstos últimos puedan llegar a confundirse con los estímulos viejos, por ejemplo, si comparten algun rasgo en particular; una vez más, dicha variabilidad en la presentación y lectura de los estímulos a evaluar se refleja en la idea de que existen distintos rangos de familiaridad que pueden ser producidos por los estímulos viejos o conocidos, con cierta distribución de probabilidad. Y de la misma forma, la emisión de un juicio se entiende como resultado de comparar la 'familiaridad' de cada estímulo a evaluar con un criterio de elección particular, donde sólo si éste se rebasa, se juzga el estímulo como 'ya antes visto'.\\

Como se mencionó en el Capítulo 1, la TDS en su forma típica define las distribuciones de ruido y señal como distribuciones Gaussianas con varianzas iguales (i.e. con una misma desviación estándar). A propósito de ello, podemos hablar de una particularidad que tiene la aplicaicón de la TDS a estudios de memoria de reconocimiento, que ha sido constante y consistentemente demostrada con los datos: la distribución de Estímulos Viejos suele mostrar mayor desviación estándar que la distribución de Ruido, justo como se muestra en la Figura~\ref{fig:RM_SDT_2}.\\


\begin{figure}[th]
\centering
\includegraphics[width=0.60\textwidth]{Figures/RM_SDT_2} 
\decoRule
\caption[SDT en Memoria de Reconocimiento (Varianzas Desiguales)]{Modelo de Detección de Señales con varianzas desiguales aplicado al estudio de Memoria de Reconocimiento}
\label{fig:RM_SDT_2}
\end{figure}


\subsection{Implicaciones y conflictos}

Me permitiré, por un momento, salirme del molde académico y hablar un poco de todo el proceso que hubo detrás de la realización del presente trabajo. Cuando comencé a revisar literatura para decidir cuál podría ser mi proyecto de tesis, descubrí la maravilla de la Teoría de Detección de Señales. 

Sin embargo, pese a la flexibilidad de la que dispone la TDS para aplicarse a, aparentemente, cualquier tarea de decisión binaria, donde se admita el papel de la incertidumbre en la emisión de un juicio de detección, su aplicación al campo de la Memoria de Reconocimiento no ha estado excenta de críticas.

Conceptualmente, es fácil pensar en una tarea de reconocimiento en términos del modelo de detección de señales. Sin embargo, cuando se revisan las implicaciones teóricas que conllevaría aceptar que la memoria de Reconocimiento funciona como un procedimiento cualquiera de detección de señales, no son del todo claras. 

\section{El Efecto Espejo}

“is usually interpreted in terms of (unequal variance) signal detection theory (SD) in which case it implies that the order of the underlying old item distributions mirrors the order of the new item distributions” (DeCarlo, L.,  2007)\\
Teoría de Atención /Verosimilitud: Un modelo de marcaje de rasgos, determinado por un muestreo diferencial dada la condición (H-frequency, L-frequency)\\
Teoría de Atención / Verosimilitud; demasiado complicada, sus supuestos no son necesarios (Decarlo, 2007; Hintzman, 1994; Murdock, 1998) Intercambio de papers Hintzman-Glanzer\\
‘The mixture model’ (DeCarlo, 2007) – Extensión de la SDT, una extensión mezclada.\\
Between vs Within condition discussion (Listas separadas o mezcladas)\\
Between condition: Problemas (1) No se puede descartar la posibilidad de que el criterio de respuesta difiera a lo largo de las condiciones. Y (2) las distribuciones subyacentes no necesariamente están escaladas de la misma forma a lo largo de las dos condiciones.\\
“one cannot compare the values of d’ across the two conditions without further assuming that the variance of the reference distributions (LN and HN) are the same, which does not appear to be the case. (DeCarlo,2007)  

%----------------------------------------------------------------------------------------
%	SECTION 1
%----------------------------------------------------------------------------------------

\subsection{Evidencia recolectada}

Teóricamente, este procedimiento nos permitiría inferir no uno, sino seis sub-criterios de elección que estarían permeando a partir de cuánta evidencia el participante se siente más, o menos seguro, de si los estímulos evaluados contienen la señal o sólo ruido. De encontrarse evidencia del Efecto Espejo en los experimentos realizados, se esperaría encontrar el mismo patrón en el promedio de los puntajes de confianza asignados a cada una de las cuatro categorías de estímulos (formadas por la combinación Condición (A o B) x Tipo de ensayo (S o R)) reportados en la literatura en Memoria de Reconocimiento \parencite{Glanzer1990, Glanzer1993}. \\

La Figura~\ref{fig:Ejem_Crit} 

\begin{figure}[th]
\centering
\includegraphics[width=0.55\textwidth]{Figures/Ejem_Criterios}
%\decoRule
\caption[Representación gráfica de los sub-Criterios para la emisión de puntajes de confianza ]{}
\label{fig:Ejem_Crit}
\end{figure}



\subsection{Relevancia e implicaciones}
A primera vista el patrón de respuestas identificado como Efecto Espejo podría parecer trivial: Si sabemos que lo que distingue a las condiciones de estímulos 

%-----------------------------------
%	SUBSECTION 1
%-----------------------------------
\subsection{Algunos modelos desarrollados para dar cuenta del Efecto Espejo}

\begin{itemize} 
\item
\item
\end{itemize}


 
% Chapter Template

\chapter{Método} % Main chapter title

\label{Cap_Exp} % Change X to a consecutive number; for referencing this chapter elsewhere, use \ref{ChapterX}

%----------------------------------------------------------------------------------------
%	SECTION 1
%----------------------------------------------------------------------------------------
\section{Planteamiento general}

%Se propone buscar evidencia del Efecto Espejo en una tarea fuera de Memoria de reconocimiento.
Al tratarse de un fenómeno exclusivamente reportado, estudiado e interpretado dentro de la literatura en Memoria de Reconocimiento, existe una tendencia a explicar el Efecto Espejo -las diferencias en la ejecución de los participantes entre las clases de estímulos contenidos en el experimento- apelando a posibles discrepancias en el procesamiento superior de los mismos durante la fase de estudio (por ejemplo, la atención, la recolección de rasgos para su futuro reconocimiento, etc.). El interés principal de la investigación aquí reportada fue buscar evidencia de los patrones de respuesta reportados en estudios de Memoria de Reconocimiento en una tarea de detección ajena a dicha área.\\

%Encontrar evidencia del Efecto Espejo en otra área, podría sugerir que el Efecto Espejo es resultado del uso de SDT para comparar dos condiciones de discriminabilidad y no necesariamente de un fenómeno de Memoria per see.
Buscar evidencia del Efecto Espejo fuera del área de Memoria de Reconocimiento permite evaluar la posibilidad de que los patrones de respuesta reportados como parte del mismo sean producto de la aplicación de la TDS como marco para el análisis y comparación de la ejecución de los participantes entre clases de estímulos que difieren en su discriminabilidad y no necesariamente de una discrepancia en su procesamiento durante la fase de estudio.\\

%Se propone una tarea perceptual ya que carece de una fase de preparacion. Se trabaja con ilusiones opticas, dado que la literatura en ellas permite anticiparnos a la d' y proponer dos niveles de dificultad.
Se decidió trabajar con una tarea de detección perceptual por dos grandes razones. La primera razón es que una tarea de este tipo implica un procedimiento más sencillo: los participantes deciden si la señal está o no presente en los estímulos mostrados sin haber interactuado con estos en una etapa previa. La señal a detectar se percibe en el estímulo, no se reconoce de la experiencia previa con el mismo. La segunda razón es que existe un cuerpo de literatura lo suficientemente amplio como para permitir la construcción de dos niveles de dificultad dentro de la tarea, específicamente, se revisó y trabajó con la literatura que aborda el fenómeno de las ilusiones ópticas.\\

\subsection{Objetivo}

%Buscar evidencia del efecto espejo en una tarea de detección perceptual.
Determinar si los patrones de respuesta identificados como parte del Efecto Espejo en Memoria de Reconocimiento aparecen también al coparar el desempeño de los participantes entre distintos niveles de discriminabilidad en una tarea de detección perceptual.\\

\section{Construcción de los Experimentos}

%5.	Jaeger, T., & Pollack, R. (1977). Effect of contrast level and temporal order on the Ebbinghaus circles illusion. Perception & Psychophysics. Vol 21 (1), 83-97.
%9.	Pressey, A. (1977). Measuring the Titchener circles and Delboeuf illusions with the method of adjustment. Bulletin of Psychonomic Society. Vol 10 (2), 118-120.
%Coren, S., & Girgus, J. (1978). Seeing is deceiving: The psychology of visual illusions. Hillsdale, NJ: Erlbaum.
%Titchener E B, 1901 Experimental Psychology: A Manual of Laboratory Practice volume I (London: MacMillan) (1901, page 169, figure 3)
%Robinson J O, 1972 The Psychology of Visual Illusion (London: Hutchinson Education)
%Rabin, J., & Adams, A. J. (1993). Size induction transcends the cardinal directions of color space. Perception, 22, 841–845.

%La tarea de detección consiste en identificar los ensayos donde el tamaño de dos círculos en pantalla es igual. Esta tarea tiene dos variaciones: En el Experimento 1, uno de los círculos a comparar es el círculo central de una ilusión de Ebbinghaus; En el Experimento 2, los dos círculos lo son. 
Se diseñó una tarea perceptual donde los participantes tenían que comparar el tamaño de dos círculos mostrados en pantalla y emitir una respuesta que señalara si los círculos tenían el mismo diámetro (señal) o no (ruido). Esta tarea se presentó en dos variaciones: En un primer caso, sólo uno de los círculos a comparar constituía el círculo central de una figura de Ebbinghaus (Experimento 1); En el segundo, ambos círculos aparecían como el componente central de una figura de Ebbinghaus distinta (Experimento 2).\\ 

%Descripcion de la Ilusión de Ebbinghausyhhhhhhhhhhvcf , la composición de la figura y cómo se explica.
La figura de Ebbinghaus -también conocida como Círculos de Titchner- está intrínsecamente relacionada con una ilusión óptica donde la percepción del tamaño de un elemento central es alterada por el contraste que tiene con elementos circundantes (la ilusión de Ebbinghaus). En la Figura~\ref{fig:Ebbinghaus} se presenta un par de ejemplos prototípicos de figuras con la ilusión de Ebbinghaus que demuestran las dos direcciones en que esta se puede presentar: la subestimación del tamaño de un estímulo (el círculo central) al estar rodeado por éstímulos más grandes (efecto de subestimación; figura izquierda), y la sobrestimación del tamaño de un estímulo rodeado por elementos de menor tamaño (efecto de sobrestimación; figura derecha); el diámetro de los círculos centrales en ambas figuras es el mismo. Esta ilusión perceptual suele explicarse como el reflejo de una tendencia en nuestro sistema a computar el tamaño de los objetos en función a su contraste con elementos similares en su entorno \parencite{Coren1971, Coren1974, Fockert2007}.\\

\begin{figure}[th]
\centering
\includegraphics[width=0.55\textwidth]{Figures/Ebbinghaus} 
\decoRule
\caption[Ilusión de Ebbinghaus: Ejemplos]{Ejemplos prototípicos de la Ilusión de Ebbinghaus. Los círculos centrales de las dos figuras mostradas tienen el mismo diámetro, sin embargo, el círculo central de la figura izquierda tiende a percibirse como más pequeño (efecto de subestimación) y el círculo central de la figura derecha suele percibirse como más grande (efecto de sobrestimación) por el contraste que ambos círculos guardan con los círculos que les rodean.}
\label{fig:Ebbinghaus}
\end{figure}

%Factores o variables que han demostrado influir en la intensidad de la ilusión.
La intensidad de la ilusión de Ebbinghaus -qué tanto se aleja el tamaño estimado del tamaño real del círculo central- suele definirse como una función de las siguientes variables \parencite{Massaro1971, Girgus1972, Roberts2005}: 

\begin{itemize}
\item El tamaño de los círculos externos.
\item La distancia entre el círculo central y el halo de círculos externos.
\item El número de círculos externos.
\end{itemize}

%Descripción del procedimiento empleado por Massaro y Anderson para probar la influencia del numero de circulos externos y el tamaño de los mismos en la Ilusión de Ebbinghaus.
La Figura~\ref{fig:Ebb_Var} presenta los resultados obtenidos en un experimento donde se evaluó el efecto de dos de las variables antes mencionadas en la intensidad de la ilusión de Ebbinghaus \parencite{Massaro1971}. En dicho estudio, se construyeron 30 figuras de Ebbinghaus con un diseño factorial de 2x3x5: dos tamaños de círculo central (13 y 17 mm), tres niveles de 'número de círculos externos' (dos, cuatro y seis) y cinco tamaños diferentes de círculos externos (-8, -4, 0, 4 y 8 mm de diferencia respecto del diámetro del círculo central); la distancia entre el círculo central y los círculos externos se mantuvo constante para todas las figuras. La tarea de los participantes era elegir dentro de un set de 27 círculos diferentes (con diámetros de 8.5 a 21.5 mm en saltos de 0.5 mm), el círculo cuyo diámetro fuera más cercano al círculo central de la figura de Ebbinghaus presentada en cada ensayo. La Figura muestra el diámetro promedio elegido por los participantes como más cercano a los círculos centrales de las figuras de Ebbinghaus construidas (se promediaron los datos obtenidos para los dos tamaños de círculo central usados), a lo largo de los distintos niveles de número de círculos externos (en el eje de las abscisas) y por cada valor de diámetro de los círculos externos (en diferentes líneas). El efecto de las manipulaciones hechas parece claro: 1) Entre más círculos externos se incluyen en las figuras de Ebbinghaus, mayor es la intensidad de la ilusión, como señala la tendencia de las lineas a mostrar distancias cada vez mayores entre el promedio de los valores reales de círculo interno y la estimación hecha por los participantes; 2) La intensidad de la ilusión es mayor mientras mayor sea la diferencia entre el tamaño del círculo central y los círculos externos, como sugiere la distancia vertical entre cada una de las líneas, cuya dirección permite distinguir con claridad entre los efectos de sobreestimación y subestimación.\\

\begin{figure}[th]
\centering
\includegraphics[width=0.55\textwidth]{Figures/Ebb_Variables} 
\decoRule
%\caption[Efecto del Numero y Tamaño de los círculos externos sobre la intensidad de la Ilusión de Ebbinghaus]{Se muestra el efecto que tienen el número de círculos externos incluídos en la ilusión de Ebbinghaus (eje x) sobre los fallos en la estimación del tamaño del círculo central (eje Y); se muestran con líneas diferentes los resultados obtenidos por ilusiones de Ebbinghaus donde los círculos externos diferían en tamaño del círculo central con los valores especificados. La figura fue extraída de la investigación conducida por \parencite{Massaro1971}; Figura 2}
\caption[Efecto del Número y Tamaño de los círculos externos en la intensidad de la Ilusión de Ebbinghaus]{Se presentan los resultados obtenidos en un experimento donde se manipuló el número y tamaño de los círculos externos de figuras de Ebbinghaus para evaluar su efecto sobre la intensidad de la ilusión perceptual evocada. La gráfica muestra el estimado del diámetro del círculo central de las figuras presentadas (promediando los datos obtenidos por todos los participantes y por los dos tamaños de círculo central usados: 13 y 17 mm), a través de los distintos niveles de número de círculos externos evaluados (eje de las abscisas) y por cada uno de los cinco tamaños de círculos externos utilizados (mostrados en líneas separadas). La tendencia de las líneas a alejarse del estimado promedio en ausencia de círculos externos -sin ilusión inducida- sugiere que a mayor número de círculos externos, mayor es la intensidad de la ilusión. La distancia vertical entre las líneas dibujadas parece indicar que la intensidad de la ilusión aumenta mientras mayor sea la diferencia entre el diámetro del círculo central y el de los círculos externos. \parencite{Massaro1971}.}
\label{fig:Ebb_Var}
\end{figure}

%Las condiciones de dificultad estarán determinadas por el número de círculos externos
Tomando en consideración los hallazgos reportados respecto de la influencia que tienen las variables inmersas en las figuras de Ebbinghaus sobre la ilusión perceptual asociada a las mismas \parencite{Massaro1971}, se decidió construir las dos condiciones de discriminabilidad para nuestra tarea de detección perceptual en búsqueda del Efecto Espejo, manipulando el número de círculos externos. Es decir, para que en la tarea de detección propuesta existiera una condición fácil -la "clase A" con una $d'$ grande- se construyeron figuras de Ebbinghaus con 'pocos' círculos externos (2 o 3 círculos); y para la condición difícil -la "clase B" con una $d'$ menor- se diseñaron figuras con 'muchos' círculos externos (7 u 8 círculos).\\ 

\subsection{Diseño de los Estímulos}

%En el experimento se trabajó con figuras de Ebbinghaus que promovieran efecto de subestimación Y sobrestimación del tamaño. 
En el diseño de las figuras de Ebbinghaus a utilizar en los experimentos propuestos se incluyeron los efectos de sobrestimación y de subestimación. Para ello se manejaron sólo dos tamaños de círculos externos, elegidos arbitrariamente de manera que fueran o más grandes que todos los tamaños de círculo central o más pequeños que la mayoría (5 cm y 1 cm).\\

%No se controló la distancia entre el círculo central y los círculos externos.
En ningún experimento se controló la distancia entre los círculos centrales y el halo de círculos externos. Los círculos externos fueron acomodados de acuerdo a los números de círculos externos considerados como parte de la condición difícil (siete y ocho), distribuyéndolos de manera uniforme y equidistante en torno al círculo central. Estos halos de siete y ocho círculos externos fueron usados como base para la construcción de los estímulos de la condición fácil (con dos o tres círculos externos), eliminando círculos y respetando la ubicación de los restantes. Este procedimiento se realizó para las figuras con efecto de subestimación y sobrestimación. Se procuró que en las figuras con dos círculos externos, estos estuvieran enfrentados en puntos opuestos del círculo central y en las figuras con tres círculos centrales, que rodearan al círculo central en ángulos de ciento veinte grados. La configuración de los halos de círculos externos resultantes permaneció constante para todos los tamaños de círculo central, haciendo que la distancia entre ambos elementos sea distinto para cada valor.\\

%El diámetro de los círculos centrales de las figuras de Ebbinghaus podía ser de 1 a 3 cm, en saltos de 0.5 cm
En ambos experimentos se utilizaron cinco valores para el tamaño de los círculos centrales (de 1 a 3 cm, en saltos de 0.5 cm). Cabe señalar que el diámetro elegido para los círculos externos en las ilusiones de sobrestimación (1 cm) es el mismo que el círculo central más pequeño. Sin embargo, esto no se considera un problema para la presente investigación porque la inclusión de los dos efectos (sobrestimación y subestimación) se hizo con la intención de prevenir la habituación y la fatiga de los participantes a la tarea, proveyendo a la misma de cierto dinamismo. El objetivo principal de la investigación realizada fue comparar el desempeño de los participantes entre las condiciones de discriminabilidad construidas con base en la literatura, y este tipo de figuras 'anómalas' se presentan de igual manera en ambas condiciones.\\

A continuación, se desarrolla con detalle la construcción de los estímulos y la distinción entre los tipos de ensayo (señal y ruido) aplicadas en cada uno de los dos experimentos llevados a cabo. Las especificaciones respecto al procedimiento y los controles realizados se explican con detalle más adelante.\\

\begin{itemize}
\item Experimento 1 : Circulo de referencia aislado vs Figura de Ebbinghaus.

%Los ensayos de la tarea de detección están compuestos por un círculo aislado constante del lado izquierdo, cuyo tamaño debe ser comparado con el círculo central de una figura de Ebbinghaus en el lado derecho de la pantalla.
En el Experimento 1 se mostró en cada ensayo una figura de Ebbinghaus en la mitad derecha de la pantalla, cuyo círculo central debía ser comparado por los participantes con un círculo de referencia aislado y con un diámetro constante de 2 cm localizado en el lado izquierdo de la pantalla. La tarea de los participantes consistía en presionar la tecla 'S' cuando ambos círculos fueran del mismo tamaño (señal) y la tecla 'N'  cundo no (ruido). Los círculos se mostraban a la misma altura de la pantalla, 14 cm a la izquierda y 10 cm a la derecha del centro de la pantalla.\\

%Desarrollo del diseño factorial de 5x2x2 utilizado para construir los estímulos del Experimento 1: 5 circulos centrales x 2 tipos de ilusion x 2 niveles de 'numero de circulos externos'
Las figuras de Ebbinghaus utilizadas en el Experimento 1 se diseñaron de acuerdo a un diseño factorial de 5x2x2, (Ver Figura~\ref{fig:Exp_1}). Se utilizaron cinco tamaños de círculo central que, partiendo del tamaño del círculo de referencia (2 cm; la combinación con la señal), se alejaban del mismo en saltos de 0.5 cm en ambas direcciones (i.e. Círculos más pequeños que la referencia, de 1 y 1.5 cm, y círculos más grandes que la referencia, de 2.5 y 3 cm de diámetro). Por cada uno de estos cinco tamaños de círculo central, se construyeron dos tipos de figuras de Ebbinghaus dependientes del tamaño de los círculos externos: grande (5 cm; efecto de subestimación) y pequeño (1 cm; efecto de sobrestimación). Y por último, por cada una de estas 10 combinaciones, se hicieron cuatro figuras diferentes de acuerdo a los niveles de 'número de círculos externos' propuestos por condición (2 y 3 círculos en la condición fácil; 7 y 8 círculos en la condición difícil). Esto nos deja con un subtotal de 20 figuras diferentes por condición y un total de 40 en todo el experimento.\\
 
%Los estímulos con ruido se presentaron 10 veces durante el experimento. Los estímulos con señal se presentaron con 40 repeticiones. La discrepancia en el número de repeticiones entre ambos tipos de estímulo se hizo para igualar la muesta de ensayos con señal y ensayos con ruido.
El conjunto de figuras de Ebbinghaus resultante contiene una mayor cantidad de estímulos con ruido (32 figuras con ruido; 16 por condición) que con señal (8 figuras con la señal; 4 por condición). Para igualar la cantidad de ensayos con señal y con ruido presentados a los participantes y preveer que la diferencia en su tasa de aparición pudiera sesgar su desempeño hacia la emisión de respuestas negativas, las figuras con señal diseñadas se repitieron un mayor número de veces que las figuras con ruido. Cada una de las 32 figuras de Ebbinghaus con ruido se presentó con diez repeticiones durante el experimento, mientras que las ocho figuras con señal tuvieron cuatro veces más repeticiones (40 en total). De esta forma, el experimento terminó compuesto por 320 ensayos con ruido y 320 ensayos con señal, 160 por cada condición de dificultad.\\

%Las repeticiones de cada figura aparecieron en cinco colores diferentes para prever la fatiga.
Los 640 estímulos contemplados en el Experimento 1 fueron presentados de manera aleatoria. Procurando evitar la fatiga de los participantes, las repeticiones de los estímulos diseñados se mostraron en cinco colores diferentes (Guinda, Anaranjado, Verde, Azul y Púrpura) en cantidades iguales (dos repeticiones de cada color para los estímulos con ruido y ocho para los estímulos con señal).\\

\item Experimento 2 : Figura de Ebbinghaus vs Figura de Ebbinghaus.

%En la tarea de detección, los participantes tenían que comparar el círculo central de dos figuras de Ebbinghaus que aparecían simultáneamente en la pantalla: una con círculos externos grandes (Efecto de subestimación)  y una con círculos externos pequeños (Efecto de sobrestimación).
En el Experimento 2 los participantes tenían que comparar el diámetro del círculo central de dos figuras de Ebbinghaus mostradas simultáneamente en pantalla y, al igual que en el Experimento 1, presionar la tecla 'S' cuando fueran del mismo tamaño (señal) y la tecla 'N' cuando no (ruido). Las parejas construidas estaban compuestas por una figura de Ebbinghaus con efecto de subestimación (con círculos externos grandes, de 5 cm de diámetro) y una figura con efecto de sobrestimación (con círculos externos pequeños, de 1 cm de diámetro). Los círculos centrales aparecían en pantalla a la misma altura, 15 cm a la izquierda y 11 cm a la derecha del centro de la pantalla.\\


%Se formaron cinco parejas con la Señal (cinco pares iguales de tamaño de círculo central) y cinco parejas con el ruido (cuatro cuyos círculos centrales diferían en 0.5 cm y una que difería en 1 cm).
La Figura~\ref{fig:Exp_2} ilustra el diseño de las figuras de Ebbinghaus utilizadas en el Experimento 2. A diferencia del Experimento 1, donde uno de los círculos a comparar era constante, en el Experimento 2 se varió el diámetro de los dos círculos a comparar. Para ello se ello se utilizaron los mismos cinco tamaños de círculo central (de 1 a 3 cm en saltos de 0.5 cm) y por lo tanto, con cinco combinaciones posibles para las Parejas-señal. Así mismo, se formaron cinco Parejas-ruido juntando arbitrariamente valores de círculo central que guardasen una diferencia de 0.5 cm entre sí -1 vs 1.5; 1.5 vs 2; 2 vs 2.5 y 2.5 vs 3 cm- con una quinta pareja con una diferencia de 1 cm entre los valores de círculo central intermedios -1.5 cm vs 2.5 cm-. Por cada una de estas 10 parejas, se crearon cuatro variaciones por condición, de acuerdo con las combinaciones posibles de niveles de 'número de círculos externos' (2 círculos externos a ambos lados, 3 círculos externos a ambos lados, 2 en izquierdo y 3 en derecho, y 3 en izquierdo y 2 en derecho en la condición fácil; 7 círculos externos a ambos lados, 8 círculos externos ambos lados, 7 círculos del lado izquierdo y 8 en el derecho y 8 círculos en el lado izquierdo y 7 en el derecho en la condición difícil). En total, el Experimento 2 estuvo compuesto por 80 parejas diferentes de figuras de Ebbinghaus cuyos círculos centrales debían compararse, 40 con la señal y 40 con el ruido y 20 de cada uno por condición.\\

%Cada pareja diseñada se repitió 8 veces; en 4 colores diferentes y contrabalanceando la ubicación de cada tipo de ilusión. En total, el Experimento 2 estuvo compuesto por 640 ensayos.
Cada una de las 80 parejas diseñadas para el Experimento 2 se presentó 8 veces, en cuatro colores diferentes (púrpura, anaranjado, azul y verde) para prevenir la fatiga de los participantes y contrabalanceando la ubicación de las ilusiones de sobrestimación y subestimación a la derecha o izquierda de la pantalla. Es decir, por cada pareja construida de figuras a comparar se incluyeron ocho ensayos en el experimento: un par de cada uno de los cuatro colores propuestos y dentro de estos, se contrabalanceó la localización de las figuras con efecto de sobrestimación y subestimación. De tal forma que el Experimento 2 estuvo compuesto por un total de 640 ensayos, 320 por cada tipo de ensayo (ruido y señal) y 160 por cada condición.\\ 

\subsection{Materiales}

%Programacion de la tarea
La tarea fue programada y ejecutada en PsychoPy v.12, un paquete de libre acceso diseñado para facilitar la generación de tareas experimentales en psicología y neurociencias con el lenguaje de programación Python.\\

%Detalles sobre la Mac y el espacio utilizado para correr el experimento.
El experimento se corrió en una computadora de escritorio Mac (pantalla de 59.5 x 34 cm), en un cubículo dentro del laboratorio 25 del Edificio D de la Facultad de Psicología de la UNAM. Los participantes se sentaron en una silla fija, situada a 1.10 m de distancia de la pantalla.\\ 


\subsection{Participantes}

%Total de participantes y distribucion entre experimentos.
Un total de cuarenta y un estudiantes de la Facultad de Psicología participaron en uno de los experimentos: veinte en el Experimento 1 y veintiuno en el Experimento 2. Los experimentos se llevaron a cabo de manera simultánea, asignando a los participantes alternadamente a uno de ellos, procurando terminar con una cantidad similar de participantes en cada uno. Los participantes nunca tuvieron conocimiento de que existiera más de un experimento.\\

%Procedencia y generalidades sobre los participantes.
Los participantes eran estudiantes de los primeros cuatro semestres de la licenciatura en Psicología en la Facultad de Psicología de la Universidad Nacional Autónoma de México, con edades entre los 18 y los 21 años. Se incentivó su participación ofreciéndoles a cambio un boleto para la rifa de una tarjeta de regalo con valor de \$300 pesos para utilizarse en la plataforma de su preferencia entre iTunes, Netflix y Amazon. Los participantes tenían visión normal o corregida hacia lo normal.\\ 

%Consentimiento informado.
Previo a su participación en el experimento se solicitó a los participantes que firmaran una carta de consentimiento donde se les informó la duración estimada de la tarea (40 minutos para cualquiera de los experimentos), se reiteraba su participación en una rifa y se les advertía de la fatiga que podrían experimentar durante el procedimiento y que, aunque su participación era voluntaria y podían dimitir en cualquier momento, se les solicitaba encarecidamente que permanecieran hasta el final dado que de lo contrario no se podrían utilizar sus datos.\\

\section{Procedimiento}

%Los experimentos propuestos difieren en el tipo de estímulos a presentar. Sin embargo, la tarea principal y el resto de los detalles del procedimiento permanecen iguales para ambos casos. 
La única diferencia entre los Experimentos 1 y 2 fue el tipo de estímulos presentados a los participantes para su comparación: en un caso se enfrentó el círculo central de una figura de Ebbinghaus contra un círculo de referencia fijo (Experimento 1) y en el otro, se mostraron simultáneamente dos figuras de Ebbinghaus (Experimento 2). Sin embargo, en ambos casos la tarea de detección perceptual es la misma (comparar el diámetro de dos círculos específicamente señalados en la pantalla para determinar si éstos eran -o no- del mismo tamaño), así como el procedimiento y su programación.\\

%La tarea de detección constaba de dos fases: 1) Una tarea de detección binaria (¿Son o no del mismo tamaño?) y 2) Una tarea con escala de confianza (¿Qué tan seguro estás de tu respuesta?)
La tarea de detección planteada se evaluó mediante dos procedimientos diferentes: 1) una pregunta binaria 'Sí/No' y 2) la puntuación de esta respuesta en una escala de confianza con valores del 1 al 3 ('1' siendo 'poco seguro' y '3', 'muy seguro'), que de acuerdo a su correspondencia con la respuesta dada em la fase binaria se transformaría y registraría en términos de una escala más informativa, con valores del 0 al 6, ('0' siendo 'totalmente seguro que no eran iguales' y '6', 'totalmente seguro que sí lo eran').\\

Tras firmar la carta de consentimiento informado, se instaló a cada participante en el espacio asignado para la realización del experimento, donde el monitor mostraba una pantalla de bienvenida que incluía la leyenda "Presiona la barra espaciadora para comenzar con las instrucciones". En el Apendice $WAWAWAWA$ se presentan capturas de pantalla que presentan las instrucciones dadas a los participantes tal y como aparecieron en el experimento. Las instrucciones finalizaban con una pantalla en blanco donde se leía 'Presiona la barra espaciadora para comenzar el experimento', dando a los participantes control sobre el momento en que se sintieran listos para comenzar el experimento.\\

El experimento se extendia a lo largo de 640 ensayos, (uno por cada estímulo construido). A continuación, se detalla la estructura de cada uno de los ensayos:\\

\begin{itemize}
\item Fase 1: Tarea de respuesta binaria 'Sí/No'

Cada ensayo comenzaba con la presentación de los estímulos a comparar, acompañados por un par de leyendas que recordaban a los participantes la pregunta de detección a responder -"¿Los círculos centrales son del mismo tamaño?"- y las teclas que debían presionar para emitir su respuesta -"S = Sí, N = No"-, en la parte superior e inferior de la pantalla, respectivamente. La Figura~\ref{fig:Ejem_YN} presenta un ejemplo de cómo se presentaban los estímulos a comparar durante la fase de respuestas binarias, por cada condición y por cada efecto-ilusión incluído en el Experimento 1.\\

\begin{figure}[th]
\centering
\includegraphics[width=0.45\textwidth]{Figures/Ejemplo_EnsayoYN_1} \includegraphics[width=0.45\textwidth]{Figures/Ejemplo_EnsayoYN_2}
\includegraphics[width=0.45\textwidth]{Figures/Ejemplo_EnsayoYN_4} \includegraphics[width=0.45\textwidth]{Figures/Ejemplo_EnsayoYN_3}
%\decoRule
\caption[Presentación de ensayos con tarea de detección binaria]{Se muestran cuatro ejemplos de la tarea de detección binaria, tal y como se les presentó a los participantes. Los paneles izquierdos presentan estímulos pertenecientes a la condición difícil y los paneles derechos, a la condición fácil. Los paneles superiores muestran ilusiones con efecto de subestimación y los paneles inferiores, con efecto de sobrestimación.}
\label{fig:Ejem_YN}
\end{figure}

Los estímulos permanecían en pantalla durante 1.5 segundos con independencia de si los participantes habían, o no, emitido una respuesta: Si el participante respondía antes, los estímulos se quedaban en pantalla hasta cumplirse el intervalo, tras el cual se pasaba inmediatamente a la segunda fase del ensayo; si el participante no había respondido, los recordatorios permanecían solos en pantalla hasta que se registrara una respuesta. Esta restricción fue incluida para preveer la posibilidad de que los participantes se habituaran a la ilusión al prolongar su observación.\\

\item Fase 2: Escala de Confianza

%En una segunda fase, los participantes tuvieron que oprimir una tecla del 1 al 3 para indicar qué tan seguros estaban de la respuesta recién emitida.
Una vez registrada la respuesta de los participantes a la tarea 'Sí/No', se les mostró una segunda pantalla donde se les solicitaba que indicaran qué tan seguros se sentían de la respuesta binaria recién dada, de acuerdo con la escala presentada en pantalla, oprimiendo la tecla '1', '2' ó '3', (ver Figura~\ref{fig:Ejem_Esc}). \\

Los puntajes registrados por los paricipantes -'1','2' o '3'- se tradujeron y registraron en una escala más grande -con valores del 1 al 6- que separa en direcciones opuestas la confianza que se tiene en los juicios de detección posibles, es decir, distingue entre la confianza que se tiene en que el estímulo evaluado contenga sólo ruido ('1 = Muy seguro de que NO son iguales'; '2 =  Más o menos seguro de que NO son iguales'; '3 = Poco seguro de que NO son iguales') y la confianza de que se trate de un estímulo con señal ('4 = Poco seguro de que son iguales'; '5 = Más o menos seguro de que son iguales'; '6 = Muy seguro de que son iguales'). La conversión de los puntajes emitidos a la escala de seis elementos se realizó en función a la respuesta binaria recién registrada, de la siguiente forma:

\begin{itemize}
\item En los ensayos en que el participante hubiera respondido 'No' a la pregunta de detección binaria '¿Los círculos centrales son del mismo tamaño?', la conversión de los puntajes de confianza asignados sería:\\
	\begin{itemize}
	\item '3 = Estoy muy seguro de mi respuesta', se traduciría en '1'.\\
	\item '2 = Estoy más o menos seguro de mi respuesta', habría conservado el valor '2'.\\
	\item '1 = Estoy poco seguro de mi respuesta', se convertiría en '3'.\\
	\end{itemize}
\\
\item En los ensayos donde el participante hubiera respondido 'Sí' en la primera fase, la transformación de los puntajes asignados habría sido la siguiente:\\
	\begin{itemize}
	\item '1 = Estoy poco seguro de mi respuesta', se transformaría en '4'.\\
	\item '2 = Estoy más o menos seguro de mi respuesta', se registraría como '5'.\\
	\item '3 = Estoy muy seguro de mi respuesta', se convertiría en '6'.\\
	\end{itemize}
\end{itemize}\\

De tal forma que los valores extremos de la escala construida representan una mayor seguridad en las respuestas emitidas en la fase anterior -'Sí' o 'No'- y los valores intermedios, una mayor incertidumbre.\\

Esta forma de obtener la escala de confianza a partir de la yuxtaposición entre las respuestas de los participantes a la tarea binaria y la valoración de su confianza en las mismas, corresponde con el método utilizado en los Experimentos 1 y 2 de una serie de Experimentos donde se reporta evidencia del Efecto Espejo  en Memoria de Reconocimiento \parencite{Glanzer1990}, con el fin de facilitar la tarea de los participantes, evitar su fatiga y garantizar la coherencia entre las respuestas obtenidas en la tarea de bisección y el puntaje asignado en la escala de confianza.\\

\begin{figure}[th]
\centering
\includegraphics[width=0.55\textwidth]{Figures/Ejemplo_Escala}
%\decoRule
\caption[Presentación de la escala de confianza]{La escala de confianza mostrada a los participantes en la segunda fase de la tarea y  a partir de la cual se les solicitaba evaluar e indicar qué tan seguros se sentían del juicio de detección emitido con su respuesta inmediatamente anterior}
\label{fig:Ejem_Esc}
\end{figure}\\

Una vez registrada la segunda respuesta del participante -el puntaje asignado en la escala de confianza- se daba por terminado el ensayo.\\

\end{itemize}

%Entre cada ensayo, se presentó una pantalla intermedia que solicitaba a los participantes presionar la barra espaciadora para indicar que estaban listos para responder al siguiente ensayo.
Entre cada ensayo se incluyó una pantalla 'de descanso' que indicaba a los participanes que debían presionar la barra espaciadora para continuar. Estas pantallas 'en pausa' otorgaban control a los participantes de su avance en el experimento y fueron incluídas para garantizar que estuvieran prestando atención durante la presentación -restringida en el tiempo- de los estímulos a comparar.\\

Al terminar los 640 ensayos, se mostraba una última pantalla donde se presentaba la retroalimentación general del desempeño de cada participante (Total de aciertos y errores cometidos), únicamente con el fin de dar cierto sentido de 'conclusión' a su participación. En ningún momento se informó a los participantes sobre el propósito de la investigación, ni sobre la existencia de los niveles de dificultad entre los cuales se compararía su ejecución.\\ 

%Se registraron tiempos de respuesta por cada respuesta registrada (Respuesta en la tarea Sí/No; Respuesta en la escala de confianza) y  la duración del experimento completo.
Además de las respuestas dadas por los participantes, se registraron también los tiempos de respuesta a lo largo del experimento (latencias). En el Apéndice $APENDICE$ se muestra y desarrolla en detalle cuáles fueron los datos obtenidos y registrados por cada uno de los participantes.\\

\begin{figure}[th]
\centering
\includegraphics[width=0.99\textwidth]{Figures/Estimulos_Experimento1} 
\decoRule
\caption[Diseño de Estimulos en el Experimento 1]{Ilustración del diseño factorial 5x2x2 utilizado para construir las figuras de Ebbinghaus presentadas en el Experimento 1. En cada ensayo los participantes compararon el tamaño de un círculo de referencia constante (2cm de diámetro, ilustrado en el lado izquierdo de la figura) con el círculo central de una figura de Ebbinghaus que podía ser de cinco tamaños diferentes, con círculos externos que inducieran efectos de sobrestimación o subestimación (señalados con signos positivos y negativos, respectivamente) y con dos variaciones del 'número de círculos externo' dependientes de la condición (2 y 3 círculos externos en la condición fácil o 7 y 8 en la condición difícil). Por cada condición, se tienen 16 estímulos con ruido, (repetidos 10 veces cada uno en cinco colores diferentes) y cuatro que contienen la señal (presentados 40 veces cada uno, en cinco colores diferentes), dejándonos con 320 ensayos por condición y un total de 640 ensayos en todo el experimento.}
\label{fig:Exp_1}
\end{figure}

\begin{figure}[th]
\centering
\includegraphics[width=0.9\textwidth]{Figures/Estimulos_Experimento2} 
\decoRule
\caption[Diseño de Estimulos en el Experimento 2]{Diseño de las parejas de figuras de Ebbinghaus mostradas en el Experimento 2, compuestas por una figura con efecto de subestimación y una con efecto de sobrestimación. Con los cinco tamaños distintos de círculo central propuestos se crearon cinco parejas iguales (señales) y cinco parejas arbitrarias desiguales (ruido). Por cada una de las diez parejas se consideró las cuatro combinaciones posibles entre los niveles de círculos externos incluídos en cada condición - 2 vs 2 o 7 vs 7; 3 vs 3 u 8 vs 8; 2 vs 3 o 7 vs 8; 3 vs 2 o 8 vs 7) Cada pareja se repitió ocho veces, contrabalanceando la posición derecha-izquierda de los efectos de sobrestimación y subestimación.}
\label{fig:Exp_2}
\end{figure}
\end{itemize}


 
% Chapter 1

\chapter{Resultados} % Main chapter title
\label{Cap_Res} % For referencing the chapter elsewhere, use \ref{Chapter1} 



\section{Datos recolectados}

%SDatos duros antes del análisis.
Antes de realizar los análisis estadísticos correspondientes para determinar si se encontró -o no- evidencia del Efecto Espejo en nuestros experimentos, los datos recopilados se exploraron de manera exhaustiva, graficando la ejecución de los participantes en relación a diferentes variables.\\ 

%Se destaca la implrtancia de revisar los datos antes de someterlos a analisis estadísticos
Graficar los datos antes de someterlos a un análisis estadístico que guíe la elaboración de conclusiones a partir de los experimentos realizados, constituye una práctica recomendable en tanto que 1) permite evaluar la pertinencia del diseño experimental a la luz de las respuestas registradas y 2) constituye un filtro para descartar la posibilidad de que los participantes estuvieran emitiendo sus respuestas de manera incongruente e inconsistente con las tareas presentadas, permitiendo una mayor confianza en las conclusiones que resulten de su análisis.\\

%Presentación de los controles graficados: Atención, El efecto del paso del tiempo y las variables externas en los estímulos. 
En esta primera sección se presenta un recorrido por las distintas gráficas realizadas con la finalidad de explorar tres grandes fuentes de ruido que, de haber tenido una influencia sobre el desempeño de los participantes, indicarían que estos no respondieron a las tareas planteadas de manera congruente con el propósito de investigación con que se diseñaron los experimentos. Dichas fuentes externas son: 

\begin{enumerate}
\item \textbf{¿Las respuestas se emiten en trenes?} (\textit{Evaluando la atención}).\\

Como un primer filtro se evaluó que los participantes estuvieran poniendo atención a las tareas presentadas al momento de registrar sus respuestas. Para ello se revisaron los patrones con que estas eran emitidas a lo largo de los ensayos.\\

Dado que durante el experimento se presentaron de manera aleatoria estímulos con Ruido o Señal de cualquiera de las dos clases de estímulos diseñadas, se esperaba encontrar una amplia variabilidad en la emisión de respuestas de los participantes (el uso constante de todas las opciones de respuesta facilitadas). De haberse encontrado que los participantes elegían una misma respuesta de manera persistente a lo largo de varios ensayos (trenes de respuesta), habría habido razones para sospechar que no estaban prestando atención a los etímulos que se les presentaron, sino que estuvieron emitiendo sus respuestas en función a aquellas que les precedieron.\\

\item \textbf{¿El Aprendizaje o la Fatiga alteran la ejecución de los participantes?}  (\textit{Evaluando cambios en el desempeño a lo largo del tiempo}).\\

Un segundo filtro estuvo relacionado con la evaluación de los posibles efectos que el paso del tiempo (y el avance entre ensayos) pudo haber tenido sobre el desempeño de los participantes.\\

Por un lado, tomando en cuenta que los experimentos realizados estuvieron conformados por un amplio número de ensayos, a lo largo de los cuales los participantes tuvieron que resolver un par de tareas de detección que podrían haber hecho el procedimiento demasiado demandante, se tenían razones para sospechar que el desempeño de los participantes pudiera verse mermado con el paso de los ensayos, a causa de la Fatiga.\\

Por otro lado, dado que los estímulos incluídos en el experimento estuvieron compuestos por variaciones de una conocida ilusión óptica, se temía que la exposición repetida a la misma redujera su impacto sobre los participantes (por Habituación o Aprendizaje) y su desempeño mejorara considerablemente con el paso del tiempo.\\

\item \textbf{¿Los participantes tienen alguna Preferencia hacia la emisión de ciertas respuestas ante cierto tipo de estímulos?} (\textit{Evaluando el efecto de las variables manipuladas en la construcción de los estímulos}).\\

Un tercer y último filtro se realizó para evaluar el posible impacto que las variables manipuladas en la construcción de los estímulos pudieran haber tenido sobre la emisión de respuestas de los participantes (por ejemplo, el color en que se presentaron las figuras).\\

\end{enumerate}

Idealmente, se esperaba que el desempeño registrado de los participantes no presentara ninguna relación con cualquiera de las fuentes de ruido presentadas anteriormente que no fuera la pertenencia de los estímulos a alguna de las clases diseñadas (clase A: figuras de Ebbinghaus con pocos círculos externos; clase B: figuras de Ebbinghaus con más círculos externos).\\






\subsection{Control 1: ¿Los participantes estaban poniendo atención a la tarea al emitir sus respuestas?}

Los experimentos realizados estuvieron compuestos de 640 ensayos, a lo largo de los cuales los participantes tuvieron que 1) decidir si los estímulos presentados cumplían con la condición que se les solicitó detectar y 2) valorar su certidumbre sobre esta primer respuesta y asignarle un puntaje. Dado lo demandante y extenso que fue el procedimiento, una primer preocupación respecto a la fiabilidad de los datos obtenidos tenía que ver con si los participantes estaban -o no- prestando atención a la tarea al momento de registrar sus respuestas -o bien, que en el proceso se agotaran y dejaran de poner atención-. Para evaluar esta posibilidad, se revisaron las respuestas emitidas ensayo a ensayo para verificar que todas las opciones de respuesta fueran utilizadas y que no se presentaran trenes de respuesta (que pudieran sugerir que la emisión de respuestas fue independiente del contenido de la tarea y dependiente de las respuestas emitidas previamente).\\

\begin{itemize}
\item Emisión de respuestas 'Sí/No' a lo largo del experimento.

Primero se graficaron las respuestas emitidas ensayo a ensayo durante la tarea de detección binaria ('Sí, los círculos son iguales' o 'No, los círculos son diferentes'). El objetivo principal de realizar estas gráficas fue el de detectar trenes de respuesta que fueran lo suficientemente largos como para que, dada la aleatoriedad con que los estímulos fueron presentados por el programa, pudieran indicar una preferencia en el participante a presionar cierta tecla de respuesta en particular, con independencia del estímulo presentado en pantalla para su evaluación.\\

%Participante representativo: Respuestas 'No' por 80 ensayos
La Figura~\ref{fig:Resp_E2_P1} presenta un ejemplo particularmente ilustrativo de este tipo de gráficas y la importancia que tiene revisar los datos antes de incluirlos en el análisis estadístico para la elaboración de conclusiones. En la figura se muestran las respuestas emitidas a la tarea de detección binaria por el Participante 1 del Experimento 2, quien pasó los primeros 80 ensayos del experimento presionando persistentemente la tecla de respuesta 'No'. Dicho tren de respuesta fue considerado lo suficientemente largo como para cuestionar la atención con que el Participante 1 respondió a la tarea, señalando la necesidad de realizar una evaluación más exhaustiva.\\ 

\begin{figure}[th]
\centering
\includegraphics[width=0.60\textwidth]{Figures/Response_Exp2_P1} 
%\decoRule
\caption[Respuesta emitida por ensayo: Ejemplo de participante sesgado]{Se muestran las respuestas "Sí/No" emitidas en cada uno de los 640 ensayos del Experimento 1 por el Participante 1. La gráfica superior muestra los primeros 320 ensayos y la gráfica inferior, los 320 restantes. En el panel superior se aprecia con claridad un tren de respuesta que se extiende a lo largo de 80 ensayos, durante los cuales el Participante 1 sólo utilizó una de las opciones de respuesta posibles.}
\label{fig:Resp_E2_P1}
\end{figure}

%Las gráficas correspondientes al resto de los participantes en los Experimentos 1 y 2, se muestran en las Figuras~\ref{fig:Response_P1} y \ref{fig:Response_E2}, respectivamente.\\

\item Correlación entre las respuestas 'Sí/No' emitidas y el tipo de estímulo presentado en cada ensayo.

A continuación, sobre estas mismas gráficas se añadieron indicadores que señalaran las características específicas de los estímulos presentados en cada ensayo -es decir, si se trataba de una Señal o Ruido y si se trataba de un estímulo Fácil o Difícil-.\\ 

Retomando el caso del Participante 1 del Experimento 2 presentado en la Figura~\ref{fig:Resp_E1_P1}, la Figura~\ref{fig:BiasResp_E1_P1} explora la posibilidad de que las respuestas registradas de hecho estuvieran relacionadas con los estímulos presentados en pantalla, (por ejemplo, en el improbable -pero posible- caso de que se le hubiera presentado una gran proporción de estímulos con Ruido durante los primeros 80 ensayos del experimento). De acuerdo con esta gráfica, parece ser que el tren de 80 respuestas 'No' consecutivas se mantuvo con independencia del tipo de estímulo presentado. Con base en ello, se decidió eliminar a dicho participante de la muestra a analizar, pues se encontró evidencia suficiente para dudar de la atención con que estuvo respondiendo a la tarea.\\

\begin{figure}[th]
\centering
\includegraphics[width=0.60\textwidth]{Figures/BiasResp_Exp2_P1} 
%\decoRule
\caption[Respuesta por Tipo de Estimulo; ejemplo de participante sesgado]{Se muestran las respuestas registradas por el Participante 1 en cada uno de los 640 ensayos del Experimento 2, indicando con diferentes colores el tipo de estímulo que se le mostraba en cada ocasión. En el panel superior se señala con colores violeta y azul si el estímulo presentado pertenecía a la categoría Difícil o Fácil, respectivamente. En el panel inferior se indica si se trataba de una señal o ruido, señalándolos con los colores verde y rojo, respectivamente.}
\label{fig:BiasResp_E1_P1}
\end{figure}

%Las Figuras~\ref{fig:BiasResp_E1} y () muestran las gráficas correspondientes al resto de los participantes en el Experimento 1 y 2, respectivamente.\\

\item Asignación de puntajes de confianza, ('1','2' y '3').

En la segunda fase del problema de detección presentado en cada ensayo, los participantes tenían que elegir entre tres opciones de respuesta (teclas 1, 2 y 3) para señalar qué tanta confianza tenían sobre la respuesta recién emitida ("poco seguro", "más o menos seguro" o "muy seguro", respectivamente). Las respuestas fueron registradas por el programa como parte de una escala mayor (con valores del 1 al 6), que permite diferenciar entre la confianza de haber rechazado correctamente un estímulo con Ruido (e.g. "1, estoy muy seguro de que los círculos eran diferentes") y la confianza de haber identificado correctamente un estímulo con Señal (e.g "6, estoy muy seguro de que los círculos son iguales"), asignando los valores intermedios (3 y 4) a los puntajes asignados para señalar una confianza baja en la respuesta emitida (e.g. "3, poco seguro de que los círculos eran diferentes" y "4, poco seguro de que los círculos eran iguales").\\

Tal y como se hizo con la tarea de detección binaria, se graficaron los puntajes de confianza registrados por los participantes en cada uno de los 640 ensayos que conformaron cada experimento. A manera de ejemplo, la Figura~\ref{fig:Rating_E2_P4} muestra los puntajes emitidos por el Participante 15 del Experimento 2 a lo largo de la tarea. Este participante, de acuerdo con lo que se esperaría encontrar, utiliza hace uso de las tres teclas de respuesta, que son registradas por el programa de acuerdo a su correspondencia con la respuesta dada a la tarea de detección binaria.\\ 
 
\begin{figure}[th]
\centering
\includegraphics[width=0.60\textwidth]{Figures/Rating_Exp2_P15} 
%\decoRule
\caption[Asignacion Puntaje de confianza: Ejemplo]{Se muestran los puntajes de confianza asignados por el Participante 15 del Experimento 2 a las respuestas emitidas en la tarea binaria, durante cada uno de los 640 ensayos que conforman el experimento. El panel superior muestra los puntajes asignados en los primeros 320 ensayos del experimento; en el panel inferior, los 320 restantes.}
\label{fig:Rating_E2_P4}
\end{figure}

%El registro ensayo a ensayo de los puntajes de confianza asignados por el resto de los participantes en los Experimentos 1 y 2 se muestran en las Figuras~\ref{fig:Rating_E1} y \ref{fig:Rating_E2}, respectivamente.\\

\end{itemize}




\subsection{Control 2: ¿La duración del experimento tuvo un impacto en la ejecución de los participantes?}

La Fatiga causada por la extensión del experimento y la posible Habituación a la ilusión óptica, fueron dos de las fuentes de ruido externo que más preocupación causaron en términos de garantizar la fiabilidad de los datos obtenidos. Para preveer su influencia sobre el desempeño de los participantes, se incluyeron un par de controles en el diseño experimental, tales como una pantalla de espera entre ensayos que daba oportunidad a los participntes de descansar e indicar cuando se sintieran listos para atender un nuevo par de estímulos, o la restricción del tiempo durante el que se mostraban los estímulos a comparar. De cualquier forma, una vez obtenidos los datos correspondientes a la ejecución de los participantes, se realizó un segundo conjunto de gráficas con el objetivo de verificar que el efecto de Fatiga o Habituación no se hayan presentado durante la tarea, como se vería reflejado por un decaimiento o mejora del desempeño de los participantes con el paso del tiempo.\\ 

\begin{itemize}
\item Aciertos y errores a lo largo del tiempo

Primero, se construyeron gráficas que permitieron observar si las respuestas emitidas por los participantes en cada ensayo, fueron registradas como aciertos o errores. Esto se realizó con el objetivo de explorar visualmente la existencia de cambios significativos en el desempeño de los participantes a lo largo del tiempo, conforme adquirían más experiencia en la tarea.\\

La Figura~\ref{fig:Success_E2_P14} muestra como ejemplo, el desempeño del Participante 14 a lo largo del Experimento 2. La gráfica superior presenta el registro acumulativo de los aciertos y errores cometidos a lo largo del experimento. En los paneles inferiores se muestra -ensayo a ensayo- si las respuestas registradas fueron clasificadas como acierto o error, dependiendo su correspondencia con el tipo de ensayo evaluado.\\

\begin{figure}[th]
\centering
\includegraphics[width=0.60\textwidth]{Figures/Success_Exp2_P14}
%\decoRule
\caption[Aciertos y errores a lo largo del tiempo: Participante ejemplar]{Aciertos y errores cometidos por el Participante 14 del Experimento 2 a lo largo de la tarea. En el panel superior se muestra el registro acumulativo de estos a lo largo del experimento, en tanto que los paneles inferiores muestran la clasificación ensayo a ensayo de las respuestas emitidas por el participante como Acierto o Error, (siendo que el panel intermedio presenta la primera mitad del experimento y el panel inferior, el resto).}
\label{fig:Success_E2_P14}
\end{figure}

%Las Figuras~\ref{fig:Success_E1} y \ref{fig:Success_E2} muestran los aciertos y errores cometidos a lo largo de los experimentos por el resto de los participantes en los Experimentos 1 y 2, respectivamente.\\


\item Resultados a lo largo del tiempo

Además de identificar los aciertos y errores cometidos por los participantes a lo largo de los experimentos, se realizaron también gráficas que señalaran qué tipo de acierto o error habían cometido los participantes en cada ensayo. Nuevamente, se presentan los datos obtenidos de la ejecución del Participante 14 del Experimento 2 en la Figura~\ref{fig:Outcome_E2_P14}, en un par de gráficos donde se señalan los resultados obtenidos a lo largo de la tarea: El panel superior muestra el registro acumulativo de los cuatro posibles resultados a obtener a lo largo del experimento y en el panel inferior se presentan los resultados obtenido en cada ensayo.\\ 

\begin{figure}[th]
\centering
\includegraphics[width=0.60\textwidth]{Figures/Outcome_Exp2_P14}
%\decoRule
\caption[Resultado obtenido a lo largo del tiempo: Ejemplo]{Resultados obtenido por el Participante 14 del Experimento 2 a lo largo del Experimento. En el panel superior se muestra la frecuencia acumulativa de los  Hits, Falsas Alarmas, Rechazos y Omisiones obtenidos durante el experimento, mientras que en el panel inferior se presenta el resultado obtenido en cada uno de los 640 ensayos.}
\label{fig:Outcome_E2_P14}
\end{figure}

%El resto de los participantes en los experimentos 1 y 2 aparecen en las Figuras~\ref{fig:Outcome_E1} y \ref{fig:Outcome_E2}, respectivamente.\\

\end{itemize}











\subsection{Control 3: ¿Las variables mezcladas para construir los estímulos están afectando el desempeño de los participantes?}

Como se recordará del Capítulo 3, donde se detalla la construcción de los estímulos utilizados en cada uno de los experimentos, la variable deliberadamente manipulada para construir las dos condiciones de dificultad entre las cuales se compara el desempeño de los participantes fue el número de círculos externos en las figuras de Ebbinghaus. Sin embargo, es importante descartar la posibilidad de que el desempeño de los participantes variara en relación a otro tipo de características de los estímulos presentados, como podrían ser los distintos colores en que fueron mostrados cada uno de los estímulos diseñado a lo largo de sus 8 o 10 presentaciones, dependiendo el experimento.\\

\begin{itemize}
\item El efecto del Color sobre la intensidad de la ilusión.

En términos del tipo de influencia que pueden tener las características de los estímulos construidos sobre la ejecución de los participantes, una primer posibilidad es que éstas pudieran estar teniendo un impacto en la intensidad de la ilusión perceptual utilizada. Por ejemplo, sería razonable tener dudas respecto de si los distintos colores en que aparecen cada uno de los estímulos diseñados podría estar alterando la manera en que los participantes responden a los mismos.\\ 

\begin{figure}[th]
\centering
\includegraphics[width=0.60\textwidth]{Figures/Color_Exp1_P14}
%\decoRule
\caption[Hits y Falsas Alarmas por Color; Ejemplo]{Se muestra el desempeño del Participante 14 del Experimento 1 en relación al color de los estímulos. En el panel izquierdo se muestra la relación entre el número de Hits obtenidos y el color de los estímulos, mientras que en el panel derecho se muestra la misma relación para las Falsas alarmas.}
\label{fig:Color_E2_P14}
\end{figure}

La Figura~\ref{fig:Color_E2_P14} muestra la relación entre los Hits y las Falsas alarmas cometidas por el Participante 14 a lo largo del Experimento 2, en relación con el color de las figuras. De acuerdo con las gráficas desplegadas, no parece ser que el color de las figuras tenga un efecto sobre el desempeño de este participante. Se muestran únicamente los Hits y las Falsas Alarmas, pensando que arrojan información suficiente respecto de los aciertos y errores cometidos a lo largo del experimento, en tanto que mantienen una relación complementaria con el número de Omisiones y Rechazos correctos, proporcionando de manera implícita la frecuencia absoluta de estos últimos.\\

%Las gráficas correspondientes a la relación entre las frecuencias absolutas de Hits y Falsas Alarmas y el color de los estímulos, para el resto de los participantes en los Experimentos 1 y 2 se encuentran en la Figura~\ref{fig:Color_E1} y \ref{fig:Color_E2}, respectivamente. 

\item El Efecto del Color sobre la respuesta de los participantes.

Una segunda forma en que el color de los estímulos podría estar alterando el desempeño de los participantes, sería si estos mostraran alguna preferencia a responder de cierta forma ante uno o más de los colores utilizados, con independencia del resto de las características presentes en los estímulos.\\

La Figura~\ref{fig:BiasCol_E1_P13} muestra la proporción de Respuestas 'Sí'/'No' emitidas por el Participante 13 del Experimento 1 para cada uno de los diferentes colores en que fueron presentados los estímulos. Como se puede ver en la figura, parece ser que en general la proporción se mantiene a lo largo de los distintos colores para este participante, por lo que no parece ser que el color esté influyendo en su manera de responder a los estímulos construidos.

\centering
\includegraphics[width=0.40\textwidth]{Figures/BiasColor_Exp1_P13}
%\decoRule
\caption[Proporción de Respuestas 'Sí'/'No' por color; Ejemplo]{Se muestra el desempeño del Participante 14 del Experimento 1 en relación al color de los estímulos. En el panel izquierdo se muestra la relación entre el número de Hits obtenidos y el color de los estímulos, mientras que en el panel dercho se muestra la misma relación para las Falsas alarmas.}
\label{fig:BiasCol_E1_P13}
\end{figure}

%Las Figuras~\ref{fig:BiasCol_E1} y ~\ref{fig:BiasColor_E2} muestran las gráficas correspondientes al resto de los participantes en los Experimentos 1 y 2, respectivamente. 

\end{itemize}


En el Apendice $APENDICE$ se incluyen, a manera de extensión del presente capítulo, todas las gráficas previamente descritas correspondientes al resto de los participantes en los Experimentos 1 y 2.























\section{Análisis estadísticos}


En las tareas contenidas en cada uno de los experimentos realizados (i.e la tarea de detección binaria y la escala de confianza), se encontró evidencia de los patrones de respuesta identificados como parte del Efecto Espejo en al menos tres cuartas partes de los participantes. En el Experimento 1, diecisiete de los veinte participantes mostraron el patrón de respuesta esperado en la tarea de detección binaria y dieciocho en la escala de confianza. A su vez, en el Experimento 2, diecinueve de los veintiun participantes mostraron los patrones asociados con el Efecto Espejo en ambas tareas. De acuerdo a una prueba binomial, todas estas proporciones son estadísticamente significativas contra el azar (p=0.0025 y p=0.0004 para las proporciones reportadas en el Experimento 1, respectivamente; p=0.0002 para el Experimento 2).\\





El análisis de datos está organizado en torno a los siguientes puntos:

\begin{itemize}
\item \textbf{Verificar que las condiciones de dificultad son de hecho diferentes.}
Dado que las condiciones de dificultad fueron diseñadas en función a la literatura revisada en ilusiones ópticas, es preciso corroborar que las manipulaciones implementadas en el diseño de las figuras de Ebbinghaus tuviesen un efecto sobre el desempeño de los participantes. En el marco de la TDS, se esperaría que las discrepancias en la 'dificultad' de la tarea a lo largo de las condiciones construidas, se reflejen en distintos valores del parámetro $d'$ -que cuantifica la discriminabilidad entre señales y ruido como la distancia entre las medias de las distribuciones correspondientes-. En general, se espera encontrar la siguiente relación:
\begin{center}
 D'(A) $>$ D'(B)\\
 \end{center}

 \item \textbf{Comparar las tasas de Hits y Falsas Alarmas entre condiciones.}
Evaluar si, tal y como se reporta en la literatura en memoria de reconocimiento, las diferencias en la ejecución de los participantes entre las dos condiciones de dificultad aparecen 'en dos sentidos', (i.e. en la condición 'fácil' se cometen tanto más aciertos como menos errores, en comparación con la condición con 'difícil'). De acuerdo con la literatura en Memoria de Reconocimiento que aborda el Efecto Espejo, considerando los Hits y las Falsas Alarmas cometidas en cada condición se tomarían como evidencia del mismo los siguientes patrones de respuesta:
\begin{center}
Hits(A) $>$ Hits(B)\\
Falsas Alarmas(B) $>$ Falsas Alarmas(A)\\
\end{center}

\item \textbf{Comparar el puntaje de confianza promedio asignado a cada tipo de ensayo entre las condiciones.}
Además de esperar que las diferencias en el desempeño de los participantes ante dos condiciones de dificultad distintas presentadas de manera ismultánea en la misma tarea se reflejen en los aciertos y errores cometidos en la tarea de detección binaria -de acuerdo con lo reportado en Memoria de Reconocimiento- se espera que estas discrepancias se extiendan a la tarea con la escala de confianza, donde los puntajes asignados por los participantes debieran reflejar sistemáticamente una mayor confianza al responder a los estímulos pertenecientes a la condición fácil que a la condición difícil. Es decir, se espera que:
\begin{center}
Confianza(A) $>$ Confianza(B)\\
\end{center}

Tomando en cuenta que los experimentos fueron programados de manera tal que las respuestas de los participantes respecto a qué tan seguros se sentían sobre la respuesta emitida a la tarea de detección binomial fueran 'traducidas' en puntajes de confianza dentro de una escala más general que distinguía la confianza en sus respuestas negativas ('1', '2' y '3') y la confianza en sus respuestas afirmativas ('4', '5' y '6'), se espera encontrar las siguientes relaciones: \\
\begin{center}
Puntaje(AS) $>$ Puntaje(BS)\\
Puntaje(AN) $<$ Puntaje(BN)\\
\end{center}

\item \textbf{Réplica de controles reportados en la literatura.}
Además de evaluar si las diferencias en la ejecución de los participantes entre las condiciones de dificultad son estadísticamente significativas, en la literatura que reporta evidencia del Efecto Espejo en diversos estudios de memoria de reconocimiento suelen encontrarse controles adicionales añadidos en el análisis de datos como una medida para reafirmar la no-trivialidad del 'efecto reportado' ($REFERENCIA$ ). La importancia de dichos controles se abordará con detalle más adelante. En el presente trabajo de tesis se retoman e incluyen los siguientes:
	\begin{itemize}
	\item Descartar relación entre el tipo de estímulo y los tiempos de respuesta.
	\item Comprobar la extensividad de las diferencias encontradas en la tarea con la escala.
	\end{itemize}
\end{itemize}






Se señala también que el análisis de datos se realizó desde dos enfoques distintos:

\begin{itemize}
\item La réplica de los análisis reportados en la literatura: ANOVA's y Pruebas t\\

Tomando en cuenta que el objetivo princial del trabajo de investigación realizado fue poner a prueba la extensividad de los patrones de respuesta reportados en Memoria de Reconocimiento -identificados como Efecto Espejo- en una tarea de detección perceptual, se consideró pertinente y necesario que los datos obtenidos en los experimentos y tareas planteadas fueran analizados de la misma forma.\\

Se utilizó como guía un artículo publicado por \parencite{Glanzer1990}, donde se reporta evidencia del Efecto Espejo en cinco experimentos en memoria de reconocimiento que difieren en las dimensiones de las palabras cuya manipulación determinó la distinción entre las dos condiciones de dificultad (Palabras poco frecuentes (A) vs Palabras muy frecuentes (B), Significado concreto (A) vs Significado abstracto (B), Interacción con las palabras (A) vs No interacción (B)). Todo el análisis de los datos recabados en la presente investigación fue hecha con una copia de dicho artículo en mano.\\ 

\item El desarrollo de modelos Bayesianos para la estimación paramétrica y la evaluación de la evidencia encontrada.\\



$CITAS LEE$

%Cognitive psychology has a rich set of models for phenomena ranging from low-level vision to high-order problem solving. To a statistician, these cognitive models remain naturally interpretable as statistical models, and in this sense modeling can be considered an elaborate form of data analysis. The dierence is that the models usually are very dierent from default statistical models like general linear models, but instead formalize processes and parameters that have stronger claims to psychological interpretability. There is no clear dividing line between a statistical and a cognitive model. Indeed, it is often possible for the same statistical model to have valid interpretations as a method of data analysis and a psychological model. Signal detection theory is a good example (e.g., Green & Swets, 1966). Originally developed as a method for analyzing binary decisions for noisy signals, in potentially entirely non-psychological contexts, it nonetheless has a natural interpretation as a model of cognitive phenomena like recognition memory. Despite this duality, the distinction between data analysis and psychological modeling is a useful one. The use of Bayesian methods to implement, apply, and evaluate cognitive models is the focus of this chapter.

% Bayesian statistical methods provide a  exible and principled framework  for relating cognitive models to behavioral data. They allow for  cognitive models to be formalized, evaluated, and applied, supporting inferences about parameters, the testing of models, and making  predictions about data. This chapter argues that Bayesian methods are most useful for cognitive modeling in allowing more ambitious accounts of cognition to be considered, including models that include ierarchical, latent-mixture, or common-cause structures. These theoretical possibilities, and the practical mechanics of using Bayesian methods implemented as graphical models, are demonstrated by means of an extended case study, involving psychophysical models of the perception of duration for auditory and visual stimuli. The case study demonstrates a number of general features of the Bayesian approach|representing uncertainty, being sensitive to model complexity, d ealing with contaminants, allowing for individual dierences, making predictions and generalizations, and so on|while emphasizing the role of informative prior distributions to capture theoretical assumptions about cognitive variables, and the complementary roles of parameter inference and model testing in answering research questions

\end{itemize}








\section{Las condiciones de dificultad son diferentes}

Las condiciones de dificultad propuestas para los experimentos realizados, se construyeron con base en los hallazgos reportados por \parencite{Massaro1971} en cuanto al efecto que tienen las diversas variables que componen las figuras de Ebbinghaus. De acuerdo con estos autores, una de las variables cuyo impacto en la intensidad de la ilusión de Ebbinghaus es más claro, es el número de círculos que aparecen en torno al círculo central.

\begin{figure}[th]
\centering
\includegraphics[width=0.80\textwidth]{Figures/Diff_D_E1yE2}
%\decoRule
\caption[Diferencias en Discriminabilidad (Verificando que las condiciones sean, de hecho, diferentes)]{Por cada uno de los experimentos realizados se muestra la relación entre las d' de cada condición de acuerdo a la ejecución de cada uno de los sujetos. En ambos experimentos, se observa una tendencia sistemática a tener mayores niveles de d' en la condición con pocos círculos en las figuras de Ebbinghaus.}
\label{fig:Diff_D}
\end{figure}

\subsection{Análisis 1: Prueba T para comparar las medias de d' por cada condición}

\begin{table}
\caption[Prueba T para evaluar diferencias en las medias de d' entre las condiciones]{Diferencias en d'}
\label{Tabla_t-HitsyFA}
\centering
\begin{tabular}{l | c c c c}
\toprule
%\tabhead{Groups} & \tabhead{Treatment X} & \tabhead{Treatment Y} \\
\textbf{Experimento} & \textbf{$\mu$ A} & \textbf{$\mu$ B} & \textbf{T}  & \textbf{P value}\\
\midrule
Experiment 1 & 3.240 & 2.448 & -3.0587 & 0.0020 \\
Experiment 2 & 1.950 & 1.022 & -3.4972 & 0.0005 \\
\bottomrule
\end{tabular}
\end{table}


\subsection{Análisis 2: Modelo jerárquico bayesiano: Modelo Delta para diferencias en d'}

Se desarrolló un modelo jerárquico bayesiano que asume que las d' y los sesgos C estimados por cada participante en cada una de las condiciones provienen de una distribución normal. El modelo incorpora un parámetro $\delta$ que representa las diferencias entre las medias de las d' estimadas por concidión.

El modelo desarrollado puede verse en la Figura~\ref{fig:Mod_Delta}, estando compuesto por los siguientes elementos:

\begin{itemize}
\item \textbf{Los datos: la materia prima que se señala con nodos sombreados.}
Dado el diseño experimental, conocemos el número total de ensayos con ruido (n) y señal (s). Adicionalmente, por cada participante sabemos cuál es el número total de Hits y Falsas Alarmas cometidos durante el experimento ($H_ij$ y $Fa_ij$, respectivamente). En el modelo, los datos se contienen en nodos cuadrados porque son variables discretas (Nodos circulares indican variables contínuas).\\

\item \textbf{Las tasas como una probabilidad oculta.} 
En la literatura clásica en TDS las tasas de hits y falsas alarmas se interpretan directamente como la proporción de las distribuciones de señal y ruido que caen por encima del criterio ($REFERENCIAS$), respectivamente. Sin embargo, el modelamiento bayesiano nos permite asumir que el número de Hits y Falsas Alarmas observado en cada participante es el resultado de una probabilidad oculta, como parte de un proceso biniomial. Es decir, se asume que el número de Hits y Falsas alarmas

\item \textbf{Sesgo y Discriminabilidad}
De acuerdo con la TDS, las tasas 

\item \textbf{Plato de participantes} 

\item \textbf{Estructura jerárquica:}

\item \textbf{Plato de Condición}

\item \textbf{Parámetro Delta}
\end{itemize} 

\begin{figure}[th]
\centering
\includegraphics[width=1.1\textwidth]{Figures/Model_Delta_Diff_D}
%\decoRule
\caption[Modelo Delta: Un modelo jerárquico bayesiano para revisar las diferencias en d']{Modelo jerárquico bayesiano con un parámetro Delta ($\delta$) que representa las diferencias en las medias de d' entre las dos condiciones. El modelo tiene priors no informativas.}
\label{fig:Mod_Delta}
\end{figure}









\section{Diferencias en las Tasas de Hits y Falsas Alarmas}




La Figura~\ref{fig:MirrorRate_E2_P4} muestra la frecuencia absoluta de Hits y Falsas Alarmas obtenidas por el Participante 4 en el Experimento 2, en cada una de las condiciones de dificultad puestas a prueba. Como se puede apreciar, este participante muestra claramente el patrón de respuestas identificado como parte del Efecto Espejo en la literatura, siendo que en la condición fácil (A) tiene más aciertos (Hits en AS > Hits en BS) y menos errores (Falsas Alarmas en AN < Falsas Alarmas en BN). Estas discrepancias en el desempeño del participante son consistentes con la idea de que existe una distribución de señal y una distribución de ruido por cada condición de dificultad, que se distribuye en el espacio de elección como un reflejo que se aleja en ambas direcciones, bajo el supuesto de que los participantes están respondiendo a la tarea utilizando un único criterio de elección y que ignoran la existencia de más de un 'tipo de estímulo'.\\

\begin{figure}[th]
\centering
\includegraphics[width=0.60\textwidth]{Figures/MirrorRate_Exp2_P4}
%\decoRule
\caption[Diferencias entre Hits y Falsas Alarmas por Condición; Ejemplo]{Se muestra el desempeño del Participante 14 del Experimento 1 en relación al color de los estímulos. En el panel izquierdo se muestra la relación entre el número de Hits obtenidos y el color de los estímulos, mientras que en el panel derecho se muestra la misma relación para las Falsas alarmas.}
\label{fig:MirrorRate_E2_P4}
\end{figure}




\begin{figure}[th]
\centering
\includegraphics[width=0.80\textwidth]{Figures/Diff_Rate_E1}\\ 
\includegraphics[width=0.80\textwidth]{Figures/Diff_Rate_E2}
%\decoRule
\caption[Diferencias en Tasas (Evaluando diferencias en el desempeño entre las condiciones)]{Comparación intrasujeto de}
\label{fig:Diff_Rate}
\end{figure}


\subsection{Análisis 1: Prueba T}



\begin{table}
\caption[Prueba T para evaluar diferencias en las medias de las tasas de ejecución (Hits y F. Alarmas) entre condiciones]{Diferencias en Hits y Falsas Alarmas entre los tipos de estímulos (Experimento 1 y 2)}
\label{Tabla_t-HitsyFA}
\centering
\begin{tabular}{l l | c c c c}
\toprule
%\tabhead{Groups} & \tabhead{Treatment X} & \tabhead{Treatment Y} \\
\textbf{Experimento} & \textbf{Tasa} & \textbf{$\mu$ A} & \textbf{$\mu$ B} & \textbf{T} & \textbf{P value}\\
\midrule
Exp 1 & Hits & 0.922 & 0.860 & -2.4348 & 0.0098 \\
Exp 1 & FA & 0.08 & 0.143 & 1.917 & 0.0314 \\
Exp 2 & Hits & 0.853 & 0.678 & -3.4757, & 0.0006 \\
Exp 2 & FA & 0.268 & 0.336 & 1.769 & 0.0425 \\
\bottomrule
\end{tabular}
\end{table}


\subsection{Modelo bayesiano}


\begin{figure}[th]
\centering
\includegraphics[width=1.1\textwidth]{Figures/Model_Tau_Diff_Tetas}
%\decoRule
\caption[Modelo Tau: Modelo Bayesiano para evaluar las diferencias entre las tasas de hits y falsas alarmas]{Comparación intrasujeto de}
\label{fig:Mod_Tau}
\end{figure}














\section{Diferencias en la asignación de Puntajes de Confianza}


La Figura~\ref{fig:MirrorRating_E1_P10} muestra el promedio de los puntajes de confianza asignados por el Participante 10 del Experimento 1 a los estímulos pertenecientes a cada una de las condiciones de dificultad construidas, separando para cada caso los ensayos con ruido y señal. En la figura puede apreciarse con claridad una tendencia ascendente que coincide con el patrón reportado en estudios de Memoria de Reconocimiento.\\

\begin{figure}[th]
\centering
\includegraphics[width=0.60\textwidth]{Figures/MirrorRating_Exp1_P10}
%\decoRule
\caption[Comparación entre Puntajes de Confianza asignados por Condición; Ejemplo]{Se muestra el desempeño del Participante 14 del Experimento 1 en relación al color de los estímulos. En el panel izquierdo se muestra la relación entre el número de Hits obtenidos y el color de los estímulos, mientras que en el panel derecho se muestra la misma relación para las Falsas alarmas.}
\label{fig:MirrorRating_E1_P10}
\end{figure}

%La comparación entre los puntajes de confianza asignados a cada grupo de estímulos por el resto de los participantes en los Experimentos 1 y 2, se presentan en las Figuras~\ref{fig:MERating_E1} y ~\ref{fig:MERating_E2}.\\





\subsection{Análisis 1: }

\begin{table}
\caption[Prueba T para evaluar diferencias en las medias de los puntajes de confianza asigandos entre condiciones]{}
\label{Tabla_t-HitsyFA}
\centering
\begin{tabular}{l l |  c c c c}
\toprule
%\tabhead{Groups} & \tabhead{Treatment X} & \tabhead{Treatment Y} \\
\textbf{Experimento} & \textbf{Ensayo} & \textbf{$\mu$ A} & \textbf{$\mu$ B} & \textbf{T} & \textbf{P value}\\
\midrule
Exp 1 & Signal & 5.445 & 5.212 & -1.7778, & 0.0418 \\
Exp 1 & Noise & 1.542 & 1.883 & -1.7208 & 0.0472 \\
Exp 2 & Signal & 5.183 & 4.342  & -3.6752, & 0.0004 \\
Exp 2 & Noise & 2.386 & 2.752 & -1.809 & 0.0391 \\
\bottomrule
\end{tabular}
\end{table}












\section{Réplica de controles reportados en la literatura}



 
% Chapter Template

\chapter{Discusión} % Main chapter title

\label{Cap_Disc} % Change X to a consecutive number; for referencing this chapter elsewhere, use \ref{ChapterX}

%----------------------------------------------------------------------------------------
%	SECTION 1
%----------------------------------------------------------------------------------------


% Chapter 1

\chapter{Conclusión} % Main chapter title

\label{Cap_Conclusion} % For referencing the chapter elsewhere, use \ref{Chapter1} 

La Teoría de Detección de Señales (SDT) presenta uno de los modelos más sólidos y ampliamente desarrollados en Psicología Experimental, que permite dar cuenta de una amplia gama de situaciones donde los organismos se enfrentan a la tarea de detectar ciertos eventos en su entorno para guiar su comportamiento de manera óptima, en función a las relaciones de contingencia anunciadas. Los supuestos de dicho modelo son lo suficientemente generales para permitir su aplicación al estudio de distintos fenómenos, dentro y fuera de la psicología, funcionando tanto como un modelo estadístico para describir la detección como problema de adaptabilidad, como una herramienta para interpretar la ejecución de sistemas evaluados experimentalmente.\\

En estudios donde el desempeño de los participantes en tareas de memoria de reconocimiento es evaluado con base en la SDT, se ha reportado consistentemente que al comparar su ejecución entre dos clases de estímulos A y B que difieren en la precisión con que sus elementos son reconocidos ($d'(A)$ $>$ $d'(B)$), las respuestas de los participantes sugieren que las distribuciones de Ruido y Señal de las clases A y B se despliegan simétricamente sobre el eje de evidencia, (las distribuciones de la clase A se encuentran en los extremos y las distribuciones de la clase B, en el área intermedia). Los patrones de respuesta relacionados con este fenómeno han sido identificados como Efecto Espejo y sus implicaciones han sido abordadas tanto en términos de su relevancia para el estudio de la memoria como de la validez que pueden tener los modelos de memoria basados en detección de señales.\\

La evidencia del Efecto Espejo ha sido reportada en una gran variedad de estudios donde las clases A y B son construidas a partir de distintas variables y donde se utilizan diferentes protocolos de detección. El presente estudio es el primero en aportar evidencia del Efecto Espejo en tareas de detección perceptual, con un par de experimentos compuestos por una tarea binaria con Escala de Confianza. El análisis de datos se realizó tanto a partir de los análisis frecuentistas reportados usualmente en la literatura, como de un análisis bayesiano sustentado en la construcción de modelos y la realización de pruebas estadísticas bayesianas. Ambas aproximaciones confirman la importancia de este proyecto como evidencia de que el Efecto Espejo puede encontrarse fuera de la memoria de reconocimiento. Sin embargo, el análisis bayesiano permitio una evaluación más precisa de su robustez, (por ejemplo, señalando que las diferencias entre los Puntajes de Confianza asignados durante la tarea con Escala de Confianza son apenas anecdóticas).\\

Los resultados obtenidos en el presente trabajo pueden ser interpretados en dos direcciones. Primero, como evidencia de que el Efecto Espejo no es un fenómeno exclusivo de la memoria de reconocimiento y debe ser abordado como una regularidad en tareas de detección con más de un nivel de $d'$. Segundo, como un referente sobre las ventajas que presenta el análisis de datos bayesiano sobre el análisis frecuentista, en tanto que permite un mejor manejo de la incertidumbre contenida en los datos, algo particularmente útil en situaciones donde se concibe una estructura probabilística tanto en el entorno como en las respuestas de los organismos.\\

%----------------------------------------------------------------------------------------
%	THESIS CONTENT - APPENDICES
%----------------------------------------------------------------------------------------

\appendix % Cue to tell LaTeX that the following "chapters" are Appendices

% Include the appendices of the thesis as separate files from the Appendices folder
% Uncomment the lines as you write the Appendices
\include{Appendices/Felisa_Consentimiento}
%% Appendix Template

\chapter{Instrucciones} % Main appendix title

\label{App_Inst} % Change X to a consecutive letter; for referencing this appendix elsewhere, use \ref{AppendixX}

(Pendiente)
%% Appendix Template

\chapter{Registro de respuesta} % Main appendix title

\label{App_Registro} % Change X to a consecutive letter; for referencing this appendix elsewhere, use \ref{AppendixX}

El experimento fue programado de manera tal que los datos obtenidos sobre la ejecución de cada participante se vaciaran en un documento en formato csv, con el título del mismo indicando el número e iniciales del participante, el día en que se este hubiera participado en el experimento y el experimento en que participó.\\

\begin{figure}[th]
\centering
\includegraphics[width=0.95\textwidth]{Figures/csv} 
\decoRule
\caption[Csv muestra]{Captura de pantalla de uno de los archivos csv generados tras la aplicación del experimento. Se ilustra el registro y clasificación de las respuestas emitidas por los participantes en cada ensayo, en función de su correspondencia con el estímulo aleatoriamente mostrado. Se muestran únicamente los primeros seis ensayos del Experimento 2, realizado el 7 de junio del 2016.}
\label{fig:csv}
\end{figure}
\end{itemize}

En la Figura se muestra como ejemplo los primeros seis ensayos registrados en uno de los csv's obtenido tras la realización de nuestro experimento. El csv se fue escribiendo por nuestro programa conforme van avanzando los ensayos, indicándonos cuál de los estímulos diseñados (Estímulo) fue seleccionado aleatoriamente para aparecer en cada ocasión (Ensayo); para señalar el estímulo presentado en cada ensayo se utilizaron números arbitrariamente asignados a cada una de las combinaciones generadas para la elaboración de las figuras de Ebbinghaus (ver las $FIGURAS DE DISEÑO DE ESTIMULOS$). La respuesta del participante se registra (Respuesta) y se clasifica como Correcta o no (se escribe 'True' o 'False' en la columna 'Correcto') de acuerdo a su correspondencia con el estímulo que apareció en pantalla. A continuación, se presentan dos columnas que van registrando la frecuencia acumulada con que el participante acierta y se equivoca a lo largo del experimento (Aciertos y Errores, respectivamente). Además de indicar si se trata de una respuesta acertada, o no, el programa identifica la respuesta emitida por el participante en función a qué tipo de acierto puede ser (Hit o Rechazo) y a qué tipo de error (Falsa alarma u Omision), llevando un contador para cada uno de estos casos que muestra la frecuencia acumulada con que el participante emite una respuesta de cada tipo (ContadorH, para los Hits; ContadorR, para los Rechazos correctos; ContadorF, para las Falsas Alarmas; ContadorM, para las omisiones). En cuanto a la segunda fase de cada ensayo, se registra el puntaje obtenido al aplicar la conversión previamente detallada al número elegido por el participante a la escala de confianza en su respuesta presentada (Confidence). Por último se muestran los Tiempos de Respuesta registrados por cada ensayo: RTime1, corresponde al tiempo de respuesta transcurrido a partir de la aparición del estímulo en pantalla; RTime1b muestra el tiempo transcurrido desde la desaparición del estímulo en pantalla y la emisión de la respuesta del participante; RTime2 presenta el tiempo de respuesta a la segunda fase de cada ensayo.\\ 


%\include{Appendices/App_Data}
%\include{Appendices/AppendixB}
%\include{Appendices/AppendixC}

%----------------------------------------------------------------------------------------
%	BIBLIOGRAPHY
%----------------------------------------------------------------------------------------

%\printbibliography[heading=bibintoc]
%\include{Bib_Adrifelcha.bib}

%\bibliographystyle{apalike}
%\bibliography{Bib_Adrifelcha}

\bibliography{Bib_Adrifelcha}
%----------------------------------------------------------------------------------------

\end{document}  
