% Chapter 1

\chapter{Introducción} % Main chapter title

\label{Chapter1} % For referencing the chapter elsewhere, use \ref{Chapter1} 

%----------------------------------------------------------------------------------------

% Define some commands to keep the formatting separated from the content 
\newcommand{\keyword}[1]{\textbf{#1}}
\newcommand{\tabhead}[1]{\textbf{#1}}
\newcommand{\code}[1]{\texttt{#1}}
\newcommand{\file}[1]{\texttt{\bfseries#1}}
\newcommand{\option}[1]{\texttt{\itshape#1}}

%----------------------------------------------------------------------------------------

El mundo está cargado de ruido e incertidumbre. Los organismos están constantemente expuestos a distintas fuentes y tipos de estimulación en su entorno que pueden, o no, dar información relevante sobre el estado de las cosas. Por ello, uno de los primeros grandes problemas a los que se enfrentan es el de ordenar el caos resultante definiendo relaciones de contingencia que les permitan ajustar su comportamiento a las reglas y restricciones operantes en su entorno. Una vez establecida la relación entre la presencia u ocurrencia de ciertos estímulos y el acceso a ciertas consecuencias, la detección de éstos se vuelve una tarea importante para que los organismos puedan guiar su comportamiento, (por ejemplo: "Sé que si como nueces se me cierra la garganta, ¿Hay nuez en este panqué? Si sí, no me lo como; si no, sí.").\\

%La detección de ciertos eventos no es una tarea sencilla. La información a evaluar suele ser ambigua.
"Detectar algo" no parecería ser un problema importante si asumiéramos que se trata de casos que aparecen con perfecta claridad, o bien, que el organismo interesado en su detección cuenta con sensores altamente precisos que garantizan su identificación. Sin embargo, la evidencia a partir de la cual juzgamos si algo está o no ocurriendo por lo general es confusa y puede llevarnos a emitir juicios erróneos. A manera de ejemplo cotidiano, imaginemos el caso de un adolescente que quiere conseguir permiso para ir de fiesta y necesita encontrar el momento ideal para pedírselo a su mamá (cuando ella esté de buen humor); los indicadores con que cuenta son imprecisos -los gestos, el tono de voz, las actividades que su madre realice durante el día, etc.- y errar en el diagnóstico del estado emocional de su madre y en consecuencia no obtener el permiso deseado al pedirlo en el momento inadecuado es un riesgo latente, ya sea por una mala lectura de los datos disponibles -que las ansias del adolescente por salir de fiesta le hagan apresurar el momento- o bien porque los datos en sí mismos son poco claros -la mamá podría ser una persona particularmente inexpresiva o, por el contrario, terriblemente variable-.\\ 

La Teoría de Detección de Señales (TDS; SDT en ingles) constituye uno de los modelos más sólidos y ampliamente estudiados en Psicología, cuyos supuestos son lo suficientemente generales para permitir su aplicación al estudio de distintos fenómenos -dentro y fuera de la psicología- donde los organismos -los sistemas a estudiar- se enfrentan a una tarea de detección a partir de la cual deben guiar su comportamiento. Funciona tanto como un modelo estadístico para describir esta clase de problemas, como una herramienta para interpretar la ejecución de los sistemas evaluados y hacer inferencias sobre la precisión con que el estímulo a detectar se distingue del ruido y los posibles sesgos que conlleven a reportar la presencia o ausencia de la señal.\\

La TDS le concede a la noción de variabilidad un papel fundamental para entender la detección de señales como un problema de adaptabilidad para los organismos. La idea básica es que las señales cuya detección resulta relevante para los organismos -al igual que cualquier otro estímulo- suelen presentarse y percibirse con cierta variabilidad en cada ocasión y además de ello, coexisten en el mundo con otros estímulos (el ruido) que dentro de su propia variabilidad pueden llegar a presentarse o percibirse con la misma evidencia que lo haría una señal, pudiendo ser confundidos con la misma. Por ejemplo, imaginemos que queremos detectar si la persona que nos acaba de contestar el teléfono es un adulto. Existe un rango de 'tonos de voz' que estaríamos dispuestos a admitir como pertenecientes a una persona mayor de edad que, a grandes rasgos, tiende a ser más grave de lo que esperaríamos escuchar en un niño. Sin embargo, sabemos que hay adultos que pueden llegar a tener una voz particularmente clara que puede llegar a confundirse con la de un menor de edad; si la persona que nos contestó el teléfono resulta ser un varón de voz aguda, no podremos estar seguros de si se trata de un adulto o no con sólo escuchar su voz y probablemente necesitemos consultarle de manera explícita.\\

La TDS es un modelo de decisión que asume que los organismos no detectan los elementos relevantes de su entorno en respuesta directa a la estimulación que reciben de manera inmediata, sino que ponderan la evidencia con la información que poseen sobre el mismo. Bajo esta visión, los organismos 'eligen' el juicio de detección que les permite guiar su comportamiento de la manera más óptima posible tomando en consideración 1) las ganancias y pérdidas en juego (que hacen más, o menos, riesgoso el cometer cierto tipo de error y más, o menos, atractivo el cometer un acierto en particular) y 2) la probabilidad con que dichos eventos se presentan en su entorno.  La noción de los umbrales amplicamente desarrollada por la Psicofísica clásica, es reemplazada por la concepción de un criterio de elección.\\ 

La generalizabilidad del modelo de la TDS al estudio de distintos fenómenos y tareas de detección se debe a lo abstracto de sus elementos: la 'señal' que interesa detectar puede ser desde un estímulo concreto o un estado del mundo hasta la pertenencia a una categoría y el 'ruido' es simplemente todo elemento presente en el entorno de la tarea que no sea la señal.\\ 

Al aplicar el modelo de Detección de Señales a tareas de Memoria de Reconocimiento, donde los participantes tienen que identificar los elementos ya antes vistos (la señal) dentro de un conjunto de ítems que incluye elementos presentados en una fase previa y elementos 'nuevos' (el ruido), se ha encontrado un patrón de respuestas consistente cuando se compara la ejecución de los participantes entre dos clases de estímulos, (una de ellas más fácil de reconocer (A) que la otra (B)), que demuestra que los participantes no sólamente son mejores reconociendo los elementos previamente mostrados en la condición A ($Hits(A)>Hits(B)$), sino que también son mejores identificando los estímulos nuevos dentro de esta misma condición ($F.alarm(A)<F.alarm(B)$). Dado que los participantes experimentales no saben que están respondiendo a más de una clase de estímulos en la tarea que se les presenta, se asume que utilizan un sólo criterio de elección para emitir sus respuestas y de acuerdo a las tasas reportadas de Hits y Falsas Alarmas, se infiere que las distribuciones de ruido y señal de cada condición se distribuyen a lo largo de un mismo eje de decisión de tal forma que parecieran reflejarse entre sí. Es por ello que en la literatura en Memoria de Reconocimiento se ha identificado dicho patrón de respuestas bajo el nombre de Efecto Espejo.\\

El Efecto Espejo sólo ha sido estudiado dentro del dominio de la Memoria de Reconocimiento, donde se ha reportado evidencia de su existencia a lo largo de una amplia variedad de procedimientos (tareas de respuesta 'Sí/No', tareas de elección forzada entre dos alternativas y protocolos con puntajes de confianza) y variables (palabras comunes vs palabras extrañas; estímulos abstractos vs estímulos concretos; palabras escritas al revés vs palabras bien escritas; imágenes a color o en blanco y negro, etc). A su vez, gran parte de los modelos y teorías desarrollados para dar cuenta de este fenómeno tienden a hacerlo en términos de la estructura 

y por tanto, ha sido explicado a partir de diferencias en la atención y procesamiento que recibe cada una de las condiciones durante la fase de estudio previa a la tarea de reconocimiento. Aquí se presenta el primer estudio en explorar su extensión a un dominio distinto, utilizando lo que se ha reportado en la literatura sobre ilusiones ópticas



El interés principal del presente trabajo de tesis fue explorar la extenssividad del Efecto Espejo en otras áreas donde se ha aplicado la TDS, poniendo a prueba el supuesto hecho en Memoria de Reconocimiento de que de


Surprisingly, evidence for the Mirror Effect has only been collected within Recognition Memory studies. And so, most of the theories and models proposed to explain it tend to do it in terms of the study phase included in any recognition memory task, where subjects interact for the first time with the stimuli and, presumably, attend/process them differently leading to the different rates of response observed in the recognition phase.
The main goal of the present study was to explore the existence of the Mirror Effect within other areas where SDT has been applied. Thus, testing the assumption that the Mirror Effect depends on a previous-study phase to appear, while it could be understood as a more basic product of any SDT-like task with two or more levels of discriminability involved.