% Chapter 1
\chapter{Marco Teórico} % Main chapter title

\label{Cap_SDT} % For referencing the chapter elsewhere, use \ref{Chapter1} 

%----------------------------------------------------------------------------------------

% Define some commands to keep the formatting separated from the content 
\newcommand{\keyword}[1]{\textbf{#1}}
\newcommand{\tabhead}[1]{\textbf{#1}}
\newcommand{\code}[1]{\texttt{#1}}
\newcommand{\file}[1]{\texttt{\bfseries#1}}
\newcommand{\option}[1]{\texttt{\itshape#1}}

%----------------------------------------------------------------------------------------

\section{Teoría de Detección de Señales}

%Detectar ciertos estados en el mundo es importante para guiar nuestro comportamiento
Uno de los problemas más frecuentes a los que se enfrentan los organismos como sistemas inmersos en entornos variables que buscan optimizar su comportamiento, es la detección de estados o eventos específicos que les informen sobre las restricciones y relaciones de contingencia vigentes.\\

%La detección es distinta de la discriminación y la categorización. Todos problemas importantes en un entorno cargado de estimulación. (El problema del embudo)
A diferencia de problemas tales como la discriminación o la categorización, donde la tarea de los organismos es evaluar la evidencia que se les presenta para asignarle una etiqueta (e.g. "¿Es A o es B?"; "¿De acuerdo a sus propiedades en tales dimensiones, se trata de un caso de..."), cuando hablamos de un problema de detección nos referimos a situaciones que pueden plantearse en términos de preguntas "Sí/No" (e.g. "¿La comida está buena? Sí/No", "¿Ese que viene es mi camión? Sí/No", "¿Este perro es hostil? Sí/No") y cuya respuesta permite guiar el comportamiento de los organismos en función de las consecuencias anunciadas (e.g. "Sí, la comida está buena, me la comeré porque es seguro", "No, ese no es mi camión, no me subiré porque acabaré en Ecatepec", "No, no es un perro hostil, puedo acariciarlo", etc.).\\ 

%Origen y expansión de la Teoría de Detección de Señales en la psicología y otras áreas
La Teoría de Detección de Señales (TDS o SDT, por sus siglas en inglés) aparece por primera vez en 1954 -como tantos otros avances científicos y tecnológicos motivados por las necesidades planteadas por la Segunda Guerra Mundial- en el contexto del estudio y desarrollo de radares para detectar señales eléctricas específicas \parencite{Peterson1954}. Muy poco tiempo después, los psicólogos John A. Swets y Wilson P. Tanner contribuyeron a la expansión de la teoría a un contexto psicológico, como un modelo para estudiar la percepción de los organismos, \parencite{Tanner1954, Swets1961}. Desde entonces, la TDS constituye uno de los modelos más estudiados, desarrollados y ampliamente aplicados en Psicología, extendiéndose desde su foco inicial en el estudio de la percepción \parencite{Rosenholtz2001, Pessoa2005, Wallis2007} hacia el estudio de cualquier fenómeno o tarea donde los organismos se enfrenten al problema de emitir -y guiar su comportamiento en función a- juicios de detección; por ejemplo, en materia de la emisión de diagnósticos clínicos \parencite{Grossberg1978, Swets2000, Boutis2010}, en el estudio de ciertas condiciones clínicas \parencite{Westermann2010, Bonnel2003, Brown1994, Naliboff1981}, en el estudio de la identificación visual de testigos \parencite{Gronlund2014, Wixted2014, Wixted2016} y un muy amplio 'etcétera' \parencite{Gordon1974, Nuechterlein1983, Harvey1992, Verghese2001}.\\ 


%La Teoría de Detección de señales como un modelo descriptivo para el problema de la detección que admite la importancia de la incertidumbre, como parte del entorno y como motor en el uso de sesgos de respuesta.
La TDS constituye un modelo estadístico que describe el problema al que se enfrentan los organismos inmersos en situaciones de detección en ambientes con incertidumbre, donde las señales -los estímulos cuya ocurrencia interesa detectar- 
coexisten con ruido -estímulos que no son la señal pero que pueden confundirse con esta-. Se trata de un modelo de decisión que entiende la detección como una tarea de elección, donde los organismos no responden simplemente con base en lo que perciben, sino que eligen el juicio de detección que les permita guiar su comportamiento de la manera mas óptima posible dada la información que poseen sobre la estructura del entorno -probabilidades y consecuencias involucradas- a la luz de la evidencia presente.\\

La generalizabilidad del modelo de la TDS al estudio de distintos fenómenos y tareas de detección se debe a lo abstracto de sus elementos: la 'señal' que interesa detectar puede ser desde un estímulo concreto -una luz o un tono- hasta la pertenencia a una categoría -una enfermedad o amenaza- y el 'ruido' es simplemente todo elemento presente en el entorno de la tarea que no sea la señal.\\ 

Los organismos compensan la incertidumbre contenida en las tareas de detección con la información que poseen sobre el entorno. En términos generales, ésta puede ser de dos tipos: 1) información probabilística y 2) información sobre las consecuencias comprometidas. Sin importar a cuál de estas categorías pertenezca, los organismos construyen la 'información sobre el entorno de decisión' con base en su experiencia con el mismo y en la información que han recibido sobre este a lo largo de su vida. Por ejemplo, imaginemos el caso de un médico que trata de decidir si los resultados obtenidos en cierta prueba clínica son evidencia suficiente para diagnosticar una enfermedad 'X' a un paciente 'Y'. La evidencia con la que el médico cuenta es imprecisa: toda prueba clínica tiene un margen de error y su lectura debe complementarse con información extraída de su historia clínica por un médico especialista. El resultado de la prueba no es lo suficientemente informativo en sí mismo y el médico debe juzgar la evidencia en función de distintos factores, por ejemplo: ¿Qué tan confiable es la prueba?, ¿cuál es su tasa de aciertos y errores?; ¿qué tan común es la enfermedad o condición cuya presencia se intenta determinar?; de acuerdo con la historia clínica del paciente, ¿qué tanto correlacionan sus características con los factores de riesgo asociados a dicha enfermedad o condición?. En otras palabras, el médico tiene que compensar la incertidumbre implícita en la evidencia a evaluar con toda la información probabilística de la que dispone. Y la historia no termina aquí.  Pese a que la inferencia probabilística contribuye a identificar la conclusión más probable, el médico no puede estar completamente seguro de su respuesta. Para optimizar su comportamiento y tomar la mejor decisión posible, el médico también debe tomar en consideración la información que posee sobre las consecuencias asociadas a cada escenario posible: a) Si el paciente tiene la enfermedad y el médico la detecta acertadamente, podrá tratarse a tiempo; b) Si tiene la enfermedad y el médico falla en detectarla, podría poner en riesgo su vida; c) Si no tiene la enfermedad y el médico le dice que sí, se gastarán recursos innecesarios en solucionar un problema que no existe, corriendo el riesgo de que el tratamiento le haga daño y d) Si no tiene la enfermedad y el médico decide no darle el diagnóstico, todo permanecerá igual. La tarea del médico es mucho más compleja de lo que parecía en un principio, puesto que no se limita a la lectura de una prueba clínica, sino a ponderar lo que sugieren los resultados de la misma con toda la información que posee sobre la probabilidad de las interpretaciones posibles y las consecuencias comprometidas.\\


\subsection{Supuestos generales del modelo}

%La TDS distingue dos grandes factores en la emisión de un juicio o respuesta: La discriminabilidad y el sesgo.
La TDS funciona como una herramienta -o marco de análisis- para traducir el desempeño observado en tareas de detección en inferencias sobre la precisión con que la señal se distingue del ruido (la discriminabilidad) y la posible preferencia -o tendencia- del sistema detector a responder en favor o en contra de esta (el sesgo). Esta distinción entre la Discriminabilidad de los estímulos comprometidos y el Sesgo del sistema, como factores que interactúan en la emisión de juicios de detección, es una de las principales propiedades de la TDS cuya importancia e implicaciones se discuten a continuación:\\

\textbf{1.- El papel de la Discriminabilidad: Siempre hay incertidumbre}\\

%Hay variabilidad en todos los estímulos implicados en las tareas de detección (en la señal y en los estímulos no-señal)
Se habla de la detección de señales como un problema de adaptación porque se asume que la variabilidad en la presentación y percepción de los estímulos en el ambiente merma la capacidad de los organismos de emitir juicios de detección que reflejen el estado del mundo con certeza. Y dado que los estímulos-señal coexisten en el mundo con estímulos-ruido, saber qué tan salientes son las señales respecto del ruido es uno de los factores más importantes para determinar qué tan difícil es su detección para los organismos. En términos de la TDS, se habla de dicha dificultad como 'la discriminabilidad' de los estímulos comprometidos en la tarea.\\

De acuerdo con la TDS, la discriminabilidad constituye el primer gran componente en la emisión de juicios de detección óptimos que reflejen el verdadero estado del mundo y permitan al organismo actuar conforme a las consecuencias vigetes. Suele explicarse en términos de:\\ %  1) la variabilidad intrínseca en la presentación de las señales y 2) el ruido con que ésta coexiste.\\

\underline{a) La Variabilidad en la Señal}\\

%Existe variabilidad en la forma en que percibimos los estímulos que nos rodean. Los sistemas sensoriales y perceptuales se comportan como instrumentos de medición (error de medida)
La noción de variabilidad ha sido uno de los principales motores para el desarrollo de modelos estadísticos en Psicología. Desde que Fechner extendiera las ideas planteadas por Gauss sobre la incertidumbre contenida en toda medición -la idea de que toda medición realizada contiene el valor 'verdadero' de aquello que se quiere medir más un 'error' aleatorio que la carga de incertidumbre- al estudio de la percepción -conceptualizando nuestros sistemas sensoriales y perceptuales como 'instrumentos de medición' que perciben las cualidades 'verdaderas' de los estímulos más un 'error' en cada observación- \parencite{Fechner, Gauss}, se sentaron las bases para el desarrollo de una amplia gama de modelos matemáticos y estadísticos en Psicofísica orientados a estudiar la relación entre las cualidades físicas -'reales'- de los estímulos y la magnitud o intensidad con que se perciben psicológicamente \parencite{Link1994}.\\

%Variabilidad en la percepción de un mismo estímulo.
En el marco de la TDS, la variabilidad se considera una propiedad intrínseca de las señales a detectar bajo el supuesto de que ningún estímulo se percibe o se presenta de manera idéntica en cada exposición. Por ejemplo, imaginemos los siguientes casos: \\

\begin{itemize}

\item Una persona es expuesta a un mismo tono en cien ocasiones distintas y tras cada presentación, asigna un valor a la intensidad percibida. El valor reportado en cada ensayo será una mezcla entre el valor real del tono y un error aleatorio; es decir, los reportes se acercarán bastante al valor real de estímulo pero presentarán cierta variabilidad en torno al mismo. Esta idea se captura en la Figura~\ref{fig:Senal_percepcion}. Imaginemos que el estímulo presentado tiene un valor real de 10 (la unidad de medición no importa para fines de este ejemplo): Es muy probable que el valor percibido y reportado coincida con -o se acerque bastante a- su valor real -la media de la distribución, $\mu$, señalada con una línea vertical roja-, pero también habrá ensayos en que aún tratándose del mismo estímulo, el valor percibido caiga por encima o por debajo de su valor real con cierta dispersión -las colas de la distribución-. Este ejemplo pretende capturar la noción de que hay variabilidad intrínseca a la percepción de los estímulos en nuestro entorno.\\

\item Supongamos el caso de una escala clínica diseñada para aportar evidencia para el diagnóstico de la Depresión. Por lo general, las pruebas clínicas sugieren que el diagnóstico se haga con base en rangos de valores. No todas las personas con depresión van a obtener exactamente el mismo puntaje. La Figura~\ref{fig:Senal_presentacion} representa de manera gráfica esta idea: Existe una serie de posibles puntajes a obtener en la prueba de depresión (los valores en el eje de las x), y se sabe que las personas con depresión suelen obtener puntajes dentro de un rango específico con cierta probabilidad (la distribución azul), habiendo puntajes más comúnes -la media de la distribución, $\mu$, señalada en rojo- que otros. Hay variabilidad en la presentación de ciertos estímulos en el entorno.\\

\end{itemize}

\begin{figure}[th]
\centering
\includegraphics[width=0.80\textwidth]{Figures/Signal_Perception} 
%\decoRule
\caption[Variabilidad en la percepción de los estímulos]{Para ilustrar la idea de que las señales a detectar no son percibidas de la misma forma en cada presentación, se plantea el ejemplo de un estímulo con intensidad de 10 (la elección de valores y la omisión de unidades de medida es arbitraria). Es muy probable que el estímulo sea percibido de acuerdo a su valor real, (la media de la distribución, $\mu$) sin embargo y con menor probabilidad, también es posible que sea percibido como ligeramente más, o menos, intenso, (siendo que los valores que más se alejan del valor 'real' son menos probables que los cercanos).}
\label{fig:Senal_percepcion}
\end{figure}

\begin{figure}[th]
\centering
\includegraphics[width=0.80\textwidth]{Figures/Signal_Presentation} 
%\decoRule
\caption[Variabilidad en la presentación de los estímulos]{Para ilustar la idea de que las señales a detectar no se presentan de la misma forma en cada ocasión, se plantea como ejemplo la detección de una condición clínica. La figura captura la noción de que en una prueba clínica diseñada para detectar casos de Depresión, esta no se identifica a partir de un sólo puntaje sino que existe un rango de valores que se asocian a dicha condición con mayor o menor probabilidad, al rededor de un valor promedio ($\mu$, señalado en rojo). Los valores representados son arbitrarios.}
\label{fig:Senal_presentacion}
\end{figure}

En general, las Figuras~\ref{fig:Senal_percepcion} y \ref{fig:Senal_presentacion} representan el supuesto más elemental descrito en la TDS: la variabilidad es intrínseca a la presentación de los estímulos, ya sea porque nuestros sistemas sensoriales no los capturan igual en cada presentación, o porque los estímulos no se nos presentan exactamente de la misma forma en cada ocasión. En otras palabras, las señales que los organismos necesitan detectar en su entorno para guiar su comportamiento son variables, en tanto que no se presentan ni son percibidas exactamente iguales en cada ocasión.\\

    \underline{b) La variabilidad en el Entorno: Ruido}\\

%La señal coexiste con el ruido y puede llegar a confundirse con el mismo.
Además de la variabilidad contenida en las señales, es necesario tomar en cuenta que coexisten en el mundo con otros estímulos o estados, mismos que pueden producir evidencia similar a las señales y ser por tanto, confundidos con las mismas. Por ejemplo, retomemos el caso de la prueba clínica para detectar casos de Depresión. Las pruebas clínicas psicológicas se aplican para distinguir entre las personas con cierta condición -la señal- y las que no la tienen -el ruido-. Y habíamos mencionado ya que las personas con depresión no obtienen un mismo puntaje en la prueba, sino que existe un rango de valores asociados a dicha condición con mayor o menor probabilidad. De la misma forma, al aplicar la prueba a personas que no tienen depresión tampoco se obtiene siempre el mismo puntaje, sino que el resultado obtenido suele presentarse dentro de un rango de valores distinto con su propia distribución de probabilidad. Esta extensión del ejemplo original se presenta en la Figura~\ref{fig:Noise}, donde nuevamente tenemos una distribución de probabilidad que señala los puntajes asociados con la condición a detectar (en azul) y una nueva distribución que señala los puntajes que las personas sin depresión suelen obtener al resolver la prueba con distinta probabilidad (en negro).\\

La Figura~\ref{fig:Noise} ilustra otros dos puntos claves considerados en la TDS. El primero, es que las señales no se presentan de manera aislada en el entorno, sino que suelen estar acompañadas de otros estímulos o estados del mundo -que constituyen lo que se conoce como 'ruido'-. El segundo, es que el ruido -al igual que las señales- es variable y puede llegar a presentarse o percibirse de la misma forma en que lo puede hacer la Señal. Este último punto se ilustra en la Figura con el área de las distribuciones presentadas donde estas se sobrelapan, relacionando los mismos valores con ambos estados del mundo -Depresión vs No depresión- con distinta probabilidad. Por ejemplo, parece ser que es posible observar un puntaje de 55 en una persona con o sin depresión, sin embargo, es mucho más probable observar este puntaje en ausencia de dicha condición (según la intersección entre el puntaje y cada una de las distribuciones); lo mismo ocurre con un puntaje de 65, es posible observarlo en ambos casos y sin embargo, es mucho más probable que se presente en una persona con depresión.\\

\begin{figure}[th]
\centering
\includegraphics[width=0.80\textwidth]{Figures/Noise} 
%\decoRule
\caption[Variabilidad en la señal y en el ruido]{Continuando con el ejemplo de la prueba de depresión, se incluyen una distribución para representar el rango de puntajes asociados con dicha condición clínica (en azul) y una segunda distribución que representa el rango de puntajes que las personas sin depresión suelen obtener (en negro). Esta figura representa la noción de que los posibles estados de mundo -la señal y el ruido- se presentan y perciben con cierta variabilidad entre cada ocurrencia. Los valores utilizados son arbitrarios.}
\label{fig:Noise}
\end{figure}

Cabe señalar que la distribución-Señal se sitúa por encima de la distribución-Ruido ya que, como se mencionó previamente, las tareas de detección implican dar respuesta a una pregunta binaria 'Sí/No' respecto a la presencia o ausencia de 'algo' -una señal- en el entorno que permite a los organismos guiar su comportamiento de manera óptima. Sea cual sea la evidencia con base en la cual se forman los juicios de detección -los valores en el eje X sobre el cual se despliegan las distribuciones ruido y señal-, se espera que la Señal tenga 'más' de dicha evidencia que el Ruido (en tanto que este último implica su ausencia). Por ejemplo, en el caso ilustrado en la Figura~\ref{fig:Noise}, tiene sentido que las personas con Depresión obtengan puntajes más altos dado que la prueba fue hecha para evaluar la presencia de dicha condición y el puntaje final obtenido suele ser reflejo de cuántas de las respuestas proporcionadas coinciden con lo que se esperaría en una persona con Depresión.\\

La variabilidad en la presentación y percepción de los posibles estados del entorno -la presencia o ausencia de la señal- constituye el elemento base sobre el cual se desarrolla la TDS y que nos lleva a pensar en la detección de señales como una tarea cargada de incertidumbre, donde los organismos no pueden confiar completamente en la evidencia que se les presenta -y no pueden emitir un juicio de detección basándose únicamente en esta- puesto que dicha evidencia puede corresponder a -o haber sido producida por- cualquiera de las interpretaciones posibles -'la señal está presente' o 'no, sólo hay ruido'-. Por ello, la TDS asume que el primer factor que incide en la ejecución y emisión de juicios de detección por parte de los sistemas inmersos en este tipo de tareas, es la discriminabilidad de la señal en relación al ruido.\\

Hablar de la discriminabilidad en tareas de detección implica cuestionar '¿con qué probabilidad la señal y el ruido producen la misma evidencia?' -o bien, '¿qué tan probable es que la señal y el ruido se confundan?'-. En términos de la representación gráfica de las distribuciones de probabilidad de ruido y señal, implica preguntarnos -y tratar de evaluar- qué tanto se sobrelapan las distribuciones. En general, el área de sobrelape de las distribuciones es un reflejo de la incertidumbre contenida en la tarea. Por ejemplo, tal y como se ilustra en la Figura~\ref{fig:Overlap}, si las distribuciones de ruido y señal están muy separadas, el sobrelape entre estas será pequeño y podemos hablar de un entorno con poca incertidumbre -alta discriminabilidad- donde ambos estados del mundo comparten -con poca probabilidad- muy poca evidencia (panel a); por otro lado, si las distribuciones están más juntas, el sobrelape en estas será mayor, indicándonos que el rango de valores-evidencia que puede relacionarse a las dos interpretaciones posibles es más grande -discriminabilidad baja- (panel b).\\

\begin{figure}[th]
\centering
\includegraphics[width=0.55\textwidth]{Figures/Overlap_Small}\\ 
\includegraphics[width=0.55\textwidth]{Figures/Overlap_Big} 
%\decoRule
\caption[El sobrelape Ruido-señal como reflejo de la incertidumbre contenida en las tareas de detección]{La incertidumbre contenida en las tareas de detección depende de la distancia que existe entre las distribuciones de ruido y señal implicadas, puesto que esta determina el área de sobrelape que existe entre las mismas, que refleja el rango de evidencia que puede ser producido y asociado con ambos estados del mundo -con su propia probabilidad-. El panel a presenta un ejemplo donde el sobrelape es pequeño al estar muy separadas las distribuciones, sugiriendo una tarea con poca incertidumbre; en el panel b, se presenta un escenario hipotético donde las distribuciones están más cerca y comparten, por tanto, más evidencia -hay más sobrelape-, cargando la tarea de una mayor incertidumbre.}
\label{fig:Overlap}
\end{figure}

Podemos pensar en la discriminabilidad como producto de la variabilidad con que la TDS asume que los posibles estados del mundo se presentan y perciben por los sistemas detectores. Se dice entonces que la discriminabilidad, como cualidad intrínseca a toda tarea de detección, depende tanto de las propiedades intrínsecas de los estímulos a evaluar -¿qué tanto comparten los estímulos con la señal vs los estímulos sin esta?- como de la precisión con que los sistemas detectores son capaces de discernir entre dichas instancias. Por ejemplo, \\

El soporte de las distribuciones -identificado en la Figura~\ref{fig:Overlap} bajo el nombre de ‘Evidencia’ ( y en las Figuras~\ref{fig:Senal_presentacion} y \ref{fig:Noise}, como 'Puntaje en prueba clínica' e 'Intensidad' en la Figura~\ref{fig:Senal_percepcion}- rara vez se define con precisión,  teniendo una concepción más bien abstracta; La idea general es que cuando queremos detectar una señal particular, comenzamos a recolectar un tipo de evidencia específico a la tarea ante la que nos encontramos. Lo más importante, es que la señal siempre va a estar asociada en mayor medida con dicha evidencia, distribuyéndose siempre en valores situados por encima (a la derecha, en la Figura 1) del ruido.\\
 
  \textbf{2.- El papel del Sesgo: La detección es decisión}\\

Una consecuencia directa de la variabilidad involucrada en el entorno de decisión, es que el desempeño de todo sistema de detección es propenso a cometer errores y emitir un juicio de presencia o ausencia de la señal, que puede no coincidir con el estado del mundo. Dependiendo la correspondencia entre el estado del mundo y el juicio emitido por el sistema de detección, la TDS maneja las clasificaciones de respuesta mostradas en la Tabla~\ref{fig:Mat_Output}; donde las celdas 2 y 3, corresponden a los errores posibles.\\

\begin{figure}[th]
\centering
\includegraphics[width=0.60\textwidth]{Figures/Matriz_Outputs} 
%\decoRule
\caption[Posibles Resultados en una Tarea de Detección]{}
\label{fig:Mat_Output}
\end{figure}

La TDS asume que el organismo fija un criterio de elección a lo largo del eje de la Evidencia, que va a determinar a partir de cuánta evidencia va a juzgar la señal como presente. Dicho criterio se va a representar como una línea transversal que atraviesa ambas distribuciones en una determinada altura, y se le va a identificar con el parámetro k. La TDS asume que los organismos van a fijar esta regla de elección, ponderando la información a la que tienen acceso con la información que poseen sobre la estructura de la tarea (i.e. cómo suele presentarse la señal, qué tan probable es que se presente, etc.)\\


    \begin{itemize}
      \item{Los errores cuestan y los aciertos pagan: Matrices de pago}\\

      \item{Estimados de Probabilidad}

     \end{itemize}

El sesgo se define como la preferencia del sistema a emitir respuestas de un tipo particular (i.e. 'Sí, detecto la señal' o 'No, no está'). La TDS cuenta con dos parámetros dedicados a la medición y evaluación del sesgo de los sistemas sometidos a tareas de detección.

Sin embargo, no todos los errores tienen el mismo costo. Imaginemos el caso de una presa en potencia que busca determinar si el sonido que acaba de escuchar en la maleza corresponde o no con el de un depredador; no hay tiempo que perder, y el costo que dicho organismo tendría que pagar por cometer una falsa alarma (gasto innecesario de energía) o una omisión (morir devorado) es sustancialmente diferente. En este escenario particular, es muy probable que la presa sea mucho más propensa a correr por su vida, juzgando la presencia del depredador a partir de valores menores de evidencia.\\

Esta discrepancia en el peso que se le da a las consecuencias posibles de emitir una u otra respuesta y obtener uno de los cuatro posibles resultados, suele representarse en términos de una matriz de pagos, que nos ayude a definir cuáles son las consecuencias que el organismo buscará evitar o promover, según sea el caso, en mayor medida.\\

Ya sea por los distintos pesos que tengan las posibles consecuencias para el organismo, o porque se tiene una preferencia o predisposición inherente a decretar la presencia o ausencia de la señal, la TDS asume que el desempeño de los organismos que se enfrentan a tareas de detección de señales va a depender tanto de la calidad de la información a la que se tiene acceso (dentro de lo que se incluye la importancia de la variabilidad, que determina tanto la discriminabilidad de la señal como la sensibilidad del sistema ante la misma), como de un sesgo de elección.\\

La localización del criterio en nuestro eje de evidencia recolectada va a estar altamente influida por el sesgo que tenga nuestro sistema. Podemos hablar entonces de dos tipos distintos de sesgo: conservador y liberal. El primero, favorece la emisión de respuestas negativas al desplazar el criterio a la derecha y requerir al sistema la recolección de mayores niveles de evidencia antes de dar por detectada la señal. El segundo, promueve la detección de la señal, situando el criterio de elección hacia la izquierda, emitiendo un juicio de detección con valores menores de evidencia. Nótese que un sistema carente de sesgo, sería aquel que situara su criterio de elección justo en el punto en que las dos distribuciones se juntan, donde la probabilidad de cometer cualquiera de los tipos de acierto y errores, son iguales entre sí.\\


\subsection{Parámetros del modelo}

Como se mencionó previamente, al realizar una tarea de detección existen dos posibles tipos de aciertos: al detectar la señal (Hits) y al rechazar el ruido (Rechazos), y dos posibles tipos de errores: los falsos positivos (Falsas alarmas) y los falsos negativos (Omisiones). La materia prima con base en la cual funciona el modelo propuesto por la TDS, son las tasas de aciertos y errores cometidos durante la tarea, de manera que por cada participante que pasa por una tarea de detección, tenemos cuatro tasas que describen su ejecución:

La Tabla 2 ilustra el cómputo de las cuatro tasas de ejecución, como una relación entre el resultado obtenido y el tipo de ensayo con base en el que se le definió como tal. Es decir, tenemos dos tasas definidas en relación al número total de ensayos con la señal (la tasa de hits y la tasa de omisiones) que nos dicen qué proporción de los ensayos con señal fueron detectados correctamente y cuáles se dejaron pasar; y tenemos dos tasas definidas en relación al total de ensayos con ruido (la tasa de falsas alarmas y la tasa de rechazos correctos) que nos describen la relación de los ensayos con ruido que fueron discriminados correctamente y aquellos que se confundieron con la señal.

Para realizar el análisis de datos, bajo el marco de la TDS, sólo necesitaremos un par de estas tasas: la tasa de hits y la tasa de falsas alarmas. Esto bajo el entendido de que las tasas de omisión y rechazos correctos no son más que su complemento, respectivamente, y que estas dos tasas contienen toda la información que necesitamos sobre el desempeño de los participantes.

La idea general de la importancia de estas tasas de ejecución, es que cada una representa el área de las distribuciones de ruido y señal que cae a la izquierda o derecha del criterio de decisión.

Para la estimación paramétrica se utiliza la misma lógica, pero se sigue el procedimiento inverso. Dado que no podemos observar ni cuantificar de manera directa el criterio usado por los participantes para responder a la tarea, qué tan juntas o separadas se encuentran las distribuciones de ruido y señal para cada participante o qué tipo de sesgo pudieran estar siguiendo, utilizamos las tasas de ejecución para hacer inferencias sobre la localización del criterio, la diferencia entre las medias de ambas distribuciones y el grado en que una respuesta se favorece sobre otra. 

A partir de ahora comenzaremos a hablar sobre cómo se calculan cada uno de los parámetros del modelo, de acuerdo a la teoría clásica que sigue los supuestos estadísticos previamente descritos.  Es importante aclarar que el Graficador de Tasas previamente expuesto no representa la teoría con entera precisión; el propósito de ese primer Graficador es simplemente ilustrar cómo describe la TDS el comportamiento de un sistema que se enfrenta ante una tarea de detección, donde existen dos distribuciones que se sobreponen. El Graficador permite manipular directamente la localización del criterio, con la simpleza que implicaría desplazar una línea vertical sobre el eje de decisión y ver qué consecuencias tiene sobre la probabilidad de obtener un tipo particular de acierto o error.


Antes de ahondar a detalle en los parámetros, hay que declarar un par de supuestos formales que hace la Teoría para facilitar la representación gráfica del modelo y la estimación paramétrica:

\begin{enumerate}
\item En su forma clásica, la TDS asume que las distribuciones de ruido y señal son distribuciones normales.
  \begin{itemize}
  \item 
  \end{itemize}
\item En su forma estándar, se asume que las distribuciones de ruido y señal son equivariantes, (compartiendo una desviación estándar de 1).
  \begin{itemize}
  \item 
  \end{itemize}
\item La distribución de ruido tiene su media en 0. 
  \begin{itemize}
  \item La estimación de todos los parámetros del modelo de detección de señales se hace tomando como referencia la distribución del ruido (con  media 0 y desviación estándar de 1)
  \end{itemize}
\end{enumerate}



\begin{itemize}
\item Discriminabilidad $(d')$

La disc

La discriminabilidad se representa en los modelos de detección de señales con un parámetro $d'$, que representa la distancia entre las medias de las distribuciones de ruido y señal. 

Para encontrar la distancia entre las medias de la distribución de ruido y señal, necesitamos saber el punto en que el criterio toca cada distribución. Para ello, calculamos las probabilidades complementarias a las tasas de hits y falsas alarmas y las traducimos a puntajes Z (Ver Fig. 3). Dado que el puntaje Z funciona como una medida de dispersión de la media, basta con restar el puntaje Z de la intersección del criterio con la distribución de señal a el puntaje Z de intersección con la distribución de ruido para conocer la localización de la media de la señal. Por definición, d’ sólo puede tener valores positivos ya que la teoría asume que la distribución de señal siempre está a la derecha de la distribución de ruido porque contiene una mayor cantidad de la evidencia con base en la cual se hace el juicio de detección de la señal.



\item Criterio  $k$

Una vez que hemos resumido el desempeño de nuestro participante en la tarea de detección, el parámetro cuya estimación resulta más sencilla y directa es el Criterio (k). Entender cómo se computa el parámetro nos requiere únicamente de mantener presente el supuesto de que el Ruido se distribuye normalmente y se va a localizar siempre a la izquierda de la señal, por lo que le asignamos una media de cero para tener un punto de referencia para estimar el espacio en que se desarrollan el resto de los parámetros. \\

Para calcular el criterio lo único que necesitamos es conocer la tasa de Falsas Alarmas, que tal y como mencionábamos en el segmento anterior, nos indica qué proporción de la distribución de ruido cae a la derecha del criterio. Dado que a la distribución de ruido, le fue asignada arbitrariamente una media de cero, podemos asignar un valor al punto en que el criterio corta la distribución de ruido y define las tasas de Rechazos y Falsas Alarmas obtenidas por el participante. Conociendo el área de la distribución de Ruido que cae bajo el criterio, (el complemento de la tasa de Falsas Alarmas, o bien, la Tasa de Rechazos correctos), y sabiendo que la distribución tiene una desviación estándar de 1, podemos convertir el valor de la tasa (que corresponde a la probabilidad de cometer un rechazo correcto, de acuerdo al área bajo la curva) en Puntajes Z y conocer la localización del criterio.\\

 El parámetro k, por lo general, va estar representado por un número natural (un número positivo), que indica en términos de Puntajes Z  la posición del criterio sobre el eje de decisión, relativo a la distribución de ruido con media cero. El criterio sólo tiene valores positivos, porque normalmente se espera que la tasa de falsas alarmas nunca tenga un valor mayor a 0.5 (las consecuencias de una tasa de Falsas Alarmas tan alta, se expondrán con más claridad en el apartado correspondiente a la d’. \\


\item Sesgo - $\beta$

El parámetro más comúnmente utilizado en la literatura para evaluar el sesgo de los participantes en estudios donde se aplica el modelo de detección de señales al análisis de tareas experimentales, es Beta ($\beta$). Se define como una razón entre la probabilidad con que la evidencia  \\

$\beta = \frac{p(Signal)}{p(Noise)}$

si $\beta<1$ o C<0, sabemos se trata de un sesgo liberal y si $\beta>1$, C>0, hablamos de un sesgo conservador.\\

\item Sesgo - $C$


\end{itemize}   %Terminan los parametros



%----------------------------------------------------------------

\subsection{Tareas de detección}

\begin{itemize}
\item Tareas de detección binaria 

En el laboratorio, la  TDS se estudia a partir  de tareas de detección donde se expone a un  sujeto  a  N  número  de  ensayos,  (comprendidos  por  n  ensayos con  sólo  ruido  y  n  ensayos donde  el  ruido  viene  acompañado  de  la  señal)  ante  los  que  se  le  pide  al  participante  que responda eligiendo una de dos opciones: Sí está la señal o No está la señal. En estos escenarios controlados,  el  experimentador  decide  la  proporción  de  ensayos  con  y  sin  señal  que  se presentarán, así como la matriz de pagos que definirán la utilidad de sus aciertos y errores. \\


\item Tareas con escala de confianza

Un segundo procedimiento comúnmente utilizado en tareas de detección implica añadir un paso adicional a la tarea de los participantes. 

\parencite{McNicol}

\item Tarea con elección forzada entre dos alternativas.



\end{itemize}














\section{Panorama general del estudio de Memoria de Reconocimiento}

\subsection{Teoría del Umbral}

Lorem ipsum dolor sit amet, consectetur adipiscing elit. Aliquam ultricies lacinia euismod. Nam tempus risus in dolor rhoncus in interdum enim tincidunt. Donec vel nunc neque. In condimentum ullamcorper quam non consequat. Fusce sagittis tempor feugiat. Fusce magna erat, molestie eu convallis ut, tempus sed arcu. Quisque molestie, ante a tincidunt ullamcorper, sapien enim dignissim lacus, in semper nibh erat lobortis purus. Integer dapibus ligula ac risus convallis pellentesque.

%-----------------------------------
%	SUBSECTION 1
%-----------------------------------
\subsection{Teoría del Procesamiento Dual}

La Teoría del Procesamiento Dual (TPD; o PDT por sus siglas en inglés) sostiene que existen dos procesos fundamentales involucrados en todo juicio de reconocimiento: la recolección y el análisis de familiaridad. El primero corresponde a la extracción de rasgos y detalles específicos del estímulo a evaluar, (i.e. el estímulo que queremos determinar si se ha visto antes, o no), un proceso que requiere cierto tiempo; el segundo, es entendido como un fenómeno más o menos automático y casi instantáneo donde el sistema identifica el estímulo como 'familiar' y decide que lo ha reconocido de alguna experiencia previa. 

%-----------------------------------
%	SUBSECTION 2
%-----------------------------------

\section{Teoría de Detección de Señales en Memoria de Reconocimiento}
Aplicar la TDS al estudio de la Memoria de Reconocimiento implica asumir que existe tal cosa como una 'fuerza de memoria' ('memory strength', en inglés) que refleja el grado en que un estímulo cualquiera es percibido como 'familiar' para el sistema que busca emitir un juicio de detección. La fuerza de memoria evocada por cada estímulo se compara con un criterio de elección para que el sistema pueda decidir si lo reconoce, o no, como 'elemento antes visto'.\\ 

\begin{figure}[th]
\centering
\includegraphics[width=0.60\textwidth]{Figures/RM_SDT_1} 
\decoRule
\caption[SDT en Memoria de Reconocimiento]{Modelo de Detección de Señales aplicado al estudio de Memoria de Reconocimiento}
\label{fig:RM_SDT_1}
\end{figure}

La Figura~\ref{fig:RM_SDT_1} ilustra la forma en que los supuestos de la TDS, expuestos a detalle en el Capítulo 1, se aplican al estudio de la Memoria de Reconocimiento. Las idea central se mantiene en escencia y simplemente sustituimos algunos conceptos: al tratarse de una tarea de reconocimiento, decimos que la 'señal' a detectar es cualquier estímulo antes visto (i.e. 'estímulos viejos', como suelen identificarse en la literatura); el 'ruido' son los estímulos nuevos, que podrían -o no- confundirse con los primeros; el 'eje de decisión' a lo largo del cual se sitúan las dos distribuciones de ruido y señal, se convierte en un 'eje de familiaridad' que va a representar distintos grados de lo que parece ser una 'fuerza de memoria'. También se mantienen las ideas centrales propuestas por el modelo: los estímulos ya antes vistos tendrán valores más altos de 'familiaridad' que aquellos nunca antes vistos, admitiendo sin embargo la posibilidad de que éstos últimos puedan llegar a confundirse con los estímulos viejos, por ejemplo, si comparten algun rasgo en particular; una vez más, dicha variabilidad en la presentación y lectura de los estímulos a evaluar se refleja en la idea de que existen distintos rangos de familiaridad que pueden ser producidos por los estímulos viejos o conocidos, con cierta distribución de probabilidad. Y de la misma forma, la emisión de un juicio se entiende como resultado de comparar la 'familiaridad' de cada estímulo a evaluar con un criterio de elección particular, donde sólo si éste se rebasa, se juzga el estímulo como 'ya antes visto'.\\

Como se mencionó en el Capítulo 1, la TDS en su forma típica define las distribuciones de ruido y señal como distribuciones Gaussianas con varianzas iguales (i.e. con una misma desviación estándar). A propósito de ello, podemos hablar de una particularidad que tiene la aplicaicón de la TDS a estudios de memoria de reconocimiento, que ha sido constante y consistentemente demostrada con los datos: la distribución de Estímulos Viejos suele mostrar mayor desviación estándar que la distribución de Ruido, justo como se muestra en la Figura~\ref{fig:RM_SDT_2}.\\


\begin{figure}[th]
\centering
\includegraphics[width=0.60\textwidth]{Figures/RM_SDT_2} 
\decoRule
\caption[SDT en Memoria de Reconocimiento (Varianzas Desiguales)]{Modelo de Detección de Señales con varianzas desiguales aplicado al estudio de Memoria de Reconocimiento}
\label{fig:RM_SDT_2}
\end{figure}


\subsection{Implicaciones y conflictos}

Me permitiré, por un momento, salirme del molde académico y hablar un poco de todo el proceso que hubo detrás de la realización del presente trabajo. Cuando comencé a revisar literatura para decidir cuál podría ser mi proyecto de tesis, descubrí la maravilla de la Teoría de Detección de Señales. 

Sin embargo, pese a la flexibilidad de la que dispone la TDS para aplicarse a, aparentemente, cualquier tarea de decisión binaria, donde se admita el papel de la incertidumbre en la emisión de un juicio de detección, su aplicación al campo de la Memoria de Reconocimiento no ha estado excenta de críticas.

Conceptualmente, es fácil pensar en una tarea de reconocimiento en términos del modelo de detección de señales. Sin embargo, cuando se revisan las implicaciones teóricas que conllevaría aceptar que la memoria de Reconocimiento funciona como un procedimiento cualquiera de detección de señales, no son del todo claras. 

\section{El Efecto Espejo}

“is usually interpreted in terms of (unequal variance) signal detection theory (SD) in which case it implies that the order of the underlying old item distributions mirrors the order of the new item distributions” (DeCarlo, L.,  2007)\\
Teoría de Atención /Verosimilitud: Un modelo de marcaje de rasgos, determinado por un muestreo diferencial dada la condición (H-frequency, L-frequency)\\
Teoría de Atención / Verosimilitud; demasiado complicada, sus supuestos no son necesarios (Decarlo, 2007; Hintzman, 1994; Murdock, 1998) Intercambio de papers Hintzman-Glanzer\\
‘The mixture model’ (DeCarlo, 2007) – Extensión de la SDT, una extensión mezclada.\\
Between vs Within condition discussion (Listas separadas o mezcladas)\\
Between condition: Problemas (1) No se puede descartar la posibilidad de que el criterio de respuesta difiera a lo largo de las condiciones. Y (2) las distribuciones subyacentes no necesariamente están escaladas de la misma forma a lo largo de las dos condiciones.\\
“one cannot compare the values of d’ across the two conditions without further assuming that the variance of the reference distributions (LN and HN) are the same, which does not appear to be the case. (DeCarlo,2007)  

%----------------------------------------------------------------------------------------
%	SECTION 1
%----------------------------------------------------------------------------------------

\subsection{Evidencia recolectada}

Teóricamente, este procedimiento nos permitiría inferir no uno, sino seis sub-criterios de elección que estarían permeando a partir de cuánta evidencia el participante se siente más, o menos seguro, de si los estímulos evaluados contienen la señal o sólo ruido. De encontrarse evidencia del Efecto Espejo en los experimentos realizados, se esperaría encontrar el mismo patrón en el promedio de los puntajes de confianza asignados a cada una de las cuatro categorías de estímulos (formadas por la combinación Condición (A o B) x Tipo de ensayo (S o R)) reportados en la literatura en Memoria de Reconocimiento \parencite{Glanzer1990, Glanzer1993}. \\

La Figura~\ref{fig:Ejem_Crit} 

\begin{figure}[th]
\centering
\includegraphics[width=0.55\textwidth]{Figures/Ejem_Criterios}
%\decoRule
\caption[Representación gráfica de los sub-Criterios para la emisión de puntajes de confianza ]{}
\label{fig:Ejem_Crit}
\end{figure}



\subsection{Relevancia e implicaciones}
A primera vista el patrón de respuestas identificado como Efecto Espejo podría parecer trivial: Si sabemos que lo que distingue a las condiciones de estímulos 

%-----------------------------------
%	SUBSECTION 1
%-----------------------------------
\subsection{Algunos modelos desarrollados para dar cuenta del Efecto Espejo}

\begin{itemize} 
\item
\item
\end{itemize}


