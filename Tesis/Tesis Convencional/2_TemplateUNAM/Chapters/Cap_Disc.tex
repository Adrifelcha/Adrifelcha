% Chapter Template

\chapter{Discusión} % Main chapter title

\label{Cap_Disc} % Change X to a consecutive number; for referencing this chapter elsewhere, use \ref{ChapterX}

En el presente trabajo se presentaron los resultados obtenidos en dos experimentos desarrollados como variaciones en la presentación de una misma tarea de detección perceptual (visual), donde se incluyeron dos clases de estímulos A y B construidas con base en una revisión de la literatura en ilusiones ópticas para que representaran dos niveles de discriminabilidad ($d'(A)>d'(B)$) entre los cuales pudiera compararse el desempeño de los participantes. En los experimentos propuestos, los participantes tuvieron que registrar sus respuestas al problema de detección a través de dos protocolos: una tarea Sí/No y una tarea con Escala de Confianza. El diseño experimental propuesto fue elaborado con el propósito de emular la estructura de los estudios en Memoria de Reconocimiento donde se ha reportado evidencia del fenómeno ahora conocido como Efecto Espejo; una regularidad en los patrones de respuesta observados en tareas de reconocimiento con dos clases de estímulos que difieren en la precisión con que sus elementos son reconocidos y que sugiere un orden simétrico en el despliegue de las distribuciones Ruido y Señal de cada clase sobre el eje de la evidencia.\\

Los patrones identificados en la literatura del Efecto Espejo en Memoria de Reconocimiento (el único dominio donde ha sido estudiado) fueron hallados en las tareas perceptuales aquí presentadas, en una proporción significativa contra el azar (entre el $85\%$ y el $90\%$ de los datos analizados).\\

Para validar la pertinencia del análisis de los datos obtenidos en los experimentos realizados como evidencia de la generalizabilidad de Efecto Espejo fuera de la Memoria de Reconocimiento, se realizó un análisis \textit{comprobatorio} que tuvo como objetivo evaluar la eficacia de la manipulación experimental propuesta con base en la literatura para el diseño de las clases de estímulos a comparar. De acuerdo con lo que se esperaba, los estímulos Señal de la clase A propuesta mostraron ser consistentemente más fáciles de detectar respecto del Ruido(A) que los estímulos Señal de la clase B. La robustez de esta afirmación a la luz de los datos fue confirmada tanto mediante la realización de pruebas t de muestras independientes para cada uno de los experimentos realizados, como a partir de la construcción de un modelo bayesiano jerárquico identificado como Modelo Delta, que incorporó al modelo bayesiano estándar de detección de señales una estructura jerárquica sobre los parámetros $d'$ y $c$ y un parámetro determinista encargado de evaluar las diferencias entre las $\mu d'$ estimadas por cada clase de estímulos.\\

Una vez establecida la validez del diseño experimental propuesto como una emulación de las tareas desarrolladas en Memoria de Reconocimiento donde se presenta evidencia del Efecto Espejo, se prosiguió a evaluar la significancia de los patrones de respuesta registrados en los experimentos realizados. El análisis de datos se realizó tanto mediante la replicación de los análisis frecuentistas reportados en la literatura, como mediante la construcción de modelos y la conducción de pruebas estadísticas bayesianas.\\

Primero, en cuanto a la evaluación de las diferencias encontradas entre las tasas de ejecución registradas por cada clase de estímulos, se realizaron un par de pruebas t de muestras independientes por cada Experimento y se desarrolló un modelo bayesiano -identificado como Modelo Tau- que agregó un par de parámetros deterministas $\tau$ al modelo bayesiano estándar, para computar las diferencias entre las tasas de Hits y Falsas Alarmas estimadas. En este primer punto de análisis se pudieron detectar algunas dferencias en la lectura de las conclusiones extraídas a partir de los datos y su robustez derivadas de las discrepancias entre el funcionamiento de los análisis frecuentistas y bayesianos realizados:\\

\begin{enumerate}
\item El análisis frecuentista se desarrolla en torno a la comparación de las tasas de ejecución registradas por los participantes ante cada clase de estímulos, definidas de forma determinsita a partir del número de Hits y Falsas Alarmas obtenidos dentro del total de ensayos con Ruido y Señal.\\

Las pruebas t realizadas para evaluar la significancia estadística de las diferencias encontradas entre los promedios de las tasas computadas, se enfrentan con el problema de que los datos a comparar se encuentran restrigidos dentro de un rango de $0$ a $1$. Más aún, de acuerdo con los supuestos de la SDT sobre el despliegue de las distribuciónes de Ruido y Señal sobre el eje de evidenncia, las tasas de ejecución se presentan dentro de rangos aún más restringidos, siendo que los aciertos (Hits y Rechazos Correctos) caen por arriba del $0.5$, y los errores (Omisiones y Falsas Alarmas), por debajo. Esto representa un problema para la comparación de los resultados encontrados, porque puede darse el caso de que las diferencias entre los datos registrados no sean lo suficientemente grandes como para considerárseles significativas, con independencia de si se presentan de forma consistente en la mayoría de los datos individuales.\\

Para solucionar el problema del efecto de suelo y techo que podría estar mermando la evaluación de los datos obtenidos como evidencia del impacto de la manipulación experimental sobre el desempeño de los participantes, los datos crudos a comparar suelen transformarse a una nueva Escala, de manera que la distancia entre estos aumente, pero manteniendo la proporción de los mismos intacta. En el caso específico de la literatura que aborda el Efecto Espejo en Memoria de Reconocimiento, se ha optado por implementar una transformación arcoseno de las tasas de respuesta registradas antes de someterlas al análisis estadístico.\\

De acuerdo con las pruebas t realizadas en el presente estudio para comparar los promedios de las transformaciones arcoseno de las tasas de ejecución registradas por cada clase de estímulo, se encontraron diferencias significativas en la ejecución de los participantes a la tarea Sí/No dependientes de la clase de estímulo a evaluar en ambos experimentos, reportándose diferencias altamente significativas entre las tasas de Hits y diferencias apenas significativas (justo por debajo de $p=0.05$) para las tasas de Falsas Alarmas.\\

\item El modelamiento bayesiano del problema de detección de señales abandona la interpretación determinista de las tasas de ejecución calculadas a partir de los datos registrados como reflejo directo del área de las distribuciones de Ruido y Señal que caen por encima del criterio de elección. En su lugar, asume que dicha CDF debe ser estimada como una probabilidad oculta que permea la observación del total de Hits y Falsas Alarmas por participante.\\

El Modelo Tau presentado en este trabajo parte del modelo bayesiano estándar de detección de señales, que estima el valor de las CDF de las distribuciones de Ruido y Señal que caen por encima del criterio de elección a partrir del número de Hits y Falsas Alarmas observados en cada participante, y añade un par de paráetros $\tau$ que computan las diferencias entre las probabilidades ocultas estimadas tras la emisión de Hits y Falsas Alarmas. Las estimaciones realizadas por el modelo acerca de las diferencias en el desempeño de los participantes para las distintas clases de estímulos (los valores de $\tau$) toman en cuenta la naturaleza probabilística del modelo de detección de señales, al asumir que los datos observados son extracciones de las distribuciones de Señal y Ruido subyacentes a la tarea. Con ello, el análisis de los datos obtenidos a la luz de los resultados arrojados por el Modelo Tau presenta una ventaja considerable sobre el análisis frecuentista, pues al tratar los datos obtenidos como resultado de un proceso probabilístico, escapa del problema de suelo y techo.\\

Los resultados arrojados por el Modelo Tau confirmaron a grandes razgos lo reportado por las pruebas t frecuentistas: la mayoría de las densidades de probabilidad posterior estimadas para los participantes de los experimentos 1 y 2 se sitúan por encima del punto de \textit{no diferencias} ($\tau = 0$). Sin embargo, y especialmente en términos de las diferencias encontradas en la emisión de Falsas Alarmas, dejó en evidencia la dispersión de las estimaciones de las $\tau$ individuales, siendo que estas presentaban valores no muy alejados de $\tau = 0$.\\
\end{enumerate}

La discrepancia en la precisión con que los análisis llevados a cabos permiten evaluar la evidencia del Efecto Espejo encontrada se atribuye a las diferencias entre la aproximación determinista adoptada por los métodos de análisis frecuentistas clásicos y el enfoque probabilístico de los métodos bayesianos. Mientras que las pruebas t frecuentista requieren de la transformación arcoseno de los datos a promediar para su análisis (las tasas de ejecución), el modelo bayesiano toma como materia prima el número total de Hits y Falsas Alarmas obtenido por los participantes y estima a partir de ellos la proporción de las distribuciones de Señal y Ruido que caen por encima del criterio. \\

Posteriormente, se evaluaron las diferencias entre los promedios de los Puntajes de Confianza asignados a los estímulos con Señal y Ruido de cada clase, mediante la realización de pruebas t de muestras independientes, frecuentistas y bayesianas. En ambos casos, se llegó a la conclusión de que los datos analizados proporcionan evidencia sobre las diferencias entre los puntajes registrados en cada clase de estímulo. En las pruebas t frecuentistas, se observaron \textit{p-values} por debajo de $0.05$ para todas las comparaciones realizadas y en las pruebas t bayesinaas, se computaron valores del Factor de Bayes que confirman la acumulación de evidencia a favor de las diferencias en el desempeño de los participantes. Sin embargo, de acuerdo con los estándares establecidos para la identificar la robustez de la información proporcionada por el Factor de Bayes, la evidencia obtenida en los dos Experimentos, en los estímulos con Señal y Ruido es apenas anecdótica (con excepción de las diferencias en los puntajes asignados a los estímulos Señal del Experimento 2, que difieren notablemente).\\

Finalmente, se revisó que no existieran diferencias en el tiempo de respuesta invertido por los participantes en cada una de las clases de estímulos. Dicha evaluación constituye uno de los análisis de control más frecuentemente reportados en los estudios de Memoria de Reconocimiento que presentan evidencia del Efecto Espejo, como una medida para garantizar que las diferencias en el desempeño de los participantes puedan atribuirse a discrepancias en la precisión con que los elementos que componen cada clase de estímulos son reconocidos y no al cuidado con que los participantes emitieron sus respuestas ante cada una. Para ello, se realizaron múltiples pruebas t de muestras independientes que evaluaron las distintas relaciones que pudieron haberse dado entre los tiempos de respuesta y el tipo de etímulos presentados en los experimentos. De acuerdo con dichas pruebas, no se encontraron diferencias significativas en el tiempo que se tomaron los participantes para responder a la tarea binaria o a la tarea con Escala de Confianza entre clases de estímulos, ni entre los estímulos con Señal y Ruido contenidos en cada clase.\\ 

Los modelos en Memoria desarrollados a partir del modelo de detección de señales, entienden las tareas de reconocimiento como instancias de un problema de detección de señales, donde los participantes tienen que detectar los estímulos que ya le habían sido presentados con anterioridad, \textit{reconociéndolos}. El Efecto Espejo, como un fenómeno exclusivamente reportado en la literatura de Memoria de Reconocimiento, ha sido abordado en la literatura tanto en términos de la  información que podría añadir a lo que se sabe sobre el funcionamiento de la Memoria de Reconocimiento, como un referente para juzgar la adecuación de los modelos en Memoria derivados de la SDT. La evidencia del Efecto Espejo encontrada en los experimentos realizados sugiere que este constituye una regularidad propia de la aplicación de la SDT en el análisis y comparación del desempeño de los participantes sometidos a un protocolo de detección con dos niveles de $d'$. A la luz de los resultados obtenidos, se enfatiza la importancia de abordar el Efecto Espejo en términos de su interacción con los supuestos planteados por la SDT y las implicaciones que trae consigo en cuanto al estudio del las situaciones de detección de señales como un problema adaptativo, común para todos los organismos como sistemas que buscan optimizar su comportamiento a traves de distintos planteamientos de dichos probleas.\\

El presente trabajo es el primero en explorar la generalizabilidad del Efecto Espejo reportado en Memoria de Reconocimiento a otras áreas dentro de la Psicología Experimental donde la SDT ha sido aplicada, siendo también el primero en presentar evidencia a favor de esta. Las implicaciones de la extensividad del Efecto Espejo a diferentes protocolos, clases de estímulos y fenómenos psicológicos en términos de su incorporación a los supuestos establecidos por la SDT deben ser revisados en investigaciones futuras. Por otro lado, también puede atribuirse al presente trabajo el ser el primero en evaluar la evidencia del Efecto Espejo mediante el uso de metodos bayesianos, con la realización de análisis estadísticos y el desarrollo de modelos. Con ello, se presenta un referente empírico sobre las ventajas que ofrece el uso de herramientas derivadas de la estadística bayesiana, especialmente en el análisis de datos obtendos en tareas donde se asume una estructura probabilística que permea en el desempeño de los participantes, como es el caso de las situaciones de detección tal y como se conciben bajo el marco de la SDT.\\









