\documentclass[letterpaper]{article}
\usepackage{graphicx}
\usepackage[utf8]{inputenc}
\usepackage{natbib}
\bibliographystyle{apalike}
%\usepackage{apalike}

\begin{document}
\title{El Efecto Espejo en Percepción: No es otro estudio en Memoria de Reconocimiento}\\
\author{por Chávez De la Peña Adriana Felisa}\\
\maketitle

En estudios de Memoria de Reconocimiento donde se ha aplicado la Teoría de Detección de Señales para comparar el desempeño de los participantes entre clases de estímulos que se distinguen por la presición con que sus elementos se reconocen al ser presentados más de una vez -habiendo una clase que se reconoce con mayor facilidad (A) que en la otra (B)-, se ha encontrado evidencia consistente de que dicha discrepancia se refleja simultáneamente en el número de Hits y Falsas Alarmas cometidas en cada una -$F.Alarmas(A) < F.Alarmas (B) < Hits(B) < Hits(A)$ - un patrón que sería identificado en la literatura como Efecto Espejo. Sin embargo, la extensión de este patrón a otras áreas no ha sido explorada todavía y su estudio e interpretación se ha restringido al dominio específico de las tareas de reconocimiento. En el presente trabajo se muestra evidencia del Efecto Espejo en una tarea de detección perceptual que incluye protocolos de respuesta binaria y de escala de confianza, donde los niveles de discriminabilidad a comparar fueron construidos con base en lo que se sabe sobre la Ilusión de Ebbinghaus. Los resultados obtenidos fueron evaluados tanto a partir de la replicación del análisis clásico (pruebas t y ANOVAS), como con la construcción de Modelos Bayesianos.\\

\textbf{Keywords:} Teoría de Detección de Señales, Memoria de Reconocimiento, Efecto Espejo, Réplica, Modelos Bayesianos.\\
-\\
-\\
-\\
-\\

\begin{itemize}
\item \textbf{Institución:} Facultad de Psicología, Universidad Nacional Autónoma de México.
\item \textbf{Modalidad:} Ponencia (Simposio).
\item \textbf{Correo Electrónico:} adrifelcha@gmail.com
\end{itemize}
\end{document}