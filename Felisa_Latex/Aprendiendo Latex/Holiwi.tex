\documentclass[letterpaper]{article}
\usepackage{graphicx}
\usepackage[utf8]{inputenc}
\usepackage{natbib}
\bibliographystyle{apalike}
%\usepackage{apalike}
\begin{document}
\title{Aprendiendo a usar Latex}
\author{por Adriana F Chávez De la Peña}
\maketitle
Hola, bienvenidos a este hermoso ejemplo de como hacer un pdf decente en latex.



Segun Marco, existen tres formas de insertar una ecuacion:



La primera, es dentro de un parrafo. PAra lo cual bsata colocar entre signos de pesos lo que queremos aparezca en letra bonita, por ejemplo: $A+B=C$


 
La segunda, es crear un espacio llamado equation dentro del texto, para que esto quede claro tendran que abrir el codigo de este documento. Dentro de un ambiente equation, se permite agregar una etiqueta a la qcuacion.

\begin{equation}
Love=\frac{1561521}{154191}
\label{LoveEq}
\end{equation}

La tercera es sin referencia
\[Love=\frac{418511248648}{464135641866}\]


Regresando a la segunda, la ventaja es que udes referirlo: Luego entonces, de acuerdo a la ecuacion \ref{LoveEq}

\begin{figure}
\centering
\includegraphics[width=0.5\textwidth]{Imagenes/Gatitos.jpeg}
\caption{Este es un Gatito Bonito, como este ejemplo}
\label{Gatito1}
\end{figure}




Como podemos ver en la imagen \ref{Gatito1}, esta es una practica bonita  \cite{chavez2016}





Todo esto queda claro en mi famosa y aclamada obra \citep{chavez2016}


\bibliography{Referenceeeees}



\end{document}