\documentclass[letterpaper]{article}
\usepackage{graphicx}
\usepackage[utf8]{inputenc}
\usepackage{natbib}
\usepackage[a4paper, total={6.5in, 10in}]{geometry}
\bibliographystyle{apalike}
%\usepackage{apalike}
\begin{document}

\begin{center}
{\Huge Crema de Zanahorias}\\\\

{\huge (sin lácteos)}\\

por mi mami\\
\end{center}

\hrulefill
\begin{center}
\textbf{\Large Ingredientes}\\
\end{center}
\hrulefill


\begin{itemize}
\item Zanahorias
	\begin{itemize}
		\item Zanahorias medianas
		\item 10 unidades alcanzan para 5 personas
	\end{itemize}
\item Retazo de pollo completo (para el caldo)
	\begin{itemize}
		\item Pedir que le quiten la grasa
		\item Pedir que le corten las puntas a las alas
	\end{itemize}
\item 1/2 diente de ajo 
\item 1/8 de cebolla
\item Galletas Bombachas ('de globo' o 'para crema')
\item Una cucharada de Knorr Suiza\\
\end{itemize}

\hrulefill \\
\begin{center}
\textbf{\Large Instrucciones}
\end{center}
\hrulefill

\begin{enumerate}
\item Limpiar las piezas del Retazo completo de pollo (quitarles los excesos de grasa y piel)
	\begin{itemize}
	\item Quitar los excesos de grasa con ayuda de unas buenas tijeras
	\item Si alguna de las alitas tiene plumas, puedes quitarlas más fácilmente si las quemas (usa un encendedor)
	\end{itemize}
\item Poner a hervir el Retazo de pollo en la \textbf{olla exprés}, con suficiente agua (aproximadamente, 1/3 de la olla) y una cucharada de Knorr Suiza.
	\begin{itemize}
	\item Dejar hirviendo y apagar la lumbre \textbf{10 minutos después} de que la olla comience a sonar.
	\item Una vez apagada la llama, espera a que la olla deje de sonar para destaparla.
		\begin{itemize}
		\item Para verificar que puedes abrir la olla sin problema, inclina la valvula de escape hasta dejar que se salga todo el vapor.
		\item Una vez que el vapor ha dejado de salir (o que la olla ya lleva un buen rato apagada y ha dejado de sonar), puedes abrirla!
		\end{itemize}
	\end{itemize}
\item Cortar los extremos de las zanahorias, pelarlas con el rallador y ponerlas a hervir en una olla con agua, hasta que se pongan blandas y puedan atravesarse con un tenedor.
\item Una vez que las zanahorias estén cocidas y el caldo de pollo esté listo, picar las zanahorias en pedazos pequeños y ponerlo a licuar junto con el 1/8 de cebolla y el 1/2 diente de ajo, con el caldo de pollo (sin el retazo), incrementando paulatinamente la proporción de caldo en la licuadora, conforme se van licuando las zanahorias.
	\begin{itemize}
		\item Según mamá, la zanahoria tiene un sabor muy suave y es peligroso poner 'demasiada cebolla' o 'demasiado ajo', porque son sabores más marcados.
		\item La cantidad de caldo de pollo que debe ponérsele a la licuadora depende de la consistencia que se observe (y desee).
	\end{itemize}
\item Derretir un poco de mantequilla en la olla donde procederemos a vaciar el contenido de la licuadora. La crema resultante debe menearse con relativa frecuencia para que no se pegue a los bordes de la olla y, de juzgarse necesario, puede agregársele otra cucharada de Knorr Suiza.
\end{enumerate}




\end{document}


