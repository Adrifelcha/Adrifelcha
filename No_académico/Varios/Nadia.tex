%%%%%%%%%%%%%%%%%%%%%%%%%%%%%%%%%%%%%%%%%
% Article Notes
% LaTeX Template
% Version 1.0 (1/10/15)
%
% This template has been downloaded from:
% http://www.LaTeXTemplates.com
%
% Authors:
% Vel (vel@latextemplates.com)
% Christopher Eliot (christopher.eliot@hofstra.edu)
% Anthony Dardis (anthony.dardis@hofstra.edu)
%
% License:
% CC BY-NC-SA 3.0 (http://creativecommons.org/licenses/by-nc-sa/3.0/)
%
%%%%%%%%%%%%%%%%%%%%%%%%%%%%%%%%%%%%%%%%%

%----------------------------------------------------------------------------------------
%	PACKAGES AND OTHER DOCUMENT CONFIGURATIONS
%----------------------------------------------------------------------------------------

\documentclass[
10pt, % Default font size is 10pt, can alternatively be 11pt or 12pt
spanish,
a4paper, % Alternatively letterpaper for US letter
twocolumn, % Alternatively onecolumn
landscape % Alternatively portrait
]{article}

\input{structure.tex} % Input the file specifying the document layout and structure

%----------------------------------------------------------------------------------------
%	ARTICLE INFORMATION
%----------------------------------------------------------------------------------------

\articletitle{Reporte de la Evaluación Preliminar de Daños} % The title of the article
\articlecitation{\cite{dubner1999assessing}} % The BibTeX citation key from your bibliography

\datenotesstarted{Declarado el 1° Abril, 2017} % The date when these notes were first made
\docdate{\datenotesstarted; rev. \today} % The date when the notes were lasted updated (automatically the current date)

\docauthor{California - Tormentas de Invierno Severas, Inundaciones y Deslindes de lodo} % Your name

%----------------------------------------------------------------------------------------

\begin{document}

\pagestyle{myheadings} % Use custom headers
\markright{\doctitle} % Place the article information into the header

%----------------------------------------------------------------------------------------
%	PRINT ARTICLE INFORMATION
%----------------------------------------------------------------------------------------

\thispagestyle{plain} % Plain formatting on the first page

\printtitle % Print the title

%----------------------------------------------------------------------------------------
%	ARTICLE NOTES
%----------------------------------------------------------------------------------------

En Marzo 19, 2017, el Gobernador Edmund G. Brown Jr. solicitó una declaración de Desastre Mayor debido a las severas tormentas de invierno, inundaciones y deslindes de lodo ocurridos durante el periodo de Febrero 1-23 del 2017. El gobernador solicitó una declaración para conseguir Asistencia Pública para 42 condados y Mitigación de Daños en todo el estado. Durante el periodo del 28 de Febrero al 15 de Marzo, 2017, se condujeron Evaluaciones Preliminares de Daños (PDAs, por sus siglas en inglés) en conjunción entre los gobiernos federal, estatal y local, en los condados solicitados y cuyos resultados se resumen acontinuación. Los PDAs estiman los daños inmediatamente después de la ocurrencia de un evento y son considerados, junto con otros tantos factores, para determinar si el desastre evaluado tiene la magnitud y es tan severo como para que la respuesta efectiva al mismo quede fuera de las capacidades del estado y los gobiernos locales afectados, haciendo necesaria la asistencia del gobierno Federal.\\

En abril 1, 2017, el Presidente Trump declaró la existencia de un Desastre Mayor en el estado de California. Con esta declaración, se dio paso a la solicitud de Asistencia Pública realizada por el gobernador, para los estados, los gobiernos locales elegibles y ciertas organizaciones privadas sin fines de lucro, bajo un acuerdo de costo-compartido en los trabajos de emergencia y la reparación o remplazo de las facilidades dañadas por las severas tormentas de invierno, inundaciones y deslindes de lodo en Alameda, Alpine, Amador, Butte, Calaveras, Colusa, Contra Costa, Del Norte, El Dorado, Flenn, Humboldt, Kings, Lake, Lassen, Marin, Mariposa, Merced, Modoc, Monterey, Napa, Nevada, Plumas, Sacramento, San Benito, San Joaquin, San Luis Obispo, San Mateo, Santa Barbara, Santa Clara, Santa Cruz, Shasta, Sierra, Siskiyou, Solano, Sonoma, Sainslaus, Sutter, Tehama, Trinity, Tuolumne, Yolo y Yuba Counties. Esta declaración también permitió que la asistencia del Programa Mayor de Mitigación de Peligros solicitada por el Gobernador, quedara disponible para que se tomaran las medidas correspondientes de mitigación del peligro a lo largo del estado.\\

\section*{Resúmen de la Información de la Evaluación de Daños usada para determinar la declaración de un Desastre Mayor}\\ % Unnumbered section\\

\textbf{Asistencia individual - (No requerida)}\\

\begin{itemize}
\item Numero total de residencias impactadas
Destruidas -\\
Con daño mayor - \\
Con daño menor - \\
Afectadas -\\
\item Porcentaje de residencias aseguradas
\item Porcentaje de hogares de bajos recursos
\item Porcentaje de hogares con dueño 
\item Estimado del costo total de asistencia individual : NA
\end{itemize}

\textbf{Asistencia Publica}

\begin{itemize}
\item Impacto primario: Medidas de protección de emergencia
\item Estimado del costo total de asistencia publica : $\$ 537,119,780$
\item Impacto a lo largo del estado per capita: $\$14,42$
\item Indicadores del impacto per capita a lo largo del estado: $\$1,43$
\item Impacto per capita a lo largo de cada condado : \textit{La lista viene en el documento :P}
\item Indicador del impacto per capita a lo largo del condado: $\$3,61$
\end{itemize}

\section*{Notas al pie}

\begin{itemize}
\item El proceso de Evaluación de Daño Preliminar (PDAs) es un mecanbismo utilizado para determinar el grado de impacto y la magnitud del daño, así como las necesidades resultantes en los individuos, negocios, sector público y la comunidad en general. La información recolectada es utilizada por el Estado como base para que el Gobernador pueda solicitar una declaración de Desastre o Emergencia Mayor, y para que el Presidente determine la respuesta que se dará a dicha solicitud del Gobernador.\\
\item \textit{Este es un punto legal que dice algo así:} Cuando el Gobernador solicita asistencia de Desastre mayor bajo la Ley de Socorro de Desastres Robert T. Stafford y la Ley de Asistencia de Emergencia, , mientras su corrección esté bajo revisión (de acuerdo a la Ley Stafford), se considera un número de factores para determinar si la asistencia se garantizará. Estos factores son señadados en las regulaciones del FEMA. El presidente tiene la discreción máxima y la autoridad para la toma de decisiones necesaria para declarar Desastres Mayores y Emergencias bajo el Acta de Stafford.\\
\item Grado de daño en las residencias impactadas:\\
\begin{itemize}
\item Destruídas - Pérdida total de la estructura, que es económicamente imposible reparar la estructura, o un fallo completo de los componentes estructurales principales (Por ejemplo, el colapso de las paredes de la planta baja, las paredes, o la azotea)\\
\item Daño Mayor- Un fallo sustancial en los elementos estructurales de la residencia (paredes, piso) o daños que pueden tomar más de treinta días en repararse.\\
\item Dali Menor - La casa está dañada e inhabitable, pero puede hacerse habitable en un breve periodo de tiempo con ciertas reparaciones
\item Afectado - Algunos daños a la estructura y su contenido, pero es habitable\\
\end{itemize}
\item Por ley, la asistencia para Desastres federal no puede duplicar la cobertura asegurada\\
\item Poblaciones especiales, como la de bajos recursos, los ancianos o los desempleados, pueden incrementar los indicadores de necesidad de ayuda.\\
\item \textit{Esta nota es algo tecnico}
\item Basada en el Censo poblacional del estado del 2010\\
\item \textit{Otra cosa tecnica}
\item \textit{Otra cosa técnica}
\end{itemize}

%----------------------------------------------------------------------------------------
%	BIBLIOGRAPHY
%----------------------------------------------------------------------------------------

\renewcommand{\refname}{Reference} % Change the default bibliography title

\bibliography{sample} % Input your bibliography file

%----------------------------------------------------------------------------------------

\end{document}