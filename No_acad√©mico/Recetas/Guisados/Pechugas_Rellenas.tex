\documentclass[letterpaper]{article}
\usepackage{graphicx}
\usepackage[utf8]{inputenc}
\usepackage{natbib}
\usepackage[a4paper, total={6.5in, 10in}]{geometry}
\bibliographystyle{apalike}
%\usepackage{apalike}
\begin{document}

\begin{center}
{\Huge Pechugas Rellenas}\\\\

{\huge de queso y piña}\\

por mi mami\\
\end{center}

\hrulefill
\begin{center}
\textbf{\Large Ingredientes}\\
\end{center}
\hrulefill


\begin{itemize}
\item N Bisteces de pechuga
	\begin{itemize}
		\item Se deben pedir 'bien aplanaditos'
	\end{itemize}
\item Un manojo pequeño de espinaca
	\begin{itemize}
		\item Considerar que por cada Pechuga se utilizarán dos o tres hojitas de espinaca
	\end{itemize}
\item Una lata de piñas en almíbar (de preferencia, en rodajas)
\item Crema 
\item Queso panela
\item Knorr Suiza
\item Aceite de Oliva
\end{itemize}

\hrulefill \\
\begin{center}
\textbf{\Large Instrucciones}
\end{center}
\hrulefill

\begin{enumerate}
\item En un pequeño recipiente preparamos un poco de sazonador, mezclando un chorrito de aceite de oliva con una cucharada de Knorr Suiza
	\begin{itemize}
	\item Este sazonador ayudará a darle más sabor a las pechugas, la idea es que se embarre en la cara interna de las pechugas rellenas, por lo que debe ser poco.
	\end{itemize}
\item Preparamos los rollos de pechuga (crudas)
	\begin{itemize}
	\item  Se unta el sazonador que acabamos de preparar con una brocha o cuchara.
	\item  Se extienden una o dos hojas de espinaca sobre esta misma cara de la pechuga
	\item  Se colocan rebanadas delgadas de queso panela y piña (opcional)
	\item  Utilizando el plástico en que vino el bistec para maniobrar la pechuga, se le hace un rollito, cuidando que en cada vuelta éste quede bien apretadito.
	\end{itemize}
\item Se asan las pechugas en el sartén, de manera que el pliegue del doblez quede viendo hacia abajo (y al cocerse, se cierre).
\end{enumerate}




\end{document}


