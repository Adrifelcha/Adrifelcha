\documentclass[letterpaper]{article}
\usepackage{graphicx}
\usepackage[utf8]{inputenc}
\usepackage[letterpaper, landscape, margin=2in]{geometry}
%\usepackage{natbib}
%\bibliographystyle{apalike}
%\usepackage{apalike}
\begin{document}
\title{Aprendiendo a usar Latex}
\author{por Adriana F Chávez De la Peña}
\maketitle

Hola, bienvenidos a este hermoso ejemplo de como hacer un pdf decente en latex.



Existen tres formas de insertar una ecuacion:



La primera, es dentro de un parrafo. Para ello bsata colocar entre signos de pesos lo que queremos aparezca en letra bonita, por ejemplo: $A+B=C$ \\

La segunda, es crear un espacio llamado equation dentro del texto, (vean el txt de este mismo documento para acceder al codigo). Una de las ventajas de crear un ambiente equation, es que permite agregar una etiqueta a la ecuacion. \\

\begin{equation}
Love=\frac{1561521}{154191}
\label{LoveEq}
\end{equation}

La tercera,  es sin referencia (ver archivo txt)\\


\[Love=\frac{418511248648}{464135641866}\]


Sin embargo, la ventaja de la segunda forma presentada, es que permite referirnos a ella en el texto así:  "Blablah...de acuerdo a la ecuacion~\ref{LoveEq}..."\\

A continuación, insertamos una imagen (ver el codigo en el archivo txt, y la imagen en la Figura~\ref{Gatito1}



\begin{figure}
\centering
\includegraphics[width=0.5\textwidth]{Imagenes/Gatitos.jpeg}
\caption{Este es un Gatito Bonito, como este ejemplo}
\label{Gatito1}
\end{figure}


Todo esto queda claro en mi famosa y aclamada obra \parencite{chavez2016}


\bibliography{Referenceeeees}



\end{document}