%%%%%%%%%%%%%%%%%%%%%%%%%%%%%%%%%%%%%%%%%
% Classicthesis-Styled CV
% LaTeX Template
% Version 1.0 (22/2/13)
%
% This template has been downloaded from:
% http://www.LaTeXTemplates.com
%
% Original author:
% Alessandro Plasmati
%
% License:
% CC BY-NC-SA 3.0 (http://creativecommons.org/licenses/by-nc-sa/3.0/)
%
%%%%%%%%%%%%%%%%%%%%%%%%%%%%%%%%%%%%%%%%%

%----------------------------------------------------------------------------------------
%	PACKAGES AND OTHER DOCUMENT CONFIGURATIONS
%----------------------------------------------------------------------------------------

\documentclass{scrartcl}

\reversemarginpar % Move the margin to the left of the page 
\newcommand{\MarginText}[1]{\marginpar{\raggedleft\itshape\small#1}} % New command defining the margin text style

\usepackage[nochapters]{classicthesis} % Use the classicthesis style for the style of the document
\usepackage[LabelsAligned]{currvita} % Use the currvita style for the layout of the document
\usepackage[spanish]{babel}
\selectlanguage{spanish}
\usepackage[utf8]{inputenc}
\renewcommand{\cvheadingfont}{\LARGE\color{Blue}} % Font color of your name at the top
\usepackage{hyperref} % Required for adding links	and customizing them
\hypersetup{colorlinks, breaklinks, urlcolor=Plum, linkcolor=Aquamarine} % Set link colors

\newlength{\datebox}\settowidth{\datebox}{September 2017} % Set the width of the date box in each block
\newcommand{\NewEntry}[3]{\noindent\hangindent=2em\hangafter=0 \parbox{\datebox}{\small \textit{#1}}\hspace{1.5em} #2 #3 % D\\\\\\efine a command for each new block - change spacing and font sizes here: #1 is the left margin, #2 is the italic date field and #3 is the position/employer/location field
\vspace{0.5em}} % Add some white space after each new entry
\newcommand{\Description}[1]{\hangindent=2em\hangafter=0\noindent\raggedright\footnotesize{#1}\par\normalsize\vspace{1em}} % Define a command for descriptions of each entry - change spacing and font sizes here

%----------------------------------------------------------------------------------------

\begin{document}
\thispagestyle{empty} % Stop the page count at the bottom of the first page

%----------------------------------------------------------------------------------------
%	NAME AND CONTACT INFORMATION SECTION
%----------------------------------------------------------------------------------------
%\noindent
%\vspace{1em}

\begin{cv}{\textbf{\spacedallcaps{Adriana Felisa Chávez De la Peña}}}\vspace{1.5em} % Your name

\hrule{}\vspace{1.5em}

\noindent\spacedsmallallcaps{\textbf{INFORMACIÓN PERSONAL}}\vspace{0.1em}\\ % Personal information heading

\NewEntry{Origen}{\textit{México, CDMX}}{7-Marzo-1993}\\ % Birthplace and date
\NewEntry{email}{\href{mailto:adrifelcha@gmail.com}{adrifelcha@gmail.com}}\\ % Email address
\NewEntry{github}{\href{https://github.com/Adrifelcha}{https://github.com/Adrifelcha}}\\ % Personal website
\NewEntry{teléfono (Casa)}{5686 3442}\\ %\ \ $\cdotp$\ \ (M) 015255 32051765} % Phone number(s)
\NewEntry{teléfono (Cel)}{044 55 3205 1765}\\%\vspace{1.5em}\\ %\ \ $\cdotp$\ \ (M) 015255 32051765} % Phone number(s)

\hrule{}\vspace{1.5em}

\noindent\spacedsmallallcaps{\textbf{ACERCA}}\vspace{0.1em}\\ % Personal information heading

\Description{Undergrad experimental psychologist interested in the study of Perception and Cognitive processes.}\vspace{1em} % Goal text

\hrule{}\vspace{1.5em}

%----------------------------------------------------------------------------------------
%	EDUCATION
%----------------------------------------------------------------------------------------


\textbf{\spacedsmallallcaps{EDUCACIÓN}}\vspace{1em}

\NewEntry{2012-2016}{Universidad Nacional Autónoma de México}

\Description{\MarginText{Licenciatura en Psicología}Promedio: 9.79/10.0\ \ $\cdotp$\ \ \textit{Título pendiente}\ \ $\cdotp$\ \ Facultad de Psicología\newline 
Tesis: \textit{Estudios en Detección de Señales (Studies on Signal Detection )}\newline
Descripción: Esta tesis exploró la extensión del fenómeno reportado en estudios de  Memoria de Reconocimiento donde la Teoría de Detección de Señales es aplicada para analizar el desempeño de los participantes experimentales, a una tarea detección perceptual.\newline
Director: Dr.~Arturo \textsc{Bouzas Riaño} \& Revisor: Dr.~Germán \textsc{Palafox Palafox}}

\vspace{1em} % Extra space between major sectionsía de Detección de Señales es aplicada para analizar el desempeño de los participantes experimentales,


%-------- detección perceptual --------------------------------------------------------------------------------
%	SCHOLARSHIPS
%----------------------------------------------------------------------------------------

\textbf{\spacedsmallallcaps{BECAS}}\vspace{1em}

\NewEntry{Oct-Dec, 2014}{Estudiante de intercambio en la \textsc{Universidad de California, Santa Barbara}}

\Description{\MarginText{University of California,\newline Santa Barbara} Estudió un cuatrimestre en la UCSB como parte de un programa de intercambio, financiado por la UNAM y por el grupo BANAMEX, como consecuencia de su alto promedio.}

\vspace{1em} % Extra space between major sections

%----------------------------------------------------------------------------------------
%	ACADEMIC EXPERIENCE
%----------------------------------------------------------------------------------------

\textbf{\spacedsmallallcaps{EXPERIENCIA ACADÉMICA}}\vspace{1em}

%\NewEntry{2015--2017}{1\textsuperscript{st} English teacher, \textsc{Instituto Mexicano Norteamericano de Cultura}}
%\Description{\MarginText{IMNAC} \\ Reference: John \textsc{McDonald}\ \ $\cdotp$\ \ +1 (000) 111 1111\ \ $\cdotp$\ \ \href{mailto:john@lehman.com}{john@lehman.com}}

\NewEntry{2014 - 2016}{\textsc{Consejera Estudiantil}}

\Description{\MarginText{UNAM: \newline Faculad de Psicología} Elegida por los estudiantes de la Facultad de Psicología para ser su representante en el H. Consejo Técnico en el periodo de Mayo del 2014 a Mayo del 2016.}

\NewEntry{2015--Present}{Miembro del Lab25, con el \textsc{Dr. Arturo Bouzas}}

\Description{\MarginText{UNAM: \newline Faculad de Psicología} Participó en los siguientes proyectos:\newline $\cdotp$ \underline{\textit{PAPIIT IN307214}}.\newline Proyecto: \textit{Aprendizaje en ambientes dinámicos} \newline $\cdotp$ \underline{\textit{PAPIME PE310016}}\newline Proyecto: \textit{Desarrollo de herramientas virtuales para la enseñanza de ciencias cgnitivas y del comportamiento.}}

\vspace{1em} % Extra space between major sections

%----------------------------------------------------------------------------------------
%	WORK EXPERIENCE
%----------------------------------------------------------------------------------------

\textbf{\spacedsmallallcaps{EXPERIENCIA LABORAL}}\vspace{1em}

\NewEntry{Jun--Nov 2017}{Becaria en la \textsc{Asociación Ciencias de la Conducta} -- Mexico City}

\Description{\MarginText{ACC:\newline Asociación Ciencias de la Conducta}Trabajó en ACC, aplicando pruebas psicométricas a niños de entre 8 y 13 años con problemas de conducta, atención y aprendizaje.}

\NewEntry{Jun-Jul 2017}{\textsc{El Buen Socio} --- Veracruz, México}

\Description{\MarginText{El Buen Socio}Trabajó en un proyecto desarrollado con el financiamiento de la Fundación Metlife, en asociación con El Observatorio de Desarrollo Regional y Promoción Social, y la Facultad de Psicología, aplicando instrumentos psicómetricos y cuestionarios sociodemográficos en los habitantes de las diferentes comunidades del estado de Veracruz. \\ Reference: Javier \textsc{Alfaro}\ \ $\cdotp$\ \ +152 (55) 5401 6575\ \ $\cdotp$\ \ \href{mailto:javieralfa@gmail.com}{javieralfa@gmail.com}}

\vspace{1em} % Extra space between major sections

%----------------------------------------------------------------------------------------
%	TEACHING EXPERIENCE
%----------------------------------------------------------------------------------------

\textbf{\spacedsmallallcaps{EXPERIENCIA EN LA ENSEÑANZA}}\vspace{1em}

\NewEntry{2015--2017}{Profesora de inglés en \textsc{Instituto Mexicano Norteamericano de Cultura}}

\Description{\MarginText{Instituto Mexicano NorteAmericano de Cultura}Impartición de cursos dirigidos a estudiantes de secundaria y preparatoria, desarrollados para ejercitar cuatro habilidades principales: lectura, redacción, escucha y discurso.}

\NewEntry{2015--Present}{Estudiante adjunta del \textsc{Dr. Arturo Bouzas Riaño}}

\Description{\MarginText{UNAM:\newline Facultad de Psicología}Estudiante adjunta para los siguientes cursos:\newline $\cdotp$ \underline{Aprendizaje y Conducta Adaptativa I} (2015) \newline $\cdotp$ \underline{Aprendizaje, Motivación y Cognición III} (2015 and 2016) \newline $\cdotp$ \underline{Taller de Investigación I, II and III} (2015-2017)}

\NewEntry{June 2017}{\textsc{Imparticion de Cursos Intersemestrales} --- Facultad de Psicología, UNAM}

\Description{\MarginText{UNAM:\newline Facultad de Psicología}$\cdotp$ Introducción al Pensamiento Estadístico.\newline Curso de una semana (4 horas al día) impartido por Adriana F. Chávez De la Peña and José Manuel Niño García\newline $\cdotp$ Introducción al Modelamiento Bayesiano.\newline Curso de una semana (4 horas al día) impartido por Uriel O. González Bravo, Adriana F. Chávez De la Peña y José Manuel Niño García\newline $\cdotp$ Un laboratorio en Python para Ciencias del Comportamiento. \newline Curso de una semana (4 horas al día) impartido por Adriana F. Chávez De la Peña, Uriel O. González Bravo and José Manuel Niño García}

%about the main basis of statistics (data recolection, data analysis, presentation of results and conclusion construction) and its use and misuse on everyday media
% which covered the basis of Bayesian modeling and its application to cognitive and behavioral sciences
% where the codes and virtual tools developed as part of a project developed within Lab25 (under the guidance of Dr. Arturo Bouzas) were presented in detail

\vspace{1em} % Extra space between major sections

%----------------------------------------------------------------------------------------

%\spacedlowsmallcaps{Publications}\vspace{1em}

%\NewEntry{January 2013}{Publication Title}
%\Description{\MarginText{Full Journal Name}Lorem ipsum dolor sit amet, consectetur adipiscing elit. Ut nisl tellus, sodales non pulvinar in, adipiscing sit amet purus. Suspendisse sed facilisis diam. Sed ornare sem nec justo adipiscing nec venenatis lectus commodo. Mauris non neque ligula. Pellentesque sed quam eu felis iaculis iaculis ac a leo. Suspendisse neque neque, placerat id adipiscing et, elementum eu sem.\\ Authors: John \textsc{Smith}, ~James \textsc{Smith}}

%\vspace{1em} % Extra space between major sections

%------------------------------------------------

%------------------------------------------------


\spacedsmallallcaps{\textbf{PRESENTACIÓN DE POSTERS}}\vspace{1em}

%\NewEntry{November 2015}{La Sensibilidad como fuente de Sesgo en una tarea de detección de señales usando la ilusión de Ebbinghaus}

%\Description{\MarginText{Simposium International of Behavior and its Applications V}\textit{About:} Classical Signal Detection Theory distinguishes between discriminability ($d'$) and system's bias ($\beta$) as two independent factors involved in the production of detection judgments. Lynn and Feldman (2014) suggested a relation between these factors that was explored by means of an experiment where the information gathered about Optical Illusions were used to construct different levels of discriminability. \\ Author: Adriana F. \textsc{Chávez De la Peña}}


\NewEntry{Nov 2015}{\MarginText{Seminario Internacional de Conducta y Aplicaciones V}\newline La Sensibilidad como fuente de Sesgo en una tarea de detección de señales usando la ilusión de Ebbinghaus}

\Description{Autora: Adriana F. \textsc{Chávez De la Peña}}

%\NewEntry{Nov, 2016}{The Mirror Effect within Perception: Not another Recognition Memory study}

%\Description{\MarginText{Object, Perception, Attention and Memory meeting}\textit{About:} A perceptual detection task which uses what is known of the Ebbinghaus illusion to design two classes of stimuli -as levels of $d'$-, is presented to explore the extension of a pattern of response known as the Mirror Effect, reported within Recognition Memory studies where Signal Detection Theory has been applied.\\ Author: Adriana F. \textsc{Chávez De la Peña}}

\NewEntry{Nov, 2016}{\MarginText{Object, Perception, Attention and Memory meeting}\newline The Mirror Effect within Perception: Not another Recognition Memory study}

\Description{\textit{(El Efecto Espejo en Percepción: No es otro estudio de Memoria de Reconocimiento)}Autora: Adriana F. \textsc{Chávez De la Peña}}


\vspace{1em} % Extra space between major sections

%------------------------------------------------

\textbf{\spacedsmallallcaps{CONFERENCIAS Y SIMPOSIOS}}\vspace{1em}

%\NewEntry{Nov, 2017}{El Efecto Espejo en Percepción: No es otro estudio de Memoria de Reconocimiento}

%\Description{\MarginText{Simposium International of Behavior and its Applications VI}\textit{Abstract:} The extensiveness of the Mirror Effect reported in Recognition Memory is explored by the results obtained in a perceptual task, which were assessed both through the replication of the analyses conducted in the literature and with the development of bayesian models.\\ Author: Adriana F. \textsc{Chávez De la Peña}}

\NewEntry{Nov, 2017}{\MarginText{Seminario Internacional de Conducta y Aplicaciones VI}\newline El Efecto Espejo en Percepción: No es otro estudio de Memoria de Reconocimiento}

\Description{Autora: Adriana F. \textsc{Chávez De la Peña}}

\vspace{1em} % Extra space between major sections

%----------------------------------------------------------------------------------------

\textbf{\spacedsmallallcaps{HABILIDADES}}\vspace{1em}

\Description{\MarginText{Psicometría} Construcción y validación de instrumentos psicológicos}
\Description{\MarginText{Comunicación} Experiencia dando clase, presentando posters y conferencias y participando como representante estudiantil}
\Description{\MarginText{Trabajo en Equipo} Trabajó en el Lab en conjunto con otros estudiantes en el desarrollo de diversos proyectos de investigación y docencia}
%\Description{\MarginText{Experimental Framework}\textsc{Pyth enon} (PsychoPy)}
\vspace{1em} % Extra space between major sections


\spacedsmallallcaps{HABILIDADES DE COMPUTACIÓN}}\vspace{1em}

\Description{\MarginText{Redacción de documentos} \LaTeX, OpenOffice}
\Description{\MarginText{Sistemas Operativos}\textsc{Mac}, \textsc{Windows}}
\Description{\MarginText{Análisis y presentación de datos}\textsc{R} (IDE: RStudio), \textsc{Python} (IDE: Spyder), \textsc{SPSS} y \textsc{JASP}}
\Description{\MarginText{Programación Experimental}\textsc{Python} (PsychoPy)}
\vspace{1em} % Extra space between major sections

%------------------------------------------------

\textbf{\spacedsmallallcaps{OTRA INFORMACIÓN}}\vspace{1em}

%\Description{\MarginText{\textbf{Awards}}2012\ \ $\cdotp$\ \ Grade perfection for a full year}
%\vspace{-0.5em} % Negative vertical space to counteract the vertical space between every \Description command
%\Description{2010\ \ $\cdotp$\ \ Top Achiever Award -- Commerce}
%\vspace{1em}
%\Description{\MarginText{\textbf{Communication Skills}}2010\ \ $\cdotp$\ \ Oral Presentation at the California Business Conference}
%\vspace{-0.5em} % Negative vertical space to counteract the vertical space between every \Description command
%\Description{2009\ \ $\cdotp$\ \ Poster at the Annual Business Conference in Oregon}
%\vspace{1em}
\newlength{\langbox} % Create a new length for the length of languages to keep them equally spaced
\settowidth{\langbox}{English} % Length equals the length of "English" - if you have a longer language in your list put it here
\Description{\MarginText{\textbf{Idiomas}}\parbox{\langbox}{\textsc{Español}}\ \ $\cdotp$\ \ \ Lengua Materna}
\vspace{-0.5em} % Negative vertical space to counteract the vertical space between every \Description command
\Description{\parbox{\langbox}{\textsc{Inglés}}\ \ $\cdotp$\ \ \ Avanzado}
\vspace{1em} % Negative vertical space to counteract the vertical space between every \Description command
%\Description{\parbox{\langbox}{\textsc{Dutch}}\ \ $\cdotp$\ \ \ Basic (simple words and phrases only)}
%\vspace{1em} % Negative vertical space to counteract the vertical space between every \Description command

%------------------------------------------------

\Description{\MarginText{\textbf{Intereses}}Psicología Cognitiva y Experimental\ \ $\cdotp$\ \ Psicofísica\ \ $\cdotp$\ \ Toma de Decisiones\ \ $\cdotp$\ \ Desarrollo Cognitivo\ \ $\cdotp$\ \ Psicología Evolutiva}


\begin{center}
\underline{\textbf{Todos los documentos probatorios están disponibles a petición}}
\end{center}

%----------------------------------------------------------------------------------------

\end{cv}

\end{document}